\kkap{DIENSTAG}

\section[DIENSTAG]{AD TERTIAM}

\rot{Hymnus} Nunc sancte nobis Spiritus \rot{p. 124.}
\vspace{0.3cm}

\setspaceafterinitial{4.2mm plus 0em minus 0em}
\setspacebeforeinitial{4.2mm plus 0em minus 0em}
\def\greinitialformat#1{{\fontsize{40}{40}\selectfont #1}}

\gresetfirstlineaboveinitial{\small \textcolor{red}{Ps 119c}}{}
\setaboveinitialseparation{0.72mm}
\setsecondannotation{\small 8. T.}
\includescore{psalmi/118/118viijcogitavi.tex}



\includescore{psalmi/tertia/horatertia_tertia.tex}

\vspace{0.3cm}

\begin{flushleft}

\versik{Dóminus non privábit bonis eos qui ámbulant in innocéntia.}{Dómine virtútum, beátus homo qui sperat in te.}

\medskip

{\rm{
\versik{Der Herr wird seine Güter nicht denen vorenthalten, die in Unschuld wandeln.}{Herr der Scharen, selig der Mensch, der auf dich hofft.}
}}
\end{flushleft}

\medskip

\section{MITTAGESHORE, I. WOCHE}

\rot{Hymnus} Rector potens, verax Deus \rot{p. 126.}

\vspace{0.3cm}
\setspaceafterinitial{4.2mm plus 0em minus 0em}
\setspacebeforeinitial{4.2mm plus 0em minus 0em}
\def\greinitialformat#1{{\fontsize{40}{40}\selectfont #1}}
\gresetfirstlineaboveinitial{\small \textcolor{red}{Ps 9b sq }}{}
\setaboveinitialseparation{0.72mm}
\setsecondannotation{\small 1. T.}


\includescore{psalmi/9/exurgedomine91011.tex}

\medskip

\includescore{psalmi/sexta/sextpsalmitertia.tex}

\begin{flushleft}

\versik{Dómine, veritátem in corde dilexísti.}{Et in occúlto sapiéntiam manifestásti mihi.}

\medskip

{\rm{
\versik{Herr, du liebst, die im Herzen voll Wahrheit sind.}{Und im Verborgenen lehrtest du mich Weisheit.}
}}
\end{flushleft}


\section{MITTAGESHORE, II. WOCHE}

\rot{Hymnus} Rector potens, verax Deus \rot{p. 126.}

\vspace{0.3cm}

\setspaceafterinitial{5.2mm plus 0em minus 0em}
\setspacebeforeinitial{4.2mm plus 0em minus 0em}
\def\greinitialformat#1{{\fontsize{40}{40}\selectfont #1}}
\gresetfirstlineaboveinitial{\small \textcolor{red}{ Ps 126sq }}{}
\setaboveinitialseparation{0.72mm}
\setsecondannotation{\small ij. T.}

\includescore{psalmi/127/beatiomnes127.tex}

\vspace{0.3cm}

\includescore{psalmi/nona/nonpsalmitertia.tex}


\begin{flushleft}

\versik{Fac cum servo tuo secúndum misericórdiam tuam, Dómine.}{Iustificatiónes tuas doce me.}

\medskip

{\rm{
\versik{Handle an deinem Knecht nach deiner Huld.}{Lehre mich deine Gesetze.}
}}
\end{flushleft}



\section[VESPERAE]{AD VESPERAS}



\setspaceafterinitial{5.2mm plus 0em minus 0em}
\setspacebeforeinitial{4.2mm plus 0em minus 0em}
\def\greinitialformat#1{{\fontsize{40}{40}\selectfont #1}}
\gresetfirstlineaboveinitial{\small \textcolor{red}{Ps 116/7}}{}
\setaboveinitialseparation{0.72mm}
\setsecondannotation{\small 3. T.}

\includescore{psalmi/115/votameadomino115.tex}

\vspace{0.6cm}

\psal{116}

\begin{quote}
\begin{verse}
Ich l\d iebe den Herrn; *\\ 
denn er hat mein lautes Flehen gehört\\ 
\vin und sein \d Ohr mir zugeneigt *\\ 
\vin an dem Tag, als ich zu ihm rief.\\ 
Mich umfingen die Fesseln des Todes, †\\
mich befielen die \d Ängste der Unterwelt, *\\  
mich trafen Bedrängnis und Kummer.\\ 
\vin Da rief ich den N\d amen des Herrn an: *\\ 
\vin «Ach Herr, rette mein Leben!»\\
Der Herr ist gn\d ädig und gerecht, *\\ 
unser Gott ist barmherzig.\\ 
\vin Der Herr behütet die schl\d ichten Herzen; *\\ 
\vin ich war in Not und er brachte mir Hilfe.\\ 
Komm wieder zur R\d uhe, mein Herz! *\\ 
Denn der Herr hat dir Gutes getan.\\ 
\vin Ja, du hast mein Leben dem Tod entrissen, †\\ 
\vin meine Tr\d änen getrocknet, *\\ 
\vin  meinen Fuß bewahrt vor dem Gleiten.\\  
So gehe ich meinen W\d eg vor dem Herrn *\\ 
im Land der Lebenden.\\ 
\vin Voll Vertrauen war ich,\\ 
\vin auch w\d enn ich sagte: *\\ 
\vin Ich bin so tief gebeugt.\\ 
In meiner Best\d ürzung sagte ich: *\\ 
Die Menschen lügen alle.\\ 
\vin Wie kann ich dem Herrn \d all das vergelten, *\\ 
\vin was er mir Gutes getan hat?\\
Ich will den Kelch des H\d eiles erheben *\\ 
und anrufen den Namen des Herrn.\\ 
\vin \textit{Ich will dem Herrn\\ 
\vin meine Gel\d übde erfüllen *\\ 
\vin offen vor seinem ganzen Volke.}\\ 
Kostbar ist in den \d Augen des Herrn *\\ 
das Sterben seiner Frommen.\\ 
\vin Ach Herr, ich bin doch dein Knecht, †\\ 
\vin dein Knecht bin ich, der S\d ohn deiner Magd.*\\ 
\vin Du hast meine Fesseln gelöst.\\ 
Ich will dir ein Opfer des D\d ankes bringen *\\ 
und anrufen den Namen des Herr\textit{e}n.\\ 
\vin Ich will dem Herrn meine Gel\d übde erfüllen *\\ 
\vin offen vor seinem ganzen Volk\textit{e}, \\
in den Vorhöfen am H\d aus des Herrn, *\\ 
in deiner Mitte, Jerusalem.\\ 

\end{verse}

\vspace{0.5cm}

\psal{117}

\begin{verse}
 Lobet den H\d errn, alle Völker,*\\
preist ihn, alle Nationen! \\
\vin Denn mächtig waltet über \d uns seine Huld,*\\
\vin die Treue des Herrn währt in Ewigkeit. \\
\end{verse}

\end{quote}

\vspace{0.5cm}

\setspaceafterinitial{4.2mm plus 0em minus 0em}
\setspacebeforeinitial{4.2mm plus 0em minus 0em}
\def\greinitialformat#1{{\fontsize{40}{40}\selectfont #1}}
\gresetfirstlineaboveinitial{\small \textcolor{red}{Ps 137}}{}
\setaboveinitialseparation{0.72mm}
\setsecondannotation{\small viij. T.}

\includescore{psalmi/136/hymnumcantatenobis136.tex}
\vspace{0.5cm}

\psal{137}

\begin{quote}
\begin{verse}
An den Strömen von Babel, †\\
da saßen wir und weinten, *\\  
wenn wir an Zion dachten. \\ 
\vin Wir hängten unsere Harfen *\\ 
\vin an die Weiden in jenem Land\textit{e}. \\
Dort verlangten von uns\\
die Zwingherren Lieder, †\\
unsere Peiniger forderten Jubel: *\\  
\textit{«Singt uns Lieder vom Zion!»}\\ 
\vin Wie könnten wir singen\\ 
\vin die Lieder des Herrn, *\\ 
\vin fern, auf fremder Erde?\\
Wenn ich dich je vergesse, Jerusalem, *\\ 
dann soll mir die rechte Hand verdorren.\\ 
\vin Die Zunge soll mir am Gaumen kleben, †\\ 
\vin wenn ich an dich nicht mehr denke, *\\ 
\vin  wenn ich Jerusalem\\ 
\vin nicht zu meiner höchsten Freude erhebe. \\
\textcolor{red}{(}Herr, vergiss den Söhnen Edoms nicht\\
den Tag von Jerusalem; *\\ 
sie sagten: «Reißt nieder,\\
bis auf den Grund reißt es nieder!»\\ 
\vin Tochter Babel, du Zerstörerin! *\\ 
\vin Wohl dem, der dir heimzahlt,\\ 
\vin was du uns getan hast!\\
Wohl dem, der deine Kinder packt *\\ 
und sie am Felsen zerschmettert!\textcolor{red}{)} \\

\end{verse}
\end{quote}

\vspace{0.6cm}

\setspaceafterinitial{10.2mm plus 0em minus 0em}
\setspacebeforeinitial{4.2mm plus 0em minus 0em}
\def\greinitialformat#1{{\fontsize{40}{40}\selectfont #1}}
\gresetfirstlineaboveinitial{\small \textcolor{red}{Ps 138}}{}
\setaboveinitialseparation{0.72mm}
\setsecondannotation{\small v. T.}

\includescore{psalmi/137/inconspectu.tex}

\vspace{0.3cm}

\psal{138}

\begin{quote}
\begin{verse}
\textit{Ich will} dir danken aus ganzem Herzen, *\\ 
\textit{dir vor den Engeln s\d ingen und spielen;} \\ 
\vin ich will mich niederwerfen\\ 
\vin zu deinem heiligen Tempel hin *\\ 
\vin und deinem Namen danken\\ 
\vin für deine H\d uld und Treue.\\ 
Denn du hast die Worte meines Mundes gehört,*\\  
deinen Namen und dein Wort\\
über \d alles verherrlicht.\\ 
\vin Du hast mich erhört an dem Tag,\\ 
\vin als ich rief; *\\ 
\vin du gabst meiner S\d eele große Kraft.\\ 
Dich sollen preisen, Herr, alle Könige der Welt,*\\ 
wenn sie die Worte deines M\d undes vernehmen.\\ 
\vin Sie sollen singen von den Wegen des Herrn; *\\ 
\vin denn groß ist die H\d errlichkeit des Herrn.\\ 
Ja, der Herr ist erhaben; †\\
doch er schaut auf die Niedrigen, *\\  
und die Stolzen erk\d ennt er von fern\textit{e}.\\ 
\vin Gehe ich auch mitten durch große Not: *\\ 
\vin du erh\d ältst mich am Leben.\\  
Du streckst die Hand aus\\
gegen meine wütenden Feinde, *\\  
und deine R\d echte hilft mir.\\ 
\vin Der Herr nimmt sich meiner an. †\\ 
\vin Herr, deine Huld währt ewig. *\\ 
\vin Lass nicht ab vom W\d erk deiner Hände!\\

\end{verse}
\end{quote}




\setspaceafterinitial{10.2mm plus 0em minus 0em}
\setspacebeforeinitial{4.2mm plus 0em minus 0em}
\def\greinitialformat#1{{\fontsize{40}{40}\selectfont #1}}
\gresetfirstlineaboveinitial{\small \textcolor{red}{Ap 4;5}}{}
\setaboveinitialseparation{0.72mm}
\setsecondannotation{\small 7. T.}

\includescore{cantica/cn/sedebitdominus.tex}

\noindent{\rm{Ant. Der Herr thront als König in Ewigkeit, der Herr segne sein Volk mit Frieden.}}

\bf
\newpage 
\cant{Ap 4,{\footnotesize{11}};5,{\footnotesize{9-12}}}

\begin{quote}
\begin{verse}
Würdig bist du, H\d err, unser Gott, *\\ 
Herrlichkeit zu empfangen und \d Ehre und Macht.\\ 
\vin Denn du bist es,\\ 
\vin der die W\d elt erschaffen hat, *\\ 
\vin durch deinen Willen war sie und wurde durch\\ 
\vin \d ihn erschaffen.\\ 
Herr, d\d u bist würdig, *\\ 
das Buch zu nehmen und seine S\d iegel zu öffnen;\\ 
\vin denn du wurdest geschlachtet †\\ 
\vin und hast mit deinem Blut Menschen für \\ 
\vin G\d ott erworben *\\ 
\vin aus allen Stämmen und Sprachen, aus allen\\ 
\vin Nati\d onen und Völkern, \\
und du hast sie zu Königen und Priestern\\ 
gemacht für \d unseren Gott; *\\  
und sie werden auf der \d Erde herrschen.\\ 
\vin Würdig ist das Lamm, das geschlachtet ist, †\\ 
\vin Macht zu empfangen, R\d eichtum \\ 
\vin und Weisheit, *\\ 
\vin  Kraft und Ehre, L\d ob und Herrlichkeit.\\  

\end{verse}
\end{quote}

\medskip
 
\noindent{\rot{Resp.br.} Adjutorium nostrum {\rot{et Hymnus}  Deus Creator \rot{vel} Lucis Creator \rot{p. 32 - 34.}}

\vspace{0.3cm}

\medskip


\begin{flushleft}

\versik{Dirigátur, Dómine, orátio mea.}{Sicut incénsum in conspéctu tuo.}

\medskip
{\rm{
\versik{Herr, mein Gebet werde gelenkt.}{Wie Weihrauch vor dein Angesicht.}
}}
\end{flushleft}



\medskip

\setspaceafterinitial{5.2mm plus 0em minus 0em}
\setspacebeforeinitial{4.2mm plus 0em minus 0em}
\def\greinitialformat#1{{\fontsize{40}{40}\selectfont #1}}
\gresetfirstlineaboveinitial{\footnotesize \textcolor{red}{Magni}}{}
\setaboveinitialseparation{0.72mm}
\setsecondannotation{\small j. T.}

\includescore{cantica/evangelica/respexisti.tex}

\medskip

\rot{Canticum} Magnificat \rot{p. 128.}

\vspace{0.5cm}


Nach dem Magnificat folgen die Fürbitten, das Vater Unser und das Tagesgebet.
Das Chorgebet endet mit einer Antiphon zu Ehren der heiligen Jungfrau.

\newpage
