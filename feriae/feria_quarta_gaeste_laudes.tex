\kkap{MITTWOCH}

\section[MITTWOCH]{AD LAUDES}

\setspaceafterinitial{5.2mm plus 0em minus 0em}
\setspacebeforeinitial{4.2mm plus 0em minus 0em}
\def\greinitialformat#1{{\fontsize{40}{40}\selectfont #1}}
\gresetfirstlineaboveinitial{\small \textcolor{red}{ Ps 64}}{}
\setaboveinitialseparation{0.72mm}
\setsecondannotation{\small ij. T.}

\includescore{psalmi/63/ps63.tex}

\vspace{0.6cm}

\psal{64}

\begin{quote}
\begin{verse}
Höre, o Gott, mein lautes Klagen, *\\
\textit{schütze mein Leben vor dem Schrecken\\
des Feindes!}\\ 
\vin Verbirg mich vor der Schar der Bösen, *\\ 
\vin vor dem Toben derer, die Unrecht tun.\\
Sie schärfen ihre Zunge wie ein Schwert, *\\
schießen giftige Worte wie Pfeile, \\ 
\vin um den Schuldlosen\\ 
\vin von ihrem Versteck aus zu treffen. *\\ 
\vin Sie schießen auf ihn,\\ 
\vin plötzlich und ohne Scheu.\\ 
Sie sind fest entschlossen zu bösem Tun. †\\
Sie planen, Fallen zu stellen, *\\ 
und sagen: «Wer sieht uns schon?»\\ 
\vin Sie haben Bosheit im Sinn, *\\ 
\vin doch halten sie ihre Pläne geheim.\\ 
Ihr Inneres ist heillos verdorben, *\\ 
ihr Herz ist ein Abgrund.\\ 
\vin Da trifft sie Gott mit seinem Pfeil; *\\ 
\vin sie werden jählings verwundet.\\
Ihre eigene Zunge bringt sie zu Fall.*\\
Alle, die es sehen, schütteln den Kopf.\\ 
\vin Dann fürchten sich alle Menschen; †\\ 
\vin sie verkünden Gottes Taten *\\ 
\vin und bedenken sein Wirken.\\
Der Gerechte freut sich am Herrn\\
und sucht bei ihm Zuflucht. *\\
Und es rühmen sich alle Menschen\\
mit redlichem Herzen.\\
\end{verse}
\end{quote}


\vspace{0.6cm}
\setspaceafterinitial{5.2mm plus 0em minus 0em}
\setspacebeforeinitial{4.2mm plus 0em minus 0em}
\def\greinitialformat#1{{\fontsize{40}{40}\selectfont #1}}
\gresetfirstlineaboveinitial{\small \textcolor{red}{Ps 65}}{}
\setaboveinitialseparation{0.72mm}
\setsecondannotation{\small 8. T.}
\includescore{psalmi/64/ps64.tex}

\vspace{0.6cm}
\psal{65}

\begin{quote}
\begin{verse}
\textit{Dir gebührt Lobgesang,\\
Gott, auf dem Zion,} *\\
dir erfüllt man Gelübde.\\ 
\vin Du erhörst die Gebete. *\\ 
\vin Alle Menschen kommen zu dir\\ 
\vin unter der Last ihrer Sünden.\\ 
Unsere Schuld ist zu groß für uns, *\\
du wirst sie vergeben.\\ 
\vin Wohl denen, die du erwählst\\ 
\vin und in deine Nähe holst, *\\ 
\vin die in den Vorhöfen\\ 
\vin deines Heiligtums wohnen.\\ 
Wir wollen uns am Gut deines Hauses sättigen,*\\ 
am Gut deines Tempels.\\ 
\vin Du vollbringst erstaunliche Taten, *\\ 
\vin erhörst uns in Treue, du Gott unsres Heiles,\\
du Zuversicht aller Enden der Erde *\\ 
und der fernsten Gestade.\\ 
\vin Du gründest die Berge in deiner Kraft, *\\ 
\vin du gürtest dich mit Stärke.\\
Du stillst das Brausen der Meere, *\\
das Brausen ihrer Wogen, das Tosen der Völker.\\ 
\vin Alle, die an den Enden der Erde wohnen, †\\ 
\vin erschauern vor deinen Zeichen; *\\ 
\vin Ost und West erfüllst du mit Jubel.\\
Du sorgst für das Land und tränkst es; *\\
du überschüttest es mit Reichtum.\\ 
\vin Der Bach Gottes ist reichlich gefüllt, *\\ 
\vin du schaffst ihnen Korn; so ordnest du alles.\\
Du tränkst die Furchen, ebnest die Schollen, *\\
machst sie weich durch Regen,\\
segnest ihre Gewächse.\\ 
\vin Du krönst das Jahr mit deiner Güte, *\\ 
\vin deinen Spuren folgt Überfluss.\\
In der Steppe prangen die Auen, *\\
die Höhen umgürten sich mit Jubel.\\ 
\vin Die Weiden schmücken sich mit Herden, †\\ 
\vin die Täler hüllen sich in Korn. *\\ 
\vin Sie jauchzen und singen.\\

\end{verse}
\end{quote}


\vspace{0.6cm}

\setspaceafterinitial{5.2mm plus 0em minus 0em}
\setspacebeforeinitial{4.2mm plus 0em minus 0em}
\def\greinitialformat#1{{\fontsize{40}{40}\selectfont #1}}
\gresetfirstlineaboveinitial{\small \textcolor{red}{Jdt 16}}{}
\setaboveinitialseparation{0.72mm}
\setsecondannotation{\small vij. T.}
\includescore{cantica/ca/iudt16.tex}

\cant{Jdt 16,1f.13-16}

\begin{quote}
\begin{verse}

Stimmt ein Lied an für meinen G\d ott\\
unter Paukenschall, *\\ 
\vin singt für den Herrn \d unter Zimbelklang! \\ 
\vin Preist ihn und s\d ing\textit{e}t sein Lob, *\\  
rühmt seinen Namen und r\d uf\textit{e}t ihn an! \\
Denn der H\d err ist ein Gott, *\\ 
der den Kr\d iegen ein Ende setzt; \\ 
\vin er führte mich heim in sein Lager\\ 
\vin inm\d itten des Volkes *\\ 
\vin  und rettete mich aus der Gew\d alt der Feinde.\\  
Ich singe meinem G\d ott ein neues Lied; *\\ 
\textit{Herr, du bist gr\d oß und} voll Herrlichkeit.\\ 
\vin \textit{Wunderbar bist du in d\d einer Stärke,} *\\ 
\vin keiner kann d\d ich übertreffen.\\  
Dienen muss dir deine g\d anze Schöpfung. *\\ 
Denn du hast gesprochen und \d alles entstand.\\ 
\vin Du sandtest deinen Geist, um den \\ \vin B\d au zu vollenden.  *\\ 
\vin  Kein Mensch kann deinem W\d ort widerstehen.\\ 
Meere und Berge erbeben in ihrem Grund, †\\
vor dir zerschmelzen die F\d elsen wie Wachs. *\\  
Doch wer dich fürchtet, der erfährt d\d eine Gnade.\\ 
\vin Zu gering ist jedes Opfer, um \\ \vin dich zu erfreuen, †\\ 
\vin alle Fettstücke sind nichts \\ \vin beim \d Opfer für dich. *\\ 
\vin  Wer den Herrn fürchtet, der \\ \vin ist gr\d oß für immer.\\ 

\end{verse}
\end{quote}

\rot{vel}

\setspaceafterinitial{5.2mm plus 0em minus 0em}
\setspacebeforeinitial{4.2mm plus 0em minus 0em}
\def\greinitialformat#1{{\fontsize{40}{40}\selectfont #1}}
\gresetfirstlineaboveinitial{\small \textcolor{red}{1 Sam 2}}{}
\setaboveinitialseparation{0.72mm}
\setsecondannotation{\small j. T.}
\includescore{cantica/ca/1sam.tex}

\cant{1Sam 2,1-10}

\begin{quote}
\begin{verse}

 Mein Herz ist voll Freude über den Herrn, *\\
große Kraft gibt mir der Herr.\\
\vin Weit öffnet sich mein Mund gegen \\ \vin meine Feinde; *\\
\vin Denn ich freue mich über deine Hilfe.\\
Niemand ist heilig, nur der Herr, †\\
Denn außer dir gibt es keinen Gott; *\\
Keiner ist ein Fels wie unser Gott.\\
\vin Redet nicht immer so vermessen, *\\
\vin Kein freches Wort entfahre eurem Mund\textit{e}.\\
Denn der Herr ist ein wissender Gott*\\
und bei ihm werden die Taten geprüft.\\

\vin Der Bogen der Helden wird zerbrochen,*\\
\vin die Wankenden aber gürten sich mit Kraft.\\
Die Satten verdingen sich um Brot,*\\
doch die Hungrigen können feiern für immer.\\
\vin Die Unfruchtbare bekommt sieben Kinder, *\\
\vin doch die Kinderreiche welkt dahin.\\
Der Herr macht tot und macht lebendig.*\\
er führt ins Totenreich hinab und \\ führt auch herauf.\\
\vin Der Herr macht arm und macht reich, *\\
\vin er erniedrigt und er erhöht.\\
Den Schwachen hebt er empor aus dem Staub *\\
und erhöht den Armen, der im Schmutz liegt;\\ 
\vin Er gibt ihm einen Sitz bei den Edlen, *\\
\vin einen Ehrenplatz weist er ihm zu.\\
Ja, dem Herrn gehören die Pfeiler der Erde; *\\
auf sie hat er den Erdkreis gegründet.\\
\vin Er behütet die Schritte seiner Frommen, *\\
\vin doch die Frevler verstummen in der  \\ \vin  Finsternis; \\
denn der Mensch ist nicht stark aus \\eigener Kraft.*\\
Wer gegen den Herrn streitet, wird zerbrechen.\\
\vin der Höchste lässt es donnern am Himmel.*\\
\vin \textit{Der Herr hält Gericht bis an die \\ \vin  Grenzen der Erde.}\\
Seinem König gebe er Kraft*\\
und erhöhe die Macht seines Gesalbten.\\
\end{verse}

\end{quote}







\vspace{0.6cm}

\setspaceafterinitial{4.2mm plus 0em minus 0em}
\setspacebeforeinitial{4.2mm plus 0em minus 0em}
\def\greinitialformat#1{{\fontsize{40}{40}\selectfont #1}}
\gresetfirstlineaboveinitial{\small \textcolor{red}{Ps 67}}{}
\setaboveinitialseparation{0.72mm}
\setsecondannotation{\small vij. T.}

\includescore{psalmi/66/deusmisereaturnostri.tex}

\vspace{0.3cm}
\psal{67}
\begin{quote}
 
\begin{verse}
 Gott sei uns gn\d ädig und segne uns.  *\\ 
Er lasse über uns sein \d Angesicht leuchten,\\ \vin
\textit{damit auf Erden sein W\d eg erkannt wird *\\ \vin 
und unter allen V\d ölkern sein Heil.}\\ 
Die Völker sollen dir d\d anken, o Gott, *\\ 
danken sollen dir die V\d ölker alle.\\ \vin 
Die Nationen sollen sich fr\d euen und jubeln. *\\ \vin 
Denn du richtest den \d Erdkreis gerecht.\\  
Du richtest die V\d ölker nach Recht *\\  
und regierst die Nati\d onen auf Erden.\\ \vin  
Die Völker sollen dir d\d anken, o Gott, *\\ \vin 
danken sollen dir die V\d ölker alle.\\ 
Das Land gab s\d einen Ertrag.  *\\ 
Es segne \d uns Gott, unser Gott.\\ \vin  
Es s\d egne uns Gott.  *\\ \vin 
Alle Welt f\d ürchte und ehre ihn.\\  


\end{verse}
\end{quote}

\noindent\rot{Resp.br.} Sana animam \rot{und Hymnus} Splendor paternæ \rot{oder} Ecce iam noctis \rot{wie am Montag pp. 18ff.}\\
\begin{flushleft}

\versik{Repléti sumus mane misericórdia tua.}{Exultávimus, et delectáti sumus.}

\medskip

{\rm{
\versik{Am Morgen sind wir erfüllt von deiner Huld.}{Wir sind (voll) Jubel und Frohsinn.}
}}
\end{flushleft}

\setspaceafterinitial{5.2mm plus 0em minus 0em}
\setspacebeforeinitial{4.2mm plus 0em minus 0em}
\def\greinitialformat#1{{\fontsize{40}{40}\selectfont #1}}
\gresetfirstlineaboveinitial{\small \textcolor{red}{Bened}}{}
\setaboveinitialseparation{0.72mm}
\setsecondannotation{\small vij. T.}

\includescore{cantica/evangelica/insanctitate.tex}

\vspace{0.2cm}

\rot{Canticum} Benedictus \rot{p. 64.}


