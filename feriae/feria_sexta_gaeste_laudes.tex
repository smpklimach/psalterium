\kkap{FREITAG}

\section[FREITAG]{AD LAUDES}

\setspaceafterinitial{3.2mm plus 0em minus 0em}
\setspacebeforeinitial{4.2mm plus 0em minus 0em}
\def\greinitialformat#1{{\fontsize{40}{40}\selectfont #1}}
\gresetfirstlineaboveinitial{\small \textcolor{red}{ Ps 51 }}{}
\setaboveinitialseparation{0.72mm}
\setsecondannotation{\small vj. T.}

\includescore{psalmi/50/misereremei.tex}

\vspace{0.3cm}
\crot{Psalm 51}

\begin{quote}
 

\begin{verse}
 \textit{Gott, sei mir gnädig} nach deiner Huld, *\\
tilge meine Frevel nach deinem r\d eichen\\ Erbarmen! \\
\vin Wasch meine Schuld von mir ab *\\
\vin und mach mich rein v\d on meiner Sünde!\\
Denn ich erkenne meine bösen Taten, *\\
meine Sünde steht mir i\d mmer vor Augen. \\
\vin Gegen dich allein habe ich gesündigt, *\\
\vin ich habe g\d etan, was dir missfällt. \\
So behältst du recht mit deinem Urteil, *\\ rein stehst d\d u da als Richter. \\
\vin Denn ich bin in Schuld geboren; *\\
\vin in Sünde hat mich meine M\d utter empfangen. \\
Lauterer Sinn im Verborgenen gefällt dir, *\\
im Geheimen l\d ehrst du mich Weisheit.\\
\vin Entsündige mich mit Ysop, dann werde \\ \vin ich rein; *\\
\vin wasche mich, dann werde \d ich weißer\\ \vin als Schnee. \\
Sättige mich mit Entzücken und Freude! *\\
Jubeln sollen die Glieder, d\d ie du \\zerschlagen hast.\\
\vin Verbirg dein Gesicht vor meinen Sünden, *\\
\vin tilge \d all meine Frevel!\\
Erschaffe mir, Gott, ein reines Herz *\\
und gib mir einen neuen, b\d eständigen Geist! \\
\vin Verwirf mich nicht von deinem Angesicht *\\
\vin und nimm deinen heilig\d en Geist nicht von mir! \\
Mach mich wieder froh mit deinem Heil *\\
mit einem willigen G\d eist rüste mich aus!\\
\vin Dann lehre ich Abtrünnige deine Wege *\\
\vin und die Sünd\d er kehren um zu dir.\\
Befrei mich von Blutschuld, Herr,\\ du Gott meines Heiles, *\\
dann wird meine Zunge jubeln über d\d eine \\Gerechtigkeit.\\
\vin Herr, öffne mir die Lippen *\\
\vin und mein Mund wird dein\d en Ruhm \\ \vin verkünden.\\
Schlachtopfer willst du nicht, ich würde sie\\ dir geben; *\\
an Brandopfern hast d\d u kein Gefallen. \\
\vin Das Opfer, das Gott gefällt, ist \\ \vin ein zerknirschter Geist, *\\
\vin ein zerbrochenes und zerschlagenes\\ \vin  Herz wirst du, G\d ott, nicht verschmähen.\\ 
In deiner Huld tu Gutes an Zion; *\\
bau die Mauern Jer\d usalems wieder auf! \\
\vin Dann hast du Freude an rechten Opfern, †\\
\vin an Brandopfern und Ganzopfern, *\\ \vin dann opfert man Stiere \d auf deinem Altar. \\

\end{verse}
\end{quote}


\vspace{0.cm}
\setspaceafterinitial{3.2mm plus 0em minus 0em}
\setspacebeforeinitial{4.2mm plus 0em minus 0em}
\def\greinitialformat#1{{\fontsize{40}{40}\selectfont #1}}
\gresetfirstlineaboveinitial{\small \textcolor{red}{Ps 92}}{}
\setaboveinitialseparation{0.72mm}
\setsecondannotation{\small 8. T.}
\includescore{psalmi/91/bonumest.tex}

\vspace{0.3cm}


\crot{Psalm 92}
\begin{quote}
\begin{verse}
Wie \textit{schön ist es, dem Herrn zu danken,} *\\
deinem Namen, du Höchster, zu singen, \\
\vin am Morgen deine Huld zu verkünden *\\
\vin und in den Nächten deine Treue\\
zur zehnsaitigen Laute, zur Harfe, *\\
zum Klang der Zither.\\
\vin Denn du hast mich durch deine Taten\\ \vin froh gemacht; *\\
\vin Herr, ich will jubeln über die Werke\\ deiner Hände.\\
Wie groß sind deine Werke, o Herr, *\\
wie tief deine Gedanken! \\
\vin Ein Mensch ohne Einsicht erkennt \\ \vin das nicht, *\\
\vin ein Tor kann es nicht verstehen. \\
Wenn auch die Frevler gedeihen †\\
und alle, die Unrecht tun, wachsen, *\\ so nur, damit du sie für immer vernichtest.\\
\vin Herr, du bist der Höchste, *\\
\vin du bleibst auf ewig.\\
Doch deine Feinde, Herr, wahrhaftig,\\ deine Feinde vergehen; *\\
auseinander getrieben werden alle, die\\ Unrecht tun. \\
\vin Du machtest mich stark wie einen Stier, *\\
\vin du salbtest mich mit frischem Öl.\\
Mein Auge blickt herab auf meine Verfolger, †\\
auf alle, die sich gegen mich erheben; *\\ mein Ohr hört vom Geschick der Bösen. \\
\vin Der Gerechte gedeiht wie die Palme, *\\
\vin er wächst wie die Zedern des Libanon. \\
Gepflanzt im Haus des Herrn, *\\
gedeihen sie in den Vorhöfen unseres Gottes. \\
\vin Sie tragen Frucht noch im Alter *\\
\vin und bleiben voll Saft und Frische;\\
sie verkünden: Gerecht ist der Herr; *\\
mein Fels ist er, an ihm ist kein Unrecht. \\

\end{verse}
\end{quote}


\vspace{0.3cm}

\setspaceafterinitial{5.2mm plus 0em minus 0em}
\setspacebeforeinitial{4.2mm plus 0em minus 0em}
\def\greinitialformat#1{{\fontsize{40}{40}\selectfont #1}}
\gresetfirstlineaboveinitial{\small \textcolor{red}{Is 45}}{}
\setaboveinitialseparation{0.72mm}
\setsecondannotation{\small v. T.}
\includescore{cantica/ca/is45.tex}

\cant{Is 45,15-25}

\begin{quote}
\begin{verse}
Wahrhaftig, du bist ein verborgener Gott. *\\
Israels G\d ott ist der Retter.\\
\vin Schmach und Schande kommt über sie alle, *\\
\vin die Götzenschmiede ger\d aten in Schande.\\
Israel aber wird vom Herrn gerettet, *\\
wird für \d immer errettet. \\
\vin Über euch kommt keine Schande und\\ \vin  Schmach mehr *\\ 
\vin für \d immer und ewig.\\
Denn so spricht der Herr, der den\\ Himmel erschuf, *\\
Ich bin der Herr und sonst niemand. \\
\vin Ich bin der Herr, der die Wahrheit spricht *\\ \vin und der verk\d ündet, was recht ist. \\
Wer hölzerne Götzen umherträgt, hat\\ keine Erkenntnis, *\\ wer einen Gott anbetet, der n\d iemanden rettet. \\
\vin Es gibt keinen Gott außer mir; *\\
 \vin außer mir gibt es keinen gerechten\\ \vin  und r\d ettenden Gott.\\
Wendet euch mir zu und lasst euch erretten, †\\
ihr Menschen aus den fernsten Ländern\\ der Erde; *\\ denn ich bin G\d ott und sonst niemand.\\
\vin Ich habe bei mir selbst geschworen †\\
\vin und mein Mund hat die Wahrheit\\ \vin gesprochen, *\\ \vin es ist ein unwiderr\d ufliches Wort: \\
Vor mir wird jedes Knie sich beugen †\\ und jede Zunge w\d ird bei mir schwören: *\\
Nur beim Herrn gibt es R\d ettung und Schutz. \\
\vin Beschämt kommen alle zu ihm, die sich ihm\\ \vin widersetzten. †\\
\textit{Alle Nachkommen Israels bekommen\\ ihr Recht *\\
und erlangen R\d uhm durch den Herren.}\\



\end{verse}

\end{quote}

\vspace{0.3cm}

\rot{vel}

\setspaceafterinitial{5.2mm plus 0em minus 0em}
\setspacebeforeinitial{4.2mm plus 0em minus 0em}
\def\greinitialformat#1{{\fontsize{40}{40}\selectfont #1}}
\gresetfirstlineaboveinitial{\small \textcolor{red}{Hab 3}}{}
\setaboveinitialseparation{0.72mm}
\setsecondannotation{\small iv. T.}
\includescore{cantica/ca/hab3.tex}

\vspace{0.3cm}

\cant{Hab 3,2f.13.15f}

\begin{quote}
\begin{verse}
 

\textit{Herr, ich höre die Kunde}, *\\ 
ich sehe, Herr, was du fr\d üher getan hast\\ \vin  
Lass es in diesen Jahren wieder geschehen, †\\ \vin offenbare es in diesen Jahren! *\\ \vin  Auch wenn du zürnst, d\d enk an\\ \vin  dein Erbarmen!\\
Gott kommt von Teman her, *\\ 
der Heilige kommt vom G\d ebirge Paran\\ \vin  
Seine Hoheit überstrahlt den Himmel, *\\ \vin  sein Ruhm \d erfüllt die Erde\\  

Er leuchtet wie das Licht der Sonne, † \\ 
ein Kranz von Strahlen umgibt ihn, *\\  in ihnen v\d erbirgt sich seine Macht\\ \vin 
Du ziehst aus, um dein Volk zu retten, *\\ \vin 
um deinem Ges\d albten zu helfen\\  
Du bahnst mit deinen Rossen den Weg\\ durch das Meer, *\\ 
durch das gewaltig sch\d äumende Wasser\\ \vin  
Ich zitterte am ganzen Leib, als \\ \vin ich es hörte, *\\ \vin 
ich vernahm d\d en Lärm und ich schrie.\\  
Fäulnis befällt meine Glieder *\\  und es wank\d en meine Schritte\\ \vin  
Doch in Ruhe erwarte ich den Tag der Not, *\\ \vin  der dem Volk bevorsteht, das \d über \\ \vin uns herfällt\\ 
Zwar blüht der Feigenbaum nicht, *\\ 
an den Reben \d ist nichts zu ernten, \\ \vin
der Ölbaum bringt keinen Ertrag, *\\ \vin  die Kornfeld\d er tragen keine Frucht;\\
 im Pferch sind keine Schafe, *\\  im St\d all steht kein Rind mehr\\ \vin  
Dennoch will ich jubeln über den Herrn *\\ \vin 
und mich freuen über G\d ott, meinen Retter\\ 
Gott, der Herr, ist meine Kraft. †\\
Er macht meine Füße schnell wie die Füße \\der Hirsche *\\  und lässt mich schreit\d en auf den Höhen.\\  
\end{verse}

\end{quote}

\vspace{0.3cm}

\setspaceafterinitial{5.2mm plus 0em minus 0em}
\setspacebeforeinitial{4.2mm plus 0em minus 0em}
\def\greinitialformat#1{{\fontsize{40}{40}\selectfont #1}}
\gresetfirstlineaboveinitial{\small \textcolor{red}{Ps 149}}{}
\setaboveinitialseparation{0.72mm}
\setsecondannotation{\small 8. T.}


\includescore{psalmi/149/ps149_filiision.tex}

\vspace{0.3cm}

\crot{Psalm 149}

\begin{quote}

\begin{verse}
 Singt dem Herrn ein neues Lied! *\\
Sein Lob erschalle in der Gemeinde \\der Frommen. \\
\vin Israel soll sich über seinen Schöpfer freuen, *\\
\vin \textit{die Kinder Zions über ihren König jauchzen.} \\
Seinen Namen sollen sie loben beim\\ Reigentanz, *\\
ihm spielen auf Pauken und Harfen. \\
\vin Der Herr hat an seinem Volk Gefallen, *\\
\vin die Gebeugten krönt er mit Sieg. \\
In festlichem Glanz sollen die Frommen \\frohlocken, *\\
auf ihren Lagern jauchzen:\\
\vin Loblieder auf Gott in ihrem Mund, *\\
\vin ein zweischneidiges Schwert in der Hand, \\
um die Vergeltung zu vollziehn an \\den Völkern, *\\
an den Nationen das Strafgericht,\\
\vin um ihre Könige mit Fesseln zu binden, *\\
\vin ihre Fürsten mit eisernen Ketten,\\
um Gericht über sie zu halten, so\\ wie geschrieben steht. *\\
Herrlich ist das für all seine Frommen. \\
\end{verse}

\end{quote}


\noindent\rot{Resp.br.} Sana animam \rot{und Hymnus} Splendor paternæ \rot{oder} Ecce iam noctis \rot{wie am Montag pp. 18ff.}\\


\begin{flushleft}

\versik{Repléti sumus mane misericórdia tua.}{Exultávimus, et delectáti sumus.}

\medskip

{\rm{
\versik{Am Morgen sind wir erfüllt von deiner Huld.}{Wir sind (voll) Jubel und Frohsinn.}
}}
\end{flushleft}
\vspace{0.3cm}
\setspaceafterinitial{5.2mm plus 0em minus 0em}
\setspacebeforeinitial{4.2mm plus 0em minus 0em}
\def\greinitialformat#1{{\fontsize{40}{40}\selectfont #1}}
\gresetfirstlineaboveinitial{\small \textcolor{red}{Benedic}}{}
\setaboveinitialseparation{0.72mm}
\setsecondannotation{\small viij. T.}

\includescore{cantica/evangelica/perviscera.tex}

\vspace{0.3cm}
\rot{Canticum} Benedictus \rot{p. 64.}