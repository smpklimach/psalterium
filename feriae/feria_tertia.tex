\thispagestyle{plain}

\kapklein{\rot{F}eria \rot{T}ertia}
\kkap{FERIA TERTIA}


\section[VIGILIAE]{AD VIGILIAS}


\setspaceafterinitial{5.2mm plus 0em minus 0em}
\setspacebeforeinitial{4.2mm plus 0em minus 0em}
\def\greinitialformat#1{{\fontsize{40}{40}\selectfont #1}}
\gresetfirstlineaboveinitial{\small \textcolor{red}{Invitat.}}{Invitat.}
\setaboveinitialseparation{0.72mm}
%\setsecondannotation{\small Ps. 125}

%\includescore{invitatoria/veniteiii.tex}

\medskip

\begin{sloppypar}
{\noindent{\rot{Invit.} Lasst uns Gott zujubeln, * unserem Heil.}}
\end{sloppypar}

\bigskip

\noindent{\rot{Hymnus} Æterne rerum Conditor \rot{vel} Nocte surgentes \rot{ut in Dominica pp. 1-3.}}


\section{VIGILIA I}

\begin{sloppypar}

{\noindent{1. Ant.} Der Herr * ist mein Licht und mein Heil. \rot{Ps. 27}\\
2. Ant. Der Herr * beschützt mein Leben. \rot{Ps. 28}\\
3. Ant. Betet an den Herrn * in seinem heiligen Tempel. \rot{Ps. 29}\\}
\end{sloppypar}

\begin{flushleft}

\versik{Kommt und seht die Werke des Herrn.}{Der Wunderbares tut auf der Erde.}

\end{flushleft}


\section{VIGILIA II}

\begin{sloppypar}
{\noindent{1. Ant.} Neigt euer Ohr * die ihr den Erdkreis bewohnt\-. \rot{Ps. 49}\\
2. Ant. Der Gott der Götter, * der Herr, spricht.\\ \rot{Ps. 50}\\
3. Ant. Wir preisen ihn * in aller Ewigkeit. \rot{Ps. 52}}
\end{sloppypar}

\begin{flushleft}

\versik{Höre mein Volk und ich werde sprechen.}{Ich der ich dein Gott bin.}

\end{flushleft}

\section{VIGILIA III}

\begin{sloppypar}

{\noindent{1. Ant.} Du hältst mich Herr * an meiner Hand.\\ \rot{Ps. 73}\\
2. Ant. Du hast dein Erbe *  und dein Volk befreit. \rot{Ps. 74}\\
3. Ant. Du bist ein Gott * der Wunder wirkt. \rot{Ps. 75}\\}

\end{sloppypar}

\begin{flushleft}

\versik{Gott, dein Weg ist heilig.}{Wer ist groß wie unser Gott?}

\end{flushleft}


\section{VIGILIA IV}

\begin{sloppypar}

{\noindent{1. Ant.} Singt dem Herrn * und preist seinen Namen. \\ \rot{Ps. 96}\\
2. Ant. Der Herr ist König * es jubelt die Erde und es freuen sich die vielen Inseln. \rot{Ps. 97}\\
3. Ant. Denn Wunderbares * hat der Herr getan.\\ \rot{Ps. 98}\\
4. Ant. Jubelt zu,  dem Herrn, unserem Gott, * und betet ihn an auf seinem heiligen Berg. \rot{Ps. 99}\\
5. Ant. Jubelt Gott zu, * alle Lande. \rot{Ps. 100}\\
6. Ant.  Dir, o Herr, will ich singen * und ich lerne deine Wege. \rot{Ps. 101}\\}

\end{sloppypar}

\begin{flushleft}

\versik{Groß ist der Herr auf Sion.}{Erhaben über alle Völker.}

\end{flushleft}


\section[LAUDES]{AD LAUDES}

\setspaceafterinitial{5.2mm plus 0em minus 0em}
\setspacebeforeinitial{4.2mm plus 0em minus 0em}
\def\greinitialformat#1{{\fontsize{40}{40}\selectfont #1}}
\gresetfirstlineaboveinitial{\small \textcolor{red}{ Ps 43}}{}
\setaboveinitialseparation{0.72mm}
\setsecondannotation{\small vj. T.}

\includescore{psalmi/42/ps42.tex}

\vspace{0.6cm}
\psal{43}

\begin{quote}
\begin{verse}
Verschaff mir Recht, o Gott, †\\
und führe meine Sache gegen ein \\
treuloses Volk! *\\  
Rette mich vor bösen und t\d ückischen Menschen!\\ 
\vin Denn du bist mein starker Gott. *\\ 
\vin Warum hast d\d u mich verstoßen? \\
Warum muss ich trauernd umhergehen, *\\  
v\d on meinem Feind bedrängt?\\ 
\vin Sende dein Licht und deine Wahrheit, *\\ 
\vin dam\d it sie mich leiten; \\
sie sollen mich führen zu deinem heiligen Berg *\\  
und z\d u deiner Wohnung. \\ 
\vin So will ich zum Altar Gottes treten,\\ 
\vin zum Gott meiner Freude. *\\ 
\vin Jauchzend will ich dich auf der Harf\d e loben,\\ 
\vin Gott, mein Gott. \\
Meine Seele, warum bist du betrübt *\\ 
und bist s\d o unruhig in mir? \\ 
\vin Harre auf Gott; denn\\ 
\vin ich werde ihm noch danken, *\\ 
\vin  meinem \textit{Gott und Retter,\\ 
\vin \d auf den ich schaue.}\\

\end{verse}
\end{quote}


\vspace{0.6cm}
\setspaceafterinitial{5.2mm plus 0em minus 0em}
\setspacebeforeinitial{4.2mm plus 0em minus 0em}
\def\greinitialformat#1{{\fontsize{40}{40}\selectfont #1}}
\gresetfirstlineaboveinitial{\small \textcolor{red}{Ps 57}}{}
\setaboveinitialseparation{0.72mm}
\setsecondannotation{\small viij. T.}
\includescore{psalmi/56/ps56.tex}

\vspace{0.6cm}
\psal{57}

\begin{quote}
\begin{verse}
 Sei mir gnädig, o Gott, sei mir gnädig; *\\
\textit{denn ich flüchte mich zu dir.}\\ 
\vin Im Schatten deiner Flügel\\ 
\vin finde ich Zuflucht, *\\ 
\vin bis das Unheil vorübergeht.\\ 
Ich rufe zu Gott, dem Höchsten, *\\
zu Gott, der mir beisteht.\\ 
\vin Er sende mir Hilfe vom Himmel; †\\ 
\vin meine Feinde schmähen mich. *\\ 
\vin Gott sende seine Huld und Treue.\\
Ich muss mich mitten unter Löwen lagern, *\\
die gierig auf Menschen sind.\\ 
\vin Ihre Zähne sind Spieße und Pfeile, *\\ 
\vin ein scharfes Schwert ihre Zunge.\\ 
Sie haben meinen Schritten ein Netz gelegt *\\ 
und meine Seele gebeugt.\\ 
\vin Sie haben mir eine Grube gegraben; *\\ 
\vin doch fielen sie selbst hinein.\\
Erheb dich über die Himmel, o Gott! *\\
Deine Herrlichkeit erscheine\\
über der ganzen Erde.\\
\vin Mein Herz ist bereit, o Gott, †\\ 
\vin mein Herz ist bereit, *\\ 
\vin ich will dir singen und spielen.\\ 
Wach auf, meine Seele! †\\
Wacht auf, Harfe und Saitenspiel! *\\ 
Ich will das Morgenrot wecken.\\ 
\vin Ich will dich vor den Völkern preisen, Herr, *\\ 
\vin dir vor den Nationen lobsingen.\\ 
Denn deine Güte reicht,\\
so weit der Himmel ist, *\\
deine Treue, so weit die Wolken ziehn.\\ 
\vin Erheb dich über die Himmel, o Gott; *\\ 
\vin deine Herrlichkeit erscheine\\ 
\vin über der ganzen Erde.\\

\end{verse}
\end{quote}


\vspace{0.6cm}

\setspaceafterinitial{5.2mm plus 0em minus 0em}
\setspacebeforeinitial{4.2mm plus 0em minus 0em}
\def\greinitialformat#1{{\fontsize{40}{40}\selectfont #1}}
\gresetfirstlineaboveinitial{\small \textcolor{red}{Tob 13}}{}
\setaboveinitialseparation{0.72mm}
\setsecondannotation{\small vij. T.}
\includescore{cantica/ca/tob13.tex}


\cant{Tob 13,2-5.7-9}

\begin{quote}
\begin{verse}
Gepriesen sei Gott, der in \d Ewigkeit lebt, *\\
sein Königtum s\d ei gepriesen.\\ 
\vin Er züchtigt und hat auch wieder Erbarmen; †\\ 
\vin er führt hinab in die Unterwelt \\ 
\vin und führt auch w\d ieder zum Leben.  *\\ 
\vin Niemand kann seiner M\d acht entfliehen.\\ 
Bekennt euch zu ihm vor allen Völkern,\\
ihr K\d inder Israels; *\\ 
denn er selbst hat uns unter die V\d ölker zerstreut.\\ 
\vin Verkündet dort seine erh\d abene Größe, *\\ 
\vin preist ihn laut vor \d allem, was lebt.\\ 
Denn er ist \d unser Herr und Gott, *\\ 
er ist unser Vater in \d alle Ewigkeit.\\ 
\vin Er züchtigt uns wegen \d unserer Sünden, *\\ 
\vin doch hat er auch w\d ieder Erbarmen.\\ 
Wenn ihr dann seht, was er für euch tut,  †\\
bekennt euch laut und \d offen zu ihm! *\\ 
Preist den Herrn der Gerechtigkeit,\\
\textit{rühmt den \d ewigen König!}\\ 
\vin Ich bekenne mich zum Herrn im L\d and \\ 
\vin der Verbannung, *\\ 
\vin ich bezeuge den Sündern seine Macht\\ 
\vin und erh\d abene Größe.\\ 
Kehrt um, ihr Sünder,  †\\
tut, was recht ist in s\d einen Augen. *\\ 
Vielleicht ist er gnädig und hat mit \d euch\\
Erbarmen.\\ 
\vin Ich will meinen Gott rühmen,\\ 
\vin den K\d önig des Himmels, *\\ 
\vin meine Seele freut sich\\
\vinüber die erhabene Größe m\d eines Gottes.\\

\end{verse}
\end{quote}



\rot{vel}


\setspaceafterinitial{5.2mm plus 0em minus 0em}
\setspacebeforeinitial{4.2mm plus 0em minus 0em}
\def\greinitialformat#1{{\fontsize{40}{40}\selectfont #1}}
\gresetfirstlineaboveinitial{\small \textcolor{red}{Is 38}}{}
\setaboveinitialseparation{0.72mm}
\setsecondannotation{\small j. T.}
\includescore{cantica/ca/is38.tex}

\cant{Is 38,10-14.17-20}

\begin{quote}
\begin{verse}
Ich sagte: In der Mitte meiner Tage †\\
muss ich hinab zu den Pforten der Unterwelt, *\\
man raubt mir den Rest meiner Jahre.\\
\vin Ich sagte: Ich darf den Herrn nicht mehr\\ 
\vin schauen im Land der Lebenden,*\\
\vin keinen Menschen mehr sehen bei den\\ 
\vin Bewohnern der Erde.\\
Meine Hütte bricht man über mir ab,*\\
man schafft sie weg wie das Zelt eines Hirten.\\
\vin Wie ein Weber hast du mein Leben \\ 
\vin zu Ende gewoben, *\\
\vin du schneidest mich ab\\ 
\vin wie ein fertig gewobenes Tuch.\\
Vom Anbruch des Tages bis in die Nacht\\
gibst du mich preis; *\\
bis zum Morgen schreie ich um Hilfe.\\
\vin Meine Augen blicken ermattet nach oben: *\\
\vin Ich bin in Not, Herr. Steh mir bei!\\
\textit{Du hast mich\\
aus meiner bitteren Not gerettet, †\\
du hast mich\\
vor dem tödlichen Abgrund bewahrt}; *\\
denn all meine Sünden\\
warfst du hinter deinen Rücken.\\
\vin Ja, in der Unterwelt dankt man dir nicht, †\\ 
\vin die Toten loben dich nicht; *\\
\vin wer ins Grab gesunken ist,\\ 
\vin kann nichts mehr von deiner Güte erhoffen.\\
Nur die Lebenden danken dir,\\
wie ich am heutigen Tag. *\\
Von deiner Treue erzählt der Vater den Kindern.\\
\vin Der Herr war bereit, mir zu helfen. *\\
\vin Wir wollen singen und spielen im Haus des\\ \vin  Herrn, 
solange wir leben! \\!

\end{verse}
\end{quote}

\medskip

\setspaceafterinitial{5.2mm plus 0em minus 0em}
\setspacebeforeinitial{4.2mm plus 0em minus 0em}
\def\greinitialformat#1{{\fontsize{40}{40}\selectfont #1}}
\gresetfirstlineaboveinitial{\small \textcolor{red}{Ps 63}}{}
\setaboveinitialseparation{0.72mm}
\setsecondannotation{\small vij. T.}


\includescore{psalmi/62/ps62.tex}

\vspace{0.5cm}
\psal{63}

\begin{quote}
\begin{verse}
Gott, du mein G\d ott, dich suche ich, *\\ 
meine Seele d\d ürstet nach dir.\\ 
\vin Nach dir schm\d achtet mein Leib *\\ 
\vin wie dürres, lechzendes L\d and ohne Wasser.\\  
\textit{Darum halte ich Ausschau\\
nach d\d ir} im Heiligtum, *\\ 
\textit{um deine Macht und Herrlichk\d eit zu sehen.}\\ 
\vin Denn deine Huld ist b\d esser als das Leben; *\\ 
\vin darum preisen dich m\d eine Lippen.\\ 
Ich will dich r\d ühmen mein Leben lang, *\\ 
in deinem Namen die H\d ände erheben.\\ 
\vin Wie an Fett und Mark wird s\d att \\ 
\vin meine Seele, *\\ 
\vin mit jubelnden Lippen soll mein M\d und \\ 
\vin dich preisen.\\  
Ich denke an dich auf n\d ächtlichem Lager *\\ 
und sinne über dich nach, w\d enn ich wache.\\ 
\vin Ja, du wurdest m\d eine Hilfe; *\\ 
\vin jubeln kann ich im Schatten d\d einer Flügel.\\  
Meine S\d eele hängt an dir, *\\ 
deine rechte H\d and hält mich fest.\\ 
\vin \textcolor{red}{\bf{(}}Viele trachten mir ohne Grund\\ 
\vin n\d ach dem Leben, *\\ 
\vin aber sie müssen hinabfahren\\ 
\vin in die T\d iefen der Erde.\\ 
Man gibt sie der Gew\d alt des Schwertes preis, *\\ 
sie werden eine Beute d\d er Schakale.\textcolor{red}{\bf{)}}\\ \vin 
Der König aber freue sich an Gott. †\\ 
\vin Wer bei ihm schwört, d\d arf sich rühmen. *\\ 
\vin  Doch allen Lügnern\\
\vin wird der M\d und verschlossen.\\  

\end{verse}
\end{quote}

\noindent\rot{Resp.br.} Sana animam \rot{ut in Feria Secunda p. 43.}\\
\noindent\rot{Hymnus} Splendor paternæ \rot{vel} Ecce iam noctis \rot{ut in Dominica p. 15-17.}

\begin{flushleft}

\versik{Repléti sumus mane misericórdia tua.}{Exultávimus, et delectáti sumus.}

\medskip

{\rm{
\versik{Am Morgen sind wir erfüllt von deiner Huld.}{Wir sind (voll) Jubel und Frohsinn.}
}}
\end{flushleft}

\medskip


\setspaceafterinitial{4.2mm plus 0em minus 0em}
\setspacebeforeinitial{4.2mm plus 0em minus 0em}
\def\greinitialformat#1{{\fontsize{40}{40}\selectfont #1}}
\gresetfirstlineaboveinitial{\footnotesize \textcolor{red}{Benedic}}{}
\setaboveinitialseparation{0.72mm}
\setsecondannotation{\small vij. T.}

\includescore{cantica/evangelica/salutemexinimicis.tex}
\vspace{0.3cm}
\rot{Canticum} Benedictus \rot{p. 196.}



\section[HORA TERTIA]{AD TERTIAM}

\rot{Hymnus} Nunc sancte nobis Spiritus \rot{p. 190.}
\vspace{0.3cm}

\setspaceafterinitial{4.2mm plus 0em minus 0em}
\setspacebeforeinitial{4.2mm plus 0em minus 0em}
\def\greinitialformat#1{{\fontsize{40}{40}\selectfont #1}}

\gresetfirstlineaboveinitial{\small \textcolor{red}{Ps 119c}}{}
\setaboveinitialseparation{0.72mm}
\setsecondannotation{\small 8. T.}
\includescore{psalmi/118/118viijcogitavi.tex}

\newpage

\includescore{psalmi/tertia/horatertia_tertia.tex}

\vspace{0.3cm}

\begin{flushleft}

\versik{Dóminus non privábit bonis eos qui ámbulant in innocéntia.}{Dómine virtútum, beátus homo qui sperat in te.}

\medskip

{\rm{
\versik{Der Herr wird seine Güter nicht denen vorenthalten, die in Unschuld wandeln.}{Herr der Scharen, selig der Mensch, der auf dich hofft.}
}}
\end{flushleft}

\medskip

\section[HORA SEXTA]{AD SEXTAM}

\rot{Hymnus} Rector potens, verax Deus \rot{p. 192.}

\vspace{0.3cm}
\setspaceafterinitial{4.2mm plus 0em minus 0em}
\setspacebeforeinitial{4.2mm plus 0em minus 0em}
\def\greinitialformat#1{{\fontsize{40}{40}\selectfont #1}}
\gresetfirstlineaboveinitial{\small \textcolor{red}{Ps 9b sq }}{}
\setaboveinitialseparation{0.72mm}
\setsecondannotation{\small 1. T.}


\includescore{psalmi/9/exurgedomine91011.tex}

\medskip

\includescore{psalmi/sexta/sextpsalmitertia.tex}

\begin{flushleft}

\versik{Dómine, veritátem in corde dilexísti.}{Et in occúlto sapiéntiam manifestásti mihi.}

\medskip

{\rm{
\versik{Herr, du liebst, die im Herzen voll Wahrheit sind.}{Und im Verborgenen lehrtest du mich Weisheit.}
}}
\end{flushleft}


\section[HORA NONA]{AD NONAM}

\rot{Hymnus} Rerum Deus tenax vigor \rot{p. 194.}

\vspace{0.3cm}

\setspaceafterinitial{5.2mm plus 0em minus 0em}
\setspacebeforeinitial{4.2mm plus 0em minus 0em}
\def\greinitialformat#1{{\fontsize{40}{40}\selectfont #1}}
\gresetfirstlineaboveinitial{\small \textcolor{red}{ Ps 126sq }}{}
\setaboveinitialseparation{0.72mm}
\setsecondannotation{\small ij. T.}

\includescore{psalmi/127/beatiomnes127.tex}

\vspace{0.3cm}

\includescore{psalmi/nona/nonpsalmitertia.tex}


\begin{flushleft}

\versik{Fac cum servo tuo secúndum misericórdiam tuam, Dómine.}{Iustificatiónes tuas doce me.}

\medskip

{\rm{
\versik{Handle an deinem Knecht nach deiner Huld.}{Lehre mich deine Gesetze.}
}}
\end{flushleft}



\section[VESPERAE]{AD VESPERAS}



\setspaceafterinitial{5.2mm plus 0em minus 0em}
\setspacebeforeinitial{4.2mm plus 0em minus 0em}
\def\greinitialformat#1{{\fontsize{40}{40}\selectfont #1}}
\gresetfirstlineaboveinitial{\small \textcolor{red}{Ps 116/7}}{}
\setaboveinitialseparation{0.72mm}
\setsecondannotation{\small 3. T.}

\includescore{psalmi/115/votameadomino115.tex}

\vspace{0.6cm}

\psal{116}

\begin{quote}
\begin{verse}
Ich l\d iebe den Herrn; *\\ 
denn er hat mein lautes Flehen gehört\\ 
\vin und sein \d Ohr mir zugeneigt *\\ 
\vin an dem Tag, als ich zu ihm rief.\\ 
Mich umfingen die Fesseln des Todes, †\\
mich befielen die \d Ängste der Unterwelt, *\\  
mich trafen Bedrängnis und Kummer.\\ 
\vin Da rief ich den N\d amen des Herrn an: *\\ 
\vin «Ach Herr, rette mein Leben!»\\
Der Herr ist gn\d ädig und gerecht, *\\ 
unser Gott ist barmherzig.\\ 
\vin Der Herr behütet die schl\d ichten Herzen; *\\ 
\vin ich war in Not und er brachte mir Hilfe.\\ 
Komm wieder zur R\d uhe, mein Herz! *\\ 
Denn der Herr hat dir Gutes getan.\\ 
\vin Ja, du hast mein Leben dem Tod entrissen, †\\ 
\vin meine Tr\d änen getrocknet, *\\ 
\vin  meinen Fuß bewahrt vor dem Gleiten.\\  
So gehe ich meinen W\d eg vor dem Herrn *\\ 
im Land der Lebenden.\\ 
\vin Voll Vertrauen war ich,\\ 
\vin auch w\d enn ich sagte: *\\ 
\vin Ich bin so tief gebeugt.\\ 
In meiner Best\d ürzung sagte ich: *\\ 
Die Menschen lügen alle.\\ 
\vin Wie kann ich dem Herrn \d all das vergelten, *\\ 
\vin was er mir Gutes getan hat?\\
Ich will den Kelch des H\d eiles erheben *\\ 
und anrufen den Namen des Herrn.\\ 
\vin \textit{Ich will dem Herrn\\ 
\vin meine Gel\d übde erfüllen *\\ 
\vin offen vor seinem ganzen Volke.}\\ 
Kostbar ist in den \d Augen des Herrn *\\ 
das Sterben seiner Frommen.\\ 
\vin Ach Herr, ich bin doch dein Knecht, †\\ 
\vin dein Knecht bin ich, der S\d ohn deiner Magd.*\\ 
\vin Du hast meine Fesseln gelöst.\\ 
Ich will dir ein Opfer des D\d ankes bringen *\\ 
und anrufen den Namen des Herr\textit{e}n.\\ 
\vin Ich will dem Herrn meine Gel\d übde erfüllen *\\ 
\vin offen vor seinem ganzen Volk\textit{e}, \\
in den Vorhöfen am H\d aus des Herrn, *\\ 
in deiner Mitte, Jerusalem.\\ 

\end{verse}

\vspace{0.5cm}

\psal{117}

\begin{verse}
 Lobet den H\d errn, alle Völker,*\\
preist ihn, alle Nationen! \\
\vin Denn mächtig waltet über \d uns seine Huld,*\\
\vin die Treue des Herrn währt in Ewigkeit. \\
\end{verse}

\end{quote}

\vspace{0.5cm}

\setspaceafterinitial{4.2mm plus 0em minus 0em}
\setspacebeforeinitial{4.2mm plus 0em minus 0em}
\def\greinitialformat#1{{\fontsize{40}{40}\selectfont #1}}
\gresetfirstlineaboveinitial{\small \textcolor{red}{Ps 137}}{}
\setaboveinitialseparation{0.72mm}
\setsecondannotation{\small viij. T.}

\includescore{psalmi/136/hymnumcantatenobis136.tex}
\vspace{0.5cm}

\psal{137}

\begin{quote}
\begin{verse}
An den Strömen von Babel, †\\
da saßen wir und weinten, *\\  
wenn wir an Zion dachten. \\ 
\vin Wir hängten unsere Harfen *\\ 
\vin an die Weiden in jenem Land\textit{e}. \\
Dort verlangten von uns\\
die Zwingherren Lieder, †\\
unsere Peiniger forderten Jubel: *\\  
\textit{«Singt uns Lieder vom Zion!»}\\ 
\vin Wie könnten wir singen\\ 
\vin die Lieder des Herrn, *\\ 
\vin fern, auf fremder Erde?\\
Wenn ich dich je vergesse, Jerusalem, *\\ 
dann soll mir die rechte Hand verdorren.\\ 
\vin Die Zunge soll mir am Gaumen kleben, †\\ 
\vin wenn ich an dich nicht mehr denke, *\\ 
\vin  wenn ich Jerusalem\\ 
\vin nicht zu meiner höchsten Freude erhebe. \\
\textcolor{red}{(}Herr, vergiss den Söhnen Edoms nicht\\
den Tag von Jerusalem; *\\ 
sie sagten: «Reißt nieder,\\
bis auf den Grund reißt es nieder!»\\ 
\vin Tochter Babel, du Zerstörerin! *\\ 
\vin Wohl dem, der dir heimzahlt,\\ 
\vin was du uns getan hast!\\
Wohl dem, der deine Kinder packt *\\ 
und sie am Felsen zerschmettert!\textcolor{red}{)} \\

\end{verse}
\end{quote}

\vspace{0.6cm}

\setspaceafterinitial{10.2mm plus 0em minus 0em}
\setspacebeforeinitial{4.2mm plus 0em minus 0em}
\def\greinitialformat#1{{\fontsize{40}{40}\selectfont #1}}
\gresetfirstlineaboveinitial{\small \textcolor{red}{Ps 138}}{}
\setaboveinitialseparation{0.72mm}
\setsecondannotation{\small v. T.}

\includescore{psalmi/137/inconspectu.tex}

\vspace{0.3cm}

\psal{138}

\begin{quote}
\begin{verse}
\textit{Ich will} dir danken aus ganzem Herzen, *\\ 
\textit{dir vor den Engeln s\d ingen und spielen;} \\ 
\vin ich will mich niederwerfen\\ 
\vin zu deinem heiligen Tempel hin *\\ 
\vin und deinem Namen danken\\ 
\vin für deine H\d uld und Treue.\\ 
Denn du hast die Worte meines Mundes gehört,*\\  
deinen Namen und dein Wort\\
über \d alles verherrlicht.\\ 
\vin Du hast mich erhört an dem Tag,\\ 
\vin als ich rief; *\\ 
\vin du gabst meiner S\d eele große Kraft.\\ 
Dich sollen preisen, Herr, alle Könige der Welt,*\\ 
wenn sie die Worte deines M\d undes vernehmen.\\ 
\vin Sie sollen singen von den Wegen des Herrn; *\\ 
\vin denn groß ist die H\d errlichkeit des Herrn.\\ 
Ja, der Herr ist erhaben; †\\
doch er schaut auf die Niedrigen, *\\  
und die Stolzen erk\d ennt er von fern\textit{e}.\\ 
\vin Gehe ich auch mitten durch große Not: *\\ 
\vin du erh\d ältst mich am Leben.\\  
Du streckst die Hand aus\\
gegen meine wütenden Feinde, *\\  
und deine R\d echte hilft mir.\\ 
\vin Der Herr nimmt sich meiner an. †\\ 
\vin Herr, deine Huld währt ewig. *\\ 
\vin Lass nicht ab vom W\d erk deiner Hände!\\

\end{verse}
\end{quote}


\newpage

\setspaceafterinitial{10.2mm plus 0em minus 0em}
\setspacebeforeinitial{4.2mm plus 0em minus 0em}
\def\greinitialformat#1{{\fontsize{40}{40}\selectfont #1}}
\gresetfirstlineaboveinitial{\small \textcolor{red}{Ap 4;5}}{}
\setaboveinitialseparation{0.72mm}
\setsecondannotation{\small 7. T.}

\includescore{cantica/cn/sedebitdominus.tex}

\noindent{\rm{Ant. Der Herr thront als König in Ewigkeit, der Herr segne sein Volk mit Frieden.}}

\bf

\cant{Ap 4,{\footnotesize{11}};5,{\footnotesize{9-12}}}

\begin{quote}
\begin{verse}
Würdig bist du, H\d err, unser Gott, *\\ 
Herrlichkeit zu empfangen und \d Ehre und Macht.\\ 
\vin Denn du bist es,\\ 
\vin der die W\d elt erschaffen hat, *\\ 
\vin durch deinen Willen war sie und wurde durch\\ 
\vin \d ihn erschaffen.\\ 
Herr, d\d u bist würdig, *\\ 
das Buch zu nehmen und seine S\d iegel zu öffnen;\\ 
\vin denn du wurdest geschlachtet †\\ 
\vin und hast mit deinem Blut Menschen für \\ 
\vin G\d ott erworben *\\ 
\vin aus allen Stämmen und Sprachen, aus allen\\ 
\vin Nati\d onen und Völkern, \\
und du hast sie zu Königen und Priestern\\ 
gemacht für \d unseren Gott; *\\  
und sie werden auf der \d Erde herrschen.\\ 
\vin Würdig ist das Lamm, das geschlachtet ist, †\\ 
\vin Macht zu empfangen, R\d eichtum \\ 
\vin und Weisheit, *\\ 
\vin  Kraft und Ehre, L\d ob und Herrlichkeit.\\  

\end{verse}
\end{quote}

\medskip
 
\noindent{\rot{Resp.br.} Adjutorium nostrum {\rot{ut in Feria Secunda p. 58.}}\\

\noindent{\rot{Hymnus} Deus Creator \rot{vel} Lucis Creator \rot{ ut in Dominica p. 32 - 33}

\vspace{0.3cm}

\medskip


\begin{flushleft}

\versik{Dirigátur, Dómine, orátio mea.}{Sicut incénsum in conspéctu tuo.}

\medskip
{\rm{
\versik{Herr, mein Gebet werde gelenkt.}{Wie Weihrauch vor dein Angesicht.}
}}
\end{flushleft}



\medskip

\setspaceafterinitial{5.2mm plus 0em minus 0em}
\setspacebeforeinitial{4.2mm plus 0em minus 0em}
\def\greinitialformat#1{{\fontsize{40}{40}\selectfont #1}}
\gresetfirstlineaboveinitial{\footnotesize \textcolor{red}{Magni}}{}
\setaboveinitialseparation{0.72mm}
\setsecondannotation{\small j. T.}

\includescore{cantica/evangelica/respexisti.tex}

\medskip

\rot{Canticum} Magnificat \rot{p. 200.}

\newpage
