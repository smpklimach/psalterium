\kkap{SONNTAG}

\section[SONNTAG]{AD TERTIAM}

\rot{Hymnus} Nunc sancte nobis Spiritus \rot{p. 124.}

\medskip



\setspaceafterinitial{5.2mm plus 0em minus 0em}
\setspacebeforeinitial{4.2mm plus 0em minus 0em}
\def\greinitialformat#1{{\fontsize{40}{40}\selectfont #1}}
\gresetfirstlineaboveinitial{\small \textcolor{red}{Ps 119a}}{}
\setaboveinitialseparation{0.72mm}
\setsecondannotation{\small ij. T.}


\includescore{alleluia/alleluia_tertia.tex}
\medskip
\includescore{psalmi/tertia/terzpsalterdom.tex}

\begin{flushleft}

\versik{Inclína cor meum, Deus, in testimónia tua.}{In via tua vivífica me.}

\medskip
{\rm{
\versik{Neige mein Herz, Herr, zu deinen Geboten.}{Auf deinen Wegen belebe mich.}
}}
\end{flushleft}
\newpage

\section{MITTAGSHORE, I. WOCHE}

\rot{Hymnus} Rector potens, verax Deus \rot{p. 126.}
\vspace{0.3cm}

\setspaceafterinitial{5.2mm plus 0em minus 0em}
\setspacebeforeinitial{4.2mm plus 0em minus 0em}
\def\greinitialformat#1{{\fontsize{40}{40}\selectfont #1}}
\gresetfirstlineaboveinitial{\small \textcolor{red}{ Ps 1 2 6 }}{}
\setaboveinitialseparation{0.72mm}
\setsecondannotation{\small viij. T.}
\includescore{alleluia/alleluia_sexta}
\medskip


\includescore{psalmi/sexta/sextpsalmidom.tex}

\medskip

\begin{flushleft}

\versik{In ætérnum, Dómine, verbum tuum.}{In generatiónem et generatiónem véritas tua.}

\medskip

{\rm{
\versik{In Ewigkeit, Herr, bleibt dein Wort.}{Von Geschlecht zu Geschlecht deine Treue.}
}}
\end{flushleft}


\newpage

\section{MITTAGSHORE, II. WOCHE}

\rot{Hymnus} Rector potens, verax Deus \rot{p. 126.}
\vspace{0.3cm}

 \setspaceafterinitial{5.2mm plus 0em minus 0em}
\setspacebeforeinitial{4.2mm plus 0em minus 0em}
\def\greinitialformat#1{{\fontsize{40}{40}\selectfont #1}}
\gresetfirstlineaboveinitial{\small \textcolor{red}{ Ps 120sq.}}{}
\setaboveinitialseparation{0.72mm}
\setsecondannotation{\small viij. T.}

\includescore{alleluia/alleluia_nona}

\medskip

\includescore{psalmi/nona/nonpsalmidom.tex}

\begin{flushleft}

\versik{Clamávi in toto corde meo, exáudi me, Dómine.}{Iustificatiónes tuas servábo.}

\medskip

{\rm{
\versik{Aus meinem ganzem Herzen habe ich gerufen, erhöre mich, Herr.}{Deiner Gerechtigkeit will ich dienen.}
}}
\end{flushleft}


\vspace{2cm}

\section[VESPERAE]{AD VESPERAS}


 \setspaceafterinitial{5.2mm plus 0em minus 0em}
\setspacebeforeinitial{4.2mm plus 0em minus 0em}
\def\greinitialformat#1{{\fontsize{40}{40}\selectfont #1}}
\gresetfirstlineaboveinitial{\small \textcolor{red}{Ps 110}}{}
\setaboveinitialseparation{0.72mm}
\setsecondannotation{\small 7. T.}

\includescore{psalmi/109/dixitdominus.tex}

\medskip

\psal{110}

\begin{quote}
\begin{verse}

\textit{So spricht der Herr zu meinem Herrn: †\\
Setze dich m\d ir zur Rechten,} *\\
und ich lege dir deine Feinde\\
als Schemel unter die Füße.\\
\vin Vom Zion strecke der Herr \\
\vin das Zepter d\d einer Macht aus: *\\
\vin ``Herrsche inmitten d\d einer Feinde!''\\
Dein ist die Herrschaft am T\d age deiner Macht, *\\
wenn du erscheinst in h\d eiligem Schmuck;\\
\vin ich habe dich gezeugt\\ 
\vin noch v\d or dem Morgenstern. *\\
\vin wie den T\d au in der Frühe.\\
Der Herr hat geschworen,\\
und n\d ie wird’s ihn reuen: *\\
``Du bist Priester auf ewig nach der \d Ordnung \\
Melchisedeks.''\\
\vin Der Herr steht d\d ir zur Seite; *\\
\vin er zerschmettert Könige\\
\vin am Tage s\d eines Zornes.\\
Er hält Gericht unter den Völkern,\\
er h\d äuft die Toten, *\\
die Häupter zerschmettert er w\d eithin auf Erden.\\
Er trinkt aus dem B\d ach am Weg\textit{e}; *\\
so kann er von neuem das H\d aupt erheben.\\!
\end{verse}
\end{quote}

\vspace{1cm}


\setspaceafterinitial{5.2mm plus 0em minus 0em}
\setspacebeforeinitial{4.2mm plus 0em minus 0em}
\def\greinitialformat#1{{\fontsize{40}{40}\selectfont #1}}
\gresetfirstlineaboveinitial{\small \textcolor{red}{Ps 111}}{}
\setaboveinitialseparation{0.72mm}
\setsecondannotation{\small iv. T.}

\includescore{psalmi/110/fidelia.tex}

\medskip

\psal{111}

\begin{quote}
\begin{verse}

Den Herrn will ich preisen von ganzem Herzen *\\
im Kreis der Frommen, inmitt\d en der Gemeinde.\\
\vin Groß sind die Werke des Herrn, *\\
\vin kostbar allen, die sich \d an ihnen freuen.\\
Er waltet in Hoheit und Pracht, *\\
seine Gerechtigkeit hat B\d estand für immer.\\
\vin Er hat ein Gedächtnis an seine Wunder\\ 
\vin gestiftet, *\\
\vin der Herr ist gnäd\d ig und barmherzig.\\
Er gibt denen Speise, die ihn fürchten, *\\
an seinen Bund d\d enkt er auf ewig.\\
\vin Er hat seinem Volk\\
\vin seine machtvollen Taten kundgetan, *\\
\vin um ihm das Erbe der V\d ölker zu geben. \\
Die Werke seiner Hände\\
sind gerecht und beständig, *\\
\textit{all seine Gebot\d e sind verlässlich.\\
\vin Sie stehen fest für immer und ewig}, *\\
\vin geschaffen in Tr\d eue und Redlichkeit.\\
Er gewährte seinem Volk Erlösung †\\
und bestimmte seinen Bund für ewige Zeiten. *\\
Furchtgebietend ist sein N\d ame und heilig. \\
\vin Die Furcht des Herrn\\ 
\vin ist der Anfang der Weisheit; †\\
\vin alle, die danach leben, sind klug. *\\
\vin Sein Ruhm hat B\d estand für immer.\\!

\end{verse}
 
\end{quote}


\vspace{0.6cm}
\setspaceafterinitial{10.2mm plus 0em minus 0em}
\setspacebeforeinitial{4.2mm plus 0em minus 0em}
\def\greinitialformat#1{{\fontsize{40}{40}\selectfont #1}}
\gresetfirstlineaboveinitial{\small \textcolor{red}{Ps 112}}{}
\setaboveinitialseparation{0.72mm}
\setsecondannotation{\small 4. T.}

\includescore{psalmi/111/inmandatiseius.tex}

\medskip

\psal{112}

\begin{quote}
\begin{verse}
Wohl dem Mann, \textit{der} den Herrn\\
fürchtet und ehrt *\\
und \textit{sich herzlich freut an s\d einen Geboten.}\\
\vin Seine Nachkommen werden\\ 
\vin mächtig im Land, *\\
\vin das Geschlecht der Redlich\d en wird gesegnet.\\
Wohlstand und Reichtum füllen sein Haus, *\\
sein Heil hat B\d estand für immer.\\
\vin Den Redlichen erstrahlt\\
\vin im Finstern ein Licht: *\\
\vin Der Gnädige, Barmherzig\d e und Gerechte.\\
Wohl dem Mann, der gütig\\
und zum Helfen bereit ist, *\\
der das Seine ordn\d et, wie es recht ist.\\
\vin Niemals gerät er ins Wanken; *\\
\vin ewig denkt man \d an den Gerechten.\\
Er fürchtet sich nicht vor Verleumdung; *\\
sein Herz ist fest, er vertr\d aut auf den Herr\textit{e}n.\\
\vin Sein Herz ist getrost, er fürchtet sich nie; *\\
\vin denn bald wird er herabschauen\\
\vin auf s\d eine Bedränger.\\
Reichlich gibt er den Armen, †\\
sein Heil hat Bestand für immer; *\\
er ist m\d ächtig und hoch geehrt.\\
\vin Voll Verdruss sieht es der Frevler, †\\
\vin er knirscht mit den Zähnen und\\ 
\vin geht zugrunde. *\\
\vin Zunichte werden die W\d ünsche der Frevler.\\
\end{verse}

\end{quote}


\vspace{1cm}
\medskip

\setspaceafterinitial{5.2mm plus 0em minus 0em}
\setspacebeforeinitial{4.2mm plus 0em minus 0em}
\def\greinitialformat#1{{\fontsize{40}{40}\selectfont #1}}
\gresetfirstlineaboveinitial{\small \textcolor{red}{Apoc 19}}{}
\setaboveinitialseparation{0.72mm}
\setsecondannotation{\small v. T.}

\includescore{cantica/cn/fecitnosdeo.tex}

\medskip

\begin{sloppypar}
{\noindent\rm{Ant. Er hat uns vor Gott, seinem Vater, zu Königen gemacht, 
der Erstgeborene der Toten und Fürst über die Könige der Erde.}}
\end{sloppypar}

\bigskip

\cant{cf. Apoc 19,1-7}

\begin{quote}
\begin{verse}


Das Heil und die Herrlichkeit\\
und die Macht ist bei unser\textit{e}m Gott.*\\
Seine Urteile sind w\d ahr und gerecht.\\
\vin Preist unsern Gott, all seine Knechte*\\
\vin und alle, die ihn fürchten, Gr\d oße und Kleine!\\
Denn König geworden ist der Herr,\\
unser Gott, *\\
der Herrscher über die g\d anze Schöpfung.\\
\vin Wir wollen uns freuen und jubeln *\\
\vin und ihm die \d Ehre erweisen!\\
Denn gekommen ist die Hochzeit des Lammes, *\\
und seine Frau hat s\d ich bereit gemacht.

\end{verse}
\end{quote}

\medskip

\setspaceafterinitial{4.2mm plus 0em minus 0em}
\setspacebeforeinitial{4.2mm plus 0em minus 0em}
\resp

\includescore{responsoria_diebusferialibus/respbrquammagnifica.tex}

\medskip

\begin{sloppypar}
{\noindent\rm{\rot{Resp.} Wie groß sind deine Werke, Herr. Alles hast du in Weisheit geschaffen.}}
\end{sloppypar}

\newpage

\crot{Hymnus}

\setspaceafterinitial{5.2mm plus 0em minus 0em}
\setspacebeforeinitial{4.2mm plus 0em minus 0em}
\def\greinitialformat#1{{\fontsize{40}{40}\selectfont #1}}
\gresetfirstlineaboveinitial{\small \textcolor{red}{hieme}}{}
\setaboveinitialseparation{0.72mm}
\setsecondannotation{\small viij. T.}
\includescore{hymni/deuscreatoromnium.tex}

\medskip

\begin{sloppypar}
{\noindent\rm{\rot{1.} Gott, aller Dinge Schöpfer und Herrscher des Himmels,
du zierst den Tag mit glänzendem Licht, die Nacht mit der Gunst des Schlafens.}}
\end{sloppypar}

\medskip

{\setlength{\columnsep}{1cm}
\begin{multicols}{2} 
\begin{verse}[\versewidth]
 
{\small{
\frot{A}rtus solútos ut quies\\
Reddat labóris úsui;\\
Mentésque fessas állevet,\\
Luctúsque solvat ánxios.\\!

\frot{G}rates perácto iam die,\\
Et noctis exórtu, preces,\\ 
Voti reos ut ádiuves,\\ 
Hymnum canéntes,\\
sólvimus.\\!

\frot{T}e cordis ima cóncinant, \\
Te vox canóra cóncrepet; \\
Te díligat castus amor, \\
Te mens adóret sóbria. \\!}}
\end{verse}

\columnbreak

\begin{verse}[\versewidth]
 
{\small\rm{\frot{2.} Dass Ruhe den erschlafften\\
Gliedern, die Arbeitskraft erneu're,\\
den müden Geist erhebe,\\
und Klagen Ängste löse.\\!

\frot{3.} Hymnen singend, lösen wir ein\\
die Pflicht des Gelobten; bringen\\
nun dar den Dank für den Tag\\
und Gebete zum Anbruch der\\
Nacht, damit du uns hilfst.\\!

\frot{4.} Des Herzens Mitte preise dich,\\
für dich erschalle wohlklingender\\
Ton, dich liebe die reine Liebe,\\
dich verehr' der verständige Geist.\\!}}
\end{verse}
\end{multicols}
}

\newpage

{\setlength{\columnsep}{1cm}
\begin{multicols}{2} 
\begin{verse}[\versewidth]

{\small{\frot{U}t cum profúnda cláuserit\\
Diem calígo nóctium;\\ 
Fides tenébras nésciat,\\
Et nox fide\\
relúceat.\\!

\frot{C}hristum rogámus et Patrem,\\ 
Christi Patrísque Spíritum;\\
Unum potens per ómnia,\\ 
Fove precántes, Trínitas.\\
Amen.\\!}}
\end{verse}

\columnbreak

\begin{verse}[\versewidth]
 
{\small\rm{\frot{5.} Dass, wenn der Abgrund uns\\
umschließt, die Finsternis der\\
Nacht den Tag, der Glaub' von\\
Dunkelheit nichts weiß und\\
Zuversicht die Nacht erhellt.\\!

\frot{6.} Christus bitten wir und\\
den Vater, Christi und des\\
Vaters Geist; Eine Macht\\
über allem, Dreifaltigkeit,\\
umarme die Bittenden. Amen.\\!}}
 
 
\end{verse}

\end{multicols}
}

\medskip

\setspaceafterinitial{5.2mm plus 0em minus 0em}
\setspacebeforeinitial{4.2mm plus 0em minus 0em}
\def\greinitialformat#1{{\fontsize{40}{40}\selectfont #1}}
\gresetfirstlineaboveinitial{\small \textcolor{red}{æstate}}{}
\setaboveinitialseparation{0.72mm}
\setsecondannotation{\small viij. T.}
\includescore{hymni/luciscreator.tex}

\medskip

\begin{sloppypar}
{\noindent\rm{\rot{1.} Des Lichtes guter Schöpfer, du holst aus Licht den Tag hervor.
Mit dem Aufgang des neuen Lichtes, bereitest du der Welt ihren Beginn.}}
\end{sloppypar}

\medskip

{\setlength{\columnsep}{1cm}
\begin{multicols}{2} 
\begin{verse}[\versewidth]
 
{\small{
\frot{Q}ui mane iunctum vésperi\\
diem vocári præcipis:\\
tætrum chaos illábitur;\\
audi preces cum flétibus.\\!}}
\end{verse}

\columnbreak

\begin{verse}[\versewidth]
{\small\rm{\frot{2.} Der du Morgen und Abend\\
vereinst, und vorsiehst, ihn Tag zu\\
nennen: Das düstre Chaos zieht\\
herauf, hör' unser flehendes Bitten.\\!}}
\end{verse}
\end{multicols}
}

{\setlength{\columnsep}{1cm}
\begin{multicols}{2} 
\begin{verse}[\versewidth]
 
{\small{
\frot{N}e mens graváta crímine\\
vitæ sit exsul múnere,\\ 
dum nil perénne cógitat\\
seséque culpis ílligat.\\!

\frot{C}ælórum pulset íntimum,\\
vitále tollat præmium;\\
vitémus omne nóxium,\\ 
purgémus omne péssimum.\\!

\frot{P}ræsta, Pater piíssime,\\ 
Patríque compar Unice,\\ 
cum Spíritu Paráclito\\
regnans per omne sæculum.\\
Amen.\\!}}

\end{verse}

\columnbreak

\begin{verse}[\versewidth]
{\small\rm{\frot{3.} Dass nicht - belastet - der Geist \\
der Gunst des Lebens sei beraubt\\
wenn er nicht beständig nachsinnt\\
und sich in Schuld verstrickt.\\!

\frot{4.} Er tromm'le an des Himmels\\
Herz, trag' hinauf den ewigen\\
Lohn; Lasset uns jede Untat\\
meiden, sühnen alle Bosheit.\\!

\frot{5.} Dies gewähre, gütigster Vater,\\
und der dem Vater Gleiche, der \\
mit dem Heiligen Geist \\
ist herrschend in Ewigkeit.\\
Amen.\\!}}
\end{verse}
\end{multicols}
}

\medskip

\begin{flushleft}
 
\versik{Dirigátur, Dómine, orátio mea.}{Sicut incénsum in conspéctu tuo.}

\medskip
{\rm{
\versik{Mein Gebet möge sich zu dir lenken, Herr.}{Wie Weihrauch gelangt vor dein Angesicht.}
}}
\end{flushleft}

\medskip




\rot{Canticum} Magnificat \rot{p. 128.}

\vspace{0.5cm}


Nach dem Magnificat folgen die Fürbitten, das Vater Unser und das Tagesgebet.
Das Chorgebet endet mit einer Antiphon zu Ehren der heiligen Jungfrau.

\newpage
