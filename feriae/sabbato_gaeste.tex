
\kkap{SAMSTAG}

\section[SAMSTAG]{AD TERTIAM}

\rot{Hymnus} Nunc sancte nobis Spiritus \rot{p. 124.}

\vspace{0.3cm}

\setspaceafterinitial{3.2mm plus 0em minus 0em}
\setspacebeforeinitial{4.2mm plus 0em minus 0em}
\def\greinitialformat#1{{\fontsize{40}{40}\selectfont #1}}
\gresetfirstlineaboveinitial{\small \textcolor{red}{Ps 119g}}{}
\setaboveinitialseparation{0.72mm}
\setsecondannotation{\small iv. T.}


\includescore{psalmi/118/videhumilitatem20.tex}


\includescore{psalmi/tertia/horatertia_sabbato.tex}


\vspace{0.3cm}


\begin{flushleft}

\versik{Dóminus non privábit bonis eos qui ámbulant in innocéntia.}{Dómine virtútum, beátus homo qui sperat in te.}

\medskip

{\rm{
\versik{Der Herr wird seine Güter nicht denen vorenthalten, die in Unschuld wandeln.}{Herr der Scharen, selig der Mensch, der auf dich hofft.}
}}
\end{flushleft}


\section{MITTAGESHORE, I. WOCHE}

\rot{Hymnus} Rector potens, verax Deus \rot{p. 126.}

\vspace{0.3cm}

\setspaceafterinitial{5.2mm plus 0em minus 0em}
\setspacebeforeinitial{4.2mm plus 0em minus 0em}
\def\greinitialformat#1{{\fontsize{40}{40}\selectfont #1}}
\gresetfirstlineaboveinitial{\small \textcolor{red}{Ps 18c sq}}{}
\setaboveinitialseparation{0.72mm}
\setsecondannotation{\small viij. T.}

\includescore{psalmi/19/exaudiatte19.tex}


\includescore{psalmi/sexta/sextpsalmisabbato.tex}
\vspace{0.3cm}



\begin{flushleft}

\versik{Dómine, veritátem in corde dilexísti.}{Et in occúlto sapiéntiam manifestásti mihi.}

\medskip

{\rm{
\versik{Herr, du liebst, die im Herzen voll Wahrheit sind.}{Und im Verborgenen lehrtest du mich Weisheit.}
}}
\end{flushleft}

\newpage

\section{MITTAGESHORE, II. WOCHE}

\rot{Hymnus} Rector potens, verax Deus \rot{p. 126.}

\vspace{0.3cm}

 \setspaceafterinitial{5.2mm plus 0em minus 0em}
\setspacebeforeinitial{4.2mm plus 0em minus 0em}
\def\greinitialformat#1{{\fontsize{40}{40}\selectfont #1}}
\gresetfirstlineaboveinitial{\small \textcolor{red}{ Ps 136 }}{}
\setaboveinitialseparation{0.72mm}
\setsecondannotation{\small 3. T.}

\includescore{psalmi/135/quoniaminaeternum135.tex}
\vspace{0.3cm}

\includescore{psalmi/nona/nonpsalmisabbato.tex}

\begin{flushleft}

\versik{Fac cum servo tuo secúndum misericórdiam tuam, Dómine.}{Iustificatiónes tuas doce me.}

\medskip

{\rm{
\versik{Handle an deinem Knecht nach deiner Huld.}{Lehre mich deine Gesetze.}
}}
\end{flushleft}



\section[VESPERAE]{AD VESPERAS}



\setspaceafterinitial{3.2mm plus 0em minus 0em}
\setspacebeforeinitial{3.2mm plus 0em minus 0em}
\def\greinitialformat#1{{\fontsize{40}{40}\selectfont #1}}
\gresetfirstlineaboveinitial{\small \textcolor{red}{Ps 146}}{}
\setaboveinitialseparation{0.72mm}
\setsecondannotation{\small iv. T.}

\includescore{psalmi/145/laudabodeum.tex}

\vspace{0.3cm}

\crot{Psalm 146}

\begin{quote}

\begin{verse}
 Lobe den Herrn, meine Seele! †\\
\textit{Ich will den Herrn loben, solange ich lebe,} *\\
meinem Gott singen und spielen,\\ sol\d ange ich da bin. \\
\vin Verlasst euch nicht auf Fürsten, *\\
\vin auf Menschen, bei denen es d\d och \\ \vin keine Hilfe gibt. \\
Haucht der Mensch sein Leben aus †\\
und kehrt er zurück zur Erde, * \\
dann ist es aus mit \d all seinen Plänen. \\
\vin Wohl dem, dessen Halt  der Gott Jakobs ist *\\
\vin und der seine Hoffnung auf den\\ \vin H\d errn, seinen Gott, setzt. \\
Der Herr hat Himmel und Erde gemacht, †\\
das Meer und alle Geschöpfe; * \\
er hält \d ewig die Treue.\\
\vin Recht verschafft er den Unterdrückten, †\\
\vin den Hungernden gibt er Brot; *\\
\vin  der Herr befr\d eit die Gefangenen. \\
Der Herr öffnet den Blinden die Augen, *\\
er richt\d et die Gebeugten auf. \\
\vin Der Herr beschützt die Fremden *\\
\vin und verhilft den Waisen \\ \vin und W\d itwen zu ihrem Recht. \\
Der Herr liebt die Gerechten, * \\
doch die Schritte der Frevler leitet \d er in die Irre.\\
\vin Der Herr ist König auf ewig, *\\
\vin dein Gott, Zion, herrscht von G\d eschlecht \\ \vin zu Geschlecht.\\

\end{verse}

\end{quote}
\vspace{0.3cm}

\setspaceafterinitial{3.2mm plus 0em minus 0em}
\setspacebeforeinitial{3.2mm plus 0em minus 0em}
\def\greinitialformat#1{{\fontsize{40}{40}\selectfont #1}}
\gresetfirstlineaboveinitial{\small \textcolor{red}{Ps 147a}}{}
\setaboveinitialseparation{0.72mm}
\setsecondannotation{\small 8. T.}

\includescore{psalmi/146/deonostro.tex}


\vspace{0.3cm}

\crot{Psalm 147a}
\begin{quote}
\begin{verse}
 Gut ist es, unserm Gott zu singen; *\\
\textit{schön ist es, ihn zu loben}. \\
\vin Der Herr baut Jerusalem wieder auf, *\\
\vin er sammelt die Versprengten Israels. \\
Er heilt die gebrochenen Herzen *\\
und verbindet ihre schmerzenden Wunden. \\
\vin Er bestimmt die Zahl der Sterne *\\
\vin und ruft sie alle mit Namen. \\
Groß ist unser Herr und gewaltig an Kraft, *\\
unermesslich ist seine Weisheit. \\
\vin Der Herr hilft den Gebeugten auf *\\
\vin und erniedrigt die Frevler.\\
Stimmt dem Herrn ein Danklied an, *\\
spielt unserm Gott auf der Harfe!\\
\vin Er bedeckt den Himmel mit Wolken, †\\
\vin spendet der Erde Regen *\\
 \vin und lässt Gras auf den Bergen sprießen. \\
Er gibt dem Vieh seine Nahrung, *\\
gibt den jungen Raben, wonach sie schreien. \\
\vin Er hat keine Freude an der Kraft \\ \vin des Pferdes, *\\
\vin kein Gefallen am schnellen Lauf des Mannes. \\
Gefallen hat der Herr an denen, die \\ ihn fürchten und ehren, *\\
die voll Vertrauen warten auf seine Huld.\\
\end{verse}

\end{quote}

\vspace{0.6cm}

\setspaceafterinitial{3.2mm plus 0em minus 0em}
\setspacebeforeinitial{4.3mm plus 0em minus 0em}
\def\greinitialformat#1{{\fontsize{40}{40}\selectfont #1}}
\gresetfirstlineaboveinitial{\small \textcolor{red}{Ps 147b}}{}
\setaboveinitialseparation{0.72mm}
\setsecondannotation{\small 1. T.}
\includescore{psalmi/147/laudajerusalem.tex}

\vspace{0.3cm}

\crot{Psalm 147b}
\begin{quote}
\begin{verse}
\textit{ Jerusalem, preise den Herrn,} *\\
lobsinge, Zion, deinem Gott! \\
\vin Denn er hat die Riegel deiner Tore\\ \vin fest gemacht, *\\
\vin die Kinder in deiner Mitte gesegnet;\\
er verschafft deinen Grenzen Frieden *\\
und sättigt dich mit bestem Weizen. \\
\vin Er sendet sein Wort zur Erde, *\\
\vin rasch eilt sein Befehl dahin. \\
Er spendet Schnee wie Wolle, *\\
streut den Reif aus wie Asche. \\
\vin Eis wirft er herab in Brocken, *\\
\vin vor seiner Kälte erstarren die Wasser.\\
Er sendet sein Wort aus und sie schmelzen, *\\
er lässt den Wind wehen, dann \\rieseln die Wasser.\\
\vin Er verkündet Jakob sein Wort, *\\
\vin Israel seine Gesetze und Rechte. \\
An keinem andern Volk hat er so gehandelt, *\\
keinem sonst seine Rechte verkündet. \\
\end{verse}
\end{quote}

\vspace{0.3cm}

\setspaceafterinitial{7.2mm plus 0em minus 0em}
\setspacebeforeinitial{5.2mm plus 0em minus 0em}
\def\greinitialformat#1{{\fontsize{40}{40}\selectfont #1}}
\gresetfirstlineaboveinitial{\small \textcolor{red}{Phil 2,1-6}}{}
\setaboveinitialseparation{0.72mm}
\setsecondannotation{\small 7. T.}

\includescore{cantica/cn/jesusnazarenus.tex}

\noindent{\rm{Ant. Jesus von Nazaret, König der Juden, Herrscher über den Himmel und Herr über alle Herrscher.}}

\cant{Phil 2,6-11}

\begin{quote}
\begin{verse}


Christus J\d esus war Gott gleich,*\\
hielt aber nicht daran f\d est, wie Gott zu sein,\\
\vin sondern er entäußerte sich und \\ \vin wurde w\d ie ein Sklave *\\
\vin \d und den Menschen gleich. \\
sein Leben war das eines Menschen; †\\
er erniedrigte sich und \\war geh\d orsam bis zum Tod,* \\
bis zum Tod am Kreuze.\\
\vin Darum hat ihn Gott über \d alle erhöht *\\
\vin und ihm den Namen verliehen, \\
\vin der über alle N\d amen erhaben ist.\\
damit alle im Himmel, auf der \\Erde und \d unter der Erde*\\
ihre Knie beugen vor den Namen Jesu\\
\vin und jeder Mund bekennt: †\\
\vin „Jesus Chr\d istus ist der Herr!“ *\\
\vin zur Ehre G\d ottes des Vaters.\\	

\end{verse}

\end{quote}

\vspace{0.3cm}



\setspaceafterinitial{4.2mm plus 0em minus 0em}
\setspacebeforeinitial{4.2mm plus 0em minus 0em}
\resp

\includescore{responsoria_diebusferialibus/respbrmagnusdominus.tex}

\rm{Resp.br. Groß ist unser Herr und gewaltig an Kraft, unermesslich ist seine Weisheit.}

\bf
\vspace{0.6cm}


 \crot{Hymnus}


\setspaceafterinitial{4.2mm plus 0em minus 0em}
\setspacebeforeinitial{4.2mm plus 0em minus 0em}
\def\greinitialformat#1{{\fontsize{45}{45}\selectfont #1}}
\gresetfirstlineaboveinitial{\small viij. T.}{}
\includescore{hymni/oluxbeata.tex}

\vspace{0.3cm}
\noindent{\rm{Hymn. 1. Schon weicht der Sonne Feuerstrahl. 
Du, ew'ges Licht, Einmütigkeit, 
Du selige Dreieinigkeit, 
gieß Lieb in uns're Herzen ein.\\ 2. Durch Lobgesang wir bitten Dich
morgens und abends inniglich
und uns'rem Lob sei zugeneigt
mit Demut in der Himmelsschar.\\
3. Dem Vater und dem Sohn zugleich
und Dir, Du Geist der Heiligkeit, 
sei, wie es schon seit jeher war, 
beständig Ehr' in Ewigkeit.}}
\begin{flushleft}

\versik{Vespertína orátio ascéndat ad te, Dómine.}{Et descéndat super nos misericórdia tua.}

\medskip
{\rm{
\versik{Unser Abendgebet steige auf zu dir, Herr.}{Und es senke sich auf uns herab dein Erbarmen.}
}}
\end{flushleft}





\setspaceafterinitial{5.2mm plus 0em minus 0em}
\setspacebeforeinitial{4.2mm plus 0em minus 0em}
\def\greinitialformat#1{{\fontsize{40}{40}\selectfont #1}}
\gresetfirstlineaboveinitial{\small \textcolor{red}{Magni.}}{}
\setaboveinitialseparation{0.72mm}
\setsecondannotation{\small viij. T.}

\includescore{cantica/evangelica/magnificatanimamea.tex}
\medskip

\rot{Canticum} Magnificat \rot{p. 128.}

\vspace{0.5cm}


Nach dem Magnificat folgen die Fürbitten, das Vater Unser und das Tagesgebet.
Das Chorgebet endet mit einer Antiphon zu Ehren der heiligen Jungfrau.
\newpage