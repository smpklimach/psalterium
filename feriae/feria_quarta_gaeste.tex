\kkap{MITTWOCH}

\section[MITTWOCH]{AD TERTIAM}

\rot{Hymnus} Nunc sancte nobis Spiritus \rot{p. 124.}

\vspace{0.3cm}

\setspaceafterinitial{7.2mm plus 0em minus 0em}
\setspacebeforeinitial{4.2mm plus 0em minus 0em}
\def\greinitialformat#1{{\fontsize{40}{40}\selectfont #1}}

\gresetfirstlineaboveinitial{\small \textcolor{red}{Ps 119d}}{}
\setaboveinitialseparation{0.72mm}
\setsecondannotation{\small viij. T.}


\includescore{psalmi/118/118xiiinaeternum.tex}


\includescore{psalmi/tertia/horatertia_quarta.tex}


\vspace{0.3cm}

\begin{flushleft}

\versik{Dóminus non privábit bonis eos qui ámbulant in innocéntia.}{Dómine virtútum, beátus homo qui sperat in te.}

\medskip

{\rm{
\versik{Der Herr wird seine Güter nicht denen vorenthalten, die in Unschuld wandeln.}{Herr der Scharen, selig der Mensch, der auf dich hofft.}
}}
\end{flushleft}




\section{MITTAGESHORE, I. WOCHE}

\rot{Hymnus} Rector potens, verax Deus \rot{p. 126.}

\vspace{0.3cm}

\setspaceafterinitial{7.2mm plus 0em minus 0em}
\setspacebeforeinitial{4.2mm plus 0em minus 0em}
\def\greinitialformat#1{{\fontsize{40}{40}\selectfont #1}}
\gresetfirstlineaboveinitial{\small \textcolor{red}{ Ps 11sq}}{}
\setaboveinitialseparation{0.72mm}
\setsecondannotation{\small 8. T.}

\includescore{psalmi/12/illumina121314.tex}
\vspace{0.3cm}

\includescore{psalmi/sexta/sextpsalmiquarta.tex}

\begin{flushleft}

\versik{Dómine, veritátem in corde dilexísti.}{Et in occúlto sapiéntiam manifestásti mihi.}

\medskip

{\rm{
\versik{Herr, du liebst, die im Herzen voll Wahrheit sind.}{Und im Verborgenen lehrtest du mich Weisheit.}
}}
\end{flushleft}







\section{MITTAGESHORE, II. WOCHE}

\rot{Hymnus} Rector potens, verax Deus \rot{p. 126.}

\vspace{0.3cm}

 \setspaceafterinitial{5.2mm plus 0em minus 0em}
\setspacebeforeinitial{4.2mm plus 0em minus 0em}
\def\greinitialformat#1{{\fontsize{40}{40}\selectfont #1}}
\gresetfirstlineaboveinitial{\small \textcolor{red}{ Ps 129sq }}{}
\setaboveinitialseparation{0.72mm}
\setsecondannotation{\small 8. T.}

\includescore{psalmi/128/benediximusvobis.tex}



\vspace{0.3cm}

\includescore{psalmi/nona/nonpsalmiquarta.tex}


\begin{flushleft}

\versik{Fac cum servo tuo secúndum misericórdiam tuam, Dómine.}{Iustificatiónes tuas doce me.}

\medskip

{\rm{
\versik{Handle an deinem Knecht nach deiner Huld.}{Lehre mich deine Gesetze.}
}}
\end{flushleft}


\section[VESPERAE]{AD VESPERAS}



\setspaceafterinitial{4.2mm plus 0em minus 0em}
\setspacebeforeinitial{4.2mm plus 0em minus 0em}
\def\greinitialformat#1{{\fontsize{40}{40}\selectfont #1}}
\gresetfirstlineaboveinitial{\small \textcolor{red}{Ps 139a}}{}
\setaboveinitialseparation{0.72mm}
\setsecondannotation{\small 3. T.}

\includescore{psalmi/138/domineprobasti.tex}

\vspace{0.3cm}
\psal{139a}
\begin{quote}
 


\begin{verse}
 \textit{Herr, du hast mich erforscht und du \\ kennst mich.} †\\
Ob ich sitze oder st\d ehe, du weißt von mir. *\\
Von fern erkennst du meine Gedanken.\\ \vin 
Ob ich gehe oder r\d uhe, es ist dir bekannt; *\\ \vin
du bist vertraut mit all meinen Wegen.\\
Noch liegt mir das Wo\d rt nicht auf der Zunge - *\\
du, Herr, kennst es bereits.\\ \vin
Du umschließt mich von \d allen Seiten *\\ \vin
und legst deine Hand auf mich.\\
Zu wunderbar ist für m\d ich dieses Wissen, *\\
zu hoch, ich kann es nicht begreifen.\\ \vin 
Wohin könnte ich fl\d iehen vor deinem Geist, *\\ \vin
wohin mich vor deinem Angesicht flüchten? \\
Steige ich hinauf in den H\d immel, so \\bist du dort; *\\
bette ich mich in der Unterwelt, bist du zugegen.\\ 
\vin Nehme ich die Fl\d ügel des Morgenrots *\\
\vin und lasse mich nieder am äußersten Meer, \\ 
auch dort wird deine H\d and mich ergreifen *\\ 
und deine Rechte mich fassen.\\
\vin Würde ich sagen: «Finsternis soll mich\\ \vin bedecken,†\\
\vin statt Licht soll N\d acht mich umgeben», *\\
\vin auch die Finsternis wäre für dich nicht finster, \\ 
die Nacht würde l\d euchten wie der Tag, *\\ 
die Finsternis wäre wie Licht.\\
\end{verse}
\end{quote}

\vspace{0.3cm}
\setspaceafterinitial{4.2mm plus 0em minus 0em}
\setspacebeforeinitial{4.2mm plus 0em minus 0em}
\def\greinitialformat#1{{\fontsize{40}{40}\selectfont #1}}
\gresetfirstlineaboveinitial{\small \textcolor{red}{Ps 139b}}{}
\setaboveinitialseparation{0.72mm}
\setsecondannotation{\small vj. T.}

\includescore{psalmi/138/mirabilia138b.tex}

\vspace{0.3cm}

\psal{139b}

\begin{quote}
 



\begin{verse}
 Denn du hast mein Inneres geschaffen, *\\
mich gewoben im Sch\d oß meiner Mutter.\\ \vin 
Ich danke dir, dass du mich so wunderbar\\ \vin  gestaltet hast. *\\ \vin
\textit{Ich weiß: Staunenswert s\d ind deine Werke.}\\
Als ich geformt wurde im Dunkeln, †\\
kunstvoll gewirkt in den Tiefen der Erde, *\\ waren meine Glieder d\d ir nicht verborgen.\\ \vin 
Deine Augen sahen, wie ich entstand, *\\ \vin
in deinem Buch war schon \d alles verzeichnet; \\
meine Tage waren schon gebildet, *\\ als noch keiner v\d on ihnen da war.\\ \vin
Wie schwierig sind für mich, o Gott,\\ \vin  deine Gedanken, *\\ \vin
wie gew\d altig ist ihre Zahl! \\
Wollte ich sie zählen, es wären mehr als \\ der Sand. *\\
Käme ich bis zum Ende, wäre ich n\d och immer \\bei dir.\\ \vin
(Wolltest du, Gott, doch den Frevler töten! *\\ \vin
Ihr blutgierigen Menschen, lasst ab von mir! \\
Sie reden über dich voll Tücke *\\
und missbrauchen deinen Namen.\\ \vin
Soll ich die nicht hassen, Herr, die dich\\ \vin hassen, *\\ \vin
die nicht verabscheuen, die sich gegen\\ \vin dich erheben?\\
Ich hasse sie mit glühendem Hass; *\\
auch mir sind sie zu Feinden geworden.)\\ \vin
Erforsche mich, Gott, und erkenne\\ \vin mein Herz, *\\ \vin 
prüfe mich und erk\d enne mein Denken! \\
Sieh her, ob ich auf dem Weg bin, der \\ dich kränkt, *\\
und leite mich auf d\d em altbewährten Weg! 

\end{verse}

\end{quote}

\vspace{0.3cm}

\setspaceafterinitial{4.2mm plus 0em minus 0em}
\setspacebeforeinitial{4.2mm plus 0em minus 0em}
\def\greinitialformat#1{{\fontsize{40}{40}\selectfont #1}}
\gresetfirstlineaboveinitial{\small \textcolor{red}{Ps 140}}{}
\setaboveinitialseparation{0.72mm}
\setsecondannotation{\small 4. T.}

\includescore{psalmi/139/aviroiniquo.tex}

\vspace{0.3cm}
\psal{140}

\begin{quote}



\begin{verse}
 \textit{Rette mich, Herr, vor bösen Menschen,} *\\
vor gewalttätig\d en Leuten schütze mich! \\ \vin
Denn sie sinnen in ihrem Herzen auf Böses, *\\ \vin
jeden T\d ag schüren sie Streit.\\
Wie die Schlangen haben sie scharfe Zungen *\\
und hinter den Lippen G\d ift wie die Nattern\\ \vin 
Behüte mich, Herr, vor den Händen\\ \vin der Frevler, †\\ \vin
vor gewalttätigen Leuten schütze mich, *\\ \vin die darauf sinnen, mich zu B\d oden zu stoßen.\\
Hochmütige legen mir heimlich Schlingen, †\\
Böse spannen ein Netz aus, *\\ stellen mir F\d allen am Wegrand\\ \vin
Ich sage zum Herrn: Du bist mein Gott *\\ \vin 
Vernimm, o Herr, m\d ein lautes Flehen! \\
Herr, mein Gebieter, meine starke Hilfe, *\\ 
du beschirmst mein Haupt \d am Tag des Kampfes.\\ \vin
Herr, erfülle nicht die Wünsche \\ \vin des Frevlers, *\\ \vin
lass seine Plän\d e nicht gelingen! \\
Die mich umzingeln, sollen das Haupt nicht\\  erheben; *\\
die Bosheit ihrer Lipp\d en treffe sie selbst\\ \vin 
Er lasse glühende Kohlen auf sie regnen, †\\ \vin
er stürze sie hinab in den Abgrund, *\\ \vin sodass sie n\d ie wieder aufstehn.\\ 
Der Verleumder soll nicht bestehen im Land, *\\
den Gewalttätigen treffe d\d as Unglück Schlag \\auf Schlag\\ \vin
Ich weiß, der Herr führt die Sache\\ \vin  des Armen, *\\ \vin
er verhilft den G\d ebeugten zum Recht\\
Deinen Namen preisen nur die Gerechten; *\\
vor deinem Angesicht dürfen nur die\\ R\d edlichen bleiben\\ 
\end{verse}

 
\end{quote}
\vspace{0.3cm}

\setspaceafterinitial{10.2mm plus 0em minus 0em}
\setspacebeforeinitial{4.2mm plus 0em minus 0em}
\def\greinitialformat#1{{\fontsize{40}{40}\selectfont #1}}
\gresetfirstlineaboveinitial{\small \textcolor{red}{Col 1, 12-20}}{}
\setaboveinitialseparation{0.72mm}
\setsecondannotation{\small viij. T.}

\includescore{cantica/cn/eccededite.tex}




\noindent{\rm{Ant. Siehe, ich mache dich zum Licht für die Völker, damit mein Heil bis an das Ende der Erde reicht.}}

\vspace{0.3cm}

\cant{Col 1, 12-20}

\begin{quote}
 


\begin{verse}


Dankt dem Vater mit Freude! *\\
Er hat euch fähig gemacht, Anteil zu haben am \\Los der Heiligen, die im Licht sind.\\ \vin 
Er hat uns der Macht der Finsternis\\ \vin  entrissen *\\ \vin
und aufgenommen in das Reich seines\\ \vin  geliebten Sohnes.\\ 
Durch ihn haben wir die Erlösung,*\\
die Vergebung der Sünden.\\ \vin 
Er ist das Ebenbild des unsichtbaren Gottes,*\\ \vin
der Erstgeborene der ganzen Schöpfung.\\
Denn in ihm wurde alles erschaffen *\\
im Himmel und auf Erden,\\ \vin
 das Sichtbare und das Unsichtbare, †\\ \vin
Throne und Herrschaften, Mächte und\\ \vin  Gewalten; *\\ \vin
 alles ist durch ihn und auf ihn hin geschaffen.\\ 
Er ist vor aller Schöpfung, *\\
in ihm hat alles Bestand.\\ \vin
Er ist das Haupt des Leibes, *\\ \vin
der Leib aber ist die Kirche.\\
 Er ist der Ursprung, † \\der Erstgeborene der Toten;*\\ so hat er in allem den Vorrang.\\ \vin 
Denn Gott wollte mit seiner ganzen Fülle in\\ \vin  ihm wohnen,*\\ \vin
um durch ihn alles zu versöhnen.\\ 
Alles im Himmel und auf Erden wollte er zu\\  Christus führen,*\\
der Friede gestiftet hat am Kreuz durch sein Blut.\\ 
\end{verse}
\end{quote}
\medskip
 
\noindent{\rot{Resp.br.} Adjutorium nostrum {\rot{et Hymnus}  Deus Creator \rot{vel} Lucis Creator \rot{p. 32 - 34.}}

\medskip


\begin{flushleft}

\versik{Dirigátur, Dómine, orátio mea.}{Sicut incénsum in conspéctu tuo.}

\medskip
{\rm{
\versik{Herr, mein Gebet werde gelenkt.}{Wie Weihrauch vor dein Angesicht.}
}}
\end{flushleft}

\setspaceafterinitial{5.2mm plus 0em minus 0em}
\setspacebeforeinitial{4.2mm plus 0em minus 0em}
\def\greinitialformat#1{{\fontsize{40}{40}\selectfont #1}}
\gresetfirstlineaboveinitial{\small \textcolor{red}{Magni.}}{}
\setaboveinitialseparation{0.72mm}
\setsecondannotation{\small j. T.}

\includescore{cantica/evangelica/quiafecitmihi.tex}

\vspace{0.2cm}

\rot{Canticum} Magnificat \rot{p. 128.}

\vspace{0.5cm}


Nach dem Magnificat folgen die Fürbitten, das Vater Unser und das Tagesgebet.
Das Chorgebet endet mit einer Antiphon zu Ehren der heiligen Jungfrau.

\newpage