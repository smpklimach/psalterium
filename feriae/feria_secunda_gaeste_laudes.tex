\kkap{MONTAG}

\section[MONTAG]{AD LAUDES}

\setspaceafterinitial{4.2mm plus 0em minus 0em}
\setspacebeforeinitial{4.2mm plus 0em minus 0em}
\def\greinitialformat#1{{\fontsize{40}{40}\selectfont #1}}
\gresetfirstlineaboveinitial{\small \textcolor{red}{ Ps 3}}{}
\setaboveinitialseparation{0.72mm}
\setsecondannotation{\small ij. T.}

\includescore{psalmi/3/ps3.tex}

\vspace{0.3cm}

\psal{3}

\begin{quote}
\begin{verse}
Herr, wie zahlreich sind meine Bedränger; *\\
so viele stehen gegen mich auf.\\ 
\vin Viele gibt es, die von mir sagen: *\\ 
\vin «Er findet keine Hilfe bei Gott.» \\
Du aber, Herr, bist ein Schild für mich, *\\
du bist meine Ehre und richtest mich auf.\\ 
\vin \textit{Ich habe laut zum Herrn gerufen; *\\ 
\vin da erhörte er mich von seinem \\
\vin heiligen Berg}e.\\ 
Ich lege mich nieder und schlafe ein, *\\
ich wache wieder auf, denn\\
der Herr beschützt mich.\\ 
\vin Viele Tausende von Kriegern\\ 
\vin fürchte ich nicht, *\\ 
\vin wenn sie mich ringsum belagern.\\ 
Herr, erhebe dich, *\\
mein Gott, bring mir Hilfe! \\ 
\vin Denn all meinen Feinden hast du \\ 
\vin den Kiefer zerschmettert, *\\ 
\vin hast den Frevlern die Zähne zerbrochen.\\
Beim Herrn findet man Hilfe.*\\
Auf dein Volk komme dein Segen! \\
\end{verse}
\end{quote}

\vspace{0.3cm}
\setspaceafterinitial{5.2mm plus 0em minus 0em}
\setspacebeforeinitial{4.2mm plus 0em minus 0em}
\def\greinitialformat#1{{\fontsize{40}{40}\selectfont #1}}
\gresetfirstlineaboveinitial{\small \textcolor{red}{Ps 5}}{}
\setaboveinitialseparation{0.72mm}
\setsecondannotation{\small viij. T.}
\includescore{psalmi/5/ps5.tex}

\vspace{0.3cm}

\psal{5}

\begin{quote}
\begin{verse}
 \textit{Höre meine Worte, Herr,} *\\ 
achte auf mein Seufzen!\\ 
\vin Vernimm mein lautes Schreien,\\ 
\vin mein König und mein Gott, *\\ 
\vin denn ich flehe zu dir.\\  
Herr, am Morgen hörst du mein Rufen, *\\ 
am Morgen rüst ich das Opfer zu, \\
halte Ausschau nach dir.\\ 
\vin Denn du bist kein Gott, dem das\\ 
\vin Unrecht gefällt; *\\ 
\vin der Frevler darf nicht bei dir weilen.\\  
Wer sich brüstet, besteht nicht\\
vor deinen Augen; *\\ 
denn dein Hass trifft alle, die Böses tun.\\ 
\vin Du lässt die Lügner zugrunde gehn, *\\ 
\vin Mörder und Betrüger sind dem Herrn\\ 
\vin ein Gräuel.\\ 
Ich aber darf dein Haus betreten *\\ 
dank deiner großen Güte,\\ 
\vin ich werfe mich nieder in Ehrfurcht *\\ 
\vin vor deinem heiligen Tempel.\\ 
Leite mich, Herr, in deiner Gerechtigkeit, †\\
meinen Feinden zum Trotz; *\\  
ebne deinen Weg vor mir!\\ 
\vin Aus ihrem Mund kommt kein wahres Wort, *\\ 
\vin ihr Inneres ist voll Verderben.\\  
Ihre Kehle ist ein offenes Grab, *\\  
aalglatt ist ihre Zunge.\\ 
\vin Gott, lass sie dafür büßen; *\\ 
\vin sie sollen fallen durch ihre eigenen Ränke.\\  
Verstoße sie wegen ihrer vielen Verbrechen; *\\  
denn sie empören sich gegen dich.\\ 
\vin Doch alle sollen sich freuen, \\ 
\vin die auf dich vertrauen, *\\ 
\vin und sollen immerfort jubeln.\\  
Beschütze alle, die deinen Namen lieben, *\\  
damit sie dich rühmen.\\ 
\vin Denn du, Herr, segnest den Gerechten. *\\ 
\vin Wie mit einem Schild deckst du ihn\\ 
\vin mit deiner Gnade.\\ 
\end{verse}
\end{quote}

\newpage

\setspaceafterinitial{5.2mm plus 0em minus 0em}
\setspacebeforeinitial{4.2mm plus 0em minus 0em}
\def\greinitialformat#1{{\fontsize{40}{40}\selectfont #1}}
\gresetfirstlineaboveinitial{\small \textcolor{red}{1 Chr 29}}{}
\setaboveinitialseparation{0.72mm}
\setsecondannotation{\small j. T.}
\includescore{cantica/ca/1chr29.tex}


\cant{1 Chr 29,10-13}

\begin{verse}[\versewidth]

Gepriesen bist du, Herr †\\
Gott unseres Vaters Israel, *\\
von Ewigkeit zu Ewigkeit!\\
\vin Dein, Herr, sind Größe und Kraft, †\\
\vin Ruhm und Glanz und Hoheit; *\\
dein ist alles im Himmel und auf Erden.\\
\vin Herr, dein ist das Königtum. *\\
\vin Du erhebst dich als Haupt über alles.\\
Reichtum und Ehre kommen von dir; *\\
du bist der Herrscher über das All.\\
\vin In deiner Hand liegen Kraft und Stärke; *\\
\vin von deiner Hand kommt alle Größe\\ 
\vin und Macht.\\
Darum danken \textit{wir} dir\textit{, unser Gott,} *\\
und \textit{rühmen deinen herrlichen Namen.}\\

\end{verse}


\rot{vel}


\setspaceafterinitial{5.2mm plus 0em minus 0em}
\setspacebeforeinitial{4.2mm plus 0em minus 0em}
\def\greinitialformat#1{{\fontsize{40}{40}\selectfont #1}}
\gresetfirstlineaboveinitial{\small \textcolor{red}{Sir 36}}{}
\setaboveinitialseparation{0.72mm}
\setsecondannotation{\small 3. T.}
\includescore{cantica/ca/sir36.tex}

\medskip

\begin{sloppypar}
{\noindent\rm{Ant. Zeige uns, Herr, das Licht deiner Barmherzigkeit.}} 
\end{sloppypar}

\medskip
\medskip



\cant{Sir 36,1-7.13-16}

\begin{quote}
\begin{verse}
Rette uns, du G\d ott des Alls, *\\  
und wirf deinen Schrecken auf alle Völker!\\ 
\vin Schwing deine Hand\\ 
\vin g\d egen das fremde Volk, *\\ 
\vin damit es deine mächtigen Taten sieht.\\ 
Wie du dich an uns vor ihren Augen\\
als h\d eilig bezeugt hast, *\\ 
so verherrliche dich an ihnen vor unseren Augen,\\ 
\vin damit sie erkennen, wie w\d ir  es erkannten: *\\ 
\vin Es gibt keinen Gott außer dir.\\ 
Erneuere die Zeichen, wiederh\d ole die Wunder, *\\ 
zeige die Macht deiner Hand\\
und die Kraft deines rechten Armes!\\ 
\vin Sammle alle St\d ämme Jakobs, *\\ 
\vin verteil den Erbbesitz wie in den\\ 
\vin Tagen der Vorzeit!\\  
Hab Erbarmen mit dem Volk, \\
das d\d einen Namen trägt, *\\ 
mit Israel, den du deinen Erstgeborenen\\
nanntest.\\ 
\vin Hab Erbarmen mit deiner h\d eiligen Stadt, *\\ 
\vin mit Jerusalem, dem Ort, wo du wohnst.\\ 
Erfülle Zion mit d\d einem Glanz *\\ 
und deinen Tempel mit deiner Herrlichkeit!\\ 
\end{verse}
\end{quote}

\medskip

\setspaceafterinitial{5.2mm plus 0em minus 0em}
\setspacebeforeinitial{4.2mm plus 0em minus 0em}
\def\greinitialformat#1{{\fontsize{40}{40}\selectfont #1}}
\gresetfirstlineaboveinitial{\small \textcolor{red}{Ps 36}}{}
\setaboveinitialseparation{0.72mm}
\setsecondannotation{\small viij. T.}

% and finally we include the score. The file must be in the same directory as this one.
\includescore{psalmi/35/ps35.tex}


\medskip
\smallskip


\psal{36}

\begin{quote}
\begin{verse}
Der Frevler spricht:\\
«Ich bin entschlossen zum Bösen.» *\\
In seinen Augen gibt es kein\\
Erschrecken vor Gott.\\ 
\vin Er gefällt sich darin, *\\ 
\vin sich schuldig zu machen und zu hassen.\\
Die Worte seines Mundes sind\\
Trug und Unheil; *\\
er hat es aufgegeben, weise und gut zu handeln.\\ 
\vin Unheil plant er auf seinem Lager, †\\ 
\vin er betritt schlimme Wege *\\ 
\vin und scheut nicht das Böse.\\ 
\textit{Herr, deine Güte reicht,\\
so weit der Himmel ist}, *\\
deine Treue, so weit die Wolken zieh\textit{e}n.\\ 
\vin Deine Gerechtigkeit steht wie\\ 
\vin die Berge Gottes, *\\ 
\vin deine Urteile sind tief wie das Meer.\\
Herr, du hilfst Menschen und Tieren. *\\
Gott, wie köstlich ist deine Huld! \\ 
\vin Die Menschen bergen sich\\ 
\vin im Schatten deiner Flügel, †\\ 
\vin sie laben sich am Reichtum deines Hauses; *\\ 
\vin du tränkst sie mit dem Strom deiner Wonnen.\\ 
Denn bei dir ist die Quelle des Lebens, *\\
in deinem Licht schauen wir das Licht.\\ 
\vin Erhalte denen, die dich kennen, deine Huld *\\ 
\vin und deine Gerechtigkeit den Menschen\\ 
\vin mit redlichem Herzen!\\
Lass mich nicht kommen unter den Fuß\\
der Stolzen; *\\
die Hand der Frevler soll mich nicht vertreiben.\\
\vin Dann brechen die Bösen zusammen, *\\ 
\vin sie werden niedergestoßen und können \\ 
\vin nie wieder aufstehn.\\ 

\end{verse}
\end{quote}

\medskip

\setspaceafterinitial{4.2mm plus 0em minus 0em}
\setspacebeforeinitial{4.2mm plus 0em minus 0em}
\resp

\includescore{responsoria_diebusferialibus/respbrsana.tex}

\medskip

\begin{sloppypar}
{\noindent\rm{\rot{Resp.} Heile meine Seele, denn ich habe vor dir gesündigt. Ich habe gesagt: Herr, erbarme dich meiner.}}
\end{sloppypar}

\newpage

\crot{Hymnus}

\setspaceafterinitial{5.2mm plus 0em minus 0em}
\setspacebeforeinitial{4.2mm plus 0em minus 0em}
\def\greinitialformat#1{{\fontsize{40}{40}\selectfont #1}}
\gresetfirstlineaboveinitial{\small \textcolor{red}{hieme}}{}
\setaboveinitialseparation{0.72mm}
\setsecondannotation{\small ij. T.}

\includescore{hymni/splendorpaternae.tex}

\medskip

\begin{sloppypar}
{\noindent\rm{\rot{1.} Du Abglanz von des Vaters Pracht,
du bringst aus Licht das Licht hervor,
du Licht vom Licht, des Lichtes Quell, du Tag, der unsern Tag erhellt.}}
\end{sloppypar}

\medskip

\begin{multicols}{2}
\begin{verse}[\versewidth]

{\small{\frot{V}erúsque sol, illábere\\
micans nitóre pérpeti, \\
iubárque Sancti Spíritus\\
infúnde nostris sénsibus.\\!

\frot{C}onfírmet actus strénuos, \\
dentes retúndat ínvidi,\\
casus secúndet ásperos, \\
donet geréndi grátiam.\\!

\frot{M}entem gubérnet et regat\\
casto, fidéli córpore,\\
fides calóre férveat, \\
fraudis venéna nésciat.\\!}}
\end{verse}

\columnbreak

\begin{verse}[\versewidth]

{\small\rm{\frot{2.} Du wahre Sonne, brich herein,\\
du Sonne, die nicht untergeht,\\
und mit des Geistes lichtem Strahl\\
dring tief in unsrer Sinne Grund.\\!

\frot{3.} Er stärke uns zu gutem Werk,\\
er leite machtvoll unser Tun,\\
er sei uns Kraft in harter Fron\\
und lenke unsern schwachen Geist.\\!

\frot{4.} Des Geistes Steuer führ’ als Herr\\
er in den keuschen, treuen Leib.\\
Der Glaube glüh’ in Liebe auf\\
und wisse nicht vom Gift des Trugs.\\!}}

\end{verse}
\end{multicols}

\newpage

\begin{multicols}{2}
\begin{verse}[\versewidth]

{\small{\frot{C}hristúsque nobis sit cibus, \\
potúsque noster sit fides;\\
læti bibámus sóbriam\\
ebrietátem Spíritus.\\!

\frot{D}eo Patri sit glória,\\
Ejúsque soli Fílio,\\
Cum Spíritu Paraćlito\\
Et nunc, et in perpétuum.\\
Amen.\\!}}

\end{verse}

\columnbreak

\begin{verse}[\versewidth]

{\small\rm{\frot{5.} Und Christus werde unser Brot,\\
und unser Glaube sei uns Trank,\\
in Freude werde uns zuteil\\
des Geistes klare Trunkenheit.\\!

\frot{6}. Gott, dem Vater sei Ehre\\
und seinem einzigen Sohn\\
mit dem Heiligen Geist\\
jetzt und in Ewigkeit.\\
Amen.\\!}}
\end{verse}

\end{multicols}

\medskip

\setspaceafterinitial{5.2mm plus 0em minus 0em}
\setspacebeforeinitial{4.2mm plus 0em minus 0em}
\def\greinitialformat#1{{\fontsize{40}{40}\selectfont #1}}
\gresetfirstlineaboveinitial{\small \textcolor{red}{æstate}}{}
\setaboveinitialseparation{0.72mm}
\setsecondannotation{\small viij. T.}
\includescore{hymni/ecceiamnoctis.tex}

\medskip

\begin{sloppypar}
{\noindent\rm{\frot{1.} Seht, wie die Schatten
dunkler Nacht verblassen: Früh\-lichtes Leuchten
rosenfarben glühet; Lasst uns aus allen Kräften
innig bitten Gott, den Allmächt’gen.}}
\end{sloppypar}

\begin{quote}

\begin{multicols}{2}
\begin{verse}[\versewidth]

{\normalsize{\frot{U}t Deus, nostri\\
miserátus, omnem\\
pellat languórem,\\
tríbuat salútem, \\
donet et nobis\\
pietáte patris\\
regna polórum.\\!}}

\end{verse}

\columnbreak

\begin{verse}[\versewidth]
{\normalsize\rm{\frot{2.} Dass er sich unser\\
liebevoll erbarme,\\
Schlaffheit vertreibe,\\
uns mit Heil begnade,\\
dass er uns schenke\\
einst voll Vatergüte\\
himmlische Reiche.\\!}}
\end{verse}

\end{multicols}
\end{quote}

\newpage

\begin{quote}
\begin{multicols}{2}
\begin{verse}[\versewidth]

{\normalsize{\frot{P}ræstet hoc nobis\\
Déitas beáta\\
Patris ac Nati,\\
paritérque Sancti\\
Spíritus, cuius\\
résonat per omnem\\
glória mundum.\\
Amen.\\!}}
\end{verse}

\columnbreak

\begin{verse}[\versewidth]
{\normalsize\rm{\frot{3.} Dies wolle uns der\\ 
hehre Gott verleihen,\\
Vater und Sohn samt\\
dem Heiligen Geiste,\\
deren Lob weithin\\
widerhallt durch alle\\ 
Zonen der Welten.\\
Amen.\\!}}

\end{verse}
\end{multicols}

\end{quote}



\medskip

\begin{flushleft}

\versik{Repléti sumus mane misericórdia tua.}{Exultávimus, et delectáti sumus.}

\medskip
{\rm{
\versik{Erfüllt sind wir am Morgen mit deiner Huld.}{Wir sind ausgelassen und fröhlich.}
}}
\end{flushleft}

\vspace{0.3cm}

\setspaceafterinitial{4.2mm plus 0em minus 0em}
\setspacebeforeinitial{4.2mm plus 0em minus 0em}
\def\greinitialformat#1{{\fontsize{40}{40}\selectfont #1}}
\gresetfirstlineaboveinitial{\footnotesize \textcolor{red}{Bendic}}{}
\setaboveinitialseparation{0.72mm}
\setsecondannotation{\small vij. T.}

\includescore{cantica/evangelica/erexitdominus.tex}

\smallskip

\rot{Canticum} Benedictus \rot{p. 64.}
