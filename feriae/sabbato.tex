\thispagestyle{plain}

\kapklein{\rot{S}abbato}
\kkap{SABBATO}


\section[VIGILIAE]{AD VIGILIAS}

\begin{sloppypar}
{\noindent{\rot{Invit.} Den Herrn, unseren Gott, * kommt, lasset uns ihn anbeten.}}
\end{sloppypar}

\bigskip

\noindent{\rot{Hymnus} Æterne rerum Conditor \rot{vel} Nocte surgentes \rot{ut in Dominica pp. 1-3.}}


\section{VIGILIA I}

\begin{sloppypar}
{\noindent{1. Ant.}  Lass mich nicht sündigen * mit meiner Zunge. \rot{Ps. 39}\\
2. Ant. Der Herr hat auf mich geschaut * und meine Bitten gehört. \rot{Ps. 40}\\
3. Ant. Heile Herr meine Seele * denn ich habe gegen dich gesündigt. \rot{Ps. 41} \\}
\end{sloppypar}

\begin{flushleft}

\versik{Mein Volk, höre auf meine Weisung.}{Neigt euer Ohr zu den Worten meines Mundes.}

\end{flushleft}

\section{VIGILIA II}

\begin{sloppypar}
{\noindent{1. Ant.} Preist unseren Gott * alle Völker. \rot{Ps. 66}\\
2. Ant. Preist den Herrn * in großer Gemeinde. \rot{Ps. 68,1-17} - Divisio -  \rot{Ps. 68, 18-36}}\\ 

\end{sloppypar}

\begin{flushleft}

\versik{Zeige, Herr, deine Macht.}{Vollende, o Gott, was du in uns begonnen hast.}

\end{flushleft}



\section{VIGILIA III}


\begin{sloppypar}
{\noindent{1. Ant.} Du allein bist der Höchste * auf der ganzen Erde. \rot{Ps. 83}\\
2. Ant. Selig * wer in deinem Hause wohnt, o Herr. \rot{Ps. 84}\\
3. Ant. Herr * mit Segen erfüllst du dein Land. \rot{Ps. 85} \\}
\end{sloppypar}

\begin{flushleft}

\versik{Gott, wir preisen dich.}{Und rufen deinen Namen an.}

\end{flushleft}

\section{VIGILIA IV}


\begin{sloppypar}
{\noindent{1. Ant.} Aus all meinen Nöten * befreie mich, o Herr. \rot{Ps. 107,23-43} \\
2. Ant. Ich preise dich vor den Völkern * denn deine Güte reicht so weit der Himmel ist. \rot{Ps. 108} \\
3. Ant. Ich preise dich, Herr, * mit meinem Mund. \rot{Ps. 109} \\}
\end{sloppypar}

\begin{flushleft}

\versik{Er sandte sein Wort und hat sie geheilt.}{Er entriss sie all ihren Ängsten}

\end{flushleft}

\section[LAUDES]{AD LAUDES}

\setspaceafterinitial{5.2mm plus 0em minus 0em}
\setspacebeforeinitial{4.2mm plus 0em minus 0em}
\def\greinitialformat#1{{\fontsize{40}{40}\selectfont #1}}
\gresetfirstlineaboveinitial{\small \textcolor{red}{ Ps 143 }}{}
\setaboveinitialseparation{0.72mm}
\setsecondannotation{\small viij. T.}

\includescore{psalmi/142/ps142.tex}

\vspace{0.3cm}

\crot{Psalm 143}

\vspace{0.3cm}
\begin{quote}

\begin{verse}
 Herr, höre mein Gebet, vernimm mein Flehen; *\\
\textit{in deiner Treue erhöre mich}, in \\deiner Gerechtigkeit!\\
\vin Geh mit deinem Knecht nicht ins Gericht; *\\
\vin denn keiner, der lebt, ist gerecht vor dir. \\
Der Feind verfolgt mich, tritt mein \\ Leben zu Boden, *\\
er lässt mich in der Finsternis wohnen wie \\ längst Verstorbene. \\
\vin Mein Geist verzagt in mir, *\\
\vin mir erstarrt das Herz in der Brust. \\
Ich denke an die vergangenen Tage, †\\
sinne nach über all deine Taten, *\\ erwäge das Werk deiner Hände. \\
\vin Ich breite die Hände aus (und bete) zu dir; *\\
\vin meine Seele dürstet nach\\ \vin  dir wie lechzendes Land. \\
Herr, erhöre mich bald, *\\
denn mein Geist wird müde; \\
\vin verbirg dein Antlitz nicht vor mir, * \\
\vin  damit ich nicht werde wie Menschen, \\ \vin die längst begraben sind.\\
Lass mich deine Huld \\erfahren am frühen Morgen; *\\
denn ich vertraue auf dich. \\
\vin Zeig mir den Weg, den ich gehen soll; *\\ \vin denn ich erhebe meine Seele zu dir. \\
Herr, entreiß mich den Feinden! *\\
Zu dir nehme ich meine Zuflucht.\\
\vin Lehre mich, deinen Willen zu tun; †\\ \vin denn  du bist mein Gott. *\\
\vin Dein guter Geist leite mich auf ebenem Pfad\textit{e}.\\ 
Um deines Namens willen, Herr, erhalt\\mich am Leben, *\\
führe mich heraus aus der Not in deiner\\ Gerechtigkeit!\\
\vin Vertilge in deiner Huld meine Feinde, †\\
\vin lass all meine Gegner untergehn! * \\ \vin Denn ich bin dein Knecht. \\


\end{verse}

\end{quote}




\vspace{0.3cm}
\setspaceafterinitial{5.2mm plus 0em minus 0em}
\setspacebeforeinitial{4.2mm plus 0em minus 0em}
\def\greinitialformat#1{{\fontsize{40}{40}\selectfont #1}}
\gresetfirstlineaboveinitial{\small \textcolor{red}{Ex 15}}{}
\setaboveinitialseparation{0.72mm}
\setsecondannotation{\small 7. T.}
\includescore{cantica/ca/ex15.tex}

\vspace{0.3cm}

\cant{Ex 15,1-18}

\begin{quote}
\begin{verse}


Ich singe dem Herrn ein Lied, †\\
denn er ist h\d och und erhaben. *\\ 
Rosse und Wagen w\d arf er ins Meer.\\
\vin \textit{Meine Stärke und mein \\ \vin L\d ied ist der Herr, *\\
\vin er ist für mich zum R\d etter geworden.} \\
Er ist mein Gott, \d ihn will ich preisen; *\\ 
den Gott meines Vaters w\d ill ich rühmen. \\
\vin Der H\d err ist ein Krieger, *\\
\vin Jahwe \d ist sein Name.\\
Pharaos Wagen und seine Streitmacht †\\
w\d arf er ins Meer. *\\ 
Seine besten Kämpfer vers\d anken im Schilfmeer.\\
\vin Fluten d\d eckten sie zu, *\\
\vin sie sanken in die T\d iefe wie Steine.\\
Deine Rechte, Herr, ist h\d errlich an Stärke; *\\
deine Rechte, Herr, zerschm\d ettert den Feind.\\
\vin In deiner erhabenen Größe wirfst  du die \\ \vin G\d egner zu Boden.*\\
\vin Du sendest deinen Zorn; er  fr\d isst sie \\ \vin wie Stoppeln.\\
Du schnaubtest vor Zorn, da  \\türmte sich Wasser, †\\
 da standen W\d ogen als Wall, *\\
Fluten erstarrten im H\d erzen des Meeres.\\
\vin Da sagte der Feind:  Ich j\d age nach, hole ein. *\\
\vin Ich teile die Beute, ich st\d ille die Gier. \\
 Ich z\d ücke mein Schwert,*\\
 meine H\d and jagt sie davon.\\
\vin Da schnaubtest du Sturm. Das  Meer \\ \vin d\d eckte sie zu. *\\
\vin Sie sanken wie Blei ins t\d osende Wasser.\\

\vspace{0.3cm}

\crot{divisio}

\vspace{0.3cm}

Wer ist wie du unter den Göttern, o Herr? †\\
Wer ist wie du gew\d altig und heilig, *\\ gepriesen als furchtbar, W\d under vollbringend?\\
\vin Du strecktest d\d eine Rechte aus, *\\
\vin da verschl\d ang sie die Erde.\\
Du lenktest in deiner Güte das\\  Volk, das d\d u erlöst hast, *\\
du führtest sie machtvoll zu deiner h\d eiligen \\Wohnung.\\
\vin Als die Völker das hörten, erz\d itterten sie, *\\
\vin die Philister p\d ackte das Schütteln.\\
Damals erschraken die Häuptlinge Edoms,†\\
die Mächtigen von Moab p\d ackte das Zittern, *\\ Kanaans Bewohner, sie \d alle verzagten.\\
\vin Schrecken und F\d urcht überfiel sie, *\\
\vin sie erstarrten zu Stein vor\\ \vin der M\d acht deines Armes, \\
bis hind\d urchzog, o Herr, dein Volk, *\\ bis hindurchzog das V\d olk, das du erschufst.\\
\vin Du brachtest sie hin und  pfl\d anztest sie ein *\\ \vin auf dem Berg d\d eines Erbes.\\
 Einen Ort, wo du thronst, Herr, h\d ast \\du gemacht; *\\ ein Heiligtum, Herr, haben deine  H\d ände \\gegründet.\\
\vin Der H\d err ist König *\\ \vin für \d immer und ewig. \\

\end{verse}
\end{quote}
\vspace{0.3cm}

\rot{vel}

\setspaceafterinitial{5.2mm plus 0em minus 0em}
\setspacebeforeinitial{4.2mm plus 0em minus 0em}
\def\greinitialformat#1{{\fontsize{40}{40}\selectfont #1}}
\gresetfirstlineaboveinitial{\small \textcolor{red}{Deut 32}}{}
\setaboveinitialseparation{0.72mm}
\setsecondannotation{\small vj. T.}
\includescore{cantica/ca/deut32.tex}

\vspace{0.3cm}

\cant{Dtn 32,1-12}
\begin{quote}
\begin{verse}

 
Hört zu, ihr Himmel, ich will reden,*\\
die Erde lausch\d e meinen Worten.\\
\vin Meine Lehre wird strömen wie Regen,*\\
\vin meine Botschaft w\d ird fallen wie Tau,\\
wie Regentropfen auf das Gras *\\
und wie Tauperl\d en auf die Pflanzen.\\
\vin Ich will den Namen des Herrn verkünden.*\\
\vin \textit{Preist die Größe \d unseres Gottes!}\\
Er heißt: der Fels. Vollkommen  ist, was er tut;*\\
denn alle sein\d e Wege sind recht.\\
\vin Er ist ein unbeirrbar treuer Gott,*\\
\vin er ist ger\d echt und gerade.\\
Ein falsches, verdrehtes \\  Geschlecht fiel von ihm ab,*\\
Verkrüppelte, die nicht m\d ehr seine Söhne sind.\\
\vin Ist das euer Dank an den Herrn,*\\
\vin du dummes, v\d erblendetes Volk?\\
Ist er nicht dein Vater, dein Schöpfer?*\\
Hat er dich nicht g\d eformt und hingestellt? \\

Denk an die Tage der Vergangenheit,*\\
lerne aus den Jahr\d en der Geschichte!\\
\vin Frag deinen Vater, er wird es dir erzählen,*\\
\vin frage die Alten, sie werd\d en es dir sagen. \\

\vspace{0.3cm}

\crot{divisio}
\vspace{0.3cm}

Als der Höchste (den Göttern) \\ die Völker übergab,*\\
als er die M\d enschheit aufteilte,\\
\vin legte er die Gebiete der Völker *\\
\vin nach d\d er Zahl der Götter fest;\\
der Herr nahm sich sein Volk als Anteil,*\\
Jakob w\d urde sein Erbland.\\
\vin Er fand ihn in der Steppe,*\\
\vin in der Wüste, wo w\d ildes Getier heult.\\
Er hüllte ihn ein, gab auf ihn Acht *\\
und hütete ihn w\d ie seinen Augenstern,\\
\vin wie der Adler, der sein Nest beschützt *\\
\vin und üb\d er seinen Jungen schwebt,\\
der seine Schwingen ausbreitet, \\ein Junges ergreift *\\
und es flügelschl\d agend davonträgt\\
\vin Der Herr allein hat Jakob geleitet,*\\
\vin kein fremder Gott st\d and ihm zur Seite.\\

 
\end{verse}
\end{quote}
\vspace{0.3cm}

\setspaceafterinitial{5.2mm plus 0em minus 0em}
\setspacebeforeinitial{4.2mm plus 0em minus 0em}
\def\greinitialformat#1{{\fontsize{40}{40}\selectfont #1}}
\gresetfirstlineaboveinitial{\small \textcolor{red}{Ps 150}}{}
\setaboveinitialseparation{0.72mm}
\setsecondannotation{\small viij. T.}

\includescore{psalmi/150/ps150.tex}

\vspace{0.3cm}

\crot{Psalm 150}
\begin{quote}
\begin{verse}
 Lobt Gott in seinem Heiligtum, *\\
lobt ihn in seiner mächtigen Feste! \\
\vin Lobt ihn für seine großen Taten, *\\
\vin \textit{lobt ihn in seiner gewaltigen Größe!}\\
Lobt ihn mit dem Schall der Hörner, *\\
lobt ihn mit Harfe und Zither! \\
\vin Lobt ihn mit Pauken und Tanz, *\\
\vin lobt ihn mit Flöten und Saitenspiel! \\
Lobt ihn mit hellen Zimbeln, *\\
lobt ihn mit klingenden Zimbeln!\\
\vin Alles, was atmet, *\\
\vin lobe den Herr\textit{e}n!\\

\end{verse}
\end{quote}


\noindent\rot{Resp.br.} Sana animam \rot{ut in Feria Secunda p. 43.}\\
\noindent\rot{Hymnus} Splendor paternæ \rot{vel} Ecce iam noctis \rot{ut in Dominica p. 15-17.}
\begin{flushleft}

\versik{Repléti sumus mane misericórdia tua.}{Exultávimus, et delectáti sumus.}

\medskip

{\rm{
\versik{Am Morgen sind wir erfüllt von deiner Huld.}{Wir sind (voll) Jubel und Frohsinn.}
}}
\end{flushleft}

\setspaceafterinitial{5.2mm plus 0em minus 0em}
\setspacebeforeinitial{4.2mm plus 0em minus 0em}
\def\greinitialformat#1{{\fontsize{40}{40}\selectfont #1}}
\gresetfirstlineaboveinitial{\small \textcolor{red}{Benedic.}}{}
\setaboveinitialseparation{0.72mm}
\setsecondannotation{\small j. T.}

\includescore{cantica/evangelica/inviapacis.tex}

\vspace{0.3cm}
\rot{Canticum} Benedictus \rot{p. 196.}


\section[HORA TERTIA]{AD TERTIAM}

\rot{Hymnus} Nunc sancte nobis Spiritus \rot{p. 190.}

\vspace{0.3cm}

\setspaceafterinitial{3.2mm plus 0em minus 0em}
\setspacebeforeinitial{4.2mm plus 0em minus 0em}
\def\greinitialformat#1{{\fontsize{40}{40}\selectfont #1}}
\gresetfirstlineaboveinitial{\small \textcolor{red}{Ps 119g}}{}
\setaboveinitialseparation{0.72mm}
\setsecondannotation{\small iv. T.}


\includescore{psalmi/118/videhumilitatem20.tex}


\includescore{psalmi/tertia/horatertia_sabbato.tex}


\vspace{0.3cm}


\begin{flushleft}

\versik{Dóminus non privábit bonis eos qui ámbulant in innocéntia.}{Dómine virtútum, beátus homo qui sperat in te.}

\medskip

{\rm{
\versik{Der Herr wird seine Güter nicht denen vorenthalten, die in Unschuld wandeln.}{Herr der Scharen, selig der Mensch, der auf dich hofft.}
}}
\end{flushleft}


\section[HORA SEXTA]{AD SEXTAM}

\rot{Hymnus} Rector potens, verax Deus \rot{p. 192.}

\vspace{0.3cm}

\setspaceafterinitial{5.2mm plus 0em minus 0em}
\setspacebeforeinitial{4.2mm plus 0em minus 0em}
\def\greinitialformat#1{{\fontsize{40}{40}\selectfont #1}}
\gresetfirstlineaboveinitial{\small \textcolor{red}{Ps 18c sq}}{}
\setaboveinitialseparation{0.72mm}
\setsecondannotation{\small viij. T.}

\includescore{psalmi/19/exaudiatte19.tex}


\includescore{psalmi/sexta/sextpsalmisabbato.tex}
\vspace{0.3cm}



\begin{flushleft}

\versik{Dómine, veritátem in corde dilexísti.}{Et in occúlto sapiéntiam manifestásti mihi.}

\medskip

{\rm{
\versik{Herr, du liebst, die im Herzen voll Wahrheit sind.}{Und im Verborgenen lehrtest du mich Weisheit.}
}}
\end{flushleft}



\section[HORA NONA]{AD NONAM}

\rot{Hymnus} Rerum Deus tenax vigor \rot{p. 194.}.

\vspace{0.3cm}

 \setspaceafterinitial{5.2mm plus 0em minus 0em}
\setspacebeforeinitial{4.2mm plus 0em minus 0em}
\def\greinitialformat#1{{\fontsize{40}{40}\selectfont #1}}
\gresetfirstlineaboveinitial{\small \textcolor{red}{ Ps 136 }}{}
\setaboveinitialseparation{0.72mm}
\setsecondannotation{\small 3. T.}

\includescore{psalmi/135/quoniaminaeternum135.tex}
\vspace{0.3cm}

\includescore{psalmi/nona/nonpsalmisabbato.tex}

\begin{flushleft}

\versik{Fac cum servo tuo secúndum misericórdiam tuam, Dómine.}{Iustificatiónes tuas doce me.}

\medskip

{\rm{
\versik{Handle an deinem Knecht nach deiner Huld.}{Lehre mich deine Gesetze.}
}}
\end{flushleft}



\section[VESPERAE]{AD VESPERAS}



\setspaceafterinitial{3.2mm plus 0em minus 0em}
\setspacebeforeinitial{3.2mm plus 0em minus 0em}
\def\greinitialformat#1{{\fontsize{40}{40}\selectfont #1}}
\gresetfirstlineaboveinitial{\small \textcolor{red}{Ps 146}}{}
\setaboveinitialseparation{0.72mm}
\setsecondannotation{\small iv. T.}

\includescore{psalmi/145/laudabodeum.tex}

\vspace{0.3cm}

\crot{Psalm 146}

\begin{quote}

\begin{verse}
 Lobe den Herrn, meine Seele! †\\
\textit{Ich will den Herrn loben, solange ich lebe,} *\\
meinem Gott singen und spielen,\\ sol\d ange ich da bin. \\
\vin Verlasst euch nicht auf Fürsten, *\\
\vin auf Menschen, bei denen es d\d och \\ \vin keine Hilfe gibt. \\
Haucht der Mensch sein Leben aus †\\
und kehrt er zurück zur Erde, * \\
dann ist es aus mit \d all seinen Plänen. \\
\vin Wohl dem, dessen Halt  der Gott Jakobs ist *\\
\vin und der seine Hoffnung auf den\\ \vin H\d errn, seinen Gott, setzt. \\
Der Herr hat Himmel und Erde gemacht, †\\
das Meer und alle Geschöpfe; * \\
er hält \d ewig die Treue.\\
\vin Recht verschafft er den Unterdrückten, †\\
\vin den Hungernden gibt er Brot; *\\
\vin  der Herr befr\d eit die Gefangenen. \\
Der Herr öffnet den Blinden die Augen, *\\
er richt\d et die Gebeugten auf. \\
\vin Der Herr beschützt die Fremden *\\
\vin und verhilft den Waisen \\ \vin und W\d itwen zu ihrem Recht. \\
Der Herr liebt die Gerechten, * \\
doch die Schritte der Frevler leitet \d er in die Irre.\\
\vin Der Herr ist König auf ewig, *\\
\vin dein Gott, Zion, herrscht von G\d eschlecht \\ \vin zu Geschlecht.\\

\end{verse}

\end{quote}
\vspace{0.3cm}

\setspaceafterinitial{3.2mm plus 0em minus 0em}
\setspacebeforeinitial{3.2mm plus 0em minus 0em}
\def\greinitialformat#1{{\fontsize{40}{40}\selectfont #1}}
\gresetfirstlineaboveinitial{\small \textcolor{red}{Ps 147a}}{}
\setaboveinitialseparation{0.72mm}
\setsecondannotation{\small 8. T.}

\includescore{psalmi/146/deonostro.tex}


\vspace{0.3cm}

\crot{Psalm 147a}
\begin{quote}
\begin{verse}
 Gut ist es, unserm Gott zu singen; *\\
\textit{schön ist es, ihn zu loben}. \\
\vin Der Herr baut Jerusalem wieder auf, *\\
\vin er sammelt die Versprengten Israels. \\
Er heilt die gebrochenen Herzen *\\
und verbindet ihre schmerzenden Wunden. \\
\vin Er bestimmt die Zahl der Sterne *\\
\vin und ruft sie alle mit Namen. \\
Groß ist unser Herr und gewaltig an Kraft, *\\
unermesslich ist seine Weisheit. \\
\vin Der Herr hilft den Gebeugten auf *\\
\vin und erniedrigt die Frevler.\\
Stimmt dem Herrn ein Danklied an, *\\
spielt unserm Gott auf der Harfe!\\
\vin Er bedeckt den Himmel mit Wolken, †\\
\vin spendet der Erde Regen *\\
 \vin und lässt Gras auf den Bergen sprießen. \\
Er gibt dem Vieh seine Nahrung, *\\
gibt den jungen Raben, wonach sie schreien. \\
\vin Er hat keine Freude an der Kraft \\ \vin des Pferdes, *\\
\vin kein Gefallen am schnellen Lauf des Mannes. \\
Gefallen hat der Herr an denen, die \\ ihn fürchten und ehren, *\\
die voll Vertrauen warten auf seine Huld.\\
\end{verse}

\end{quote}

\vspace{0.6cm}

\setspaceafterinitial{3.2mm plus 0em minus 0em}
\setspacebeforeinitial{4.3mm plus 0em minus 0em}
\def\greinitialformat#1{{\fontsize{40}{40}\selectfont #1}}
\gresetfirstlineaboveinitial{\small \textcolor{red}{Ps 147b}}{}
\setaboveinitialseparation{0.72mm}
\setsecondannotation{\small 1. T.}
\includescore{psalmi/147/laudajerusalem.tex}

\vspace{0.3cm}

\crot{Psalm 147b}
\begin{quote}
\begin{verse}
\textit{ Jerusalem, preise den Herrn,} *\\
lobsinge, Zion, deinem Gott! \\
\vin Denn er hat die Riegel deiner Tore\\ \vin fest gemacht, *\\
\vin die Kinder in deiner Mitte gesegnet;\\
er verschafft deinen Grenzen Frieden *\\
und sättigt dich mit bestem Weizen. \\
\vin Er sendet sein Wort zur Erde, *\\
\vin rasch eilt sein Befehl dahin. \\
Er spendet Schnee wie Wolle, *\\
streut den Reif aus wie Asche. \\
\vin Eis wirft er herab in Brocken, *\\
\vin vor seiner Kälte erstarren die Wasser.\\
Er sendet sein Wort aus und sie schmelzen, *\\
er lässt den Wind wehen, dann \\rieseln die Wasser.\\
\vin Er verkündet Jakob sein Wort, *\\
\vin Israel seine Gesetze und Rechte. \\
An keinem andern Volk hat er so gehandelt, *\\
keinem sonst seine Rechte verkündet. \\
\end{verse}
\end{quote}

\vspace{0.3cm}

\setspaceafterinitial{7.2mm plus 0em minus 0em}
\setspacebeforeinitial{5.2mm plus 0em minus 0em}
\def\greinitialformat#1{{\fontsize{40}{40}\selectfont #1}}
\gresetfirstlineaboveinitial{\small \textcolor{red}{Phil 2,1-6}}{}
\setaboveinitialseparation{0.72mm}
\setsecondannotation{\small 7. T.}

\includescore{cantica/cn/jesusnazarenus.tex}

\noindent{\rm{Ant. Jesus von Nazaret, König der Juden, Herrscher über den Himmel und Herr über alle Herrscher.}}

\cant{Phil 2,6-11}

\begin{quote}
\begin{verse}


Christus J\d esus war Gott gleich,*\\
hielt aber nicht daran f\d est, wie Gott zu sein,\\
\vin sondern er entäußerte sich und \\ \vin wurde w\d ie ein Sklave *\\
\vin \d und den Menschen gleich. \\
sein Leben war das eines Menschen; †\\
er erniedrigte sich und \\war geh\d orsam bis zum Tod,* \\
bis zum Tod am Kreuze.\\
\vin Darum hat ihn Gott über \d alle erhöht *\\
\vin und ihm den Namen verliehen, \\
\vin der über alle N\d amen erhaben ist.\\
damit alle im Himmel, auf der \\Erde und \d unter der Erde*\\
ihre Knie beugen vor den Namen Jesu\\
\vin und jeder Mund bekennt: †\\
\vin „Jesus Chr\d istus ist der Herr!“ *\\
\vin zur Ehre G\d ottes des Vaters.\\	

\end{verse}

\end{quote}

\vspace{0.3cm}



\setspaceafterinitial{4.2mm plus 0em minus 0em}
\setspacebeforeinitial{4.2mm plus 0em minus 0em}
\resp

\includescore{responsoria_diebusferialibus/respbrmagnusdominus.tex}

\rm{Resp.br. Groß ist unser Herr und gewaltig an Kraft, unermesslich ist seine Weisheit.}

\bf
\vspace{0.6cm}


 \crot{Hymnus}


\setspaceafterinitial{4.2mm plus 0em minus 0em}
\setspacebeforeinitial{4.2mm plus 0em minus 0em}
\def\greinitialformat#1{{\fontsize{45}{45}\selectfont #1}}
\gresetfirstlineaboveinitial{\small viij. T.}{}
\includescore{hymni/oluxbeata.tex}

\vspace{0.3cm}
\noindent{\rm{Hymn. 1. Schon weicht der Sonne Feuerstrahl. 
Du, ew'ges Licht, Einmütigkeit, 
Du selige Dreieinigkeit, 
gieß Lieb in uns're Herzen ein.\\ 2. Durch Lobgesang wir bitten Dich
morgens und abends inniglich
und uns'rem Lob sei zugeneigt
mit Demut in der Himmelsschar.\\
3. Dem Vater und dem Sohn zugleich
und Dir, Du Geist der Heiligkeit, 
sei, wie es schon seit jeher war, 
beständig Ehr' in Ewigkeit.}}
\begin{flushleft}

\versik{Vespertína orátio ascéndat ad te, Dómine.}{Et descéndat super nos misericórdia tua.}

\medskip
{\rm{
\versik{Unser Abendgebet steige auf zu dir, Herr.}{Und es senke sich auf uns herab dein Erbarmen.}
}}
\end{flushleft}





\setspaceafterinitial{5.2mm plus 0em minus 0em}
\setspacebeforeinitial{4.2mm plus 0em minus 0em}
\def\greinitialformat#1{{\fontsize{40}{40}\selectfont #1}}
\gresetfirstlineaboveinitial{\small \textcolor{red}{Magni.}}{}
\setaboveinitialseparation{0.72mm}
\setsecondannotation{\small viij. T.}

\includescore{cantica/evangelica/magnificatanimamea.tex}
\medskip

\rot{Canticum} Magnificat \rot{p. 200.}


\newpage