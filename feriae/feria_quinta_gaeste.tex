
\kkap{DONNERSTAG}


\section[DONNERSTAG]{AD TERTIAM}

\rot{Hymnus} Nunc sancte nobis Spiritus \rot{p. 124.}

\vspace{0.3cm}

\setspaceafterinitial{5.2mm plus 0em minus 0em}
\setspacebeforeinitial{4.2mm plus 0em minus 0em}
\def\greinitialformat#1{{\fontsize{40}{40}\selectfont #1}}
\gresetfirstlineaboveinitial{\small \textcolor{red}{Ps 119e}}{}
\setaboveinitialseparation{0.72mm}
\setsecondannotation{\small viij. T.}

\includescore{psalmi/118/adiuvame14.tex}

\includescore{psalmi/tertia/horatertia_quinta.tex}

\vspace{0.3cm}

\begin{flushleft}

\versik{Dóminus non privábit bonis eos qui ámbulant in innocéntia.}{Dómine virtútum, beátus homo qui sperat in te.}

\medskip

{\rm{
\versik{Der Herr wird seine Güter nicht denen vorenthalten, die in Unschuld wandeln.}{Herr der Scharen, selig der Mensch, der auf dich hofft.}
}}
\end{flushleft}

\section{MITTAGESHORE, I. WOCHE}

\rot{Hymnus} Rector potens, verax Deus \rot{p. 126.}

\vspace{0.3cm}

\setspaceafterinitial{5.2mm plus 0em minus 0em}
\setspacebeforeinitial{4.2mm plus 0em minus 0em}
\def\greinitialformat#1{{\fontsize{40}{40}\selectfont #1}}
\gresetfirstlineaboveinitial{\small \textcolor{red}{ Ps 16-18a}}{}
\setaboveinitialseparation{0.72mm}
\setsecondannotation{\small ij. T.}

\includescore{psalmi/15/conserva1516117.tex}


\vspace{0.3cm}

\includescore{psalmi/sexta/sextpsalmiquinta.tex}

\begin{flushleft}

\versik{Dómine, veritátem in corde dilexísti.}{Et in occúlto sapiéntiam manifestásti mihi.}

\medskip

{\rm{
\versik{Herr, du liebst, die im Herzen voll Wahrheit sind.}{Und im Verborgenen lehrtest du mich Weisheit.}
}}
\end{flushleft}


\section{MITTAGESHORE, II. WOCHE}

\rot{Hymnus} Rector potens, verax Deus \rot{p. 126.}

\vspace{0.3cm}

 \setspaceafterinitial{5.2mm plus 0em minus 0em}
\setspacebeforeinitial{4.2mm plus 0em minus 0em}
\def\greinitialformat#1{{\fontsize{40}{40}\selectfont #1}}
\gresetfirstlineaboveinitial{\small \textcolor{red}{Ps 132sq}}{}
\setaboveinitialseparation{0.72mm}
\setsecondannotation{\small 1. T.}

\includescore{psalmi/131/etomnismansutudiniseius.tex}
\vspace{0.3cm}


\includescore{psalmi/nona/nonpsalmiquinta.tex}



\begin{flushleft}

\versik{Fac cum servo tuo secúndum misericórdiam tuam, Dómine.}{Iustificatiónes tuas doce me.}

\medskip

{\rm{
\versik{Handle an deinem Knecht nach deiner Huld.}{Lehre mich deine Gesetze.}
}}
\end{flushleft}



\section[VESPERAE]{AD VESPERAS}

\setspaceafterinitial{4.2mm plus 0em minus 0em}
\setspacebeforeinitial{4.2mm plus 0em minus 0em}
\def\greinitialformat#1{{\fontsize{40}{40}\selectfont #1}}
\gresetfirstlineaboveinitial{\small \textcolor{red}{Ps 141}}{}
\setaboveinitialseparation{0.72mm}
\setsecondannotation{\small 8. T.}

\includescore{psalmi/140/domineclamavi.tex}

\vspace{0.3cm}
\crot{Psalm 141}

\begin{quote}


 
\begin{verse}

\textit{ Herr, ich rufe zu dir}. Eile mir zu Hilfe; *\\
\textit{höre auf meine Stimme}, wenn ich zu dir rufe.\\ \vin 
Wie ein Rauchopfer steige mein Gebet\\ \vin  vor dir auf; *\\ \vin 
als Abendopfer gelte vor dir, wenn ich\\ \vin  meine Hände erhebe.\\ 
Herr, stell eine Wache vor meinen Mund, *\\
eine Wehr vor das Tor meiner Lippen! \\ \vin 
Gib, dass mein Herz sich bösen Worten\\ \vin  nicht zuneigt, *\\ \vin 
dass ich nichts tue, was schändlich ist, \\
zusammen mit Menschen, die Unrecht tun. *\\ Von ihren Leckerbissen will ich nicht kosten.\\
\vin Der Gerechte mag mich schlagen aus Güte: *\\
\vin Wenn er mich bessert, ist es Salböl für\\ \vin mein Haupt; \\  
da wird sich mein Haupt nicht sträuben. *\\   Ist er in Not, will ich stets für ihn beten.\\
\vin Haben ihre Richter sich auch die Felsen\\  \vin hinabgestürzt, *\\ 
\vin sie sollen hören, dass mein Wort für sie\\ \vin freundlich ist.\\ 
Wie wenn man Furchen zieht und das\\   Erdreich aufreißt, *\\ 
so sind unsre Glieder hingestreut an \\ den Rand der Unterwelt.\\
\vin Mein Herr und Gott, meine Augen richten\\ \vin sich auf dich; *\\
\vin bei dir berge ich mich. Gieß mein\\ \vin Leben nicht aus! \\
Vor der Schlinge, die sie mir legten,  bewahre \\mich, *\\
vor den Fallen derer, die Unrecht tun! \\
\vin(Die Frevler sollen sich in ihren eigenen\\ \vin Netzen fangen, *\\
\vin während ich heil entkomme.)\\ 



\end{verse}


\vspace{0.3cm}

\setspaceafterinitial{4.2mm plus 0em minus 0em}
\setspacebeforeinitial{4.2mm plus 0em minus 0em}
\def\greinitialformat#1{{\fontsize{40}{40}\selectfont #1}}
\gresetfirstlineaboveinitial{\small \textcolor{red}{Ps 142}}{}
\setaboveinitialseparation{0.72mm}
\setsecondannotation{\small 8. T.}

\includescore{psalmi/141/portiomea.tex}

\vspace{0.3cm}

\psal{142}

\begin{verse}
 Mit lauter Stimme schrei ich zum Herrn, *\\ 
laut flehe ich zum Herrn um Gnade.\\ \vin
Ich schütte vor ihm meine Klagen aus, *\\ \vin
eröffne ihm meine Not.\\ 
Wenn auch mein Geist in mir verzagt, *\\ 
du kennst meinen Pfad.\\ \vin 
Auf dem Weg, den ich gehe, *\\ \vin legten sie mir Schlingen.\\  
Ich blicke nach rechts und schaue aus, *\\ 
doch niemand ist da, der mich beachtet.\\ \vin 
Mir ist jede Zuflucht genommen, *\\ \vin niemand fragt nach meinem Leben.\\ 
Herr, ich schreie zu dir, †\\
ich sage: Meine Zuflucht bist du, *\\ \textit{ mein Anteil im Land der Lebenden}.\\ \vin 
Vernimm doch mein Flehen; *\\ \vin
denn ich bin arm und elend.\\  
Meinen Verfolgern entreiß mich; *\\  sie sind viel stärker als ich.\\ \vin
Führe mich heraus aus dem Kerker, *\\ \vin
damit ich deinen Namen preise.\\  
Die Gerechten scharen sich um mich, *\\  weil du mir Gutes tust.\\  
\end{verse}



\vspace{0.3cm}

\setspaceafterinitial{5.2mm plus 0em minus 0em}
\setspacebeforeinitial{4.2mm plus 0em minus 0em}
\def\greinitialformat#1{{\fontsize{40}{40}\selectfont #1}}
\gresetfirstlineaboveinitial{\small \textcolor{red}{Ps 144}}{}
\setaboveinitialseparation{0.72mm}
\setsecondannotation{\small vj. T.}
\includescore{psalmi/143/benedictusdominusdeusmeus.tex}

\vspace{0.3cm}

\psal{144a}

 
\begin{verse}
 \textit{Gelobt sei der Herr, der mein Fels ist}, *\\
der meine Hände den Kampf gelehrt hat, mein\d e\\ Finger den Krieg.\\ \vin  
Du bist meine Huld und Burg, * \\ \vin 
meine F\d estung, mein Retter, \\
mein Schild, dem ich vertraue. *\\ Er macht m\d ir Völker untertan. \\ \vin 
Herr, was ist der Mensch, dass du dich um \\ \vin ihn kümmerst, *\\ \vin 
des Menschen Kind, dass d\d u es beachtest? \\
Der Mensch gleicht einem Hauch, *\\
seine Tage sind wie ein fl\d üchtiger Schatten.\\ \vin  
Herr, neig deinen Himmel und steig herab, *\\ \vin 
rühre die Berge an, s\d o dass sie rauchen.\\ 
Schleudre Blitze und zerstreue die Feinde, *\\
schieß deine Pfeile ab \d und jag sie dahin!\\ \vin 
Streck deine Hände aus der Höhe herab und\\ \vin  befreie mich; †\\ \vin 
reiß mich heraus aus gewaltigen Wassern, *\\ \vin  aus d\d er Hand der Fremden!\\
Alles, was ihr Mund sagt, ist Lüge, *\\
Meineide schw\d ört ihre Rechte.\\
\end{verse}



\vspace{0.3cm}

\setspaceafterinitial{6.2mm plus 0em minus 0em}
\setspacebeforeinitial{4.2mm plus 0em minus 0em}
\def\greinitialformat#1{{\fontsize{40}{40}\selectfont #1}}
\gresetfirstlineaboveinitial{\small \textcolor{red}{Ap 11;12}}{}
\setaboveinitialseparation{0.72mm}
\setsecondannotation{\small j. T.}

\includescore{cantica/cn/dediteidominus.tex}

\noindent{\rm{Ant. Der Herr hat ihm Herrschaft, Würde und Königtum gegeben. Alle Völker, Nationen und Sprachen müssen ihm dienen.}}

\cant{Ap 11;12}

\begin{verse}
 Wir danken dir, Herr, †\\
 Gott und Herrscher über die ganze Schöpfung,*\\
 der du bist und der du warst;\\
 \vin denn du hast deine große Macht in Anspruch\\ \vin  genommen *\\
 \vin und die Herrschaft angetreten. \\
 Die Völker gerieten in Zorn, †\\
 Da kam dein Zorn *\\
 und die Zeit, die Toten zu richten:\\
 \vin die Zeit, deine Knechte zu belohnen, †\\
 \vin die Propheten und die Heiligen *\\
 \vin  und alle, die deinen Namen fürchten, \\ \vin die Kleinen und die Großen, \\

 
 Jetzt ist er da, der rettende Sieg †\\
und die Macht und die Herrschaft unseres\\ Gottes *\\
und die Vollmacht seines Gesalbten.\\

\vin Denn gestürzt ist, der unsere Brüder\\ \vin  verklagte,*\\
\vin der sie bei Tag und bei Nacht vor unserem \\ \vin Gott verklagte.\\
Sie haben ihn besiegt durch das Blut\\ des Lammes *\\
und durch ihr Wort und Zeugnis,\\ 
\vin sie hielten ihr Leben nicht fest,*\\
\vin  bis hinein in den Tod.\\
Darum jubelt, ihr Himmel *\\
und alle, die darin wohnen.\\


\end{verse}

 
\end{quote}
\medskip
 
\noindent{\rot{Resp.br.} Adjutorium nostrum {\rot{et Hymnus}  Deus Creator \rot{vel} Lucis Creator \rot{p. 32 - 34.}}

\medskip



\begin{flushleft}

\versik{Dirigátur, Dómine, orátio mea.}{Sicut incénsum in conspéctu tuo.}

\medskip
{\rm{
\versik{Herr, mein Gebet werde gelenkt.}{Wie Weihrauch vor dein Angesicht.}
}}
\end{flushleft}

\vspace{0.6cm}

\setspaceafterinitial{5.2mm plus 0em minus 0em}
\setspacebeforeinitial{4.2mm plus 0em minus 0em}
\def\greinitialformat#1{{\fontsize{40}{40}\selectfont #1}}
\gresetfirstlineaboveinitial{\small \textcolor{red}{Magni.}}{}
\setaboveinitialseparation{0.72mm}
\setsecondannotation{\small iij. T.}

\includescore{cantica/evangelica/deposuit.tex}

\vspace{0.2cm}

\rot{Canticum} Magnificat \rot{p. 128.}
\vspace{0.5cm}


Nach dem Magnificat folgen die Fürbitten, das Vater Unser und das Tagesgebet.
Das Chorgebet endet mit einer Antiphon zu Ehren der heiligen Jungfrau.

\newpage