\kkap{DONNERSTAG}

\section[DONNERSTAG]{AD LAUDES}

\setspaceafterinitial{5.2mm plus 0em minus 0em}
\setspacebeforeinitial{4.2mm plus 0em minus 0em}
\def\greinitialformat#1{{\fontsize{40}{40}\selectfont #1}}
\gresetfirstlineaboveinitial{\small \textcolor{red}{Ps 76}}{}
\setaboveinitialseparation{0.72mm}
\setsecondannotation{\small viij. T.}

\includescore{psalmi/75/ps75.tex}
\vspace{0.3cm}


\crot{Psalm 76}

\begin{quote}
 


\begin{verse}
Gott gab sich zu erkennen in Juda, *\\
\textit{sein Name ist groß in Israel.}\\
\vin Sein Zelt erstand in Salem, *\\
\vin seine Wohnung auf dem Zion.\\ 
Dort zerbrach er die blitzenden Pfeile des\\ Bogens, *\\
Schild und Schwert, die Waffen des Krieges.\\
\vin Du bist furchtbar und herrlich, *\\
\vin mehr als die ewigen Berge.\\
Ausgeplündert sind die tapferen Streiter, †\\
sie sinken hin in den Schlaf; *\\ 
allen Helden versagen die Hände.\\ 
\vin Wenn du drohst, Gott Jakobs, *\\
\vin erstarren Rosse und Wagen.\\
Furchtbar bist du. Wer kann bestehen vor dir, *\\
vor der Gewalt deines Zornes? \\
\vin Vom Himmel her machst du das Urteil\\ \vin  bekannt; *\\
\vin Furcht packt die Erde, und sie verstummt,\\
wenn Gott sich erhebt zum Gericht, *\\
um allen Gebeugten auf der Erde zu helfen.\\ 
\vin Denn auch der Mensch voll Trotz muss \\ \vin dich preisen *\\
\vin und der Rest der Völker dich feiern.\\
Legt Gelübde ab und erfüllt sie dem\\ Herrn, eurem Gott! *\\
Ihr alle ringsum, bringt Gaben ihm, den ihr\\ fürchtet.\\ 
\vin Er nimmt den Fürsten den Mut; *\\
\vin Furcht erregend ist er für die Könige der Erde.\\


\end{verse}

\end{quote}



\vspace{0.3cm}
\setspaceafterinitial{4.2mm plus 0em minus 0em}
\setspacebeforeinitial{4.2mm plus 0em minus 0em}
\def\greinitialformat#1{{\fontsize{40}{40}\selectfont #1}}
\gresetfirstlineaboveinitial{\small \textcolor{red}{Ps 88}}{}
\setaboveinitialseparation{0.72mm}
\setsecondannotation{\small v. T.}
\includescore{psalmi/87/intretoratio.tex}

\vspace{0.3cm}
\crot{Psalm 88}

\begin{quote}
 


\begin{verse}
 Herr, du Gott meines Heils, *\\
zu dir schreie ich am T\d ag und bei Nacht.\\ \vin  
\textit{Lass mein Gebet zu dir dringen,} *\\ \vin 
wende dein \d Ohr meinem Flehen zu!\\ 
Denn meine Seele ist gesättigt mit Leid, *\\
mein Leben ist dem T\d otenreich nahe.\\ \vin  
Schon zähle ich zu denen, die hinabsinken \\ \vin ins Grab, *\\ \vin 
bin wie ein Mann, dem alle Kr\d aft \\ \vin genommen ist.\\ 
Ich bin zu den Toten hinweggerafft *\\
wie Erschlagene, die im Gr\d abe ruhen;\\
\vin an sie denkst du nicht mehr, *\\ 
\vin denn sie sind deiner H\d and entzogen.\\  
Du hast mich ins tiefste Grab gebracht, *\\ 
tief hinab in f\d instere Nacht.\\
 \vin Schwer lastet dein Grimm auf mir, *\\
 \vin all deine Wogen stürzen über m\d ir zusammen.\\   
Die Freunde hast du mir entfremdet, †\\  
mich ihrem Abscheu ausgesetzt; *\\  ich bin gefangen und k\d ann nicht heraus.\\ 
\vin Mein Auge wird trübe vor Elend.  †\\
\vin Jeden Tag, Herr, ruf ich zu dir; *\\ \vin ich strecke nach d\d ir meine Hände aus.\\
 Wirst du an den Toten Wunder tun, *\\
 werden Schatten aufstehn, um d\d ich zu preisen? \\ 
\vin Erzählt man im Grab von deiner Huld, *\\
\vin von deiner Tr\d eue im Totenreich?\\
 Werden deine Wunder in der Finsternis \\  bekannt, *\\
 deine Gerechtigkeit im L\d and des Vergessens?\\
\vin Herr, darum schreie ich zu dir, *\\
\vin früh am Morgen tritt mein Geb\d et \\ \vin vor dich hin.\\ 
 Warum, o Herr, verwirfst du mich, *\\
 warum verbirgst du d\d ein Gesicht vor mir? \\
\vin Gebeugt bin ich und todkrank von früher\\ \vin  Jugend an, *\\
\vin deine Schrecken lasten auf mir und \d ich \\ \vin  bin zerquält.\\ 
 Über mich fuhr die Glut deines Zorns dahin, *\\
 deine Schr\d ecken vernichten mich.\\
\vin Sie umfluten mich allzeit wie Wasser *\\
\vin und dringen auf mich ein von \d allen Seiten.\\ 
 Du hast mir die Freunde und Gefährten\\ entfremdet; *\\
mein Vertrauter ist n\d ur noch die Finsternis.\\ 

\end{verse}

\end{quote}

\vspace{0.3cm}

\setspaceafterinitial{5.2mm plus 0em minus 0em}
\setspacebeforeinitial{4.2mm plus 0em minus 0em}
\def\greinitialformat#1{{\fontsize{40}{40}\selectfont #1}}
\gresetfirstlineaboveinitial{\small \textcolor{red}{Ier 31}}{}
\setaboveinitialseparation{0.72mm}
\setsecondannotation{\small iv. T.}
\includescore{cantica/ca/ier31.tex}

\cant{Ier 31,10-14}

\begin{quote}
 
\begin{verse}
Hört, ihr Völker, das Wort des Herrn,*\\
verkündet es auf den fernst\d en Inseln und sagt:\\ 
\vin Er, der Israel zerstreut hat, wird es \\ \vin auch sammeln*\\
\vin und hüten wie ein H\d irt seine Herde.\\
Denn der Herr wird Jakob erlösen*\\
und ihn befreien aus d\d er Hand des Stärkeren.\\
\vin Sie kommen und jubeln auf Zions Höhe, †\\
\vin sie strahlen vor Freude über die \\ \vin Gaben des Herrn,*\\ 
\vin über Korn, Wein und Öl, über L\d ämmer\\ \vin und Rinder.\\ 
Sie werden wie ein bewässerter Garten sein*\\ 
und n\d ie mehr verschmachten.\\
\vin Dann freut sich das Mädchen beim\\ \vin  Reigentanz,*\\
\vin Jung \d und Alt sind fröhlich.\\ 
Ich verwandle ihre Trauer in Jubel,*\\
tröste und erfreue sie n\d ach ihrem Kummer.\\
\vin Ich labe die Priester mit Opferfett *\\
\vin und \textit{mein Volk wird satt \d an meinen Gaben.}\\

\end{verse}

\end{quote}

\rot{vel}

\setspaceafterinitial{5.2mm plus 0em minus 0em}
\setspacebeforeinitial{4.2mm plus 0em minus 0em}
\def\greinitialformat#1{{\fontsize{40}{40}\selectfont #1}}
\gresetfirstlineaboveinitial{\small \textcolor{red}{Is 12}}{}
\setaboveinitialseparation{0.72mm}
\setsecondannotation{\small viij. T.}
\includescore{cantica/ca/is12.tex}

\cant{Is 12,1-6}

\begin{quote}
 



\begin{verse}
 
Ich danke dir, Herr.\\ Du hast mir gezürnt, † \\ doch \textit{ dein Zorn hat sich gewendet *\\ und du hast mich getröstet}.\\ \vin
Ja, Gott ist meine Rettung; *\\ \vin
ihm will ich vertrauen und niemals verzagen.\\ 
Denn meine Stärke und mein \\ Lied ist der Herr. *\\ Er ist für mich zum Retter geworden.\\ \vin 
Ihr werdet Wasser schöpfen voll Freude *\\ \vin
aus den Quellen des Heil\textit{e}s.\\
An jenem Tag werdet ihr sagen: *\\
Dankt dem Herrn! Ruft seinen Namen an!\\ \vin 
Macht seine Taten unter den \\ \vin Völkern bekannt, *\\ \vin verkündet: Sein Name ist groß und erhaben!\\
Preist den Herrn; †\\
denn herrliche Taten hat er vollbracht; *\\ auf der ganzen Erde soll man es wissen.\\ \vin
Jauchzt und jubelt, ihr Bewohner von Zion; *\\ \vin
denn groß ist in eurer Mitte der Heilige Israels.\\
\end{verse}

\end{quote}

\vspace{0.6cm}

\vspace{0.6cm}

\setspaceafterinitial{4.2mm plus 0em minus 0em}
\setspacebeforeinitial{4.2mm plus 0em minus 0em}
\def\greinitialformat#1{{\fontsize{40}{40}\selectfont #1}}
\gresetfirstlineaboveinitial{\small \textcolor{red}{Ps 90}}{}
\setaboveinitialseparation{0.72mm}
\setsecondannotation{\small vj. T.}

% and finally we include the score. The file must be in the same directory as this one.
\includescore{psalmi/89/ps89.tex}

\vspace{0.3cm}

\crot{Psalm 90}
\begin{quote}
 


\begin{verse}
 \textit{Herr, du warst unsere Zuflucht} *\\
von G\d eschlecht zu Geschlecht.\\
\vin Ehe die Berge geboren wurden, †\\
\vin die Erde entstand und das Weltall, *\\ \vin bist du, o Gott, von Ew\d igkeit zu Ewigkeit.\\
Du lässt die Menschen zurückkehren \\  zum Staub *\\
und sprichst: «Kommt w\d ieder, ihr Menschen!» \\
\vin Denn tausend Jahre sind für dich 
wie der Tag, \\ \vin der gestern vergangen ist, *\\ \vin wie ein\d e Wache in der Nacht.\\ 
Von Jahr zu Jahr säst du die Menschen aus; *\\
sie gleichen d\d em sprossenden Gras.\\
\vin Am Morgen grünt es und blüht, *\\
\vin am Abend wird es g\d eschnitten und welkt.\\
Denn wir vergehen durch deinen Zorn, *\\
werden vern\d ichtet durch deinen Grimm.\\
\vin Du hast unsre Sünden vor dich hingestellt, *\\
\vin unsere geheime Schuld in das L\d icht deines \\ \vin  Angesichts.\\ 
Denn all unsre Tage gehn hin unter \\ deinem Zorn, *\\
wir beenden unsere Jahre w\d ie einen Seufzer.\\ 
\vin Unser Leben währt siebzig Jahre, *\\
\vin und wenn es hoch k\d ommt, sind es achtzig.\\ 
Das Beste daran ist nur Mühsal\\ und Beschwer, *\\ rasch geht es vorbei, w\d ir fliegen dahin.\\ 
\vin Wer kennt die Gewalt deines Zornes *\\
\vin und fürcht\d et sich vor deinem Grimm? \\
Unsre Tage zu zählen, lehre uns! *\\
Dann gewinn\d en wir ein weises Herz.\\ 
\vin Herr, wende dich uns doch endlich zu! *\\
\vin Hab Mitleid m\d it deinen Knechten!\\
Sättige uns am Morgen mit deiner Huld! *\\
Dann wollen wir jubeln und uns freuen\\ \d all unsre Tage.\\ 
\vin Erfreue uns so viele Tage, wie du uns \\ \vin gebeugt hast, *\\
\vin so viele Jahre, wie wir \d Unglück erlitten.\\
Zeig deinen Knechten deine Taten *\\
und ihren Kindern deine \d erhabene Macht! \\
\vin Es komme über uns die Güte des Herrn, \\ \vin unsres Gottes. †\\
\vin Lass das Werk unsrer Hände gedeihen, *\\ \vin ja, lass gedeihen das W\d erk unsrer Hände!\\
\end{verse}

\end{quote}

\noindent\rot{Resp.br.} Sana animam \rot{und Hymnus} Splendor paternæ \rot{oder} Ecce iam noctis \rot{wie am Montag pp. 18ff.}\\
\begin{flushleft}

\versik{Repléti sumus mane misericórdia tua.}{Exultávimus, et delectáti sumus.}

\medskip

{\rm{
\versik{Am Morgen sind wir erfüllt von deiner Huld.}{Wir sind (voll) Jubel und Frohsinn.}
}}
\end{flushleft}

\setspaceafterinitial{5.2mm plus 0em minus 0em}
\setspacebeforeinitial{4.2mm plus 0em minus 0em}
\def\greinitialformat#1{{\fontsize{40}{40}\selectfont #1}}
\gresetfirstlineaboveinitial{\small \textcolor{red}{Benedic.}}{}
\setaboveinitialseparation{0.72mm}
\setsecondannotation{\small viij. T.}

\includescore{cantica/evangelica/addandam.tex}

\rot{Canticum} Benedictus \rot{p. 64.}

