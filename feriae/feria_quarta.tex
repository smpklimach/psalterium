\thispagestyle{plain}

\kapklein{\rot{F}eria \rot{Q}uarta}
\kkap{FERIA QUARTA}





\section[VIGILIAE]{AD VIGILIAS}


\setspaceafterinitial{5.2mm plus 0em minus 0em}
\setspacebeforeinitial{4.2mm plus 0em minus 0em}
\def\greinitialformat#1{{\fontsize{40}{40}\selectfont #1}}
\gresetfirstlineaboveinitial{\small \textcolor{red}{Invitat.}}{Invitat.}
\setaboveinitialseparation{0.72mm}
%\setsecondannotation{\small Ps. 125}

%\includescore{invitatoria/veniteiv.tex}

\medskip

\begin{sloppypar}
{\noindent{\rot{Invit.} Alle Länder der Erde * sind, o Herr, in deiner Hand.}}
\end{sloppypar}

\bigskip

\noindent{\rot{Hymnus} Æterne rerum Conditor \rot{vel} Nocte surgentes \rot{ut in Dominica pp. 1-3.}}


\section{VIGILIA I}

\begin{sloppypar}
{\noindent{1. Ant.} Ich preise dich, Herr, * denn du hast mich aufgenommen. \rot{Ps. 30}\\
2. Ant. Befreie mich Herr * in deiner Gerechtigkeit. \rot{Ps. 31}\\
3. Ant. Freut euch im Herrn * und jauchzt ihr Gerechten. \rot{Ps. 32}\\}
\end{sloppypar}

\begin{flushleft}

\versik{Ich will hören, was Gott redet.}{Frieden verkündet der Herr seinem Volk.}

\end{flushleft}


\section{VIGILIA II}

\begin{sloppypar}
{\noindent{1. Ant.} Gewendet hat der Herr * die Gefangenschaft seines Volkes. \rot{Ps. 53} \\
2. Ant. Der Herr ist mein Helfer * und nimmt meine Seele auf. \rot{Ps. 54} \\
3. Ant. Wende dich mir zu * und erhöre mich, o Herr. \rot{Ps. 55} \\}
\end{sloppypar}

\begin{flushleft}

\versik{Die Erklärung deiner Worte bringt Erleuchtung.}{Und gibt den Einfältigen Verstand.}

\end{flushleft}

\section{VIGILIA III}

\begin{sloppypar}
{\noindent{1. Ant.} Mit lauter Stimme rief ich zum Herrn * und er hat nicht vergessen, mir gnädig zu sein. \rot{Ps. 77}\\
2. Neigt euer Ohr * zu den Worten meines Mundes. \rot{Ps. 78, 1-31}\\}
\end{sloppypar}

\begin{flushleft}

\versik{Wir preisen dich, Herr.}{Und rufen deinen Namen an.}

\end{flushleft}


\section{VIGILIA IV}

\begin{sloppypar}
{\noindent{1. Ant.} Mein Schreien * dringe zu dir, o Gott.\\ \rot{Ps. 102}\\
2. Ant. Meine Seele * preise den Herrn. \rot{Ps. 103}\\
3. Ant. Du bist mit Hoheit und Pracht bekleidet * du hüllst dich in Licht wie in ein Kleid. \rot{Ps. 104,1-18}\\}
\end{sloppypar}

\begin{flushleft}

\versik{Herr, bei dir ist die Quelle des Lebens.}{In deinem Licht schauen wir das Licht.}

\end{flushleft}

\section[LAUDES]{AD LAUDES}

\setspaceafterinitial{5.2mm plus 0em minus 0em}
\setspacebeforeinitial{4.2mm plus 0em minus 0em}
\def\greinitialformat#1{{\fontsize{40}{40}\selectfont #1}}
\gresetfirstlineaboveinitial{\small \textcolor{red}{ Ps 64}}{}
\setaboveinitialseparation{0.72mm}
\setsecondannotation{\small ij. T.}

\includescore{psalmi/63/ps63.tex}

\vspace{0.6cm}

\psal{64}

\begin{quote}
\begin{verse}
Höre, o Gott, mein lautes Klagen, *\\
\textit{schütze mein Leben vor dem Schrecken\\
des Feindes!}\\ 
\vin Verbirg mich vor der Schar der Bösen, *\\ 
\vin vor dem Toben derer, die Unrecht tun.\\
Sie schärfen ihre Zunge wie ein Schwert, *\\
schießen giftige Worte wie Pfeile, \\ 
\vin um den Schuldlosen\\ 
\vin von ihrem Versteck aus zu treffen. *\\ 
\vin Sie schießen auf ihn,\\ 
\vin plötzlich und ohne Scheu.\\ 
Sie sind fest entschlossen zu bösem Tun. †\\
Sie planen, Fallen zu stellen, *\\ 
und sagen: «Wer sieht uns schon?»\\ 
\vin Sie haben Bosheit im Sinn, *\\ 
\vin doch halten sie ihre Pläne geheim.\\ 
Ihr Inneres ist heillos verdorben, *\\ 
ihr Herz ist ein Abgrund.\\ 
\vin Da trifft sie Gott mit seinem Pfeil; *\\ 
\vin sie werden jählings verwundet.\\
Ihre eigene Zunge bringt sie zu Fall.*\\
Alle, die es sehen, schütteln den Kopf.\\ 
\vin Dann fürchten sich alle Menschen; †\\ 
\vin sie verkünden Gottes Taten *\\ 
\vin und bedenken sein Wirken.\\
Der Gerechte freut sich am Herrn\\
und sucht bei ihm Zuflucht. *\\
Und es rühmen sich alle Menschen\\
mit redlichem Herzen.\\
\end{verse}
\end{quote}


\vspace{0.6cm}
\setspaceafterinitial{5.2mm plus 0em minus 0em}
\setspacebeforeinitial{4.2mm plus 0em minus 0em}
\def\greinitialformat#1{{\fontsize{40}{40}\selectfont #1}}
\gresetfirstlineaboveinitial{\small \textcolor{red}{Ps 65}}{}
\setaboveinitialseparation{0.72mm}
\setsecondannotation{\small 8. T.}
\includescore{psalmi/64/ps64.tex}

\vspace{0.6cm}
\psal{65}

\begin{quote}
\begin{verse}
\textit{Dir gebührt Lobgesang,\\
Gott, auf dem Zion,} *\\
dir erfüllt man Gelübde.\\ 
\vin Du erhörst die Gebete. *\\ 
\vin Alle Menschen kommen zu dir\\ 
\vin unter der Last ihrer Sünden.\\ 
Unsere Schuld ist zu groß für uns, *\\
du wirst sie vergeben.\\ 
\vin Wohl denen, die du erwählst\\ 
\vin und in deine Nähe holst, *\\ 
\vin die in den Vorhöfen\\ 
\vin deines Heiligtums wohnen.\\ 
Wir wollen uns am Gut deines Hauses sättigen,*\\ 
am Gut deines Tempels.\\ 
\vin Du vollbringst erstaunliche Taten, *\\ 
\vin erhörst uns in Treue, du Gott unsres Heiles,\\
du Zuversicht aller Enden der Erde *\\ 
und der fernsten Gestade.\\ 
\vin Du gründest die Berge in deiner Kraft, *\\ 
\vin du gürtest dich mit Stärke.\\
Du stillst das Brausen der Meere, *\\
das Brausen ihrer Wogen, das Tosen der Völker.\\ 
\vin Alle, die an den Enden der Erde wohnen, †\\ 
\vin erschauern vor deinen Zeichen; *\\ 
\vin Ost und West erfüllst du mit Jubel.\\
Du sorgst für das Land und tränkst es; *\\
du überschüttest es mit Reichtum.\\ 
\vin Der Bach Gottes ist reichlich gefüllt, *\\ 
\vin du schaffst ihnen Korn; so ordnest du alles.\\
Du tränkst die Furchen, ebnest die Schollen, *\\
machst sie weich durch Regen,\\
segnest ihre Gewächse.\\ 
\vin Du krönst das Jahr mit deiner Güte, *\\ 
\vin deinen Spuren folgt Überfluss.\\
In der Steppe prangen die Auen, *\\
die Höhen umgürten sich mit Jubel.\\ 
\vin Die Weiden schmücken sich mit Herden, †\\ 
\vin die Täler hüllen sich in Korn. *\\ 
\vin Sie jauchzen und singen.\\

\end{verse}
\end{quote}


\vspace{0.6cm}

\setspaceafterinitial{5.2mm plus 0em minus 0em}
\setspacebeforeinitial{4.2mm plus 0em minus 0em}
\def\greinitialformat#1{{\fontsize{40}{40}\selectfont #1}}
\gresetfirstlineaboveinitial{\small \textcolor{red}{Jdt 16}}{}
\setaboveinitialseparation{0.72mm}
\setsecondannotation{\small vij. T.}
\includescore{cantica/ca/iudt16.tex}

\cant{Jdt 16,1f.13-16}

\begin{quote}
\begin{verse}

Stimmt ein Lied an für meinen G\d ott\\
unter Paukenschall, *\\ 
\vin singt für den Herrn \d unter Zimbelklang! \\ 
\vin Preist ihn und s\d ing\textit{e}t sein Lob, *\\  
rühmt seinen Namen und r\d uf\textit{e}t ihn an! \\
Denn der H\d err ist ein Gott, *\\ 
der den Kr\d iegen ein Ende setzt; \\ 
\vin er führte mich heim in sein Lager\\ 
\vin inm\d itten des Volkes *\\ 
\vin  und rettete mich aus der Gew\d alt der Feinde.\\  
Ich singe meinem G\d ott ein neues Lied; *\\ 
\textit{Herr, du bist gr\d oß und} voll Herrlichkeit.\\ 
\vin \textit{Wunderbar bist du in d\d einer Stärke,} *\\ 
\vin keiner kann d\d ich übertreffen.\\  
Dienen muss dir deine g\d anze Schöpfung. *\\ 
Denn du hast gesprochen und \d alles entstand.\\ 
\vin Du sandtest deinen Geist, um den \\ \vin B\d au zu vollenden.  *\\ 
\vin  Kein Mensch kann deinem W\d ort widerstehen.\\ 
Meere und Berge erbeben in ihrem Grund, †\\
vor dir zerschmelzen die F\d elsen wie Wachs. *\\  
Doch wer dich fürchtet, der erfährt d\d eine Gnade.\\ 
\vin Zu gering ist jedes Opfer, um \\ \vin dich zu erfreuen, †\\ 
\vin alle Fettstücke sind nichts \\ \vin beim \d Opfer für dich. *\\ 
\vin  Wer den Herrn fürchtet, der \\ \vin ist gr\d oß für immer.\\ 

\end{verse}
\end{quote}

\rot{vel}

\setspaceafterinitial{5.2mm plus 0em minus 0em}
\setspacebeforeinitial{4.2mm plus 0em minus 0em}
\def\greinitialformat#1{{\fontsize{40}{40}\selectfont #1}}
\gresetfirstlineaboveinitial{\small \textcolor{red}{1 Sam 2}}{}
\setaboveinitialseparation{0.72mm}
\setsecondannotation{\small j. T.}
\includescore{cantica/ca/1sam.tex}

\cant{1Sam 2,1-10}

\begin{quote}
\begin{verse}

 Mein Herz ist voll Freude über den Herrn, *\\
große Kraft gibt mir der Herr.\\
\vin Weit öffnet sich mein Mund gegen \\ \vin meine Feinde; *\\
\vin Denn ich freue mich über deine Hilfe.\\
Niemand ist heilig, nur der Herr, †\\
Denn außer dir gibt es keinen Gott; *\\
Keiner ist ein Fels wie unser Gott.\\
\vin Redet nicht immer so vermessen, *\\
\vin Kein freches Wort entfahre eurem Mund\textit{e}.\\
Denn der Herr ist ein wissender Gott*\\
und bei ihm werden die Taten geprüft.\\

\vin Der Bogen der Helden wird zerbrochen,*\\
\vin die Wankenden aber gürten sich mit Kraft.\\
Die Satten verdingen sich um Brot,*\\
doch die Hungrigen können feiern für immer.\\
\vin Die Unfruchtbare bekommt sieben Kinder, *\\
\vin doch die Kinderreiche welkt dahin.\\
Der Herr macht tot und macht lebendig.*\\
er führt ins Totenreich hinab und \\ führt auch herauf.\\
\vin Der Herr macht arm und macht reich, *\\
\vin er erniedrigt und er erhöht.\\
Den Schwachen hebt er empor aus dem Staub *\\
und erhöht den Armen, der im Schmutz liegt;\\ 
\vin Er gibt ihm einen Sitz bei den Edlen, *\\
\vin einen Ehrenplatz weist er ihm zu.\\
Ja, dem Herrn gehören die Pfeiler der Erde; *\\
auf sie hat er den Erdkreis gegründet.\\
\vin Er behütet die Schritte seiner Frommen, *\\
\vin doch die Frevler verstummen in der  \\ \vin  Finsternis; \\
denn der Mensch ist nicht stark aus \\eigener Kraft.*\\
Wer gegen den Herrn streitet, wird zerbrechen.\\
\vin der Höchste lässt es donnern am Himmel.*\\
\vin \textit{Der Herr hält Gericht bis an die \\ \vin  Grenzen der Erde.}\\
Seinem König gebe er Kraft*\\
und erhöhe die Macht seines Gesalbten.\\
\end{verse}

\end{quote}







\vspace{0.6cm}

\setspaceafterinitial{4.2mm plus 0em minus 0em}
\setspacebeforeinitial{4.2mm plus 0em minus 0em}
\def\greinitialformat#1{{\fontsize{40}{40}\selectfont #1}}
\gresetfirstlineaboveinitial{\small \textcolor{red}{Ps 67}}{}
\setaboveinitialseparation{0.72mm}
\setsecondannotation{\small vij. T.}

\includescore{psalmi/66/deusmisereaturnostri.tex}

\vspace{0.3cm}
\psal{67}
\begin{quote}
 
\begin{verse}
 Gott sei uns gn\d ädig und segne uns.  *\\ 
Er lasse über uns sein \d Angesicht leuchten,\\ \vin
\textit{damit auf Erden sein W\d eg erkannt wird *\\ \vin 
und unter allen V\d ölkern sein Heil.}\\ 
Die Völker sollen dir d\d anken, o Gott, *\\ 
danken sollen dir die V\d ölker alle.\\ \vin 
Die Nationen sollen sich fr\d euen und jubeln. *\\ \vin 
Denn du richtest den \d Erdkreis gerecht.\\  
Du richtest die V\d ölker nach Recht *\\  
und regierst die Nati\d onen auf Erden.\\ \vin  
Die Völker sollen dir d\d anken, o Gott, *\\ \vin 
danken sollen dir die V\d ölker alle.\\ 
Das Land gab s\d einen Ertrag.  *\\ 
Es segne \d uns Gott, unser Gott.\\ \vin  
Es s\d egne uns Gott.  *\\ \vin 
Alle Welt f\d ürchte und ehre ihn.\\  


\end{verse}
\end{quote}

\noindent\rot{Resp.br.} Sana animam \rot{ut in Feria Secunda p. 43.}\\
\noindent\rot{Hymnus} Splendor paternæ \rot{vel} Ecce iam noctis \rot{ut in Dominica p. 15-17.}
\begin{flushleft}

\versik{Repléti sumus mane misericórdia tua.}{Exultávimus, et delectáti sumus.}

\medskip

{\rm{
\versik{Am Morgen sind wir erfüllt von deiner Huld.}{Wir sind (voll) Jubel und Frohsinn.}
}}
\end{flushleft}

\setspaceafterinitial{5.2mm plus 0em minus 0em}
\setspacebeforeinitial{4.2mm plus 0em minus 0em}
\def\greinitialformat#1{{\fontsize{40}{40}\selectfont #1}}
\gresetfirstlineaboveinitial{\small \textcolor{red}{Bened}}{}
\setaboveinitialseparation{0.72mm}
\setsecondannotation{\small vij. T.}

\includescore{cantica/evangelica/insanctitate.tex}

\vspace{0.2cm}

\rot{Canticum} Benedictus \rot{p. 196.}




\section[HORA TERTIA]{AD TERTIAM}

\rot{Hymnus} Nunc sancte nobis Spiritus \rot{p. 190.}

\vspace{0.3cm}

\setspaceafterinitial{7.2mm plus 0em minus 0em}
\setspacebeforeinitial{4.2mm plus 0em minus 0em}
\def\greinitialformat#1{{\fontsize{40}{40}\selectfont #1}}

\gresetfirstlineaboveinitial{\small \textcolor{red}{Ps 119d}}{}
\setaboveinitialseparation{0.72mm}
\setsecondannotation{\small viij. T.}


\includescore{psalmi/118/118xiiinaeternum.tex}


\includescore{psalmi/tertia/horatertia_quarta.tex}


\vspace{0.3cm}

\begin{flushleft}

\versik{Dóminus non privábit bonis eos qui ámbulant in innocéntia.}{Dómine virtútum, beátus homo qui sperat in te.}

\medskip

{\rm{
\versik{Der Herr wird seine Güter nicht denen vorenthalten, die in Unschuld wandeln.}{Herr der Scharen, selig der Mensch, der auf dich hofft.}
}}
\end{flushleft}




\section[HORA SEXTA]{AD SEXTAM}

\rot{Hymnus} Rector potens, verax Deus \rot{p. 192.}

\vspace{0.3cm}

\setspaceafterinitial{7.2mm plus 0em minus 0em}
\setspacebeforeinitial{4.2mm plus 0em minus 0em}
\def\greinitialformat#1{{\fontsize{40}{40}\selectfont #1}}
\gresetfirstlineaboveinitial{\small \textcolor{red}{ Ps 11sq}}{}
\setaboveinitialseparation{0.72mm}
\setsecondannotation{\small 8. T.}

\includescore{psalmi/12/illumina121314.tex}
\vspace{0.3cm}

\includescore{psalmi/sexta/sextpsalmiquarta.tex}

\begin{flushleft}

\versik{Dómine, veritátem in corde dilexísti.}{Et in occúlto sapiéntiam manifestásti mihi.}

\medskip

{\rm{
\versik{Herr, du liebst, die im Herzen voll Wahrheit sind.}{Und im Verborgenen lehrtest du mich Weisheit.}
}}
\end{flushleft}







\section[HORA NONA]{AD NONAM}

\rot{Hymnus} Rerum Deus tenax vigor \rot{p. 194.}

\vspace{0.3cm}

 \setspaceafterinitial{5.2mm plus 0em minus 0em}
\setspacebeforeinitial{4.2mm plus 0em minus 0em}
\def\greinitialformat#1{{\fontsize{40}{40}\selectfont #1}}
\gresetfirstlineaboveinitial{\small \textcolor{red}{ Ps 129sq }}{}
\setaboveinitialseparation{0.72mm}
\setsecondannotation{\small 8. T.}

\includescore{psalmi/128/benediximusvobis.tex}



\vspace{0.3cm}

\includescore{psalmi/nona/nonpsalmiquarta.tex}


\begin{flushleft}

\versik{Fac cum servo tuo secúndum misericórdiam tuam, Dómine.}{Iustificatiónes tuas doce me.}

\medskip

{\rm{
\versik{Handle an deinem Knecht nach deiner Huld.}{Lehre mich deine Gesetze.}
}}
\end{flushleft}


\section[VESPERAE]{AD VESPERAS}



\setspaceafterinitial{4.2mm plus 0em minus 0em}
\setspacebeforeinitial{4.2mm plus 0em minus 0em}
\def\greinitialformat#1{{\fontsize{40}{40}\selectfont #1}}
\gresetfirstlineaboveinitial{\small \textcolor{red}{Ps 139a}}{}
\setaboveinitialseparation{0.72mm}
\setsecondannotation{\small 3. T.}

\includescore{psalmi/138/domineprobasti.tex}

\vspace{0.3cm}
\psal{139a}
\begin{quote}
 


\begin{verse}
 \textit{Herr, du hast mich erforscht und du \\ kennst mich.} †\\
Ob ich sitze oder st\d ehe, du weißt von mir. *\\
Von fern erkennst du meine Gedanken.\\ \vin 
Ob ich gehe oder r\d uhe, es ist dir bekannt; *\\ \vin
du bist vertraut mit all meinen Wegen.\\
Noch liegt mir das Wo\d rt nicht auf der Zunge - *\\
du, Herr, kennst es bereits.\\ \vin
Du umschließt mich von \d allen Seiten *\\ \vin
und legst deine Hand auf mich.\\
Zu wunderbar ist für m\d ich dieses Wissen, *\\
zu hoch, ich kann es nicht begreifen.\\ \vin 
Wohin könnte ich fl\d iehen vor deinem Geist, *\\ \vin
wohin mich vor deinem Angesicht flüchten? \\
Steige ich hinauf in den H\d immel, so \\bist du dort; *\\
bette ich mich in der Unterwelt, bist du zugegen.\\ 
\vin Nehme ich die Fl\d ügel des Morgenrots *\\
\vin und lasse mich nieder am äußersten Meer, \\ 
auch dort wird deine H\d and mich ergreifen *\\ 
und deine Rechte mich fassen.\\
\vin Würde ich sagen: «Finsternis soll mich\\ \vin bedecken,†\\
\vin statt Licht soll N\d acht mich umgeben», *\\
\vin auch die Finsternis wäre für dich nicht finster, \\ 
die Nacht würde l\d euchten wie der Tag, *\\ 
die Finsternis wäre wie Licht.\\
\end{verse}
\end{quote}

\vspace{0.3cm}
\setspaceafterinitial{4.2mm plus 0em minus 0em}
\setspacebeforeinitial{4.2mm plus 0em minus 0em}
\def\greinitialformat#1{{\fontsize{40}{40}\selectfont #1}}
\gresetfirstlineaboveinitial{\small \textcolor{red}{Ps 139b}}{}
\setaboveinitialseparation{0.72mm}
\setsecondannotation{\small vj. T.}

\includescore{psalmi/138/mirabilia138b.tex}

\vspace{0.3cm}

\psal{139b}

\begin{quote}
 



\begin{verse}
 Denn du hast mein Inneres geschaffen, *\\
mich gewoben im Sch\d oß meiner Mutter.\\ \vin 
Ich danke dir, dass du mich so wunderbar\\ \vin  gestaltet hast. *\\ \vin
\textit{Ich weiß: Staunenswert s\d ind deine Werke.}\\
Als ich geformt wurde im Dunkeln, †\\
kunstvoll gewirkt in den Tiefen der Erde, *\\ waren meine Glieder d\d ir nicht verborgen.\\ \vin 
Deine Augen sahen, wie ich entstand, *\\ \vin
in deinem Buch war schon \d alles verzeichnet; \\
meine Tage waren schon gebildet, *\\ als noch keiner v\d on ihnen da war.\\ \vin
Wie schwierig sind für mich, o Gott,\\ \vin  deine Gedanken, *\\ \vin
wie gew\d altig ist ihre Zahl! \\
Wollte ich sie zählen, es wären mehr als \\ der Sand. *\\
Käme ich bis zum Ende, wäre ich n\d och immer \\bei dir.\\ \vin
(Wolltest du, Gott, doch den Frevler töten! *\\ \vin
Ihr blutgierigen Menschen, lasst ab von mir! \\
Sie reden über dich voll Tücke *\\
und missbrauchen deinen Namen.\\ \vin
Soll ich die nicht hassen, Herr, die dich\\ \vin hassen, *\\ \vin
die nicht verabscheuen, die sich gegen\\ \vin dich erheben?\\
Ich hasse sie mit glühendem Hass; *\\
auch mir sind sie zu Feinden geworden.)\\ \vin
Erforsche mich, Gott, und erkenne\\ \vin mein Herz, *\\ \vin 
prüfe mich und erk\d enne mein Denken! \\
Sieh her, ob ich auf dem Weg bin, der \\ dich kränkt, *\\
und leite mich auf d\d em altbewährten Weg! 

\end{verse}

\end{quote}

\vspace{0.3cm}

\setspaceafterinitial{4.2mm plus 0em minus 0em}
\setspacebeforeinitial{4.2mm plus 0em minus 0em}
\def\greinitialformat#1{{\fontsize{40}{40}\selectfont #1}}
\gresetfirstlineaboveinitial{\small \textcolor{red}{Ps 140}}{}
\setaboveinitialseparation{0.72mm}
\setsecondannotation{\small 4. T.}

\includescore{psalmi/139/aviroiniquo.tex}

\vspace{0.3cm}
\psal{140}

\begin{quote}



\begin{verse}
 \textit{Rette mich, Herr, vor bösen Menschen,} *\\
vor gewalttätig\d en Leuten schütze mich! \\ \vin
Denn sie sinnen in ihrem Herzen auf Böses, *\\ \vin
jeden T\d ag schüren sie Streit.\\
Wie die Schlangen haben sie scharfe Zungen *\\
und hinter den Lippen G\d ift wie die Nattern\\ \vin 
Behüte mich, Herr, vor den Händen\\ \vin der Frevler, †\\ \vin
vor gewalttätigen Leuten schütze mich, *\\ \vin die darauf sinnen, mich zu B\d oden zu stoßen.\\
Hochmütige legen mir heimlich Schlingen, †\\
Böse spannen ein Netz aus, *\\ stellen mir F\d allen am Wegrand\\ \vin
Ich sage zum Herrn: Du bist mein Gott *\\ \vin 
Vernimm, o Herr, m\d ein lautes Flehen! \\
Herr, mein Gebieter, meine starke Hilfe, *\\ 
du beschirmst mein Haupt \d am Tag des Kampfes.\\ \vin
Herr, erfülle nicht die Wünsche \\ \vin des Frevlers, *\\ \vin
lass seine Plän\d e nicht gelingen! \\
Die mich umzingeln, sollen das Haupt nicht\\  erheben; *\\
die Bosheit ihrer Lipp\d en treffe sie selbst\\ \vin 
Er lasse glühende Kohlen auf sie regnen, †\\ \vin
er stürze sie hinab in den Abgrund, *\\ \vin sodass sie n\d ie wieder aufstehn.\\ 
Der Verleumder soll nicht bestehen im Land, *\\
den Gewalttätigen treffe d\d as Unglück Schlag \\auf Schlag\\ \vin
Ich weiß, der Herr führt die Sache\\ \vin  des Armen, *\\ \vin
er verhilft den G\d ebeugten zum Recht\\
Deinen Namen preisen nur die Gerechten; *\\
vor deinem Angesicht dürfen nur die\\ R\d edlichen bleiben\\ 
\end{verse}

 
\end{quote}
\vspace{0.3cm}

\setspaceafterinitial{10.2mm plus 0em minus 0em}
\setspacebeforeinitial{4.2mm plus 0em minus 0em}
\def\greinitialformat#1{{\fontsize{40}{40}\selectfont #1}}
\gresetfirstlineaboveinitial{\small \textcolor{red}{Col 1, 12-20}}{}
\setaboveinitialseparation{0.72mm}
\setsecondannotation{\small viij. T.}

\includescore{cantica/cn/eccededite.tex}




\noindent{\rm{Ant. Siehe, ich mache dich zum Licht für die Völker, damit mein Heil bis an das Ende der Erde reicht.}}

\vspace{0.3cm}

\cant{Col 1, 12-20}

\begin{quote}
 


\begin{verse}


Dankt dem Vater mit Freude! *\\
Er hat euch fähig gemacht, Anteil zu haben am \\Los der Heiligen, die im Licht sind.\\ \vin 
Er hat uns der Macht der Finsternis\\ \vin  entrissen *\\ \vin
und aufgenommen in das Reich seines\\ \vin  geliebten Sohnes.\\ 
Durch ihn haben wir die Erlösung,*\\
die Vergebung der Sünden.\\ \vin 
Er ist das Ebenbild des unsichtbaren Gottes,*\\ \vin
der Erstgeborene der ganzen Schöpfung.\\
Denn in ihm wurde alles erschaffen *\\
im Himmel und auf Erden,\\ \vin
 das Sichtbare und das Unsichtbare, †\\ \vin
Throne und Herrschaften, Mächte und\\ \vin  Gewalten; *\\ \vin
 alles ist durch ihn und auf ihn hin geschaffen.\\ 
Er ist vor aller Schöpfung, *\\
in ihm hat alles Bestand.\\ \vin
Er ist das Haupt des Leibes, *\\ \vin
der Leib aber ist die Kirche.\\
 Er ist der Ursprung, † \\der Erstgeborene der Toten;*\\ so hat er in allem den Vorrang.\\ \vin 
Denn Gott wollte mit seiner ganzen Fülle in\\ \vin  ihm wohnen,*\\ \vin
um durch ihn alles zu versöhnen.\\ 
Alles im Himmel und auf Erden wollte er zu\\  Christus führen,*\\
der Friede gestiftet hat am Kreuz durch sein Blut.\\ 
\end{verse}
\end{quote}
\medskip
 
\noindent{\rot{Resp.br.} Adjutorium nostrum {\rot{ p. 58.}}\\
\noindent{\rot{Hymnus} Deus Creator omnium \rot{vel} Lucis Creator optime \rot{ut in Dominica p. 32-33 sq.}}

\medskip


\begin{flushleft}

\versik{Dirigátur, Dómine, orátio mea.}{Sicut incénsum in conspéctu tuo.}

\medskip
{\rm{
\versik{Herr, mein Gebet werde gelenkt.}{Wie Weihrauch vor dein Angesicht.}
}}
\end{flushleft}

\setspaceafterinitial{5.2mm plus 0em minus 0em}
\setspacebeforeinitial{4.2mm plus 0em minus 0em}
\def\greinitialformat#1{{\fontsize{40}{40}\selectfont #1}}
\gresetfirstlineaboveinitial{\small \textcolor{red}{Magni.}}{}
\setaboveinitialseparation{0.72mm}
\setsecondannotation{\small j. T.}

\includescore{cantica/evangelica/quiafecitmihi.tex}

\vspace{0.2cm}

\rot{Canticum} Magnificat \rot{p. 200.}

\newpage