\kkap{MONTAG}

\section[MONTAG]{AD TERTIAM}

\rot{Hymnus} Nunc sancte nobis Spiritus \rot{p. 1124.}
\vspace{0.3cm}

\setspaceafterinitial{5.2mm plus 0em minus 0em}
\setspacebeforeinitial{4.2mm plus 0em minus 0em}
\def\greinitialformat#1{{\fontsize{40}{40}\selectfont #1}}
\gresetfirstlineaboveinitial{\small \textcolor{red}{Ps 119b}}{}
\setaboveinitialseparation{0.72mm}
\setsecondannotation{\small ij. T.}
\includescore{psalmi/118/118vdeducmedomine.tex}



\medskip

\includescore{psalmi/tertia/horatertia_secunda.tex}


\begin{flushleft}

\versik{Adiútor meus es tu, ne me reícias.}{Neque derelínquas me, Deus salútis meæ.}

\medskip

{\rm{
\versik{Du bist mein Helfer, verstoß mich nicht.}{Und lass mich nicht im Stich, Gott, meines Heiles.}
}}
\end{flushleft}

\section{MITTAGESHORE, I. WOCHE}

\rot{Hymnus} Rector potens, verax Deus \rot{p. 126.}

\vspace{0.3cm}

\setspaceafterinitial{5.2mm plus 0em minus 0em}
\setspacebeforeinitial{4.2mm plus 0em minus 0em}
\def\greinitialformat#1{{\fontsize{40}{40}\selectfont #1}}
\gresetfirstlineaboveinitial{\small \textcolor{red}{ Ps 7 8 9a}}{}
\setaboveinitialseparation{0.72mm}
\setsecondannotation{\small j. T.}

\includescore{psalmi/7/ps7dominedeus789.tex}



\includescore{psalmi/sexta/sextpsalmisecunda}

\begin{flushleft}

\versik{Benedícam Dóminum in omni témpore.}{Semper laus eius in ore meo.}

\medskip
{\rm{
\versik{Ich will den Herrn allezeit preisen.}{Immer sei sein Lob in meinem Mund.}
}}
\end{flushleft}

\section{MITTAGESHORE, II. WOCHE}

\rot{Hymnus} Rector potens, verax Deus \rot{p. 126.}

\vspace{0.3cm}

 \setspaceafterinitial{4.2mm plus 0em minus 0em}
\setspacebeforeinitial{4.2mm plus 0em minus 0em}
\def\greinitialformat#1{{\fontsize{40}{40}\selectfont #1}}
\gresetfirstlineaboveinitial{\small \textcolor{red}{Ps  123sq }}{}
\setaboveinitialseparation{0.72mm}
\setsecondannotation{\small viij. T.}

\includescore{psalmi/122/quihabitas.tex}

\medskip

\includescore{psalmi/nona/nonpsalmisecunda}


\begin{flushleft}

\versik{Rédime me, Dómine, et miserére mei.}{In ecclésiis benedícam Dómino.}

\medskip
{\rm{
\versik{Erlöse mich und sei mir gnädig.}{Den Herrn will ich preisen in der Gemeinde.}
}}
\end{flushleft}



\section[VESPERAE]{AD VESPERAS}

\setspaceafterinitial{4.2mm plus 0em minus 0em}
\setspacebeforeinitial{4.2mm plus 0em minus 0em}
\def\greinitialformat#1{{\fontsize{40}{40}\selectfont #1}}
\gresetfirstlineaboveinitial{\small \textcolor{red}{Ps 113}}{}
\setaboveinitialseparation{0.72mm}
\setsecondannotation{\small 7. T.}

\includescore{psalmi/112/sitnomendomini.tex}
\vspace{0.3cm}

\psal{113}

\begin{quote}

\begin{verse}
Lobet, ihr Kn\d echte des Herrn, *\\
lobt den N\d amen des Herr\textit{e}n!\\ 
\vin Der \textit{ Name des H\d errn sei gepriesen} *\\ 
\vin von nun an b\d is \textit{in Ewigkeit.}\\
Vom Aufgang der Sonne b\d is zum Untergang *\\
sei der Name des H\d err{\textit{e}}n gelobt.\\ 
\vin Der Herr ist erhaben über \d alle Völker, *\\ 
\vin seine Herrlichkeit überr\d agt die Himmel.\\ 
Wer gleicht dem H\d errn, unserm Gott, *\\
im Himmel \d und auf Erden,\\ 
\vin ihm, der \d in der Höhe thront, *\\ 
\vin der hinabschaut \d in die Tiefe, \\
der den Schwachen aus dem St\d aub emporhebt *\\
und den Armen erhöht, d\d er im Schmutz liegt?\\ 
\vin Er gibt ihm einen S\d itz bei den Edlen, *\\ 
\vin bei den Edlen s\d eines Volkes.\\ 
Die Frau, die kinderlos war, lässt er\\
im H\d ause wohnen; *\\
sie wird Mutter und freut sich an \d ihren Kindern.\\ 

\end{verse}
\end{quote}

\vspace{0.3cm}


\setspaceafterinitial{4.2mm plus 0em minus 0em}
\setspacebeforeinitial{4.2mm plus 0em minus 0em}
\def\greinitialformat#1{{\fontsize{40}{40}\selectfont #1}}
\gresetfirstlineaboveinitial{\small \textcolor{red}{Ps 114}}{}
\setaboveinitialseparation{0.72mm}
\setsecondannotation{\small 1. T. irr.}

\includescore{psalmi/113/afacie.tex}
\vspace{0.3cm}



\psal{114}

\begin{quote}
\begin{verse}
Als Israel aus Ägypten \d auszog, *\\
Jakobs Haus aus dem Volk mit fr\d emder Sprache, \\ 
\vin da wurde Juda Gottes H\d eiligtum, *\\ 
\vin Israel das Gebiet s\d einer Herrschaft.\\  
Das Meer sah \d es und floh, *\\
der Jord\d an wich zurück.\\ 
\vin Die Berge hüpften wie W\d idder, *\\ 
\vin die Hügel wie j\d unge Lämmer.\\  
Was ist mit dir, Meer, dass \d du fliehst, *\\
und mit dir, Jordan, dass \d du zurückweichst?\\ 
\vin Ihr Berge, was hüpft ihr wie W\d idder, *\\ 
\vin und ihr Hügel, wie j\d unge Lämmer?\\  
Vor dem Herrn \textit{erbebe, du \d Erde, *\\
vor dem Antlitz des G\d ottes} Jakobs,\\ 
\vin der den Fels zur Wasserflut w\d andelt *\\ 
\vin und Kieselgestein zu quell\d endem Wasser.\\  
\end{verse}
\end{quote}



\setspaceafterinitial{7.2mm plus 0em minus 0em}
\setspacebeforeinitial{4.2mm plus 0em minus 0em}
\def\greinitialformat#1{{\fontsize{40}{40}\selectfont #1}}
\gresetfirstlineaboveinitial{\small \textcolor{red}{Ps 115}}{}
\setaboveinitialseparation{0.72mm}
\setsecondannotation{\small 1. T. irr.}

\includescore{psalmi/114/nosquivivimus.tex}

\vspace{0.3cm}

\psal{115}

\begin{quote}
\begin{verse}
Nicht uns, o Herr, bring zu Ehr\d en, †\\
nicht uns, sondern deinen N\d amen, *\\  
in deiner H\d uld und Treue! \\ 
\vin Warum sollen die Völker s\d agen: *\\ 
\vin «Wo \d ist denn ihr Gott?» \\
Unser Gott ist im H\d immel; *\\ 
alles, was ihm gefällt, d\d as vollbringt er.\\ 
\vin Die Götzen der Völker sind nur \\ 
\vin Silber \d und Gold, *\\ 
\vin ein Machw\d erk von Menschenhand.\\ 
Sie haben einen Mund und r\d eden nicht, *\\ 
Aug\d en und sehen nicht;\\ 
\vin sie haben Ohren und h\d ören nicht, *\\ 
\vin eine Nas\d e und riechen nicht;\\
mit ihren Händen können sie nicht greif\d en, †\\
mit den Füßen nicht g\d ehen, *\\  
sie bringen keinen Laut hervor aus \d ihrer Kehle.\\ 
\vin Die sie gemacht haben, sollen\\ 
\vin ihrem Machwerk gl\d eichen, *\\ 
\vin alle, die den Götz\d en vertrauen.\\ 
Israel, vertrau auf den H\d err\textit{e}n! *\\ 
Er ist für euch H\d elfer und Schild.\\ 
\vin Haus Aaron, vertrau auf den H\d err\textit{e}n! *\\ 
\vin Er ist für euch H\d elfer und Schild.\\ 
Alle, die ihr den Herrn fürchtet, \\
vertraut auf den H\d err\textit{e}n! *\\ 
Er ist für euch H\d elfer und Schild.\\ 
\vin Der Herr denkt an uns, er wird uns segn\d en, †\\ 
\vin er wird das Haus Israel s\d egnen, *\\ 
\vin  er wird das Haus \d Aaron segnen.\\ 
Der Herr wird alle segnen, die ihn f\d ürchten, *\\ 
segnen Klein\d e und Große.\\ 
\vin Es mehre euch d\d er Herr, *\\ 
\vin euch und \d eure Kinder.\\  
Seid gesegnet vom H\d err\textit{e}n, *\\ 
der Himmel und Erd\d e gemacht hat.\\ 
\vin Der Himmel ist der Himmel des H\d err\textit{e}n, *\\ 
\vin die Erde aber gab \d er den Menschen.\\  
Tote können den Herrn nicht mehr l\d oben, *\\ 
keiner, der ins Schweig\d en hinabfuhr.\\ 
\vin \textit{Wir aber preisen den H\d err\textit{e}n} *\\ 
\vin von nun an b\d is in Ewigkeit.\\ 

\end{verse}
\end{quote}

\medskip
\vspace{0.3cm}
\setspaceafterinitial{5.2mm plus 0em minus 0em}
\setspacebeforeinitial{4.2mm plus 0em minus 0em}
\def\greinitialformat#1{{\fontsize{40}{40}\selectfont #1}}
\gresetfirstlineaboveinitial{\small \textcolor{red}{ Eph 1}}{}
\setaboveinitialseparation{0.72mm}
\setsecondannotation{\small 3. T.}

\includescore{cantica/cn/sanguisjesu.tex}



\cant{Eph 1,3-10}

\begin{quote}
\begin{verse}
Gepr\d iesen sei Gott,*\\
der Gott und Vater unseres Herrn\\
Jesus Christus:\\
\vin Er hat uns mit allem Segen \\ 
\vin seines G\d eistes gesegnet *\\
\vin durch unsere Gemeinschaft mit Christus\\ 
\vin im Himmel.\\
Denn in ihm hat er uns erwählt vor der\\
Ersch\d affung der Welt,*\\
damit wir heilig und untadelig leben vor Gott;\\
\vin er hat uns aus Liebe\\ 
\vin im Voraus d\d azu bestimmt,*\\
\vin seine Söhne zu werden \textit{durch Jesus Christus,}\\
und nach seinem gnädigen Willen zu \d ihm \\
zu gelangen, *\\
zum Lob seiner herrlichen Gnade.\\
\vin Er hat sie uns geschenkt in seinem geliebten\\ 
\vin Sohn; †\\
\vin \textit{durch sein Blut haben wir} die Erlösung \\ 
\vin \textit{die Verg\d ebung der Sünden},*\\
\vin nach dem Reichtum seiner Gnade. \\
Durch sie hat er uns r\d eich beschenkt *\\
mit aller Weisheit und Einsicht \\
\vin und hat uns das Geheimnis seines\\ 
\vin W\d illens kundgetan,* \\
\vin wie er es gnädig im Voraus bestimmt hat: \\
Er hat beschlossen, \\
die Fülle der Zeiten heraufzuführen, †\\
in Christus \d alles zu vereinen, *\\
alles, was im Himmel und auf Erden ist. \\!

\end{verse}
\end{quote}

\medskip



\setspaceafterinitial{4.2mm plus 0em minus 0em}
\setspacebeforeinitial{4.2mm plus 0em minus 0em}
\resp

\includescore{responsoria_diebusferialibus/respbradiutorium.tex}

\medskip

\begin{sloppypar}
{\noindent\rm{\rot{Resp.} Unsere Hilfe ist im Namen des Herrn. Der Himmel und Erde gemacht hat.}}
\end{sloppypar}

\vspace{0.3cm}

\crot{Hymnus}

\setspaceafterinitial{5.2mm plus 0em minus 0em}
\setspacebeforeinitial{4.2mm plus 0em minus 0em}
\def\greinitialformat#1{{\fontsize{40}{40}\selectfont #1}}
\gresetfirstlineaboveinitial{\small \textcolor{red}{hieme}}{}
\setaboveinitialseparation{0.72mm}
\setsecondannotation{\small viij. T.}
\includescore{hymni/deuscreatoromnium.tex}

\medskip

\begin{sloppypar}
{\noindent\rm{\rot{1.} Gott, aller Dinge Schöpfer und Herrscher des Himmels,
du zierst den Tag mit glänzendem Licht, die Nacht mit der Gunst des Schlafens.}}
\end{sloppypar}

\medskip

{\setlength{\columnsep}{1cm}
\begin{multicols}{2} 
\begin{verse}[\versewidth]
 
{\small{
\frot{A}rtus solútos ut quies\\
Reddat labóris úsui;\\
Mentésque fessas állevet,\\
Luctúsque solvat ánxios.\\!

\frot{G}rates perácto iam die,\\
Et noctis exórtu, preces,\\ 
Voti reos ut ádiuves,\\ 
Hymnum canéntes,\\
sólvimus.\\!

\frot{T}e cordis ima cóncinant, \\
Te vox canóra cóncrepet; \\
Te díligat castus amor, \\
Te mens adóret sóbria. \\!}}
\end{verse}

\columnbreak

\begin{verse}[\versewidth]
 
{\small\rm{\frot{2.} Dass Ruhe den erschlafften\\
Gliedern, die Arbeitskraft erneu're,\\
den müden Geist erhebe,\\
und Klagen Ängste löse.\\!

\frot{3.} Hymnen singend, lösen wir ein\\
die Pflicht des Gelobten; bringen\\
nun dar den Dank für den Tag\\
und Gebete zum Anbruch der\\
Nacht, damit du uns hilfst.\\!

\frot{4.} Des Herzens Mitte preise dich,\\
für dich erschalle wohlklingender\\
Ton, dich liebe die reine Liebe,\\
dich verehr' der verständige Geist.\\!}}
\end{verse}
\end{multicols}
}



{\setlength{\columnsep}{1cm}
\begin{multicols}{2} 
\begin{verse}[\versewidth]

{\small{\frot{U}t cum profúnda cláuserit\\
Diem calígo nóctium;\\ 
Fides tenébras nésciat,\\
Et nox fide\\
relúceat.\\!

\frot{C}hristum rogámus et Patrem,\\ 
Christi Patrísque Spíritum;\\
Unum potens per ómnia,\\ 
Fove precántes, Trínitas.\\
Amen.\\!}}
\end{verse}

\columnbreak

\begin{verse}[\versewidth]
 
{\small\rm{\frot{5.} Dass, wenn der Abgrund uns\\
umschließt, die Finsternis der\\
Nacht den Tag, der Glaub' von\\
Dunkelheit nichts weiß und\\
Zuversicht die Nacht erhellt.\\!

\frot{6.} Christus bitten wir und\\
den Vater, Christi und des\\
Vaters Geist; Eine Macht\\
über allem, Dreifaltigkeit,\\
umarme die Bittenden. Amen.\\!}}
 
 
\end{verse}

\end{multicols}
}

\medskip

\setspaceafterinitial{5.2mm plus 0em minus 0em}
\setspacebeforeinitial{4.2mm plus 0em minus 0em}
\def\greinitialformat#1{{\fontsize{40}{40}\selectfont #1}}
\gresetfirstlineaboveinitial{\small \textcolor{red}{æstate}}{}
\setaboveinitialseparation{0.72mm}
\setsecondannotation{\small viij. T.}
\includescore{hymni/luciscreator.tex}

\medskip

\begin{sloppypar}
{\noindent\rm{\rot{1.} Des Lichtes guter Schöpfer, du holst aus Licht den Tag hervor.
Mit dem Aufgang des neuen Lichtes, bereitest du der Welt ihren Beginn.}}
\end{sloppypar}

\medskip

{\setlength{\columnsep}{1cm}
\begin{multicols}{2} 
\begin{verse}[\versewidth]
 
{\small{
\frot{Q}ui mane iunctum vésperi\\
diem vocári præcipis:\\
tætrum chaos illábitur;\\
audi preces cum flétibus.\\!}}
\end{verse}






\begin{verse}[\versewidth]
 
{\small{
\frot{N}e mens graváta crímine\\
vitæ sit exsul múnere,\\ 
dum nil perénne cógitat\\
seséque culpis ílligat.\\!

\frot{C}ælórum pulset íntimum,\\
vitále tollat præmium;\\
vitémus omne nóxium,\\ 
purgémus omne péssimum.\\!



\frot{P}ræsta, Pater piíssime,\\ 
Patríque compar Unice,\\ 
cum Spíritu Paráclito\\
regnans per omne sæculum.\\
Amen.\\!}}

\end{verse}

\begin{verse}[\versewidth]
{\small\rm{\frot{2.} Der du Morgen und Abend\\
vereinst, und vorsiehst, ihn Tag zu\\
nennen: Das düstre Chaos zieht\\
herauf, hör' unser flehendes Bitten.\\!}}
\end{verse}



\begin{verse}[\versewidth]
{\small\rm{\frot{3.} Dass nicht - belastet - der Geist \\
der Gunst des Lebens sei beraubt\\
wenn er nicht beständig nachsinnt\\
und sich in Schuld verstrickt.\\!

\frot{4.} Er tromm'le an des Himmels\\
Herz, trag' hinauf den ewigen\\
Lohn; Lasset uns jede Untat\\
meiden, sühnen alle Bosheit.\\!

\frot{5.} Dies gewähre, gütigster Vater,\\
und der dem Vater Gleiche, der \\
mit dem Heiligen Geist \\
ist herrschend in Ewigkeit.\\
Amen.\\!}}
\end{verse}
\end{multicols}
}


\vspace{0.3cm}

\begin{flushleft}

\versik{Dirigátur, Dómine, orátio mea.}{Sicut incénsum in conspéctu tuo.}

\medskip
{\rm{
\versik{Lenke, Herr, mein Gebet.}{Wie Weihrauch zu deinem Angesicht.}
}}
\end{flushleft}

\medskip



\setspaceafterinitial{5.2mm plus 0em minus 0em}
\setspacebeforeinitial{4.2mm plus 0em minus 0em}
\def\greinitialformat#1{{\fontsize{40}{40}\selectfont #1}}
\gresetfirstlineaboveinitial{\footnotesize \textcolor{red}{Magni}}{}
\setaboveinitialseparation{0.72mm}
\setsecondannotation{\small {\textcolor{white}{v}}v. T.}

\includescore{cantica/evangelica/exultet.tex}
\vspace{0.3cm}

\rot{Canticum} Magnificat \rot{p. 128.}

\vspace{0.5cm}


Nach dem Magnificat folgen die Fürbitten, das Vater Unser und das Tagesgebet.
Das Chorgebet endet mit einer Antiphon zu Ehren der heiligen Jungfrau.

\newpage