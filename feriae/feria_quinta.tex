\thispagestyle{plain}

\kapklein{\rot{F}eria \rot{Q}uinta}
\kkap{FERIA QUINTA}

\section[VIGILIAE]{AD VIGILIAS}


\setspaceafterinitial{5.2mm plus 0em minus 0em}
\setspacebeforeinitial{4.2mm plus 0em minus 0em}
\def\greinitialformat#1{{\fontsize{40}{40}\selectfont #1}}
\gresetfirstlineaboveinitial{\small \textcolor{red}{Invitat.}}{Invit.}
\setaboveinitialseparation{0.72mm}
%\setsecondannotation{\small Ps. 125}

%\includescore{invitatoria/venitev.tex}


\begin{sloppypar}
{\noindent{\rot{Invit.} Lasset uns anbeten den Herrn * denn er hat uns erschaffen.}}
\end{sloppypar}

\bigskip

\noindent{\rot{Hymnus} Æterne rerum Conditor \rot{vel} Nocte surgentes \rot{ut in Dominica pp. 1-3.}}


\section{VIGILIA I}


\begin{sloppypar}
{\noindent{1. Ant.}  Für den Frommen * ziemt es sich Gott zu loben. \rot{Ps. 33}\\
2. Ant. Der Engel des Herrn * umschirmt alle, die ihn fürchten und ehren, und  er befreit sie. \rot{Ps. 34}\\
3. Ant. Bekämpfe alle * die mich bekämpfen. \rot{Ps. 35}\\}
\end{sloppypar}

\begin{flushleft}

\versik{Der Herr möge uns seine Wege lehren.}{Und wir gehen auf seinen Pfaden.}

\end{flushleft}

\section{VIGILIA II}

\begin{sloppypar}
{\noindent{1. Ant.} Ich vertraue auf Gott * und fürchte nicht, was die Menschen mir antun. \rot{Ps. 56} \\
2. Ant. Richtet gerecht * ihr Menschen. \rot{Ps. 58} \\
3. Ant. Befreie mich * von den Bösen, denn sie wollen sich meiner Seele bemächtigen. \rot{Ps. 59} \\}
\end{sloppypar}

\begin{flushleft}

\versik{Herr, deine Wahrheit reicht bis zu den Wolken.}{Deine Gerichte sind tief wie der Abgrund.}

\end{flushleft}


\section{VIGILIA III}

\begin{sloppypar}
{\noindent{1. Ant.} Er vergab ihnen * voll Erbarmen die Schuld. \rot{Ps. 78, 32-72}\\
2. Ant. Sei nachsichtig * mit unseren Sünden, o Herr. \rot{Ps. 79}\\}
\end{sloppypar}

\begin{flushleft}

\versik{Preist Gott, unseren Helfer.}{Jubelt dem Gott Jakobs zu.}

\end{flushleft}


\section{VIGILIA IV}

\begin{sloppypar}
{\noindent{1. Ant.} Herr * wie zahlreich sind deine Werke! \\ \rot{Ps. 104,19-35}\\
2. Ant. Freuen soll sich das Herz derer * die den Herrn suchen. \rot{Ps. 105,1-22} - Divisio -  \rot{Ps. 105, 23-45}}\\ 

\end{sloppypar}

\begin{flushleft}

\versik{Sucht den Herrn und preist ihn.}{Sucht beständig sein Antlitz.}

\end{flushleft}



\section[LAUDES]{AD LAUDES}

\setspaceafterinitial{5.2mm plus 0em minus 0em}
\setspacebeforeinitial{4.2mm plus 0em minus 0em}
\def\greinitialformat#1{{\fontsize{40}{40}\selectfont #1}}
\gresetfirstlineaboveinitial{\small \textcolor{red}{Ps 76}}{}
\setaboveinitialseparation{0.72mm}
\setsecondannotation{\small viij. T.}

\includescore{psalmi/75/ps75.tex}
\vspace{0.3cm}

\newpage
\crot{Psalm 76}

\begin{quote}
 


\begin{verse}
Gott gab sich zu erkennen in Juda, *\\
\textit{sein Name ist groß in Israel.}\\
\vin Sein Zelt erstand in Salem, *\\
\vin seine Wohnung auf dem Zion.\\ 
Dort zerbrach er die blitzenden Pfeile des\\ Bogens, *\\
Schild und Schwert, die Waffen des Krieges.\\
\vin Du bist furchtbar und herrlich, *\\
\vin mehr als die ewigen Berge.\\
Ausgeplündert sind die tapferen Streiter, †\\
sie sinken hin in den Schlaf; *\\ 
allen Helden versagen die Hände.\\ 
\vin Wenn du drohst, Gott Jakobs, *\\
\vin erstarren Rosse und Wagen.\\
Furchtbar bist du. Wer kann bestehen vor dir, *\\
vor der Gewalt deines Zornes? \\
\vin Vom Himmel her machst du das Urteil\\ \vin  bekannt; *\\
\vin Furcht packt die Erde, und sie verstummt,\\
wenn Gott sich erhebt zum Gericht, *\\
um allen Gebeugten auf der Erde zu helfen.\\ 
\vin Denn auch der Mensch voll Trotz muss \\ \vin dich preisen *\\
\vin und der Rest der Völker dich feiern.\\
Legt Gelübde ab und erfüllt sie dem\\ Herrn, eurem Gott! *\\
Ihr alle ringsum, bringt Gaben ihm, den ihr\\ fürchtet.\\ 
\vin Er nimmt den Fürsten den Mut; *\\
\vin Furcht erregend ist er für die Könige der Erde.\\


\end{verse}

\end{quote}



\vspace{0.3cm}
\setspaceafterinitial{4.2mm plus 0em minus 0em}
\setspacebeforeinitial{4.2mm plus 0em minus 0em}
\def\greinitialformat#1{{\fontsize{40}{40}\selectfont #1}}
\gresetfirstlineaboveinitial{\small \textcolor{red}{Ps 88}}{}
\setaboveinitialseparation{0.72mm}
\setsecondannotation{\small v. T.}
\includescore{psalmi/87/intretoratio.tex}

\vspace{0.3cm}
\crot{Psalm 88}

\begin{quote}
 


\begin{verse}
 Herr, du Gott meines Heils, *\\
zu dir schreie ich am T\d ag und bei Nacht.\\ \vin  
\textit{Lass mein Gebet zu dir dringen,} *\\ \vin 
wende dein \d Ohr meinem Flehen zu!\\ 
Denn meine Seele ist gesättigt mit Leid, *\\
mein Leben ist dem T\d otenreich nahe.\\ \vin  
Schon zähle ich zu denen, die hinabsinken \\ \vin ins Grab, *\\ \vin 
bin wie ein Mann, dem alle Kr\d aft \\ \vin genommen ist.\\ 
Ich bin zu den Toten hinweggerafft *\\
wie Erschlagene, die im Gr\d abe ruhen;\\
\vin an sie denkst du nicht mehr, *\\ 
\vin denn sie sind deiner H\d and entzogen.\\  
Du hast mich ins tiefste Grab gebracht, *\\ 
tief hinab in f\d instere Nacht.\\
 \vin Schwer lastet dein Grimm auf mir, *\\
 \vin all deine Wogen stürzen über m\d ir zusammen.\\   
Die Freunde hast du mir entfremdet, †\\  
mich ihrem Abscheu ausgesetzt; *\\  ich bin gefangen und k\d ann nicht heraus.\\ 
\vin Mein Auge wird trübe vor Elend.  †\\
\vin Jeden Tag, Herr, ruf ich zu dir; *\\ \vin ich strecke nach d\d ir meine Hände aus.\\
 Wirst du an den Toten Wunder tun, *\\
 werden Schatten aufstehn, um d\d ich zu preisen? \\ 
\vin Erzählt man im Grab von deiner Huld, *\\
\vin von deiner Tr\d eue im Totenreich?\\
 Werden deine Wunder in der Finsternis \\  bekannt, *\\
 deine Gerechtigkeit im L\d and des Vergessens?\\
\vin Herr, darum schreie ich zu dir, *\\
\vin früh am Morgen tritt mein Geb\d et \\ \vin vor dich hin.\\ 
 Warum, o Herr, verwirfst du mich, *\\
 warum verbirgst du d\d ein Gesicht vor mir? \\
\vin Gebeugt bin ich und todkrank von früher\\ \vin  Jugend an, *\\
\vin deine Schrecken lasten auf mir und \d ich \\ \vin  bin zerquält.\\ 
 Über mich fuhr die Glut deines Zorns dahin, *\\
 deine Schr\d ecken vernichten mich.\\
\vin Sie umfluten mich allzeit wie Wasser *\\
\vin und dringen auf mich ein von \d allen Seiten.\\ 
 Du hast mir die Freunde und Gefährten\\ entfremdet; *\\
mein Vertrauter ist n\d ur noch die Finsternis.\\ 

\end{verse}

\end{quote}

\vspace{0.3cm}

\setspaceafterinitial{5.2mm plus 0em minus 0em}
\setspacebeforeinitial{4.2mm plus 0em minus 0em}
\def\greinitialformat#1{{\fontsize{40}{40}\selectfont #1}}
\gresetfirstlineaboveinitial{\small \textcolor{red}{Ier 31}}{}
\setaboveinitialseparation{0.72mm}
\setsecondannotation{\small iv. T.}
\includescore{cantica/ca/ier31.tex}

\cant{Ier 31,10-14}

\begin{quote}
 
\begin{verse}
Hört, ihr Völker, das Wort des Herrn,*\\
verkündet es auf den fernst\d en Inseln und sagt:\\ 
\vin Er, der Israel zerstreut hat, wird es \\ \vin auch sammeln*\\
\vin und hüten wie ein H\d irt seine Herde.\\
Denn der Herr wird Jakob erlösen*\\
und ihn befreien aus d\d er Hand des Stärkeren.\\
\vin Sie kommen und jubeln auf Zions Höhe, †\\
\vin sie strahlen vor Freude über die \\ \vin Gaben des Herrn,*\\ 
\vin über Korn, Wein und Öl, über L\d ämmer\\ \vin und Rinder.\\ 
Sie werden wie ein bewässerter Garten sein*\\ 
und n\d ie mehr verschmachten.\\
\vin Dann freut sich das Mädchen beim\\ \vin  Reigentanz,*\\
\vin Jung \d und Alt sind fröhlich.\\ 
Ich verwandle ihre Trauer in Jubel,*\\
tröste und erfreue sie n\d ach ihrem Kummer.\\
\vin Ich labe die Priester mit Opferfett *\\
\vin und \textit{mein Volk wird satt \d an meinen Gaben.}\\

\end{verse}

\end{quote}

\rot{vel}

\setspaceafterinitial{5.2mm plus 0em minus 0em}
\setspacebeforeinitial{4.2mm plus 0em minus 0em}
\def\greinitialformat#1{{\fontsize{40}{40}\selectfont #1}}
\gresetfirstlineaboveinitial{\small \textcolor{red}{Is 12}}{}
\setaboveinitialseparation{0.72mm}
\setsecondannotation{\small viij. T.}
\includescore{cantica/ca/is12.tex}

\cant{Is 12,1-6}

\begin{quote}
 



\begin{verse}
 
Ich danke dir, Herr.\\ Du hast mir gezürnt, † \\ doch \textit{ dein Zorn hat sich gewendet *\\ und du hast mich getröstet}.\\ \vin
Ja, Gott ist meine Rettung; *\\ \vin
ihm will ich vertrauen und niemals verzagen.\\ 
Denn meine Stärke und mein \\ Lied ist der Herr. *\\ Er ist für mich zum Retter geworden.\\ \vin 
Ihr werdet Wasser schöpfen voll Freude *\\ \vin
aus den Quellen des Heil\textit{e}s.\\
An jenem Tag werdet ihr sagen: *\\
Dankt dem Herrn! Ruft seinen Namen an!\\ \vin 
Macht seine Taten unter den \\ \vin Völkern bekannt, *\\ \vin verkündet: Sein Name ist groß und erhaben!\\
Preist den Herrn; †\\
denn herrliche Taten hat er vollbracht; *\\ auf der ganzen Erde soll man es wissen.\\ \vin
Jauchzt und jubelt, ihr Bewohner von Zion; *\\ \vin
denn groß ist in eurer Mitte der Heilige Israels.\\
\end{verse}

\end{quote}

\vspace{0.6cm}

\vspace{0.6cm}

\setspaceafterinitial{4.2mm plus 0em minus 0em}
\setspacebeforeinitial{4.2mm plus 0em minus 0em}
\def\greinitialformat#1{{\fontsize{40}{40}\selectfont #1}}
\gresetfirstlineaboveinitial{\small \textcolor{red}{Ps 90}}{}
\setaboveinitialseparation{0.72mm}
\setsecondannotation{\small vj. T.}

% and finally we include the score. The file must be in the same directory as this one.
\includescore{psalmi/89/ps89.tex}

\vspace{0.3cm}

\crot{Psalm 90}
\begin{quote}
 


\begin{verse}
 \textit{Herr, du warst unsere Zuflucht} *\\
von G\d eschlecht zu Geschlecht.\\
\vin Ehe die Berge geboren wurden, †\\
\vin die Erde entstand und das Weltall, *\\ \vin bist du, o Gott, von Ew\d igkeit zu Ewigkeit.\\
Du lässt die Menschen zurückkehren \\  zum Staub *\\
und sprichst: «Kommt w\d ieder, ihr Menschen!» \\
\vin Denn tausend Jahre sind für dich 
wie der Tag, \\ \vin der gestern vergangen ist, *\\ \vin wie ein\d e Wache in der Nacht.\\ 
Von Jahr zu Jahr säst du die Menschen aus; *\\
sie gleichen d\d em sprossenden Gras.\\
\vin Am Morgen grünt es und blüht, *\\
\vin am Abend wird es g\d eschnitten und welkt.\\
Denn wir vergehen durch deinen Zorn, *\\
werden vern\d ichtet durch deinen Grimm.\\
\vin Du hast unsre Sünden vor dich hingestellt, *\\
\vin unsere geheime Schuld in das L\d icht deines \\ \vin  Angesichts.\\ 
Denn all unsre Tage gehn hin unter \\ deinem Zorn, *\\
wir beenden unsere Jahre w\d ie einen Seufzer.\\ 
\vin Unser Leben währt siebzig Jahre, *\\
\vin und wenn es hoch k\d ommt, sind es achtzig.\\ 
Das Beste daran ist nur Mühsal\\ und Beschwer, *\\ rasch geht es vorbei, w\d ir fliegen dahin.\\ 
\vin Wer kennt die Gewalt deines Zornes *\\
\vin und fürcht\d et sich vor deinem Grimm? \\
Unsre Tage zu zählen, lehre uns! *\\
Dann gewinn\d en wir ein weises Herz.\\ 
\vin Herr, wende dich uns doch endlich zu! *\\
\vin Hab Mitleid m\d it deinen Knechten!\\
Sättige uns am Morgen mit deiner Huld! *\\
Dann wollen wir jubeln und uns freuen\\ \d all unsre Tage.\\ 
\vin Erfreue uns so viele Tage, wie du uns \\ \vin gebeugt hast, *\\
\vin so viele Jahre, wie wir \d Unglück erlitten.\\
Zeig deinen Knechten deine Taten *\\
und ihren Kindern deine \d erhabene Macht! \\
\vin Es komme über uns die Güte des Herrn, \\ \vin unsres Gottes. †\\
\vin Lass das Werk unsrer Hände gedeihen, *\\ \vin ja, lass gedeihen das W\d erk unsrer Hände!\\
\end{verse}

\end{quote}

\noindent\rot{Resp.br.} Sana animam \rot{ut in Feria Secunda p. 43.}\\
\noindent\rot{Hymnus} Splendor paternæ \rot{vel} Ecce iam noctis \rot{ut in Dominica p. 15-17.}
\begin{flushleft}

\versik{Repléti sumus mane misericórdia tua.}{Exultávimus, et delectáti sumus.}

\medskip

{\rm{
\versik{Am Morgen sind wir erfüllt von deiner Huld.}{Wir sind (voll) Jubel und Frohsinn.}
}}
\end{flushleft}

\setspaceafterinitial{5.2mm plus 0em minus 0em}
\setspacebeforeinitial{4.2mm plus 0em minus 0em}
\def\greinitialformat#1{{\fontsize{40}{40}\selectfont #1}}
\gresetfirstlineaboveinitial{\small \textcolor{red}{Benedic.}}{}
\setaboveinitialseparation{0.72mm}
\setsecondannotation{\small viij. T.}

\includescore{cantica/evangelica/addandam.tex}

\rot{Canticum} Benedictus \rot{p. 196.}



\section[HORA TERTIA]{AD TERTIAM}

\rot{Hymnus} Nunc sancte nobis Spiritus \rot{p. 190.}

\vspace{0.3cm}

\setspaceafterinitial{5.2mm plus 0em minus 0em}
\setspacebeforeinitial{4.2mm plus 0em minus 0em}
\def\greinitialformat#1{{\fontsize{40}{40}\selectfont #1}}
\gresetfirstlineaboveinitial{\small \textcolor{red}{Ps 119e}}{}
\setaboveinitialseparation{0.72mm}
\setsecondannotation{\small viij. T.}

\includescore{psalmi/118/adiuvame14.tex}

\includescore{psalmi/tertia/horatertia_quinta.tex}

\vspace{0.3cm}

\begin{flushleft}

\versik{Dóminus non privábit bonis eos qui ámbulant in innocéntia.}{Dómine virtútum, beátus homo qui sperat in te.}

\medskip

{\rm{
\versik{Der Herr wird seine Güter nicht denen vorenthalten, die in Unschuld wandeln.}{Herr der Scharen, selig der Mensch, der auf dich hofft.}
}}
\end{flushleft}

\section[HORA SEXTA]{AD SEXTAM}

\rot{Hymnus} Rector potens, verax Deus \rot{p. 192.}

\vspace{0.3cm}

\setspaceafterinitial{5.2mm plus 0em minus 0em}
\setspacebeforeinitial{4.2mm plus 0em minus 0em}
\def\greinitialformat#1{{\fontsize{40}{40}\selectfont #1}}
\gresetfirstlineaboveinitial{\small \textcolor{red}{ Ps 16-18a}}{}
\setaboveinitialseparation{0.72mm}
\setsecondannotation{\small ij. T.}

\includescore{psalmi/15/conserva1516117.tex}


\vspace{0.3cm}

\includescore{psalmi/sexta/sextpsalmiquinta.tex}

\begin{flushleft}

\versik{Dómine, veritátem in corde dilexísti.}{Et in occúlto sapiéntiam manifestásti mihi.}

\medskip

{\rm{
\versik{Herr, du liebst, die im Herzen voll Wahrheit sind.}{Und im Verborgenen lehrtest du mich Weisheit.}
}}
\end{flushleft}


\section[HORA NONA]{AD NONAM}

\rot{Hymnus} Rerum Deus tenax vigor \rot{p. 194.}

\vspace{0.3cm}

 \setspaceafterinitial{5.2mm plus 0em minus 0em}
\setspacebeforeinitial{4.2mm plus 0em minus 0em}
\def\greinitialformat#1{{\fontsize{40}{40}\selectfont #1}}
\gresetfirstlineaboveinitial{\small \textcolor{red}{Ps 132sq}}{}
\setaboveinitialseparation{0.72mm}
\setsecondannotation{\small 1. T.}

\includescore{psalmi/131/etomnismansutudiniseius.tex}
\vspace{0.3cm}
\newpage

\includescore{psalmi/nona/nonpsalmiquinta.tex}



\begin{flushleft}

\versik{Fac cum servo tuo secúndum misericórdiam tuam, Dómine.}{Iustificatiónes tuas doce me.}

\medskip

{\rm{
\versik{Handle an deinem Knecht nach deiner Huld.}{Lehre mich deine Gesetze.}
}}
\end{flushleft}



\section[VESPERAE]{AD VESPERAS}

\setspaceafterinitial{4.2mm plus 0em minus 0em}
\setspacebeforeinitial{4.2mm plus 0em minus 0em}
\def\greinitialformat#1{{\fontsize{40}{40}\selectfont #1}}
\gresetfirstlineaboveinitial{\small \textcolor{red}{Ps 141}}{}
\setaboveinitialseparation{0.72mm}
\setsecondannotation{\small 8. T.}

\includescore{psalmi/140/domineclamavi.tex}

\vspace{0.3cm}
\crot{Psalm 141}

\begin{quote}

 
\begin{verse}

\textit{ Herr, ich rufe zu dir}. Eile mir zu Hilfe; *\\
\textit{höre auf meine Stimme}, wenn ich zu dir rufe.\\ \vin 
Wie ein Rauchopfer steige mein Gebet\\ \vin  vor dir auf; *\\ \vin 
als Abendopfer gelte vor dir, wenn ich\\ \vin  meine Hände erhebe.\\ 
Herr, stell eine Wache vor meinen Mund, *\\
eine Wehr vor das Tor meiner Lippen! \\ \vin 
Gib, dass mein Herz sich bösen Worten\\ \vin  nicht zuneigt, *\\ \vin 
dass ich nichts tue, was schändlich ist, \\
zusammen mit Menschen, die Unrecht tun. *\\ Von ihren Leckerbissen will ich nicht kosten.\\
\vin Der Gerechte mag mich schlagen aus Güte: *\\
\vin Wenn er mich bessert, ist es Salböl für\\ \vin mein Haupt; \\  
da wird sich mein Haupt nicht sträuben. *\\   Ist er in Not, will ich stets für ihn beten.\\
\vin Haben ihre Richter sich auch die Felsen\\  \vin hinabgestürzt, *\\ 
\vin sie sollen hören, dass mein Wort für sie\\ \vin freundlich ist.\\ 
Wie wenn man Furchen zieht und das\\   Erdreich aufreißt, *\\ 
so sind unsre Glieder hingestreut an \\ den Rand der Unterwelt.\\
\vin Mein Herr und Gott, meine Augen richten\\ \vin sich auf dich; *\\
\vin bei dir berge ich mich. Gieß mein\\ \vin Leben nicht aus! \\
Vor der Schlinge, die sie mir legten,  bewahre \\mich, *\\
vor den Fallen derer, die Unrecht tun! \\
\vin(Die Frevler sollen sich in ihren eigenen\\ \vin Netzen fangen, *\\
\vin während ich heil entkomme.)\\ 



\end{verse}


\vspace{0.3cm}

\setspaceafterinitial{4.2mm plus 0em minus 0em}
\setspacebeforeinitial{4.2mm plus 0em minus 0em}
\def\greinitialformat#1{{\fontsize{40}{40}\selectfont #1}}
\gresetfirstlineaboveinitial{\small \textcolor{red}{Ps 142}}{}
\setaboveinitialseparation{0.72mm}
\setsecondannotation{\small 8. T.}

\includescore{psalmi/141/portiomea.tex}

\vspace{0.3cm}

\psal{142}

\begin{verse}
 Mit lauter Stimme schrei ich zum Herrn, *\\ 
laut flehe ich zum Herrn um Gnade.\\ \vin
Ich schütte vor ihm meine Klagen aus, *\\ \vin
eröffne ihm meine Not.\\ 
Wenn auch mein Geist in mir verzagt, *\\ 
du kennst meinen Pfad.\\ \vin 
Auf dem Weg, den ich gehe, *\\ \vin legten sie mir Schlingen.\\  
Ich blicke nach rechts und schaue aus, *\\ 
doch niemand ist da, der mich beachtet.\\ \vin 
Mir ist jede Zuflucht genommen, *\\ \vin niemand fragt nach meinem Leben.\\ 
Herr, ich schreie zu dir, †\\
ich sage: Meine Zuflucht bist du, *\\ \textit{ mein Anteil im Land der Lebenden}.\\ \vin 
Vernimm doch mein Flehen; *\\ \vin
denn ich bin arm und elend.\\  
Meinen Verfolgern entreiß mich; *\\  sie sind viel stärker als ich.\\ \vin
Führe mich heraus aus dem Kerker, *\\ \vin
damit ich deinen Namen preise.\\  
Die Gerechten scharen sich um mich, *\\  weil du mir Gutes tust.\\  
\end{verse}



\vspace{0.3cm}

\setspaceafterinitial{5.2mm plus 0em minus 0em}
\setspacebeforeinitial{4.2mm plus 0em minus 0em}
\def\greinitialformat#1{{\fontsize{40}{40}\selectfont #1}}
\gresetfirstlineaboveinitial{\small \textcolor{red}{Ps 144}}{}
\setaboveinitialseparation{0.72mm}
\setsecondannotation{\small vj. T.}
\includescore{psalmi/143/benedictusdominusdeusmeus.tex}

\vspace{0.3cm}

\psal{144a}

 
\begin{verse}
 \textit{Gelobt sei der Herr, der mein Fels ist}, *\\
der meine Hände den Kampf gelehrt hat, mein\d e\\ Finger den Krieg.\\ \vin  
Du bist meine Huld und Burg, * \\ \vin 
meine F\d estung, mein Retter, \\
mein Schild, dem ich vertraue. *\\ Er macht m\d ir Völker untertan. \\ \vin 
Herr, was ist der Mensch, dass du dich um \\ \vin ihn kümmerst, *\\ \vin 
des Menschen Kind, dass d\d u es beachtest? \\
Der Mensch gleicht einem Hauch, *\\
seine Tage sind wie ein fl\d üchtiger Schatten.\\ \vin  
Herr, neig deinen Himmel und steig herab, *\\ \vin 
rühre die Berge an, s\d o dass sie rauchen.\\ 
Schleudre Blitze und zerstreue die Feinde, *\\
schieß deine Pfeile ab \d und jag sie dahin!\\ \vin 
Streck deine Hände aus der Höhe herab und\\ \vin  befreie mich; †\\ \vin 
reiß mich heraus aus gewaltigen Wassern, *\\ \vin  aus d\d er Hand der Fremden!\\
Alles, was ihr Mund sagt, ist Lüge, *\\
Meineide schw\d ört ihre Rechte.\\
\end{verse}

\end{quote}

\vspace{0.3cm}

\setspaceafterinitial{6.2mm plus 0em minus 0em}
\setspacebeforeinitial{4.2mm plus 0em minus 0em}
\def\greinitialformat#1{{\fontsize{40}{40}\selectfont #1}}
\gresetfirstlineaboveinitial{\small \textcolor{red}{Ap 11;12}}{}
\setaboveinitialseparation{0.72mm}
\setsecondannotation{\small j. T.}

\includescore{cantica/cn/dediteidominus.tex}

\noindent{\rm{Ant. Der Herr hat ihm Herrschaft, Würde und Königtum gegeben. Alle Völker, Nationen und Sprachen müssen ihm dienen.}}

\cant{Ap 11;12}

\begin{verse}
 Wir danken dir, Herr, †\\
 Gott und Herrscher über die ganze Schöpfung,*\\
 der du bist und der du warst;\\
 \vin denn du hast deine große Macht in Anspruch\\ \vin  genommen *\\
 \vin und die Herrschaft angetreten. \\
 Die Völker gerieten in Zorn, †\\
 Da kam dein Zorn *\\
 und die Zeit, die Toten zu richten:\\
 \vin die Zeit, deine Knechte zu belohnen, †\\
 \vin die Propheten und die Heiligen *\\
 \vin  und alle, die deinen Namen fürchten, \\ \vin die Kleinen und die Großen, \\

 
 Jetzt ist er da, der rettende Sieg †\\
und die Macht und die Herrschaft unseres\\ Gottes *\\
und die Vollmacht seines Gesalbten.\\

\vin Denn gestürzt ist, der unsere Brüder\\ \vin  verklagte,*\\
\vin der sie bei Tag und bei Nacht vor unserem \\ \vin Gott verklagte.\\
Sie haben ihn besiegt durch das Blut\\ des Lammes *\\
und durch ihr Wort und Zeugnis,\\ 
\vin sie hielten ihr Leben nicht fest,*\\
\vin  bis hinein in den Tod.\\
Darum jubelt, ihr Himmel *\\
und alle, die darin wohnen.\\


\end{verse}


\medskip
 
\noindent{\rot{Resp.br.} Adjutorium nostrum {\rot{ p. 58.}}\\
\noindent{\rot{Hymnus} Deus Creator omnium \rot{vel} Lucis Creator optime \rot{ut in Dominica p. 32-33 sq.}}

\medskip



\begin{flushleft}

\versik{Dirigátur, Dómine, orátio mea.}{Sicut incénsum in conspéctu tuo.}

\medskip
{\rm{
\versik{Herr, mein Gebet werde gelenkt.}{Wie Weihrauch vor dein Angesicht.}
}}
\end{flushleft}

\vspace{0.6cm}

\setspaceafterinitial{5.2mm plus 0em minus 0em}
\setspacebeforeinitial{4.2mm plus 0em minus 0em}
\def\greinitialformat#1{{\fontsize{40}{40}\selectfont #1}}
\gresetfirstlineaboveinitial{\small \textcolor{red}{Magni.}}{}
\setaboveinitialseparation{0.72mm}
\setsecondannotation{\small iij. T.}

\includescore{cantica/evangelica/deposuit.tex}

\vspace{0.2cm}

\rot{Canticum} Magnificat \rot{p. 200.}

\newpage