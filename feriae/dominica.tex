\thispagestyle{plain}
\bf
\kapklein{\rot{D}ominica}
\kkap{DOMINICA}


\section[VIGILIAE]{AD VIGILIAS}

\crot{Invitatorium}

\vspace{0.2cm}

\setspaceafterinitial{4.2mm plus 0em minus 0em}
\setspacebeforeinitial{4.2mm plus 0em minus 0em}
\def\greinitialformat#1{{\fontsize{50}{50}\selectfont #1}}
%\gresetfirstlineaboveinitial{\small \textcolor{red}{Invitat.}}{Invitat.}
\setaboveinitialseparation{0.72mm}
%\setsecondannotation{\small Ps. 125}

\includescore{invitatoria/venitedom.tex}

\medskip

\begin{sloppypar}
{\noindent{\rot{Invit.} Lasset uns anbeten den Herrn, * der uns erschaffen hat.}}
\end{sloppypar}

\vspace{0.3cm}

\crot{Hymnus}

\setspaceafterinitial{5.2mm plus 0em minus 0em}
\setspacebeforeinitial{4.2mm plus 0em minus 0em}
\def\greinitialformat#1{{\fontsize{40}{40}\selectfont #1}}
\gresetfirstlineaboveinitial{\small \textcolor{red}{hieme}}{}
\setaboveinitialseparation{0.72mm}
\setsecondannotation{\small ij. T.}
\includescore{hymni/aeternererum.tex}

\medskip


\begin{sloppypar}
{\noindent\rm{\frot{1.} O ew'ger Schöpfer aller Welt,
des Walten Tag und Nacht regiert, du setzt den Zeiten ihre Zeit,
schenkst Wechsel in der Zeiten Lauf.}}
\end{sloppypar}

{\setlength{\columnsep}{0.5cm}

\begin{multicols}{2}
\begin{verse}[\versewidth]
 

{\small{\frot{2.} Præco diéi iam sonat, \\
noctis profúndæ pérvigil, \\
noctúrna lux viántibus\\
a nocte noctem ségregans.\\!

\frot{3.} Iesu, labántes réspice\\
et nos vidéndo córrige;\\
si réspicis, labes cadunt\\
fletúque culpa sólvitur.\\!

\frot{4.} Tu, lux, refúlge sénsibus\\
mentísque somnum díscute;\\
te nostra vox primum sonet\\
et vota solvámus tibi.\\!

\frot{5.} Sit, Christe, rex piíssime, \\
tibi Patríque glória\\
cum Spíritu Paráclito,\\ 
in sempitérna s\'{æ}cula. \frot{ A}men.\\!}}

\end{verse}

\columnbreak
 
\begin{verse}[\versewidth]
 
{\footnotesize\rm{\frot{2.} Der Hahn, des Tages Herold, ruft,\\
der Wächter in der Finsternis.\\
Sein Schrei trennt\\
von der Nacht die Nacht,\\
dem Wanderer zur Nacht ein Licht.\\!

\frot{3.} Herr, wenn wir fallen, sieh uns an\\
und heile uns durch deinen Blick.\\
Dein Blick löscht Fehl und Sünde aus,\\
in Tränen löst sich unsre Schuld.\\!

\frot{4.} Du Licht durchdringe\\
unsern Geist, von unsern Herzen\\
scheuch den Schlaf,\\
dir sei das erste Wort geweiht,\\
dich preise unser Morgenlob.\\!

\frot{5.} Christus, gütigster König, es sei\\
dir und dem Vater Lobpreis,\\
mit dem Heiligen Geiste\\
in alle Ewigkeit. \frot{A}men.\\!}}


\end{verse} 
\end{multicols}
}

\smallskip


\setspaceafterinitial{5.2mm plus 0em minus 0em}
\setspacebeforeinitial{4.2mm plus 0em minus 0em}
\def\greinitialformat#1{{\fontsize{40}{40}\selectfont #1}}
\gresetfirstlineaboveinitial{\small \textcolor{red}{æstate}}{}
\setaboveinitialseparation{0.72mm}
\setsecondannotation{\small viij. T.}
\includescore{hymni/noctesurgentes.tex}

\medskip


\begin{sloppypar}
{\noindent\rm{\frot{1.} Uns in der Nacht erhebend, wachen wir alle; 
wir meditieren beständig Psalmen und singen mit allen Kräften dem Herrn liebliche Hymnen;}}
\end{sloppypar}

\newpage

\begin{quote}

\begin{multicols}{2}
\begin{verse}[\versewidth]


{\small{\frot{U}t, pio regi\\
páriter canéntes,\\ 
cum suis sanctis\\
mereámur aulam\\
íngredi cæli,\\
simul et beátam\\
dúcere vitam.\\
{\textcolor{white}x}\\!
	
\frot{P}ræstet hoc nobis\\
Déitas beáta\\
Patris ac Nati,\\
paritérque Sancti\\
Spíritus, cuius\\
résonat per omnem\\
glória mundum.\\
Amen.\\!}}

\end{verse}

\columnbreak

\begin{verse}[\versewidth]

{\small\rm{\frot{2.} Damit vom gütigen\\
König die gemeinsam \\
Singenden für würdig\\
befunden werden, mit\\
seinen Heiligen in den\\
Palast der Himmel\\
einzutreten und ein\\
seliges Leben zu führen.\\!

\frot{3.} Dies gewähre uns\\
die selige Gottheit,\\ 
des Vaters und des\\
Eingeborenen\\
zusammen mit\\
dem Heiligen Geiste,\\
dessen Ruhm in aller\\
Welt erschallt.\\
Amen.\\}}

\end{verse}
\end{multicols}

\end{quote}

\section{VIGILIA I}

\begin{sloppypar}

{\noindent{1. Ant.} O Herr, * an deiner Macht freut sich der König. \rot{Ps. 21}\\
2. Ant. Sie teilten unter sich meine Kleider * und warfen das Los um mein Gewand. \rot{Ps. 22}\\
3. Ant. Der Herr ist mein Hirte, * nichts wird mir fehlen. \rot{Ps. 23}\\}
\end{sloppypar}

\medskip

\begin{flushleft}

\versik{In der Nacht denke ich an deinen Namen, Herr.}{Und bewahre dein Gesetz.}

\end{flushleft}

\newpage

\section{VIGILIA II}

\begin{sloppypar}
{\noindent{1. Ant.} Meine Seele dürstet nach dem lebendigen Gott. * Wann darf ich kommen und dein Antlitz schau\-en? \rot{Ps. 42}\\
2. Ant. Du hast uns gerettet, Herr, * und wir preisen deinen Namen in Ewigkeit. \rot{Ps. 44}\\
3. Ant. Mein Herz * fließt über von froher Kunde. \rot{Ps. 45}}
\end{sloppypar}

\begin{flushleft}

\versik{Erhebe dich, Herr, in deiner Macht.}{Wir wollen deine Machttaten besingen.}

\end{flushleft}

\section{VIGILIA III}

\begin{sloppypar}
{\noindent{1. Ant.} Sucht den Herrn, * und eure Seele wird leben.
       \indent{\rot{Ps. 69, 2-20} - Divisio -  \rot{Ps. 69, 21-37}}\\         
2. Ant. Herr, mein Gott, * komm mir zu Hilfe. \rot{Ps.70}}
\end{sloppypar}

\begin{flushleft}

\versik{Führe mich, Herr, in deiner Wahrheit und lehre mich.}{Denn du bist der Gott meines Heiles.}

\end{flushleft}

\section{VIGILIA IV}

\begin{sloppypar}
{\noindent{1. Ant.} Neige dein Ohr, o Herr, * und erhöre mich. \rot{Ps. 86}\\
2. Ant. Herrliches sagt man von dir, * Stadt unseres Gottes. \rot{Ps. 87}\\
3. Ant. Gepriesen sei der Herr * in Ewigkeit.\\ \rot{Ps 89, 2-19}\\}
\end{sloppypar}

\begin{flushleft}

\versik{Selig der Mann, den du, Herr, erziehst.}{Und den du deine Gesetze lehrst.}

\end{flushleft}

\section[CANTICA]{CANTICA}

\crot{Per Annum.}

\vspace{0.2cm}

\setspaceafterinitial{5.2mm plus 0em minus 0em}
\setspacebeforeinitial{4.2mm plus 0em minus 0em}
\def\greinitialformat#1{{\fontsize{40}{40}\selectfont #1}}
\gresetfirstlineaboveinitial{\small \textcolor{red}{Eccl 43}}{}
\setaboveinitialseparation{0.72mm}
\setsecondannotation{\small viij. T.}

\includescore{iiinocalleluia.tex}

\cantvig{Eccl 43, 28-30}

\begin{quote}
\begin{verse}
 
\vin Wir können (ihn) nur loben, †\\
\vin aber nie erfassen, *\\
\vin ist er doch größer als alle seine Werke.\\
Überaus ehrfurchtgebietend ist der Herr,*\\
unbegreiflich ist seine Stärke.\\
\vin Ihr, die ihr den Herrn lobt, singt laut, †\\
\vin soviel ihr könnt; *\\
\vin denn nie wird es genügen.\\
Ihr, die ihr ihn preist, schöpft neue Kraft, †\\
werdet nicht müde;*\\
denn fassen könnt ihr es nie. \\

\end{verse}
\end{quote}

\begin{flushleft}

\versik{In der Mitte der Nacht stehe ich auf, um dich zu preisen.}{Wegen deiner gerechten Entscheide.}

\end{flushleft}

\newpage

\crot{Tempore Quadragesimæ.}

\vspace{0.2cm}


\setspaceafterinitial{5.2mm plus 0em minus 0em}
\setspacebeforeinitial{4.2mm plus 0em minus 0em}
\def\greinitialformat#1{{\fontsize{40}{40}\selectfont #1}}
\gresetfirstlineaboveinitial{\small \textcolor{red}{Is 1}}{}
\setaboveinitialseparation{0.72mm}
\setsecondannotation{\small v. T.}

\includescore{noninsolo.tex}

\medskip

\begin{sloppypar}
{\noindent\rm{Ant. Nicht vom Brot allein * lebt der Mensch, sondern von jedem Wort, das aus dem Munde Gottes kommt.}}
\end{sloppypar}

\medskip

\cantvig{Is 1,18-20}

\begin{quote}
\begin{verse}
„Kommt her, wir wollen sehen, † \\
wer von uns recht hat“,*\\
spricht der Herr.\\
\vin „Wären eure Sünden auch rot\\
\vin wie Scharlach,*\\
\vin sie sollen weiß werden wie Schnee.\\
Wären sie rot wie Purpur,*\\
sie sollen weiß werden wie Wolle.\\
\vin Wenn ihr bereit seid zu hören,*\\
\vin sollt ihr den Ertrag des Landes genießen.\\
Wenn ihr aber trotzig seid und euch weigert, †\\
werdet ihr vom Schwert gefressen.“*\\
Ja, der Mund des Herrn hat gesprochen.
 
\end{verse}
\end{quote}

\begin{flushleft}

\versik{Kehrt um zum Herrn, eurem Gott.}{Denn er ist gnädig und barmherzig.}

\end{flushleft}


\section[LAUDES]{AD LAUDES}

\setspaceafterinitial{5.2mm plus 0em minus 0em}
\setspacebeforeinitial{4.2mm plus 0em minus 0em}
\def\greinitialformat#1{{\fontsize{40}{40}\selectfont #1}}
\gresetfirstlineaboveinitial{\small \textcolor{red}{ Ps 118a}}{}
\setaboveinitialseparation{0.72mm}
\setsecondannotation{\small 3. T.}

\includescore{psalmi/117/confitemini117.tex}

\medskip

\psal{118a}

\begin{quote}
\begin{verse}
 

\textit{Danket dem Herrn,} denn \d er ist gütig, *\\
\textit{denn seine Huld währt ewig.}\\
\vin So soll \d Israel sagen: *\\
\vin Denn seine Huld währt ewig.\\
So soll das Haus \d Aaron sagen: *\\
Denn seine Huld währt ewig.\\
\vin So sollen alle sagen,\\
\vin die den Herrn f\d ürchten und ehren: *\\
\vin Denn seine Huld währt ewig.\\
In der Bedrängnis r\d ief ich zum Herrn; *\\ 
der Herr hat mich erhört und mich frei gemacht.\\
\vin Der Herr ist bei mir, †\\
\vin ich f\d ürchte mich nicht. *\\
\vin Was können Menschen mir antun?\\
Der Herr ist bei mir, \d er ist mein Helfer; *\\
ich aber schaue auf meine Hasser herab.\\
\vin Besser, sich zu b\d ergen beim Herrn, *\\
\vin als auf Menschen zu bauen.\\
Besser, sich zu b\d ergen beim Herrn, *\\
als auf Fürsten zu bauen.\\
\vin Alle V\d ölker umringen mich; *\\
\vin ich wehre sie ab im Namen des Herrn.\\
Sie umringen, j\d a, sie umringen mich; *\\
ich wehre sie ab im Namen des Herrn.\\
\vin Sie umschwirren mich wie Bienen, †\\
\vin wie ein Str\d ohfeuer verlöschen sie; *\\
\vin ich wehre sie ab im Namen des Herrn.\\
Sie stießen mich hart, †\\
sie w\d ollten mich stürzen; *\\
der Herr aber hat mir geholfen.\\
\vin Meine Stärke und mein L\d ied ist der Herr; *\\
\vin er ist für mich zum Retter geworden.\\

\end{verse}
\end{quote}

\medskip

\setspaceafterinitial{5.2mm plus 0em minus 0em}
\setspacebeforeinitial{4.2mm plus 0em minus 0em}
\def\greinitialformat#1{{\fontsize{40}{40}\selectfont #1}}
\gresetfirstlineaboveinitial{\small \textcolor{red}{ Ps 118b}}{}
\setaboveinitialseparation{0.72mm}
\setsecondannotation{\small viij. T.}

\includescore{psalmi/117/dexteradomini.tex}

\medskip



\psal{118b}

\begin{quote}
\begin{verse}

Frohlocken und Jubel erschallt *\\
in den Zelten der Gerechten:\\
\vin ``\textit{Die Rechte des Herrn wirkt mit Macht! †\\
\vin Die Rechte des Herrn ist erhoben}, *\\
\vin die Rechte des Herrn wirkt mit Macht!''\\
Ich werde nicht sterben, sondern leben, *\\
um die Taten des Herrn zu verkünden.\\
\vin Der Herr hat mich hart gezüchtigt, *\\
\vin doch er hat mich nicht dem Tod übergeben.\\
Öffnet mir die Tore zur Gerechtigkeit, *\\
damit ich eintrete, um dem Herrn zu danken.\\
\vin Das ist das Tor zum Herrn, *\\
\vin nur Gerechte treten hier ein.\\
Ich danke dir, daß du mich erhört hast; *\\
du bist für mich zum Retter geworden.\\
\vin Der Stein, den die Bauleute verwarfen, *\\
\vin er ist zum Eckstein geworden.\\
Das hat der Herr vollbracht, *\\
vor unseren Augen geschah dieses Wunder.\\
\vin Dies ist der Tag, den der Herr gemacht hat; *\\
\vin wir wollen jubeln und uns an ihm freuen.\\
Ach, Herr, bring doch Hilfe! *\\
Ach, Herr, gib doch Gelingen!\\
\vin Gesegnet sei er,\\
\vin der kommt im Namen des Herrn. †\\ 
\vin Wir segnen euch, vom Haus des Herrn her. *\\
\vin Gott, der Herr, erleuchte uns.\\
Mit Zweigen in den Händen\\
schließt euch zusammen zum Reigen, *\\
bis zu den Hörnern des Altars!\\
\vin Du bist mein Gott, dir will ich danken; *\\
\vin mein Gott, dich will ich rühmen.\\
Danket dem Herrn, denn er ist gütig, *\\
denn seine Huld währt ewig.\\

\end{verse}
\end{quote}

\medskip

\setspaceafterinitial{5.2mm plus 0em minus 0em}
\setspacebeforeinitial{4.2mm plus 0em minus 0em}
\def\greinitialformat#1{{\fontsize{40}{40}\selectfont #1}}
\gresetfirstlineaboveinitial{\textcolor{red}{Dan 3}}{}
\setaboveinitialseparation{0.72mm}
\setsecondannotation{\small viij. T.}
\includescore{cantica/ca/dan3,52.tex}
\smallskip

\begin{sloppypar}
{\noindent\rm{Ant. Lasst uns Hymnen singen * dem Herrn, unserem Gott, halleluja.}}
\end{sloppypar}

\medskip



\cant{Dan 3,57-88}

\begin{quote}
\begin{verse}

Preist den Herrn, all ihr Werke des Herrn; *\\
lobt und rühmt ihn in Ewigkeit!\\
\vin Preist den Herrn, ihr Himmel; *\\
\vin preist den Herrn, ihr Engel des Herr\textit{e}n!\\
All ihr Wasser über dem Himmel,\\
preiset den Herrn; *\\
all ihr Mächte des Herrn, preiset den Herr\textit{e}n!\\
\vin Preist den Herrn, Sonne und Mond; *\\
\vin preist den Herrn, ihr Sterne am Himmel!\\
Preist den Herrn, aller Regen und Tau; *\\
preist den Herrn, all ihr Winde!\\
\vin Preist den Herrn, Feuer und Glut; *\\
\vin preist den Herrn, Frost und Hitze!\\
Preist den Herrn, Tau und Schnee; *\\
preist den Herrn, Eis und Kälte!\\
\vin Preist den Herrn, ihr Nächte und Tage; *\\
\vin preist den Herrn, Licht und Dunkel!\\
Preist den Herrn, Rauhreif und Schnee; *\\
Preist den Herrn, ihr Blitze und Wolken!\\
\vin Die Erde preise den Herrn; *\\
\vin sie lobe und rühme ihn in Ewigkeit!\\
Preist den Herrn, ihr Berge und Hügel; *\\
preist den Herrn, all ihr Gewächse auf Erden!\\
\vin Preist den Herrn, ihr Meere und Flüsse; *\\
\vin preist den Herrn, ihr Quellen!\\
Preist den Herrn, ihr Tiere des Meeres †\\
und alles, was sich regt im Wasser; *\\
preist den Herrn, all ihr Vögel am Himmel!\\
\vin Preist den Herrn all ihr Tiere,\\
\vin wilde und zahme; *\\
\vin preist den Herrn, ihr Menschen!\\
Preist den Herrn, ihr Israeliten; *\\
lobt und rühmt ihn in Ewigkeit!\\
\vin Preist den Herrn, ihr seine Priester; *\\
\vin preist den Herrn, ihr seine Knechte!\\
Ihr Geister und Seelen der Gerechten,\\
preiset den Herrn; *\\
ihr Demütigen und Frommen, preiset\\ den Herr\textit{e}n!\\
\vin Preist den Herrn, Hananja,\\
\vin Asarja und Mischaël; *\\
\vin lobt und rühmt ihn in Ewigkeit!\\
Lasst uns preisen den Vater und den Sohn\\
mit dem Heiligen Geist, *\\
ihn loben und rühmen in Ewigkeit.\\ 

\end{verse}
\end{quote}

\medskip

{\frot{vel}}


\medskip

\setspaceafterinitial{5.5mm plus 0em minus 0em}
\setspacebeforeinitial{4.2mm plus 0em minus 0em}
\def\greinitialformat#1{{\fontsize{40}{40}\selectfont #1}}
\gresetfirstlineaboveinitial{\textcolor{red}{Dan 3}}{}
\setaboveinitialseparation{0.72mm}
\setsecondannotation{\small 1. T.}
\includescore{cantica/ca/dan3.tex}

\smallskip

\cant{Dan 3,52-57}


\begin{quote}
\begin{verse}
Gepriesen bist du, Herr, du\\
Gott unserer Väter, *\\
gelobt und gerühmt in Ewigkeit.\\
\vin Gepriesen ist dein heiliger, herrlicher Name,*\\
\vin hoch gelobt und verherrlicht in Ewigkeit.\\
Gepriesen bist du im Tempel deiner \\
heiligen Herrlichkeit, *\\
hoch gerühmt und verherrlicht in Ewigkeit.\\
\vin Gepriesen bist du, †\\
\vin der in die Tiefen schaut \\
\vin und auf Kerubim thront, *\\
\vin gelobt und gerühmt in Ewigkeit.\\
Gepriesen bist du \\
auf dem Thron deiner Herrschaft, *\\
hoch gerühmt und gefeiert in Ewigkeit.\\
\vin \textit{Gepriesen bist du\\ 
\vin am Gewölbe des Himmels, *\\
\vin gerühmt} und verherrlicht \textit{in Ewigkeit.}\\
Gepriesen ist der Vater und der Sohn\\
mit dem Heiligen Geiste, *\\
gelobt und hoch erhoben in Ewigkeit.\\
\vin Preiset den Herrn, all ihr Werke des Herrn; *\\
\vin lobt und erhebt ihn in Ewigkeit!\\

\end{verse}
\end{quote}

\vspace{0.6cm}

\setspaceafterinitial{5.2mm plus 0em minus 0em}
\setspacebeforeinitial{4.2mm plus 0em minus 0em}
\def\greinitialformat#1{{\fontsize{40}{40}\selectfont #1}}
\gresetfirstlineaboveinitial{\small \textcolor{red}{Ps 148}}{}
\setaboveinitialseparation{0.72mm}
\setsecondannotation{\small 1. T.}

% and finally we include the score. The file must be in the same directory as this one.
\includescore{psalmi/148/ps148.tex}


\psal{148}

\begin{quote}
\begin{verse}

\textit{Lobet den Herrn vom Himmel her}, *\\
lobt ihn in den Höhen:\\
\vin Lobt ihn, all seine Engel, *\\
\vin lobt ihn, all seine Scharen;\\
lobt ihn, Sonne und Mond, *\\
lobt ihn, all ihr leuchtenden Sterne;\\
\vin lobt ihn, alle Himmel *\\
\vin und ihr Wasser über dem Himmel!\\
Loben sollen sie den Namen des Herrn; *\\
denn er gebot, und sie waren erschaffen.\\
\vin Er stellte sie hin für immer und ewig, *\\
\vin er gab ihnen ein Gesetz,\\
\vin das sie nicht übertreten.\\
Lobt den Herrn, ihr auf der Erde, *\\
ihr Seeungeheuer und all ihr Tiefen,\\
\vin Feuer und Hagel, Schnee und Nebel, *\\
\vin du Sturmwind, der sein Wort vollzieht,\\
ihr Berge und all ihr Hügel, *\\
ihr Fruchtbäume und alle Zedern,\\
\vin ihr wilden Tiere und alles Vieh, *\\
\vin Kriechtiere und gefiederte Vögel,\\
ihr Könige der Erde und alle Völker, *\\
ihr Fürsten und alle Richter auf Erden,\\
\vin ihr jungen Männer und auch ihr Mädchen,*\\
\vin ihr Alten mit den Jungen!\\
Loben sollen sie den Namen des Herrn; †\\
denn sein Name allein ist erhaben, *\\
seine Hoheit strahlt über Erde und Himmel.\\
\vin Seinem Volk verleiht er Macht, †\\
\vin das ist ein Ruhm für all seine Frommen, *\\
\vin für Israels Kinder, das Volk,\\ 
\vin das ihm nahen darf.\\!
\end{verse}
\end{quote}

\medskip

\setspaceafterinitial{4.2mm plus 0em minus 0em}
\setspacebeforeinitial{4.2mm plus 0em minus 0em}
\resp

\includescore{responsoria_diebusferialibus/respbrhaecestdies.tex}

\medskip

\begin{sloppypar}
{\noindent\rm{\rot{Resp.} Dies ist der Tag, den der Herr gemacht hat. Lasst uns jubeln und in ihm uns freuen.}}
\end{sloppypar}

\newpage

\crot{Hymnus}

\setspaceafterinitial{5.2mm plus 0em minus 0em}
\setspacebeforeinitial{4.2mm plus 0em minus 0em}
\def\greinitialformat#1{{\fontsize{40}{40}\selectfont #1}}
\gresetfirstlineaboveinitial{\small \textcolor{red}{hieme}}{}
\setaboveinitialseparation{0.72mm}
\setsecondannotation{\small ij. T.}

\includescore{hymni/splendorpaternae.tex}

\medskip

\begin{sloppypar}
{\noindent\rm{\rot{1.} Du Abglanz von des Vaters Pracht,
du bringst aus Licht das Licht hervor,
du Licht vom Licht, des Lichtes Quell, du Tag, der unsern Tag erhellt.}}
\end{sloppypar}

\medskip

\begin{multicols}{2}
\begin{verse}[\versewidth]

{\small{\frot{V}erúsque sol, illábere\\
micans nitóre pérpeti, \\
iubárque Sancti Spíritus\\
infúnde nostris sénsibus.\\!

\frot{C}onfírmet actus strénuos, \\
dentes retúndat ínvidi,\\
casus secúndet ásperos, \\
donet geréndi grátiam.\\!

\frot{M}entem gubérnet et regat\\
casto, fidéli córpore,\\
fides calóre férveat, \\
fraudis venéna nésciat.\\!}}
\end{verse}

\columnbreak

\begin{verse}[\versewidth]

{\small\rm{\frot{2.} Du wahre Sonne, brich herein,\\
du Sonne, die nicht untergeht,\\
und mit des Geistes lichtem Strahl\\
dring tief in unsrer Sinne Grund.\\!

\frot{3.} Er stärke uns zu gutem Werk,\\
er leite machtvoll unser Tun,\\
er sei uns Kraft in harter Fron\\
und lenke unsern schwachen Geist.\\!

\frot{4.} Des Geistes Steuer führ’ als Herr\\
er in den keuschen, treuen Leib.\\
Der Glaube glüh’ in Liebe auf\\
und wisse nicht vom Gift des Trugs.\\!}}

\end{verse}
\end{multicols}

\newpage

\begin{multicols}{2}
\begin{verse}[\versewidth]

{\small{\frot{C}hristúsque nobis sit cibus, \\
potúsque noster sit fides;\\
læti bibámus sóbriam\\
ebrietátem Spíritus.\\!

\frot{D}eo Patri sit glória,\\
Ejúsque soli Fílio,\\
Cum Spíritu Paraćlito\\
Et nunc, et in perpétuum.\\
Amen.\\!}}

\end{verse}

\columnbreak

\begin{verse}[\versewidth]

{\small\rm{\frot{5.} Und Christus werde unser Brot,\\
und unser Glaube sei uns Trank,\\
in Freude werde uns zuteil\\
des Geistes klare Trunkenheit.\\!

\frot{6}. Gott, dem Vater sei Ehre\\
und seinem einzigen Sohn\\
mit dem Heiligen Geist\\
jetzt und in Ewigkeit.\\
Amen.\\!}}
\end{verse}

\end{multicols}

\medskip

\setspaceafterinitial{5.2mm plus 0em minus 0em}
\setspacebeforeinitial{4.2mm plus 0em minus 0em}
\def\greinitialformat#1{{\fontsize{40}{40}\selectfont #1}}
\gresetfirstlineaboveinitial{\small \textcolor{red}{æstate}}{}
\setaboveinitialseparation{0.72mm}
\setsecondannotation{\small viij. T.}
\includescore{hymni/ecceiamnoctis.tex}

\medskip

\begin{sloppypar}
{\noindent\rm{\frot{1.} Seht, wie die Schatten
dunkler Nacht verblassen: Früh\-lichtes Leuchten
rosenfarben glühet; Lasst uns aus allen Kräften
innig bitten Gott, den Allmächt’gen.}}
\end{sloppypar}

\begin{quote}

\begin{multicols}{2}
\begin{verse}[\versewidth]

{\normalsize{\frot{U}t Deus, nostri\\
miserátus, omnem\\
pellat languórem,\\
tríbuat salútem, \\
donet et nobis\\
pietáte patris\\
regna polórum.\\!}}

\end{verse}

\columnbreak

\begin{verse}[\versewidth]
{\normalsize\rm{\frot{2.} Dass er sich unser\\
liebevoll erbarme,\\
Schlaffheit vertreibe,\\
uns mit Heil begnade,\\
dass er uns schenke\\
einst voll Vatergüte\\
himmlische Reiche.\\!}}
\end{verse}

\end{multicols}
\end{quote}

\newpage

\begin{quote}
\begin{multicols}{2}
\begin{verse}[\versewidth]

{\normalsize{\frot{P}ræstet hoc nobis\\
Déitas beáta\\
Patris ac Nati,\\
paritérque Sancti\\
Spíritus, cuius\\
résonat per omnem\\
glória mundum.\\
Amen.\\!}}
\end{verse}

\columnbreak

\begin{verse}[\versewidth]
{\normalsize\rm{\frot{3.} Dies wolle uns der\\ 
hehre Gott verleihen,\\
Vater und Sohn samt\\
dem Heiligen Geiste,\\
deren Lob weithin\\
widerhallt durch alle\\ 
Zonen der Welten.\\
Amen.\\!}}

\end{verse}
\end{multicols}

\end{quote}

\begin{flushleft}

\versik{Dóminus regnávit, decórem índuit.}{Induit Dóminus fortitúdinem, et præcínxit se virtúte.}

\medskip
{\rm{
\versik{Der Herr ist König geworden, mit Pracht bekleidet.}{Der Herr ist mit Stärke bekleidet und hat sich mit Tugend umgürtet.}
}}
\end{flushleft}

\medskip

\cantben{Lc 1,68-79}



\rot{Antiphona de Dominica currente.}

\rot{Canticum} Benedictus \rot{p. 196.}

\medskip

\section[HORA TERTIA]{AD TERTIAM}

\rot{Hymnus} Nunc sancte nobis Spiritus \rot{p. 190.}

\medskip



\setspaceafterinitial{5.2mm plus 0em minus 0em}
\setspacebeforeinitial{4.2mm plus 0em minus 0em}
\def\greinitialformat#1{{\fontsize{40}{40}\selectfont #1}}
\gresetfirstlineaboveinitial{\small \textcolor{red}{Ps 119a}}{}
\setaboveinitialseparation{0.72mm}
\setsecondannotation{\small ij. T.}


\includescore{alleluia/alleluia_tertia.tex}
\medskip
\includescore{psalmi/tertia/terzpsalterdom.tex}

\begin{flushleft}

\versik{Inclína cor meum, Deus, in testimónia tua.}{In via tua vivífica me.}

\medskip
{\rm{
\versik{Neige mein Herz, Herr, zu deinen Geboten.}{Auf deinen Wegen belebe mich.}
}}
\end{flushleft}


\section[HORA SEXTA]{AD SEXTAM}

\rot{Hymnus} Rector potens, verax Deus \rot{p. 192.}
\vspace{0.3cm}

\setspaceafterinitial{5.2mm plus 0em minus 0em}
\setspacebeforeinitial{4.2mm plus 0em minus 0em}
\def\greinitialformat#1{{\fontsize{40}{40}\selectfont #1}}
\gresetfirstlineaboveinitial{\small \textcolor{red}{ Ps 1 2 6 }}{}
\setaboveinitialseparation{0.72mm}
\setsecondannotation{\small viij. T.}
\includescore{alleluia/alleluia_sexta}
\medskip

\newpage
\includescore{psalmi/sexta/sextpsalmidom.tex}

\medskip

\begin{flushleft}

\versik{In ætérnum, Dómine, verbum tuum.}{In generatiónem et generatiónem véritas tua.}

\medskip

{\rm{
\versik{In Ewigkeit, Herr, bleibt dein Wort.}{Von Geschlecht zu Geschlecht deine Treue.}
}}
\end{flushleft}


\newpage

\section[HORA NONA]{AD NONAM}

\rot{Hymnus} Rerum Deus tenax vigor \rot{p. 194.}
\vspace{0.3cm}

 \setspaceafterinitial{5.2mm plus 0em minus 0em}
\setspacebeforeinitial{4.2mm plus 0em minus 0em}
\def\greinitialformat#1{{\fontsize{40}{40}\selectfont #1}}
\gresetfirstlineaboveinitial{\small \textcolor{red}{ Ps 120sq.}}{}
\setaboveinitialseparation{0.72mm}
\setsecondannotation{\small viij. T.}

\includescore{alleluia/alleluia_nona}

\medskip

\includescore{psalmi/nona/nonpsalmidom.tex}

\begin{flushleft}

\versik{Clamávi in toto corde meo, exáudi me, Dómine.}{Iustificatiónes tuas servábo.}

\medskip

{\rm{
\versik{Aus meinem ganzem Herzen habe ich gerufen, erhöre mich, Herr.}{Deiner Gerechtigkeit will ich dienen.}
}}
\end{flushleft}


\vspace{2cm}

\section[VESPERAE]{AD VESPERAS}


 \setspaceafterinitial{5.2mm plus 0em minus 0em}
\setspacebeforeinitial{4.2mm plus 0em minus 0em}
\def\greinitialformat#1{{\fontsize{40}{40}\selectfont #1}}
\gresetfirstlineaboveinitial{\small \textcolor{red}{Ps 110}}{}
\setaboveinitialseparation{0.72mm}
\setsecondannotation{\small 7. T.}

\includescore{psalmi/109/dixitdominus.tex}

\medskip

\psal{110}

\begin{quote}
\begin{verse}

\textit{So spricht der Herr zu meinem Herrn: †\\
Setze dich m\d ir zur Rechten,} *\\
und ich lege dir deine Feinde\\
als Schemel unter die Füße.\\
\vin Vom Zion strecke der Herr \\
\vin das Zepter d\d einer Macht aus: *\\
\vin ``Herrsche inmitten d\d einer Feinde!''\\
Dein ist die Herrschaft am T\d age deiner Macht, *\\
wenn du erscheinst in h\d eiligem Schmuck;\\
\vin ich habe dich gezeugt\\ 
\vin noch v\d or dem Morgenstern. *\\
\vin wie den T\d au in der Frühe.\\
Der Herr hat geschworen,\\
und n\d ie wird’s ihn reuen: *\\
``Du bist Priester auf ewig nach der \d Ordnung \\
Melchisedeks.''\\
\vin Der Herr steht d\d ir zur Seite; *\\
\vin er zerschmettert Könige\\
\vin am Tage s\d eines Zornes.\\
Er hält Gericht unter den Völkern,\\
er h\d äuft die Toten, *\\
die Häupter zerschmettert er w\d eithin auf Erden.\\
Er trinkt aus dem B\d ach am Weg\textit{e}; *\\
so kann er von neuem das H\d aupt erheben.\\!
\end{verse}
\end{quote}

\vspace{1cm}


\setspaceafterinitial{5.2mm plus 0em minus 0em}
\setspacebeforeinitial{4.2mm plus 0em minus 0em}
\def\greinitialformat#1{{\fontsize{40}{40}\selectfont #1}}
\gresetfirstlineaboveinitial{\small \textcolor{red}{Ps 111}}{}
\setaboveinitialseparation{0.72mm}
\setsecondannotation{\small iv. T.}

\includescore{psalmi/110/fidelia.tex}

\medskip

\psal{111}

\begin{quote}
\begin{verse}

Den Herrn will ich preisen von ganzem Herzen *\\
im Kreis der Frommen, inmitt\d en der Gemeinde.\\
\vin Groß sind die Werke des Herrn, *\\
\vin kostbar allen, die sich \d an ihnen freuen.\\
Er waltet in Hoheit und Pracht, *\\
seine Gerechtigkeit hat B\d estand für immer.\\
\vin Er hat ein Gedächtnis an seine Wunder\\ 
\vin gestiftet, *\\
\vin der Herr ist gnäd\d ig und barmherzig.\\
Er gibt denen Speise, die ihn fürchten, *\\
an seinen Bund d\d enkt er auf ewig.\\
\vin Er hat seinem Volk\\
\vin seine machtvollen Taten kundgetan, *\\
\vin um ihm das Erbe der V\d ölker zu geben. \\
Die Werke seiner Hände\\
sind gerecht und beständig, *\\
\textit{all seine Gebot\d e sind verlässlich.\\
\vin Sie stehen fest für immer und ewig}, *\\
\vin geschaffen in Tr\d eue und Redlichkeit.\\
Er gewährte seinem Volk Erlösung †\\
und bestimmte seinen Bund für ewige Zeiten. *\\
Furchtgebietend ist sein N\d ame und heilig. \\
\vin Die Furcht des Herrn\\ 
\vin ist der Anfang der Weisheit; †\\
\vin alle, die danach leben, sind klug. *\\
\vin Sein Ruhm hat B\d estand für immer.\\!

\end{verse}
 
\end{quote}


\vspace{0.6cm}
\setspaceafterinitial{10.2mm plus 0em minus 0em}
\setspacebeforeinitial{4.2mm plus 0em minus 0em}
\def\greinitialformat#1{{\fontsize{40}{40}\selectfont #1}}
\gresetfirstlineaboveinitial{\small \textcolor{red}{Ps 112}}{}
\setaboveinitialseparation{0.72mm}
\setsecondannotation{\small 4. T.}

\includescore{psalmi/111/inmandatiseius.tex}

\medskip

\psal{112}

\begin{quote}
\begin{verse}
Wohl dem Mann, \textit{der} den Herrn\\
fürchtet und ehrt *\\
und \textit{sich herzlich freut an s\d einen Geboten.}\\
\vin Seine Nachkommen werden\\ 
\vin mächtig im Land, *\\
\vin das Geschlecht der Redlich\d en wird gesegnet.\\
Wohlstand und Reichtum füllen sein Haus, *\\
sein Heil hat B\d estand für immer.\\
\vin Den Redlichen erstrahlt\\
\vin im Finstern ein Licht: *\\
\vin Der Gnädige, Barmherzig\d e und Gerechte.\\
Wohl dem Mann, der gütig\\
und zum Helfen bereit ist, *\\
der das Seine ordn\d et, wie es recht ist.\\
\vin Niemals gerät er ins Wanken; *\\
\vin ewig denkt man \d an den Gerechten.\\
Er fürchtet sich nicht vor Verleumdung; *\\
sein Herz ist fest, er vertr\d aut auf den Herr\textit{e}n.\\
\vin Sein Herz ist getrost, er fürchtet sich nie; *\\
\vin denn bald wird er herabschauen\\
\vin auf s\d eine Bedränger.\\
Reichlich gibt er den Armen, †\\
sein Heil hat Bestand für immer; *\\
er ist m\d ächtig und hoch geehrt.\\
\vin Voll Verdruss sieht es der Frevler, †\\
\vin er knirscht mit den Zähnen und\\ 
\vin geht zugrunde. *\\
\vin Zunichte werden die W\d ünsche der Frevler.\\
\end{verse}

\end{quote}


\vspace{1cm}
\medskip

\setspaceafterinitial{5.2mm plus 0em minus 0em}
\setspacebeforeinitial{4.2mm plus 0em minus 0em}
\def\greinitialformat#1{{\fontsize{40}{40}\selectfont #1}}
\gresetfirstlineaboveinitial{\small \textcolor{red}{Apoc 19}}{}
\setaboveinitialseparation{0.72mm}
\setsecondannotation{\small v. T.}

\includescore{cantica/cn/fecitnosdeo.tex}

\medskip

\begin{sloppypar}
{\noindent\rm{Ant. Er hat uns vor Gott, seinem Vater, zu Königen gemacht, 
der Erstgeborene der Toten und Fürst über die Könige der Erde.}}
\end{sloppypar}

\bigskip

\cant{cf. Apoc 19,1-7}

\begin{quote}
\begin{verse}


Das Heil und die Herrlichkeit\\
und die Macht ist bei unser\textit{e}m Gott.*\\
Seine Urteile sind w\d ahr und gerecht.\\
\vin Preist unsern Gott, all seine Knechte*\\
\vin und alle, die ihn fürchten, Gr\d oße und Kleine!\\
Denn König geworden ist der Herr,\\
unser Gott, *\\
der Herrscher über die g\d anze Schöpfung.\\
\vin Wir wollen uns freuen und jubeln *\\
\vin und ihm die \d Ehre erweisen!\\
Denn gekommen ist die Hochzeit des Lammes, *\\
und seine Frau hat s\d ich bereit gemacht.

\end{verse}
\end{quote}

\medskip

\setspaceafterinitial{4.2mm plus 0em minus 0em}
\setspacebeforeinitial{4.2mm plus 0em minus 0em}
\resp

\includescore{responsoria_diebusferialibus/respbrquammagnifica.tex}

\medskip

\begin{sloppypar}
{\noindent\rm{\rot{Resp.} Wie groß sind deine Werke, Herr. Alles hast du in Weisheit geschaffen.}}
\end{sloppypar}

\newpage

\crot{Hymnus}

\setspaceafterinitial{5.2mm plus 0em minus 0em}
\setspacebeforeinitial{4.2mm plus 0em minus 0em}
\def\greinitialformat#1{{\fontsize{40}{40}\selectfont #1}}
\gresetfirstlineaboveinitial{\small \textcolor{red}{hieme}}{}
\setaboveinitialseparation{0.72mm}
\setsecondannotation{\small viij. T.}
\includescore{hymni/deuscreatoromnium.tex}

\medskip

\begin{sloppypar}
{\noindent\rm{\rot{1.} Gott, aller Dinge Schöpfer und Herrscher des Himmels,
du zierst den Tag mit glänzendem Licht, die Nacht mit der Gunst des Schlafens.}}
\end{sloppypar}

\medskip

{\setlength{\columnsep}{1cm}
\begin{multicols}{2} 
\begin{verse}[\versewidth]
 
{\small{
\frot{A}rtus solútos ut quies\\
Reddat labóris úsui;\\
Mentésque fessas állevet,\\
Luctúsque solvat ánxios.\\!

\frot{G}rates perácto iam die,\\
Et noctis exórtu, preces,\\ 
Voti reos ut ádiuves,\\ 
Hymnum canéntes,\\
sólvimus.\\!

\frot{T}e cordis ima cóncinant, \\
Te vox canóra cóncrepet; \\
Te díligat castus amor, \\
Te mens adóret sóbria. \\!}}
\end{verse}

\columnbreak

\begin{verse}[\versewidth]
 
{\small\rm{\frot{2.} Dass Ruhe den erschlafften\\
Gliedern, die Arbeitskraft erneu're,\\
den müden Geist erhebe,\\
und Klagen Ängste löse.\\!

\frot{3.} Hymnen singend, lösen wir ein\\
die Pflicht des Gelobten; bringen\\
nun dar den Dank für den Tag\\
und Gebete zum Anbruch der\\
Nacht, damit du uns hilfst.\\!

\frot{4.} Des Herzens Mitte preise dich,\\
für dich erschalle wohlklingender\\
Ton, dich liebe die reine Liebe,\\
dich verehr' der verständige Geist.\\!}}
\end{verse}
\end{multicols}
}

\newpage

{\setlength{\columnsep}{1cm}
\begin{multicols}{2} 
\begin{verse}[\versewidth]

{\small{\frot{U}t cum profúnda cláuserit\\
Diem calígo nóctium;\\ 
Fides tenébras nésciat,\\
Et nox fide\\
relúceat.\\!

\frot{C}hristum rogámus et Patrem,\\ 
Christi Patrísque Spíritum;\\
Unum potens per ómnia,\\ 
Fove precántes, Trínitas.\\
Amen.\\!}}
\end{verse}

\columnbreak

\begin{verse}[\versewidth]
 
{\small\rm{\frot{5.} Dass, wenn der Abgrund uns\\
umschließt, die Finsternis der\\
Nacht den Tag, der Glaub' von\\
Dunkelheit nichts weiß und\\
Zuversicht die Nacht erhellt.\\!

\frot{6.} Christus bitten wir und\\
den Vater, Christi und des\\
Vaters Geist; Eine Macht\\
über allem, Dreifaltigkeit,\\
umarme die Bittenden. Amen.\\!}}
 
 
\end{verse}

\end{multicols}
}

\medskip

\setspaceafterinitial{5.2mm plus 0em minus 0em}
\setspacebeforeinitial{4.2mm plus 0em minus 0em}
\def\greinitialformat#1{{\fontsize{40}{40}\selectfont #1}}
\gresetfirstlineaboveinitial{\small \textcolor{red}{æstate}}{}
\setaboveinitialseparation{0.72mm}
\setsecondannotation{\small viij. T.}
\includescore{hymni/luciscreator.tex}

\medskip

\begin{sloppypar}
{\noindent\rm{\rot{1.} Des Lichtes guter Schöpfer, du holst aus Licht den Tag hervor.
Mit dem Aufgang des neuen Lichtes, bereitest du der Welt ihren Beginn.}}
\end{sloppypar}

\medskip

{\setlength{\columnsep}{1cm}
\begin{multicols}{2} 
\begin{verse}[\versewidth]
 
{\small{
\frot{Q}ui mane iunctum vésperi\\
diem vocári præcipis:\\
tætrum chaos illábitur;\\
audi preces cum flétibus.\\!}}
\end{verse}

\columnbreak

\begin{verse}[\versewidth]
{\small\rm{\frot{2.} Der du Morgen und Abend\\
vereinst, und vorsiehst, ihn Tag zu\\
nennen: Das düstre Chaos zieht\\
herauf, hör' unser flehendes Bitten.\\!}}
\end{verse}
\end{multicols}
}

{\setlength{\columnsep}{1cm}
\begin{multicols}{2} 
\begin{verse}[\versewidth]
 
{\small{
\frot{N}e mens graváta crímine\\
vitæ sit exsul múnere,\\ 
dum nil perénne cógitat\\
seséque culpis ílligat.\\!

\frot{C}ælórum pulset íntimum,\\
vitále tollat præmium;\\
vitémus omne nóxium,\\ 
purgémus omne péssimum.\\!

\frot{P}ræsta, Pater piíssime,\\ 
Patríque compar Unice,\\ 
cum Spíritu Paráclito\\
regnans per omne sæculum.\\
Amen.\\!}}

\end{verse}

\columnbreak

\begin{verse}[\versewidth]
{\small\rm{\frot{3.} Dass nicht - belastet - der Geist \\
der Gunst des Lebens sei beraubt\\
wenn er nicht beständig nachsinnt\\
und sich in Schuld verstrickt.\\!

\frot{4.} Er tromm'le an des Himmels\\
Herz, trag' hinauf den ewigen\\
Lohn; Lasset uns jede Untat\\
meiden, sühnen alle Bosheit.\\!

\frot{5.} Dies gewähre, gütigster Vater,\\
und der dem Vater Gleiche, der \\
mit dem Heiligen Geist \\
ist herrschend in Ewigkeit.\\
Amen.\\!}}
\end{verse}
\end{multicols}
}

\medskip

\begin{flushleft}
 
\versik{Dirigátur, Dómine, orátio mea.}{Sicut incénsum in conspéctu tuo.}

\medskip
{\rm{
\versik{Mein Gebet möge sich zu dir lenken, Herr.}{Wie Weihrauch gelangt vor dein Angesicht.}
}}
\end{flushleft}

\medskip



\rot{Antiphona de Dominica currente.}

\rot{Canticum} Magnificat \rot{p. 200.}

\newpage









