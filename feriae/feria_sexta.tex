\thispagestyle{plain}

\kapklein{\rot{F}eria \rot{S}exta}
\kkap{FERIA SEXTA}





\section[vigiliae]{AD VIGILIAS}

\begin{sloppypar}
{\noindent{\rot{Invit.} Kommt herbei * jubelt dem Herrn.}}
\end{sloppypar}

\bigskip

\noindent{\rot{Hymnus} Æterne rerum Conditor \rot{vel} Nocte surgentes \rot{ut in Dominica pp. 1-3.}}

\section{VIGILIA I}



\begin{sloppypar}
{\noindent{1. Ant.}  Befiehl  dem Herrn * deinen Weg. \rot{Ps. 37, 1-20} - Divisio -  \rot{Ps. 37, 21-40}\\  
2. Ant. O Herr * züchtige mich nicht in deinem Zorn. \rot{Ps. 38}\\}
\end{sloppypar}

\begin{flushleft}

\versik{Freu dich am Herrn.}{Er erfüllt dir die Wünsche deines Herzens.}

\end{flushleft}



\section{VIGILIA II}



\begin{sloppypar}
{\noindent{1. Ant.} Sei du unsere Hilfe, Herr * in der Bedrängnis. \rot{Ps. 60} \\
2. Ant. Du gibst das Erbe * denen, die deinen Namen fürchten, O Herr. \rot{Ps. 61} \\
3. Ant. Bei Gott ist mein Heil * und meine Ehre. \rot{Ps. 62} \\}
\end{sloppypar}

\begin{flushleft}

\versik{Wie süß sind meinen Lippen deine Worte, o Herr.}{Süßer als Honig für meinen Mund.}

\end{flushleft}


\section{VIGILIA III}

\begin{sloppypar}
{\noindent{1. Ant.} Wecke auf deine Macht,  o Herr * und rette uns. \rot{Ps. 80}\\
2. Ant. Preist Gott *  unseren Helfer. \rot{Ps. 81}\\
3. Ant. Du hast dich auf den Thron gesetzt * Gott, du gerechter Richter. \rot{Ps. 82} \\}
\end{sloppypar}

\begin{flushleft}

\versik{Selig, die in deinem Hause wohnen.}{Sie loben dich in Ewigkeit.}

\end{flushleft}

\section{VIGILIA IV}



\begin{sloppypar}
{\noindent{1. Ant.}  Denk an uns, Herr, * aus Liebe zu deinem Volk.  \rot{Ps. 106,1-23} - Divisio -  \rot{Ps. 106, 24-48}\\
2. Ant. Reiß mich heraus * aus all meinen Nöten, o Herr. \rot{Ps. 107,1-22}}\\ 

\end{sloppypar}

\begin{flushleft}

\versik{Lehre mich Güte und Weisheit.}{Denn ich vertraue auf deine Gebote.}

\end{flushleft}



\section[LAUDES]{AD LAUDES}

\setspaceafterinitial{3.2mm plus 0em minus 0em}
\setspacebeforeinitial{4.2mm plus 0em minus 0em}
\def\greinitialformat#1{{\fontsize{40}{40}\selectfont #1}}
\gresetfirstlineaboveinitial{\small \textcolor{red}{ Ps 51 }}{}
\setaboveinitialseparation{0.72mm}
\setsecondannotation{\small vj. T.}

\includescore{psalmi/50/misereremei.tex}

\vspace{0.3cm}
\crot{Psalm 51}

\begin{quote}
 

\begin{verse}
 \textit{Gott, sei mir gnädig} nach deiner Huld, *\\
tilge meine Frevel nach deinem r\d eichen\\ Erbarmen! \\
\vin Wasch meine Schuld von mir ab *\\
\vin und mach mich rein v\d on meiner Sünde!\\
Denn ich erkenne meine bösen Taten, *\\
meine Sünde steht mir i\d mmer vor Augen. \\
\vin Gegen dich allein habe ich gesündigt, *\\
\vin ich habe g\d etan, was dir missfällt. \\
So behältst du recht mit deinem Urteil, *\\ rein stehst d\d u da als Richter. \\
\vin Denn ich bin in Schuld geboren; *\\
\vin in Sünde hat mich meine M\d utter empfangen. \\
Lauterer Sinn im Verborgenen gefällt dir, *\\
im Geheimen l\d ehrst du mich Weisheit.\\
\vin Entsündige mich mit Ysop, dann werde \\ \vin ich rein; *\\
\vin wasche mich, dann werde \d ich weißer\\ \vin als Schnee. \\
Sättige mich mit Entzücken und Freude! *\\
Jubeln sollen die Glieder, d\d ie du \\zerschlagen hast.\\
\vin Verbirg dein Gesicht vor meinen Sünden, *\\
\vin tilge \d all meine Frevel!\\
Erschaffe mir, Gott, ein reines Herz *\\
und gib mir einen neuen, b\d eständigen Geist! \\
\vin Verwirf mich nicht von deinem Angesicht *\\
\vin und nimm deinen heilig\d en Geist nicht von mir! \\
Mach mich wieder froh mit deinem Heil *\\
mit einem willigen G\d eist rüste mich aus!\\
\vin Dann lehre ich Abtrünnige deine Wege *\\
\vin und die Sünd\d er kehren um zu dir.\\
Befrei mich von Blutschuld, Herr,\\ du Gott meines Heiles, *\\
dann wird meine Zunge jubeln über d\d eine \\Gerechtigkeit.\\
\vin Herr, öffne mir die Lippen *\\
\vin und mein Mund wird dein\d en Ruhm \\ \vin verkünden.\\
Schlachtopfer willst du nicht, ich würde sie\\ dir geben; *\\
an Brandopfern hast d\d u kein Gefallen. \\
\vin Das Opfer, das Gott gefällt, ist \\ \vin ein zerknirschter Geist, *\\
\vin ein zerbrochenes und zerschlagenes\\ \vin  Herz wirst du, G\d ott, nicht verschmähen.\\ 
In deiner Huld tu Gutes an Zion; *\\
bau die Mauern Jer\d usalems wieder auf! \\
\vin Dann hast du Freude an rechten Opfern, †\\
\vin an Brandopfern und Ganzopfern, *\\ \vin dann opfert man Stiere \d auf deinem Altar. \\

\end{verse}
\end{quote}


\vspace{0.cm}
\setspaceafterinitial{3.2mm plus 0em minus 0em}
\setspacebeforeinitial{4.2mm plus 0em minus 0em}
\def\greinitialformat#1{{\fontsize{40}{40}\selectfont #1}}
\gresetfirstlineaboveinitial{\small \textcolor{red}{Ps 92}}{}
\setaboveinitialseparation{0.72mm}
\setsecondannotation{\small 8. T.}
\includescore{psalmi/91/bonumest.tex}

\vspace{0.3cm}


\crot{Psalm 92}
\begin{quote}
\begin{verse}
Wie \textit{schön ist es, dem Herrn zu danken,} *\\
deinem Namen, du Höchster, zu singen, \\
\vin am Morgen deine Huld zu verkünden *\\
\vin und in den Nächten deine Treue\\
zur zehnsaitigen Laute, zur Harfe, *\\
zum Klang der Zither.\\
\vin Denn du hast mich durch deine Taten\\ \vin froh gemacht; *\\
\vin Herr, ich will jubeln über die Werke\\ deiner Hände.\\
Wie groß sind deine Werke, o Herr, *\\
wie tief deine Gedanken! \\
\vin Ein Mensch ohne Einsicht erkennt \\ \vin das nicht, *\\
\vin ein Tor kann es nicht verstehen. \\
Wenn auch die Frevler gedeihen †\\
und alle, die Unrecht tun, wachsen, *\\ so nur, damit du sie für immer vernichtest.\\
\vin Herr, du bist der Höchste, *\\
\vin du bleibst auf ewig.\\
Doch deine Feinde, Herr, wahrhaftig,\\ deine Feinde vergehen; *\\
auseinander getrieben werden alle, die\\ Unrecht tun. \\
\vin Du machtest mich stark wie einen Stier, *\\
\vin du salbtest mich mit frischem Öl.\\
Mein Auge blickt herab auf meine Verfolger, †\\
auf alle, die sich gegen mich erheben; *\\ mein Ohr hört vom Geschick der Bösen. \\
\vin Der Gerechte gedeiht wie die Palme, *\\
\vin er wächst wie die Zedern des Libanon. \\
Gepflanzt im Haus des Herrn, *\\
gedeihen sie in den Vorhöfen unseres Gottes. \\
\vin Sie tragen Frucht noch im Alter *\\
\vin und bleiben voll Saft und Frische;\\
sie verkünden: Gerecht ist der Herr; *\\
mein Fels ist er, an ihm ist kein Unrecht. \\

\end{verse}
\end{quote}


\vspace{0.3cm}

\setspaceafterinitial{5.2mm plus 0em minus 0em}
\setspacebeforeinitial{4.2mm plus 0em minus 0em}
\def\greinitialformat#1{{\fontsize{40}{40}\selectfont #1}}
\gresetfirstlineaboveinitial{\small \textcolor{red}{Is 45}}{}
\setaboveinitialseparation{0.72mm}
\setsecondannotation{\small v. T.}
\includescore{cantica/ca/is45.tex}

\cant{Is 45,15-25}

\begin{quote}
\begin{verse}
Wahrhaftig, du bist ein verborgener Gott. *\\
Israels G\d ott ist der Retter.\\
\vin Schmach und Schande kommt über sie alle, *\\
\vin die Götzenschmiede ger\d aten in Schande.\\
Israel aber wird vom Herrn gerettet, *\\
wird für \d immer errettet. \\
\vin Über euch kommt keine Schande und\\ \vin  Schmach mehr *\\ 
\vin für \d immer und ewig.\\
Denn so spricht der Herr, der den\\ Himmel erschuf, *\\
Ich bin der Herr und sonst niemand. \\
\vin Ich bin der Herr, der die Wahrheit spricht *\\ \vin und der verk\d ündet, was recht ist. \\
Wer hölzerne Götzen umherträgt, hat\\ keine Erkenntnis, *\\ wer einen Gott anbetet, der n\d iemanden rettet. \\
\vin Es gibt keinen Gott außer mir; *\\
 \vin außer mir gibt es keinen gerechten\\ \vin  und r\d ettenden Gott.\\
Wendet euch mir zu und lasst euch erretten, †\\
ihr Menschen aus den fernsten Ländern\\ der Erde; *\\ denn ich bin G\d ott und sonst niemand.\\
\vin Ich habe bei mir selbst geschworen †\\
\vin und mein Mund hat die Wahrheit\\ \vin gesprochen, *\\ \vin es ist ein unwiderr\d ufliches Wort: \\
Vor mir wird jedes Knie sich beugen †\\ und jede Zunge w\d ird bei mir schwören: *\\
Nur beim Herrn gibt es R\d ettung und Schutz. \\
\vin Beschämt kommen alle zu ihm, die sich ihm\\ \vin widersetzten. †\\
\textit{Alle Nachkommen Israels bekommen\\ ihr Recht *\\
und erlangen R\d uhm durch den Herren.}\\



\end{verse}

\end{quote}

\vspace{0.3cm}

\rot{vel}

\setspaceafterinitial{5.2mm plus 0em minus 0em}
\setspacebeforeinitial{4.2mm plus 0em minus 0em}
\def\greinitialformat#1{{\fontsize{40}{40}\selectfont #1}}
\gresetfirstlineaboveinitial{\small \textcolor{red}{Hab 3}}{}
\setaboveinitialseparation{0.72mm}
\setsecondannotation{\small iv. T.}
\includescore{cantica/ca/hab3.tex}

\vspace{0.3cm}

\cant{Hab 3,2f.13.15f}

\begin{quote}
\begin{verse}
 

\textit{Herr, ich höre die Kunde}, *\\ 
ich sehe, Herr, was du fr\d üher getan hast\\ \vin  
Lass es in diesen Jahren wieder geschehen, †\\ \vin offenbare es in diesen Jahren! *\\ \vin  Auch wenn du zürnst, d\d enk an\\ \vin  dein Erbarmen!\\
Gott kommt von Teman her, *\\ 
der Heilige kommt vom G\d ebirge Paran\\ \vin  
Seine Hoheit überstrahlt den Himmel, *\\ \vin  sein Ruhm \d erfüllt die Erde\\  

Er leuchtet wie das Licht der Sonne, † \\ 
ein Kranz von Strahlen umgibt ihn, *\\  in ihnen v\d erbirgt sich seine Macht\\ \vin 
Du ziehst aus, um dein Volk zu retten, *\\ \vin 
um deinem Ges\d albten zu helfen\\  
Du bahnst mit deinen Rossen den Weg\\ durch das Meer, *\\ 
durch das gewaltig sch\d äumende Wasser\\ \vin  
Ich zitterte am ganzen Leib, als \\ \vin ich es hörte, *\\ \vin 
ich vernahm d\d en Lärm und ich schrie.\\  
Fäulnis befällt meine Glieder *\\  und es wank\d en meine Schritte\\ \vin  
Doch in Ruhe erwarte ich den Tag der Not, *\\ \vin  der dem Volk bevorsteht, das \d über \\ \vin uns herfällt\\ 
Zwar blüht der Feigenbaum nicht, *\\ 
an den Reben \d ist nichts zu ernten, \\ \vin
der Ölbaum bringt keinen Ertrag, *\\ \vin  die Kornfeld\d er tragen keine Frucht;\\
 im Pferch sind keine Schafe, *\\  im St\d all steht kein Rind mehr\\ \vin  
Dennoch will ich jubeln über den Herrn *\\ \vin 
und mich freuen über G\d ott, meinen Retter\\ 
Gott, der Herr, ist meine Kraft. †\\
Er macht meine Füße schnell wie die Füße \\der Hirsche *\\  und lässt mich schreit\d en auf den Höhen.\\  
\end{verse}

\end{quote}

\vspace{0.3cm}

\setspaceafterinitial{5.2mm plus 0em minus 0em}
\setspacebeforeinitial{4.2mm plus 0em minus 0em}
\def\greinitialformat#1{{\fontsize{40}{40}\selectfont #1}}
\gresetfirstlineaboveinitial{\small \textcolor{red}{Ps 149}}{}
\setaboveinitialseparation{0.72mm}
\setsecondannotation{\small 8. T.}


\includescore{psalmi/149/ps149_filiision.tex}

\vspace{0.3cm}

\crot{Psalm 149}

\begin{quote}

\begin{verse}
 Singt dem Herrn ein neues Lied! *\\
Sein Lob erschalle in der Gemeinde \\der Frommen. \\
\vin Israel soll sich über seinen Schöpfer freuen, *\\
\vin \textit{die Kinder Zions über ihren König jauchzen.} \\
Seinen Namen sollen sie loben beim\\ Reigentanz, *\\
ihm spielen auf Pauken und Harfen. \\
\vin Der Herr hat an seinem Volk Gefallen, *\\
\vin die Gebeugten krönt er mit Sieg. \\
In festlichem Glanz sollen die Frommen \\frohlocken, *\\
auf ihren Lagern jauchzen:\\
\vin Loblieder auf Gott in ihrem Mund, *\\
\vin ein zweischneidiges Schwert in der Hand, \\
um die Vergeltung zu vollziehn an \\den Völkern, *\\
an den Nationen das Strafgericht,\\
\vin um ihre Könige mit Fesseln zu binden, *\\
\vin ihre Fürsten mit eisernen Ketten,\\
um Gericht über sie zu halten, so\\ wie geschrieben steht. *\\
Herrlich ist das für all seine Frommen. \\
\end{verse}

\end{quote}


\noindent\rot{Resp.br.} Sana animam \rot{ut in Feria Secunda p. 43.}\\
\noindent\rot{Hymnus} Splendor paternæ \rot{vel} Ecce iam noctis \rot{ut in Dominica p. 15-17.}

\begin{flushleft}

\versik{Repléti sumus mane misericórdia tua.}{Exultávimus, et delectáti sumus.}

\medskip

{\rm{
\versik{Am Morgen sind wir erfüllt von deiner Huld.}{Wir sind (voll) Jubel und Frohsinn.}
}}
\end{flushleft}
\vspace{0.3cm}
\setspaceafterinitial{5.2mm plus 0em minus 0em}
\setspacebeforeinitial{4.2mm plus 0em minus 0em}
\def\greinitialformat#1{{\fontsize{40}{40}\selectfont #1}}
\gresetfirstlineaboveinitial{\small \textcolor{red}{Benedic}}{}
\setaboveinitialseparation{0.72mm}
\setsecondannotation{\small viij. T.}

\includescore{cantica/evangelica/perviscera.tex}

\vspace{0.3cm}
\rot{Canticum} Benedictus \rot{p. 196.}


\section[HORA TERTIA]{AD TERTIAM}

\rot{Hymnus} Nunc sancte nobis Spiritus \rot{p. 190.}

\vspace{0.3cm}

\setspaceafterinitial{5.2mm plus 0em minus 0em}
\setspacebeforeinitial{4.2mm plus 0em minus 0em}
\def\greinitialformat#1{{\fontsize{40}{40}\selectfont #1}}
\gresetfirstlineaboveinitial{\small \textcolor{red}{Ps 119f}}{}
\setaboveinitialseparation{0.72mm}
\setsecondannotation{\small 8. T.}

\includescore{psalmi/118/aspiceinme17.tex}

\includescore{psalmi/tertia/horatertia_sexta.tex}


\vspace{0.3cm}


\begin{flushleft}

\versik{Dóminus non privábit bonis eos qui ámbulant in innocéntia.}{Dómine virtútum, beátus homo qui sperat in te.}

\medskip

{\rm{
\versik{Der Herr wird seine Güter nicht denen vorenthalten, die in Unschuld wandeln.}{Herr der Scharen, selig der Mensch, der auf dich hofft.}
}}
\end{flushleft}




\section[HORA SEXTA]{AD SEXTAM}

\rot{Hymnus} Rector potens, verax Deus \rot{p. 192.}

\vspace{0.3cm}

\setspaceafterinitial{5.2mm plus 0em minus 0em}
\setspacebeforeinitial{4.2mm plus 0em minus 0em}
\def\greinitialformat#1{{\fontsize{40}{40}\selectfont #1}}
\setspaceafterinitial{5.2mm plus 0em minus 0em}
\setspacebeforeinitial{4.2mm plus 0em minus 0em}
\def\greinitialformat#1{{\fontsize{40}{40}\selectfont #1}}
\gresetfirstlineaboveinitial{\small \textcolor{red}{Ps  18b }}{}
\setaboveinitialseparation{0.72mm}
\setsecondannotation{\small 7. T.}


\includescore{psalmi/17/vivitdominus171819.tex}
\vspace{0.3cm}

\includescore{psalmi/sexta/sextpsalmisexta.tex}

\begin{flushleft}

\versik{Dómine, veritátem in corde dilexísti.}{Et in occúlto sapiéntiam manifestásti mihi.}

\medskip

{\rm{
\versik{Herr, du liebst, die im Herzen voll Wahrheit sind.}{Und im Verborgenen lehrtest du mich Weisheit.}
}}
\end{flushleft}







\section[HORA NONA]{AD NONAM}

\rot{Hymnus} Rerum Deus tenax vigor \rot{p. 194.}

\vspace{0.3cm}

 \setspaceafterinitial{5.2mm plus 0em minus 0em}
\setspacebeforeinitial{4.2mm plus 0em minus 0em}
\def\greinitialformat#1{{\fontsize{40}{40}\selectfont #1}}
\gresetfirstlineaboveinitial{\small \textcolor{red}{ Ps 135 }}{}
\setaboveinitialseparation{0.72mm}
\setsecondannotation{\small 3. T.}

\includescore{psalmi/134/omniaquaecumque.tex}
\vspace{0.3cm}
\includescore{psalmi/nona/nonpsalmisexta.tex}

\begin{flushleft}

\versik{Fac cum servo tuo secúndum misericórdiam tuam, Dómine.}{Iustificatiónes tuas doce me.}

\medskip

{\rm{
\versik{Handle an deinem Knecht nach deiner Huld.}{Lehre mich deine Gesetze.}
}}
\end{flushleft}


\section[VESPERAE]{AD VESPERAS}

\setspaceafterinitial{3.2mm plus 0em minus 0em}
\setspacebeforeinitial{4.2mm plus 0em minus 0em}
\def\greinitialformat#1{{\fontsize{40}{40}\selectfont #1}}
\gresetfirstlineaboveinitial{\small \textcolor{red}{Ps 144b}}{}
\setaboveinitialseparation{0.72mm}
\setsecondannotation{\small 8. T.}

\includescore{psalmi/143/beatuspopulus.tex}

\vspace{0.3cm}
\crot{Psalm 144b}
\begin{quote}
 


\begin{verse}

 Ein neues Lied will ich, o Gott, dir singen, *\\
auf der zehnsaitigen Harfe will ich dir spielen,\\ \vin 
der du den Königen den Sieg verleihst *\\ \vin 
und David, deinen Knecht, errettest.\\ 
Vor dem bösen Schwert errette mich, *\\
entreiß mich der Hand der Fremden!\\ \vin 
Alles, was ihr Mund sagt, ist Lüge, *\\ \vin  Meineide schwört ihre Rechte.\\
Unsre Söhne seien wie junge Bäume, *\\
hoch gewachsen in ihrer Jugend, \\ \vin 
unsre Töchter wie schlanke Säulen, *\\ \vin  die geschnitzt sind für den Tempel.\\ 
Unsre Speicher seien gefüllt, *\\
überquellend von vielerlei Vorrat; \\ \vin 
unsre Herden mögen sich tausendfach\\ \vin mehren, *\\ \vin  vieltausendfach auf unsren Fluren.\\ 
Unsre Kühe mögen tragen, ohne zu verwerfen\\  und ohne Unfall; *\\
kein Wehgeschrei werde laut auf unsern Straßen.\\ \vin  
Wohl dem Volk, dem es so ergeht, *\\ \vin 
\textit{glücklich das Volk, dessen Gott der Herr ist!} \\
\end{verse}

\end{quote}
\vspace{0.3cm}

\setspaceafterinitial{8.2mm plus 0em minus 0em}
\setspacebeforeinitial{4.2mm plus 0em minus 0em}
\def\greinitialformat#1{{\fontsize{40}{40}\selectfont #1}}
\gresetfirstlineaboveinitial{\small \textcolor{red}{Ps 145a}}{}
\setaboveinitialseparation{0.72mm}
\setsecondannotation{\small 8. T.}

\includescore{psalmi/144/inaeternum.tex}
\vspace{0.3cm}
\crot{Psalm 145a}

\begin{quote}
\begin{verse}
 Ich will dich rühmen, mein Gott und König, *\\
und deinen Namen preisen \textit{immer und ewig;}\\
\vin ich will dich preisen Tag für Tag *\\
\vin und deinen Namen loben immer und ewig. \\
Groß ist der Herr und hoch zu loben, *\\
seine Größe ist unerforschlich. \\
\vin Ein Geschlecht verkünde dem andern den \\ \vin Ruhm deiner Werke *\\
\vin und erzähle von deinen gewaltigen Taten. \\
Sie sollen vom herrlichen Glanz deiner \\ Hoheit reden; *\\
ich will deine Wunder besingen.\\
\vin Sie sollen sprechen von der Gewalt\\ \vin  deiner erschreckenden Taten; *\\
\vin ich will von deinen großen Taten berichten.\\
Sie sollen die Erinnerung an deine große \\ Güte wecken *\\
und über deine Gerechtigkeit jubeln.\\
\vin Der Herr ist gnädig und barmherzig, *\\
\vin langmütig und reich an Gnade. \\
Der Herr ist gütig zu allen, *\\
sein Erbarmen waltet über all seinen Werken. \\
\vin Danken sollen dir, Herr, all deine Werke *\\
\vin und deine Frommen dich preisen. \\
Sie sollen von der Herrlichkeit deines \\ Königtums reden, *\\
sollen sprechen von deiner Macht, \\
\vin den Menschen deine machtvollen \\ \vin Taten verkünden *\\
\vin und den herrlichen Glanz deines Königtums. \\
Dein Königtum ist ein Königtum für\\ ewige Zeiten, *\\
deine Herrschaft währt von Geschlecht\\ zu Geschlecht\textit{e}.\\ 
\end{verse}

\end{quote}

\vspace{0.3cm}

\setspaceafterinitial{3.2mm plus 0em minus 0em}
\setspacebeforeinitial{4.2mm plus 0em minus 0em}
\def\greinitialformat#1{{\fontsize{40}{40}\selectfont #1}}
\gresetfirstlineaboveinitial{\small \textcolor{red}{Ps 145b}}{}
\setaboveinitialseparation{0.72mm}
\setsecondannotation{\small 4. T.}

\includescore{psalmi/144/custoditdominus.tex}
\vspace{0.3cm}
\crot{Psalm 145b}
\begin{quote}
\begin{verse}
 Der Herr ist treu in all seinen Worten, *\\ voll Huld in \d all seinen Taten.\\
\vin Der Herr stützt alle, die fallen, *\\
\vin und richtet \d alle Gebeugten auf. \\
Aller Augen warten auf dich *\\
und du gibst ihnen Sp\d eise zur rechten Zeit. \\
\vin Du öffnest deine Hand *\\
\vin und sättigst alles, was lebt, nach\\ \vin d\d einem Gefallen. \\
Gerecht ist der Herr in allem, was er tut, *\\
voll Huld in \d all seinen Werken. \\
\vin Der Herr ist allen, die ihn anrufen, nahe, *\\
\vin allen, die zu ihm \d aufrichtig rufen. \\
Die Wünsche derer, die ihn fürchten,\\ erfüllt er, *\\
er hört ihr Schr\d eien und rettet sie. \\
\vin \textit{Alle, die ihn lieben, behütet der Herr,} *\\
\vin doch alle Fr\d evler vernichtet er. \\
Mein Mund verkünde das Lob des Herrn. *\\
Alles, was lebt, preise seinen heiligen Namen \\\d immer und ewig! \\
\end{verse}

 \end{quote}





\vspace{0.3cm}

\setspaceafterinitial{4.2mm plus 0em minus 0em}
\setspacebeforeinitial{4.2mm plus 0em minus 0em}
\def\greinitialformat#1{{\fontsize{40}{40}\selectfont #1}}
\gresetfirstlineaboveinitial{\small \textcolor{red}{Ap 15, 3-4}}{}
\setaboveinitialseparation{0.72mm}
\setsecondannotation{\small iv. T.}

\includescore{cantica/cn/omnesgentes.tex}

\cant{Ap 15, 3-4}

\begin{quote}
\begin{verse}


Groß und wunderbar sind deine Taten,*\\
Herr und Gott, du Herrscher über d\d ie ganze\\ Schöpfung!\\
\vin Gerecht und zuverlässig sind deine Wege,*\\
\vin du K\d önig der Völker.\\
Wer wird dich nicht fürchten, Herr, †\\
wer wird  deinen Namen nicht preisen?*\\
Denn du \d allein bist heilig.\\
\vin Ja, \textit{alle Völker werden kommen und\\ \vin  beten dich an},*\\
\vin denn offenbar geworden sind deine\\ \vin  g\d erechten Taten.\\
\end{verse}
\end{quote}


\medskip
 
\noindent{\rot{Resp.br.} Adjutorium nostrum {\rot{ut in Feria Secunda p. 58.}}\\

\noindent{\rot{Hymnus} Deus Creator \rot{vel} Lucis Creator \rot{ ut in Dominica p. 32 - 33}

\vspace{0.3cm}

\medskip


\begin{flushleft}

\versik{Dirigátur, Dómine, orátio mea.}{Sicut incénsum in conspéctu tuo.}

\medskip
{\rm{
\versik{Herr, mein Gebet werde gelenkt.}{Wie Weihrauch vor dein Angesicht.}
}}
\end{flushleft}



\medskip

\setspaceafterinitial{5.2mm plus 0em minus 0em}
\setspacebeforeinitial{4.2mm plus 0em minus 0em}
\def\greinitialformat#1{{\fontsize{40}{40}\selectfont #1}}
\gresetfirstlineaboveinitial{\small \textcolor{red}{Magni}}{}
\setaboveinitialseparation{0.72mm}
\setsecondannotation{\small viij. T.}

\includescore{cantica/evangelica/suscepit.tex}

\medskip

\rot{Canticum} Magnificat \rot{p. 200.}

\newpage