
\kkap{SAMSTAG}


\section[SAMSTAG]{AD LAUDES}

\setspaceafterinitial{5.2mm plus 0em minus 0em}
\setspacebeforeinitial{4.2mm plus 0em minus 0em}
\def\greinitialformat#1{{\fontsize{40}{40}\selectfont #1}}
\gresetfirstlineaboveinitial{\small \textcolor{red}{ Ps 143 }}{}
\setaboveinitialseparation{0.72mm}
\setsecondannotation{\small viij. T.}

\includescore{psalmi/142/ps142.tex}

\vspace{0.3cm}

\crot{Psalm 143}

\vspace{0.3cm}
\begin{quote}

\begin{verse}
 Herr, höre mein Gebet, vernimm mein Flehen; *\\
\textit{in deiner Treue erhöre mich}, in \\deiner Gerechtigkeit!\\
\vin Geh mit deinem Knecht nicht ins Gericht; *\\
\vin denn keiner, der lebt, ist gerecht vor dir. \\
Der Feind verfolgt mich, tritt mein \\ Leben zu Boden, *\\
er lässt mich in der Finsternis wohnen wie \\ längst Verstorbene. \\
\vin Mein Geist verzagt in mir, *\\
\vin mir erstarrt das Herz in der Brust. \\
Ich denke an die vergangenen Tage, †\\
sinne nach über all deine Taten, *\\ erwäge das Werk deiner Hände. \\
\vin Ich breite die Hände aus (und bete) zu dir; *\\
\vin meine Seele dürstet nach\\ \vin  dir wie lechzendes Land. \\
Herr, erhöre mich bald, *\\
denn mein Geist wird müde; \\
\vin verbirg dein Antlitz nicht vor mir, * \\
\vin  damit ich nicht werde wie Menschen, \\ \vin die längst begraben sind.\\
Lass mich deine Huld \\erfahren am frühen Morgen; *\\
denn ich vertraue auf dich. \\
\vin Zeig mir den Weg, den ich gehen soll; *\\ \vin denn ich erhebe meine Seele zu dir. \\
Herr, entreiß mich den Feinden! *\\
Zu dir nehme ich meine Zuflucht.\\
\vin Lehre mich, deinen Willen zu tun; †\\ \vin denn  du bist mein Gott. *\\
\vin Dein guter Geist leite mich auf ebenem Pfad\textit{e}.\\ 
Um deines Namens willen, Herr, erhalt\\mich am Leben, *\\
führe mich heraus aus der Not in deiner\\ Gerechtigkeit!\\
\vin Vertilge in deiner Huld meine Feinde, †\\
\vin lass all meine Gegner untergehn! * \\ \vin Denn ich bin dein Knecht. \\


\end{verse}

\end{quote}




\vspace{0.3cm}
\setspaceafterinitial{5.2mm plus 0em minus 0em}
\setspacebeforeinitial{4.2mm plus 0em minus 0em}
\def\greinitialformat#1{{\fontsize{40}{40}\selectfont #1}}
\gresetfirstlineaboveinitial{\small \textcolor{red}{Ex 15}}{}
\setaboveinitialseparation{0.72mm}
\setsecondannotation{\small 7. T.}
\includescore{cantica/ca/ex15.tex}

\vspace{0.3cm}

\cant{Ex 15,1-18}

\begin{quote}
\begin{verse}


Ich singe dem Herrn ein Lied, †\\
denn er ist h\d och und erhaben. *\\ 
Rosse und Wagen w\d arf er ins Meer.\\
\vin \textit{Meine Stärke und mein \\ \vin L\d ied ist der Herr, *\\
\vin er ist für mich zum R\d etter geworden.} \\
Er ist mein Gott, \d ihn will ich preisen; *\\ 
den Gott meines Vaters w\d ill ich rühmen. \\
\vin Der H\d err ist ein Krieger, *\\
\vin Jahwe \d ist sein Name.\\
Pharaos Wagen und seine Streitmacht †\\
w\d arf er ins Meer. *\\ 
Seine besten Kämpfer vers\d anken im Schilfmeer.\\
\vin Fluten d\d eckten sie zu, *\\
\vin sie sanken in die T\d iefe wie Steine.\\
Deine Rechte, Herr, ist h\d errlich an Stärke; *\\
deine Rechte, Herr, zerschm\d ettert den Feind.\\
\vin In deiner erhabenen Größe wirfst  du die \\ \vin G\d egner zu Boden.*\\
\vin Du sendest deinen Zorn; er  fr\d isst sie \\ \vin wie Stoppeln.\\
Du schnaubtest vor Zorn, da  \\türmte sich Wasser, †\\
 da standen W\d ogen als Wall, *\\
Fluten erstarrten im H\d erzen des Meeres.\\
\vin Da sagte der Feind:  Ich j\d age nach, hole ein. *\\
\vin Ich teile die Beute, ich st\d ille die Gier. \\
 Ich z\d ücke mein Schwert,*\\
 meine H\d and jagt sie davon.\\
\vin Da schnaubtest du Sturm. Das  Meer \\ \vin d\d eckte sie zu. *\\
\vin Sie sanken wie Blei ins t\d osende Wasser.\\

\vspace{0.3cm}

\crot{divisio}

\vspace{0.3cm}

Wer ist wie du unter den Göttern, o Herr? †\\
Wer ist wie du gew\d altig und heilig, *\\ gepriesen als furchtbar, W\d under vollbringend?\\
\vin Du strecktest d\d eine Rechte aus, *\\
\vin da verschl\d ang sie die Erde.\\
Du lenktest in deiner Güte das\\  Volk, das d\d u erlöst hast, *\\
du führtest sie machtvoll zu deiner h\d eiligen \\Wohnung.\\
\vin Als die Völker das hörten, erz\d itterten sie, *\\
\vin die Philister p\d ackte das Schütteln.\\
Damals erschraken die Häuptlinge Edoms,†\\
die Mächtigen von Moab p\d ackte das Zittern, *\\ Kanaans Bewohner, sie \d alle verzagten.\\
\vin Schrecken und F\d urcht überfiel sie, *\\
\vin sie erstarrten zu Stein vor\\ \vin der M\d acht deines Armes, \\
bis hind\d urchzog, o Herr, dein Volk, *\\ bis hindurchzog das V\d olk, das du erschufst.\\
\vin Du brachtest sie hin und  pfl\d anztest sie ein *\\ \vin auf dem Berg d\d eines Erbes.\\
 Einen Ort, wo du thronst, Herr, h\d ast \\du gemacht; *\\ ein Heiligtum, Herr, haben deine  H\d ände \\gegründet.\\
\vin Der H\d err ist König *\\ \vin für \d immer und ewig. \\

\end{verse}
\end{quote}
\vspace{0.3cm}

\rot{vel}

\setspaceafterinitial{5.2mm plus 0em minus 0em}
\setspacebeforeinitial{4.2mm plus 0em minus 0em}
\def\greinitialformat#1{{\fontsize{40}{40}\selectfont #1}}
\gresetfirstlineaboveinitial{\small \textcolor{red}{Deut 32}}{}
\setaboveinitialseparation{0.72mm}
\setsecondannotation{\small vj. T.}
\includescore{cantica/ca/deut32.tex}

\vspace{0.3cm}

\cant{Dtn 32,1-12}
\begin{quote}
\begin{verse}

 
Hört zu, ihr Himmel, ich will reden,*\\
die Erde lausch\d e meinen Worten.\\
\vin Meine Lehre wird strömen wie Regen,*\\
\vin meine Botschaft w\d ird fallen wie Tau,\\
wie Regentropfen auf das Gras *\\
und wie Tauperl\d en auf die Pflanzen.\\
\vin Ich will den Namen des Herrn verkünden.*\\
\vin \textit{Preist die Größe \d unseres Gottes!}\\
Er heißt: der Fels. Vollkommen  ist, was er tut;*\\
denn alle sein\d e Wege sind recht.\\
\vin Er ist ein unbeirrbar treuer Gott,*\\
\vin er ist ger\d echt und gerade.\\
Ein falsches, verdrehtes \\  Geschlecht fiel von ihm ab,*\\
Verkrüppelte, die nicht m\d ehr seine Söhne sind.\\
\vin Ist das euer Dank an den Herrn,*\\
\vin du dummes, v\d erblendetes Volk?\\
Ist er nicht dein Vater, dein Schöpfer?*\\
Hat er dich nicht g\d eformt und hingestellt? \\

Denk an die Tage der Vergangenheit,*\\
lerne aus den Jahr\d en der Geschichte!\\
\vin Frag deinen Vater, er wird es dir erzählen,*\\
\vin frage die Alten, sie werd\d en es dir sagen. \\

\vspace{0.3cm}

\crot{divisio}
\vspace{0.3cm}

Als der Höchste (den Göttern) \\ die Völker übergab,*\\
als er die M\d enschheit aufteilte,\\
\vin legte er die Gebiete der Völker *\\
\vin nach d\d er Zahl der Götter fest;\\
der Herr nahm sich sein Volk als Anteil,*\\
Jakob w\d urde sein Erbland.\\
\vin Er fand ihn in der Steppe,*\\
\vin in der Wüste, wo w\d ildes Getier heult.\\
Er hüllte ihn ein, gab auf ihn Acht *\\
und hütete ihn w\d ie seinen Augenstern,\\
\vin wie der Adler, der sein Nest beschützt *\\
\vin und üb\d er seinen Jungen schwebt,\\
der seine Schwingen ausbreitet, \\ein Junges ergreift *\\
und es flügelschl\d agend davonträgt\\
\vin Der Herr allein hat Jakob geleitet,*\\
\vin kein fremder Gott st\d and ihm zur Seite.\\

 
\end{verse}
\end{quote}
\vspace{0.3cm}

\setspaceafterinitial{5.2mm plus 0em minus 0em}
\setspacebeforeinitial{4.2mm plus 0em minus 0em}
\def\greinitialformat#1{{\fontsize{40}{40}\selectfont #1}}
\gresetfirstlineaboveinitial{\small \textcolor{red}{Ps 150}}{}
\setaboveinitialseparation{0.72mm}
\setsecondannotation{\small viij. T.}

\includescore{psalmi/150/ps150.tex}

\vspace{0.3cm}

\crot{Psalm 150}
\begin{quote}
\begin{verse}
 Lobt Gott in seinem Heiligtum, *\\
lobt ihn in seiner mächtigen Feste! \\
\vin Lobt ihn für seine großen Taten, *\\
\vin \textit{lobt ihn in seiner gewaltigen Größe!}\\
Lobt ihn mit dem Schall der Hörner, *\\
lobt ihn mit Harfe und Zither! \\
\vin Lobt ihn mit Pauken und Tanz, *\\
\vin lobt ihn mit Flöten und Saitenspiel! \\
Lobt ihn mit hellen Zimbeln, *\\
lobt ihn mit klingenden Zimbeln!\\
\vin Alles, was atmet, *\\
\vin lobe den Herr\textit{e}n!\\

\end{verse}
\end{quote}


\noindent\rot{Resp.br.} Sana animam \rot{und Hymnus} Splendor paternæ \rot{oder} Ecce iam noctis \rot{wie am Montag pp. 18ff.}\\
\begin{flushleft}

\versik{Repléti sumus mane misericórdia tua.}{Exultávimus, et delectáti sumus.}

\medskip

{\rm{
\versik{Am Morgen sind wir erfüllt von deiner Huld.}{Wir sind (voll) Jubel und Frohsinn.}
}}
\end{flushleft}

\setspaceafterinitial{5.2mm plus 0em minus 0em}
\setspacebeforeinitial{4.2mm plus 0em minus 0em}
\def\greinitialformat#1{{\fontsize{40}{40}\selectfont #1}}
\gresetfirstlineaboveinitial{\small \textcolor{red}{Benedic.}}{}
\setaboveinitialseparation{0.72mm}
\setsecondannotation{\small j. T.}

\includescore{cantica/evangelica/inviapacis.tex}

\vspace{0.3cm}
\rot{Canticum} Benedictus \rot{p. 64.}