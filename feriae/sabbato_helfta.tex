\thispagestyle{plain}

\kapklein{\rot{S}abbato}
\kkap{sabbato}





\section[VIGILIAE]{AD VIGILIAS}


\setspaceafterinitial{5.2mm plus 0em minus 0em}
\setspacebeforeinitial{4.2mm plus 0em minus 0em}
\def\greinitialformat#1{{\fontsize{40}{40}\selectfont #1}}
\gresetfirstlineaboveinitial{\small \textcolor{red}{Invitat.}}{Invitat.}
\setaboveinitialseparation{0.72mm}
%\setsecondannotation{\small Ps. 125}

\includescore{invitatoria/venitesab.tex}



\vspace{0.6cm}

\rot{Hymnus ut in Dominica.}


\section{VIGILIAE I}


\setspaceafterinitial{5.2mm plus 0em minus 0em}
\setspacebeforeinitial{4.2mm plus 0em minus 0em}
\def\greinitialformat#1{{\fontsize{40}{40}\selectfont #1}}
\gresetfirstlineaboveinitial{\small \textcolor{red}{ Ps 38 (39)}}{}
\setaboveinitialseparation{0.72mm}
\setsecondannotation{\small 1. T.}

\includescore{psalmi/38/utnondelinquam.tex}

Ant. Lass mich nicht sündigen * mit meiner Zunge.


\vspace{0.6cm}

\setspaceafterinitial{5.2mm plus 0em minus 0em}
\setspacebeforeinitial{4.2mm plus 0em minus 0em}
\def\greinitialformat#1{{\fontsize{40}{40}\selectfont #1}}
\gresetfirstlineaboveinitial{\small \textcolor{red}{ Ps 39 (40)}}{}
\setaboveinitialseparation{0.72mm}
\setsecondannotation{\small viij. T.}
\includescore{psalmi/39/respexitme39.tex}

Ant. Der Herr hat auf mich geschaut * und meine Bitten gehört.

\vspace{0.6cm}

\setspaceafterinitial{5.2mm plus 0em minus 0em}
\setspacebeforeinitial{4.2mm plus 0em minus 0em}
\def\greinitialformat#1{{\fontsize{40}{40}\selectfont #1}}
\gresetfirstlineaboveinitial{\small \textcolor{red}{ Ps 40 (41)}}{}
\setaboveinitialseparation{0.72mm}
\setsecondannotation{\small ij. T.}
\includescore{psalmi/40/sanadomine4041.tex}

Ant. Heile Herr meine Seele * denn ich habe gegen dich gesündigt.

\vspace{0.3cm}

\versik{Atténdite, pópule meus, doctrínam meam.}{Inclináte aurem vestram in verba oris mei.}

\versik{Mein Volk, höre auf meine Weisung.}{Neigt euer Ohr zu den Worten meines Mundes.}



\section{VIGILIAE II}



\setspaceafterinitial{5.2mm plus 0em minus 0em}
\setspacebeforeinitial{4.2mm plus 0em minus 0em}
\def\greinitialformat#1{{\fontsize{40}{40}\selectfont #1}}
\gresetfirstlineaboveinitial{\small \textcolor{red}{ Ps 65 (66)}}{}
\setaboveinitialseparation{0.72mm}
\setsecondannotation{\small vj. T.}

\includescore{psalmi/65/benedicite6165.tex}

Ant. Preist unseren Gott * alle Völker.




\vspace{0.6cm}
%2.Antiphon

\setspaceafterinitial{5.2mm plus 0em minus 0em}
\setspacebeforeinitial{4.2mm plus 0em minus 0em}
\def\greinitialformat#1{{\fontsize{40}{40}\selectfont #1}}
\gresetfirstlineaboveinitial{\small \textcolor{red}{ Ps 67 (68)}}{}
\setaboveinitialseparation{0.72mm}
\setsecondannotation{\small viij. T.}
\includescore{psalmi/67/inecclesiis67.tex}

Ant. Preist den Herrn * in großer Gemeinde.

\vspace{0.3cm}

\versik{Manda, Deus, virtúti tuae.}{Confirma hoc, Deus, quod operátus es in nobis.}

\versik{Zeige, Herr, deine Macht.}{Vollende, o Gott, was du in uns begonnen hast.}




\section{VIGILIAE III}

\setspaceafterinitial{5.2mm plus 0em minus 0em}
\setspacebeforeinitial{4.2mm plus 0em minus 0em}
\def\greinitialformat#1{{\fontsize{40}{40}\selectfont #1}}
\gresetfirstlineaboveinitial{\small \textcolor{red}{ Ps 82 (83)}}{}
\setaboveinitialseparation{0.72mm}
\setsecondannotation{\small j. T.}

\includescore{psalmi/82/tusolus8182.tex}

Ant. Du allein bist der Höchste * auf der ganzen Erde.
\vspace{0.6cm}

\setspaceafterinitial{5.2mm plus 0em minus 0em}
\setspacebeforeinitial{4.2mm plus 0em minus 0em}
\def\greinitialformat#1{{\fontsize{40}{40}\selectfont #1}}
\gresetfirstlineaboveinitial{\small \textcolor{red}{ Ps 83 (84)}}{}
\setaboveinitialseparation{0.72mm}
\setsecondannotation{\small viij. T.}

\includescore{psalmi/83/beatiquihabitant83.tex}

Ant. Selig * wer in deinem Hause wohnt, o Herr.

\vspace{0.6cm}
\setspaceafterinitial{5.2mm plus 0em minus 0em}
\setspacebeforeinitial{4.2mm plus 0em minus 0em}
\def\greinitialformat#1{{\fontsize{40}{40}\selectfont #1}}
\gresetfirstlineaboveinitial{\small \textcolor{red}{ Ps 84 (85)}}{}
\setaboveinitialseparation{0.72mm}
\setsecondannotation{\small vj. T.}

\includescore{psalmi/84/benedixisti8384.tex}


Ant. Herr * mit Segen erfüllst du dein Land.


\vspace{0.3cm}
\versik{Confitébimur tibi, Deus.}{Et invocábimus nomen tuum.}

\versik{Gott, wir preisen dich.}{Und rufen deinen Namen an.}


\section{VIGILIAE IV}

\setspaceafterinitial{5.2mm plus 0em minus 0em}
\setspacebeforeinitial{4.2mm plus 0em minus 0em}
\def\greinitialformat#1{{\fontsize{40}{40}\selectfont #1}}
\gresetfirstlineaboveinitial{\small \textcolor{red}{Ps 106(107b)}}{}
\setaboveinitialseparation{0.72mm}
\setsecondannotation{\small v. T.}

\includescore{psalmi/106/denecessitatibus.tex}

Ant. Aus all meinen Nöten * befreie mich, o Herr.


\vspace{0.6cm}

\setspaceafterinitial{5.2mm plus 0em minus 0em}
\setspacebeforeinitial{4.2mm plus 0em minus 0em}
\def\greinitialformat#1{{\fontsize{40}{40}\selectfont #1}}
\gresetfirstlineaboveinitial{\small \textcolor{red}{ Ps 107(108)}}{}
\setaboveinitialseparation{0.72mm}
\setsecondannotation{\small 4. T.}
\includescore{psalmi/107/psallamtibi107.tex}

Ant. Ich preise dich vor den Völkern * denn deine Güte reicht so weit der Himmel ist.


\vspace{0.6cm}

\setspaceafterinitial{5.2mm plus 0em minus 0em}
\setspacebeforeinitial{4.2mm plus 0em minus 0em}
\def\greinitialformat#1{{\fontsize{40}{40}\selectfont #1}}
\gresetfirstlineaboveinitial{\small \textcolor{red}{ Ps 108(109)}}{}
\setaboveinitialseparation{0.72mm}
\setsecondannotation{\small v. T.}
\includescore{psalmi/108/confitebor107108.tex}

Ant. Ich preise dich, Herr, * mit meinem Mund.



\vspace{0.3cm}

\versik{Misit verbum suum, et sanávit eos.}{Et erípuit eos de interitiónibus eórum.}

\versik{Er sandte sein Wort und hat sie geheilt.}{Er entriss sie all ihren Ängsten}



\section[LAUDES]{AD LAUDES}

\setspaceafterinitial{5.2mm plus 0em minus 0em}
\setspacebeforeinitial{4.2mm plus 0em minus 0em}
\def\greinitialformat#1{{\fontsize{40}{40}\selectfont #1}}
\gresetfirstlineaboveinitial{\small \textcolor{red}{ Ps 143 }}{}
\setaboveinitialseparation{0.72mm}
\setsecondannotation{\small viij. T.}

\includescore{psalmi/142/ps142.tex}

\vspace{0.3cm}

\crot{Psalm 143}

\vspace{0.3cm}

\begin{verse}[\versewidth]
 Herr, höre mein Gebet, vernimm\\ mein Flehen; *\\
\textit{in deiner Treue erhöre mich}, in \\deiner Gerechtigkeit!\\
\vin Geh mit deinem Knecht nicht\\ \vin ins Gericht; *\\
\vin denn keiner, der lebt, ist gerecht vor dir. \\
Der Feind verfolgt mich, tritt mein \\ Leben zu Boden, *\\
er lässt mich in der Finsternis wohnen wie \\ längst Verstorbene. \\
\vin Mein Geist verzagt in mir, *\\
\vin mir erstarrt das Herz in der Brust. \\
Ich denke an die vergangenen Tage, †\\
sinne nach über all deine Taten, *\\ erwäge das Werk deiner Hände. \\
\vin Ich breite die Hände aus (und bete)\\ \vin  zu dir; *\\
\vin meine Seele dürstet nach\\ \vin  dir wie lechzendes Land. \\
Herr, erhöre mich bald, *\\
denn mein Geist wird müde; \\
\vin verbirg dein Antlitz nicht vor mir, * \\
\vin  damit ich nicht werde wie Menschen, \\ \vin die längst begraben sind.\\
Lass mich deine Huld \\erfahren am frühen Morgen; *\\
denn ich vertraue auf dich. \\
\vin Zeig mir den Weg, den ich gehen soll; *\\ \vin denn ich erhebe meine Seele zu dir. \\
Herr, entreiß mich den Feinden! *\\
Zu dir nehme ich meine Zuflucht.\\
\vin Lehre mich, deinen Willen zu tun; denn \\ \vin du bist mein Gott. *\\
\vin Dein guter Geist leite mich auf ebenem Pfad.\\ 
Um deines Namens willen, Herr, erhalt\\mich am Leben, *\\
führe mich heraus aus der Not in \\deiner Gerechtigkeit!\\
\vin Vertilge in deiner Huld meine Feinde, †\\
\vin lass all meine Gegner untergehn! * \\ \vin Denn ich bin dein Knecht. \\


\end{verse}






\vspace{0.3cm}
\setspaceafterinitial{5.2mm plus 0em minus 0em}
\setspacebeforeinitial{4.2mm plus 0em minus 0em}
\def\greinitialformat#1{{\fontsize{40}{40}\selectfont #1}}
\gresetfirstlineaboveinitial{\small \textcolor{red}{Ex 15}}{}
\setaboveinitialseparation{0.72mm}
\setsecondannotation{\small 7. T.}
\includescore{cantica/ca/ex15.tex}

\vspace{0.3cm}

\cant{Ex 15,1-18}
\begin{verse}[\versewidth]


Ich singe dem Herrn ein Lied, †\\
denn er ist h\d och und erhaben. *\\ 
Rosse und Wagen w\d arf er ins Meer.\\
\vin \textit{Meine Stärke und mein \\ \vin L\d ied ist der Herr, *\\
\vin er ist für mich zum R\d etter geworden.} \\
Er ist mein Gott, \d ihn will ich preisen; *\\ 
den Gott meines Vaters w\d ill ich rühmen. \\
\vin Der H\d err ist ein Krieger, *\\
\vin Jahwe \d ist sein Name.\\
Pharaos Wagen und seine Streitmacht †\\
w\d arf er ins Meer. *\\ 
Seine besten Kämpfer vers\d anken im Schilfmeer.\\
\vin Fluten d\d eckten sie zu, *\\
\vin sie sanken in die T\d iefe wie Steine.\\
Deine Rechte, Herr, ist h\d errlich an Stärke; *\\
deine Rechte, Herr, zerschm\d ettert den Feind.\\
\vin In deiner erhabenen Größe wirfst \\ \vin du die G\d egner zu Boden.*\\
\vin Du sendest deinen Zorn; er\\ \vin  fr\d isst sie wie Stoppeln.\\
Du schnaubtest vor Zorn, da \\ türmte sich Wasser, †\\
 da standen W\d ogen als Wall, *\\
Fluten erstarrten im H\d erzen des Meeres.\\
\vin Da sagte der Feind: \\ \vin Ich j\d age nach, hole ein. *\\
\vin Ich teile die Beute, ich st\d ille die Gier. \\
 Ich z\d ücke mein Schwert,*\\
 meine H\d and jagt sie davon.\\
\vin Da schnaubtest du Sturm. Das \\ \vin Meer d\d eckte sie zu. *\\
\vin Sie sanken wie Blei ins t\d osende Wasser.\\
\end{verse}

\vspace{0.3cm}

\begin{center}
\frot{- DIVISIO -}
\end{center}

\vspace{0.3cm}

\begin{verse}[\versewidth]

Wer ist wie du unter den Göttern, o Herr? †\\
Wer ist wie du gew\d altig und heilig, *\\ gepriesen als furchtbar, W\d under vollbringend?\\
\vin Du strecktest d\d eine Rechte aus, *\\
\vin da verschl\d ang sie die Erde.\\
Du lenktest in deiner Güte das\\  Volk, das d\d u erlöst hast, *\\
du führtest sie machtvoll zu deiner \\h\d eiligen Wohnung.\\
\vin Als die Völker das hörten, erz\d itterten sie, *\\
\vin die Philister p\d ackte das Schütteln.\\
Damals erschraken die Häuptlinge Edoms,†\\
die Mächtigen von Moab p\d ackte das Zittern, *\\ Kanaans Bewohner, sie \d alle verzagten.\\
\vin Schrecken und F\d urcht überfiel sie, *\\
\vin sie erstarrten zu Stein vor\\ \vin der M\d acht deines Armes, \\
bis hind\d urchzog, o Herr, dein Volk, *\\ bis hindurchzog das V\d olk, das du erschufst.\\
\vin Du brachtest sie hin und \\ \vin pfl\d anztest sie ein *\\ \vin auf dem Berg d\d eines Erbes.\\
 Einen Ort, wo du thronst, Herr,\\ h\d ast du gemacht; *\\ ein Heiligtum, Herr, haben\\ deine H\d ände gegründet.\\
\vin Der H\d err ist König *\\ \vin für \d immer und ewig. \\

\end{verse}

\vspace{0.3cm}

\rot{vel}

\setspaceafterinitial{5.2mm plus 0em minus 0em}
\setspacebeforeinitial{4.2mm plus 0em minus 0em}
\def\greinitialformat#1{{\fontsize{40}{40}\selectfont #1}}
\gresetfirstlineaboveinitial{\small \textcolor{red}{Deut 32}}{}
\setaboveinitialseparation{0.72mm}
\setsecondannotation{\small vj. T.}
\includescore{cantica/ca/deut32.tex}

\vspace{0.3cm}

\cant{Dtn 32,1-12}

\begin{verse}[\versewidth]
 
Hört zu, ihr Himmel, ich will reden,*\\
die Erde lausch\d e meinen Worten.\\
\vin Meine Lehre wird strömen wie Regen,*\\
\vin meine Botschaft w\d ird fallen wie Tau,\\
wie Regentropfen auf das Gras *\\
und wie Tauperl\d en auf die Pflanzen.\\
\vin Ich will den Namen des Herrn verkünden.*\\
\vin \textit{Preist die Größe \d unseres Gottes!}\\
Er heißt: der Fels. Vollkommen \\ ist, was er tut;*\\
denn alle sein\d e Wege sind recht.\\
\vin Er ist ein unbeirrbar treuer Gott,*\\
\vin er ist ger\d echt und gerade.\\
Ein falsches, verdrehtes \\  Geschlecht fiel von ihm ab,*\\
Verkrüppelte, die nicht m\d ehr seine Söhne sind.\\
\vin Ist das euer Dank an den Herrn,*\\
\vin du dummes, v\d erblendetes Volk?\\
Ist er nicht dein Vater, dein Schöpfer?*\\
Hat er dich nicht g\d eformt und hingestellt? \\

Denk an die Tage der Vergangenheit,,*\\
lerne aus den Jahr\d en der Geschichte!\\
\vin Frag deinen Vater, er wird es dir erzählen,*\\
\vin frage die Alten, sie werd\d en es dir sagen. \\

\vspace{0.3cm}

\crot{divisio}
\vspace{0.3cm}

Als der Höchste (den Göttern) \\ die Völker übergab,*\\
als er die M\d enschheit aufteilte,\\
\vin legte er die Gebiete der Völker *\\
\vin nach d\d er Zahl der Götter fest;\\
der Herr nahm sich sein Volk als Anteil,*\\
Jakob w\d urde sein Erbland.\\
\vin Er fand ihn in der Steppe,*\\
\vin in der Wüste, wo w\d ildes Getier heult.\\
Er hüllte ihn ein, gab auf ihn Acht *\\
und hütete ihn w\d ie seinen Augenstern,\\
\vin wie der Adler, der sein Nest beschützt *\\
\vin und üb\d er seinen Jungen schwebt,\\
der seine Schwingen ausbreitet, \\ein Junges ergreift *\\
und es flügelschl\d agend davonträgt\\
\vin Der Herr allein hat Jakob geleitet,*\\
\vin kein fremder Gott st\d and ihm zur Seite.\\

 
\end{verse}

\vspace{0.3cm}

\setspaceafterinitial{5.2mm plus 0em minus 0em}
\setspacebeforeinitial{4.2mm plus 0em minus 0em}
\def\greinitialformat#1{{\fontsize{40}{40}\selectfont #1}}
\gresetfirstlineaboveinitial{\small \textcolor{red}{Ps 150}}{}
\setaboveinitialseparation{0.72mm}
\setsecondannotation{\small viij. T.}

\includescore{psalmi/150/ps150.tex}

\vspace{0.3cm}

\crot{Psalm 150}

\begin{verse}[\versewidth]
 Lobt Gott in seinem Heiligtum, *\\
lobt ihn in seiner mächtigen Feste! \\
\vin Lobt ihn für seine großen Taten, *\\
\vin \textit{lobt ihn in seiner gewaltigen Größe!}\\
Lobt ihn mit dem Schall der Hörner, *\\
lobt ihn mit Harfe und Zither! \\
\vin Lobt ihn mit Pauken und Tanz, *\\
\vin lobt ihn mit Flöten und Saitenspiel! \\
Lobt ihn mit hellen Zimbeln, *\\
lobt ihn mit klingenden Zimbeln!\\
\vin Alles, was atmet, *\\
\vin lob\d e den Herrn!\\

\end{verse}


 \rot{Responsorium breve et Hymnus ut in Feria Secunda.}

\vspace{0.6cm}



\versik{Repléti sumus mane misericórdia tua.}{Exultávimus, et delectáti sumus.}

\vspace{0.6cm}

\setspaceafterinitial{5.2mm plus 0em minus 0em}
\setspacebeforeinitial{4.2mm plus 0em minus 0em}
\def\greinitialformat#1{{\fontsize{40}{40}\selectfont #1}}
\gresetfirstlineaboveinitial{\small \textcolor{red}{Benedic.}}{}
\setaboveinitialseparation{0.72mm}
\setsecondannotation{\small j. T.}

\includescore{cantica/evangelica/inviapacis.tex}




\section[horae minores]{ad tertiam}

\rot{Hymnus} Nunc sancte nobis Spiritus.

\vspace{0.3cm}

\setspaceafterinitial{3.2mm plus 0em minus 0em}
\setspacebeforeinitial{4.2mm plus 0em minus 0em}
\def\greinitialformat#1{{\fontsize{40}{40}\selectfont #1}}
\gresetfirstlineaboveinitial{\small \textcolor{red}{ xx - xxii }}{}
\setaboveinitialseparation{0.72mm}
\setsecondannotation{\small iv. T.}


\includescore{psalmi/118/videhumilitatem20.tex}





\vspace{0.3cm}


\versik{Dóminus non privábit bonis eos qui ámbulant in innocéntia.}{Dómine virtútum, beátus homo qui sperat in te.}




\section[horae minores]{ad sextam}

\rot{Hymnus} Rector potens, verax Deus.

\vspace{0.3cm}

\setspaceafterinitial{5.2mm plus 0em minus 0em}
\setspacebeforeinitial{4.2mm plus 0em minus 0em}
\def\greinitialformat#1{{\fontsize{40}{40}\selectfont #1}}
\gresetfirstlineaboveinitial{\small \textcolor{red}{ Ps 18c 19 20) }}{}
\setaboveinitialseparation{0.72mm}
\setsecondannotation{\small viij. T.}

\includescore{psalmi/19/exaudiatte19.tex}



\vspace{0.3cm}



\versik{Dómine, veritátem in corde dilexísti.}{Et in occúlto sapiéntiam manifestásti mihi.}



\section[horae minores]{ad nonam}

\rot{Hymnus} Rerum Deus tenax vigor.

\vspace{0.3cm}

 \setspaceafterinitial{5.2mm plus 0em minus 0em}
\setspacebeforeinitial{4.2mm plus 0em minus 0em}
\def\greinitialformat#1{{\fontsize{40}{40}\selectfont #1}}
\gresetfirstlineaboveinitial{\small \textcolor{red}{ Ps 135 (135) }}{}
\setaboveinitialseparation{0.72mm}
\setsecondannotation{\small 3. T.}

\includescore{psalmi/135/quoniaminaeternum135.tex}
\vspace{0.3cm}


\versik{Fac cum servo tuo secúndum misericórdiam tuam, Dómine.}{Iustificatiónes tuas doce me.}


\newpage



\section[vesperae]{ad vesperas}



\setspaceafterinitial{3.2mm plus 0em minus 0em}
\setspacebeforeinitial{3.2mm plus 0em minus 0em}
\def\greinitialformat#1{{\fontsize{40}{40}\selectfont #1}}
\gresetfirstlineaboveinitial{\small \textcolor{red}{Ps 146}}{}
\setaboveinitialseparation{0.72mm}
\setsecondannotation{\small iv. T.}

\includescore{psalmi/145/laudabodeum.tex}

\vspace{0.3cm}

\crot{Psalm 146}



\begin{verse}[\versewidth]
 Lobe den Herrn, meine Seele! †\\
\textit{Ich will den Herrn loben, solange ich lebe,} *\\
meinem Gott singen und spielen,\\ sol\d ange ich da bin. \\
\vin Verlasst euch nicht auf Fürsten, *\\
\vin auf Menschen, bei denen es d\d och \\ \vin keine Hilfe gibt. \\
Haucht der Mensch sein Leben aus †\\
und kehrt er zurück zur Erde, * \\
dann ist es aus mit \d all seinen Plänen. \\
\vin Wohl dem, dessen Halt \\ \vin der Gott Jakobs ist *\\
\vin und der seine Hoffnung auf den\\ \vin H\d errn, seinen Gott, setzt. \\
Der Herr hat Himmel und Erde gemacht, †\\
das Meer und alle Geschöpfe; * \\
er hält \d ewig die Treue.\\
\vin Recht verschafft er den Unterdrückten, †\\
\vin den Hungernden gibt er Brot; *\\
\vin  der Herr befr\d eit die Gefangenen. \\
Der Herr öffnet den Blinden die Augen, *\\
er richt\d et die Gebeugten auf. \\
\vin Der Herr beschützt die Fremden *\\
\vin und verhilft den Waisen \\ \vin und W\d itwen zu ihrem Recht. \\
Der Herr liebt die Gerechten, * \\
doch die Schritte der Frevler leitet\\ \d er in die Irre.\\
\vin Der Herr ist König auf ewig, *\\
\vin dein Gott, Zion, herrscht von G\d eschlecht \\ \vin zu Geschlecht.\\

\end{verse}


\vspace{0.3cm}

\setspaceafterinitial{3.2mm plus 0em minus 0em}
\setspacebeforeinitial{3.2mm plus 0em minus 0em}
\def\greinitialformat#1{{\fontsize{40}{40}\selectfont #1}}
\gresetfirstlineaboveinitial{\small \textcolor{red}{Ps 147a}}{}
\setaboveinitialseparation{0.72mm}
\setsecondannotation{\small 8. T.}

\includescore{psalmi/146/deonostro.tex}
Ant. Unser Lobgesang möge unserem Gott gefallen.

\vspace{0.3cm}

\crot{Psalm 147a}

\begin{verse}[\versewidth]
 Gut ist es, unserm Gott zu singen; *\\
schön ist es, ihn zu loben. \\
\vin Der Herr baut Jerusalem wieder auf, *\\
\vin er sammelt die Versprengten Israels. \\
Er heilt die gebrochenen Herzen *\\
und verbindet ihre schmerzenden Wunden. \\
\vin Er bestimmt die Zahl der Sterne *\\
\vin und ruft sie alle mit Namen. \\
Groß ist unser Herr und gewaltig an Kraft, *\\
unermesslich ist seine Weisheit. \\
\vin Der Herr hilft den Gebeugten auf *\\
\vin und erniedrigt die Frevler.\\
Stimmt dem Herrn ein Danklied an, *\\
spielt unserm Gott auf der Harfe!\\
\vin Er bedeckt den Himmel mit Wolken, †\\
\vin spendet der Erde Regen *\\
 \vin und lässt Gras auf den Bergen sprießen. \\
Er gibt dem Vieh seine Nahrung, *\\
gibt den jungen Raben, wonach sie schreien. \\
\vin Er hat keine Freude an der Kraft des Pferdes, *\\
\vin kein Gefallen am schnellen Lauf des Mannes. \\
Gefallen hat der Herr an denen, die \\ ihn fürchten und ehren, *\\
die voll Vertrauen warten auf seine Huld.\\
\end{verse}



\vspace{0.6cm}

\setspaceafterinitial{3.2mm plus 0em minus 0em}
\setspacebeforeinitial{4.3mm plus 0em minus 0em}
\def\greinitialformat#1{{\fontsize{40}{40}\selectfont #1}}
\gresetfirstlineaboveinitial{\small \textcolor{red}{Ps 147b}}{}
\setaboveinitialseparation{0.72mm}
\setsecondannotation{\small 1. T.}
\includescore{psalmi/147/laudajerusalem.tex}

\vspace{0.3cm}

\crot{Psalm 147b}

\begin{verse}[\versewidth]
\textit{ Jerusalem, preise den Herrn,} *\\
lobsinge, Z\d ion, deinem Gott! \\
\vin Denn er hat die Riegel deiner Tore\\ \vin fest gemacht, *\\
\vin die Kinder in deiner Mitte gesegnet;\\
er verschafft deinen Grenzen Frieden *\\
und sättigt dich mit bestem Weizen. \\
\vin Er sendet sein Wort zur Erde, *\\
\vin rasch eilt s\d ein Befehl dahin. \\
Er spendet Schnee wie Wolle, *\\
streut den Reif aus wie Asche. \\
\vin Eis wirft er herab in Brocken, *\\
\vin vor seiner Kälte erstarren die Wasser.\\
Er sendet sein Wort aus und sie schmelzen, *\\
er lässt den Wind wehen, dann \\rieseln die Wasser.\\
\vin Er verkündet Jakob sein Wort, *\\
\vin Israel seine Gesetze und Rechte. \\
An keinem andern Volk hat er so gehandelt, *\\
keinem sonst seine Rechte verkündet. \\
\end{verse}

\vspace{0.3cm}

\setspaceafterinitial{7.2mm plus 0em minus 0em}
\setspacebeforeinitial{5.2mm plus 0em minus 0em}
\def\greinitialformat#1{{\fontsize{40}{40}\selectfont #1}}
\gresetfirstlineaboveinitial{\small \textcolor{red}{Phil 2,1-6}}{}
\setaboveinitialseparation{0.72mm}
\setsecondannotation{\small 7. T.}

\includescore{cantica/cn/jesusnazarenus.tex}

\cant{Phil 2,6-11}

\begin{verse}[\versewidth]


Christus J\d esus war Gott gleich,*\\
hielt aber nicht daran f\d est, wie Gott zu sein,\\
\vin sondern er entäußerte sich und \\ \vin wurde w\d ie ein Sklave *\\
\vin \d und den Menschen gleich. \\
sein Leben war das eines Menschen; †\\
er erniedrigte sich und \\war geh\d orsam bis zum Tod,* \\
bis zum Tod am Kreuze.\\
\vin Darum hat ihn Gott über \d alle erhöht *\\
\vin und ihm den Namen verliehen, \\
\vin der über alle Namen erhaben ist.\\
damit alle im Himmel, auf der \\Erde und \d unter der Erde*\\
ihre Knie beugen vor den Namen Jesu\\
\vin und jeder Mund bekennt: †\\
\vin „Jesus Chr\d istus ist der Herr!“\\
\vin zur Ehre Gottes des Vaters.\\	

\end{verse}


\vspace{0.3cm}



\setspaceafterinitial{4.2mm plus 0em minus 0em}
\setspacebeforeinitial{4.2mm plus 0em minus 0em}
\resp

\includescore{responsoria_diebusferialibus/respbrmagnusdominus.tex}

\vspace{0.6cm}


 \crot{Hymnus}


\setspaceafterinitial{4.2mm plus 0em minus 0em}
\setspacebeforeinitial{4.2mm plus 0em minus 0em}
\def\greinitialformat#1{{\fontsize{45}{45}\selectfont #1}}
\gresetfirstlineaboveinitial{\small viij. T.}{}
\includescore{hymni/oluxbeata.tex}

\vspace{0.3cm}


\versik{Vespertína orátio ascéndat ad te, Dómine,}{Et descéndat super nos misericórdia tua.}

\vspace{0.3cm}



\setspaceafterinitial{5.2mm plus 0em minus 0em}
\setspacebeforeinitial{4.2mm plus 0em minus 0em}
\def\greinitialformat#1{{\fontsize{40}{40}\selectfont #1}}
\gresetfirstlineaboveinitial{\small \textcolor{red}{Magni.}}{}
\setaboveinitialseparation{0.72mm}
\setsecondannotation{\small viij. T.}

\includescore{magnificatanimamea.tex}


\newpage