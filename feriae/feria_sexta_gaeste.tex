
\kkap{FREITAG}

\section[FREITAG]{AD TERTIAM}

\rot{Hymnus} Nunc sancte nobis Spiritus \rot{p. 124.}

\vspace{0.3cm}

\setspaceafterinitial{5.2mm plus 0em minus 0em}
\setspacebeforeinitial{4.2mm plus 0em minus 0em}
\def\greinitialformat#1{{\fontsize{40}{40}\selectfont #1}}
\gresetfirstlineaboveinitial{\small \textcolor{red}{Ps 119f}}{}
\setaboveinitialseparation{0.72mm}
\setsecondannotation{\small 8. T.}

\includescore{psalmi/118/aspiceinme17.tex}

\includescore{psalmi/tertia/horatertia_sexta.tex}


\vspace{0.3cm}


\begin{flushleft}

\versik{Dóminus non privábit bonis eos qui ámbulant in innocéntia.}{Dómine virtútum, beátus homo qui sperat in te.}

\medskip

{\rm{
\versik{Der Herr wird seine Güter nicht denen vorenthalten, die in Unschuld wandeln.}{Herr der Scharen, selig der Mensch, der auf dich hofft.}
}}
\end{flushleft}




\section{MITTAGESHORE, I. WOCHE}

\rot{Hymnus} Rector potens, verax Deus \rot{p. 126.}

\vspace{0.3cm}

\setspaceafterinitial{5.2mm plus 0em minus 0em}
\setspacebeforeinitial{4.2mm plus 0em minus 0em}
\def\greinitialformat#1{{\fontsize{40}{40}\selectfont #1}}
\setspaceafterinitial{5.2mm plus 0em minus 0em}
\setspacebeforeinitial{4.2mm plus 0em minus 0em}
\def\greinitialformat#1{{\fontsize{40}{40}\selectfont #1}}
\gresetfirstlineaboveinitial{\small \textcolor{red}{Ps  18b }}{}
\setaboveinitialseparation{0.72mm}
\setsecondannotation{\small 7. T.}


\includescore{psalmi/17/vivitdominus171819.tex}
\vspace{0.3cm}

\includescore{psalmi/sexta/sextpsalmisexta.tex}

\begin{flushleft}

\versik{Dómine, veritátem in corde dilexísti.}{Et in occúlto sapiéntiam manifestásti mihi.}

\medskip

{\rm{
\versik{Herr, du liebst, die im Herzen voll Wahrheit sind.}{Und im Verborgenen lehrtest du mich Weisheit.}
}}
\end{flushleft}







\section{MITTAGESHORE, II. WOCHE}

\rot{Hymnus} Rector potens, verax Deus \rot{p. 126.}

\vspace{0.3cm}

 \setspaceafterinitial{5.2mm plus 0em minus 0em}
\setspacebeforeinitial{4.2mm plus 0em minus 0em}
\def\greinitialformat#1{{\fontsize{40}{40}\selectfont #1}}
\gresetfirstlineaboveinitial{\small \textcolor{red}{ Ps 135 }}{}
\setaboveinitialseparation{0.72mm}
\setsecondannotation{\small 3. T.}

\includescore{psalmi/134/omniaquaecumque.tex}
\vspace{0.3cm}
\includescore{psalmi/nona/nonpsalmisexta.tex}

\begin{flushleft}

\versik{Fac cum servo tuo secúndum misericórdiam tuam, Dómine.}{Iustificatiónes tuas doce me.}

\medskip

{\rm{
\versik{Handle an deinem Knecht nach deiner Huld.}{Lehre mich deine Gesetze.}
}}
\end{flushleft}


\section[VESPERAE]{AD VESPERAS}

\setspaceafterinitial{3.2mm plus 0em minus 0em}
\setspacebeforeinitial{4.2mm plus 0em minus 0em}
\def\greinitialformat#1{{\fontsize{40}{40}\selectfont #1}}
\gresetfirstlineaboveinitial{\small \textcolor{red}{Ps 144b}}{}
\setaboveinitialseparation{0.72mm}
\setsecondannotation{\small 8. T.}

\includescore{psalmi/143/beatuspopulus.tex}

\vspace{0.3cm}
\crot{Psalm 144b}
\begin{quote}
 


\begin{verse}

 Ein neues Lied will ich, o Gott, dir singen, *\\
auf der zehnsaitigen Harfe will ich dir spielen,\\ \vin 
der du den Königen den Sieg verleihst *\\ \vin 
und David, deinen Knecht, errettest.\\ 
Vor dem bösen Schwert errette mich, *\\
entreiß mich der Hand der Fremden!\\ \vin 
Alles, was ihr Mund sagt, ist Lüge, *\\ \vin  Meineide schwört ihre Rechte.\\
Unsre Söhne seien wie junge Bäume, *\\
hoch gewachsen in ihrer Jugend, \\ \vin 
unsre Töchter wie schlanke Säulen, *\\ \vin  die geschnitzt sind für den Tempel.\\ 
Unsre Speicher seien gefüllt, *\\
überquellend von vielerlei Vorrat; \\ \vin 
unsre Herden mögen sich tausendfach\\ \vin mehren, *\\ \vin  vieltausendfach auf unsren Fluren.\\ 
Unsre Kühe mögen tragen, ohne zu verwerfen\\  und ohne Unfall; *\\
kein Wehgeschrei werde laut auf unsern Straßen.\\ \vin  
Wohl dem Volk, dem es so ergeht, *\\ \vin 
\textit{glücklich das Volk, dessen Gott der Herr ist!} \\
\end{verse}

\end{quote}
\vspace{0.3cm}

\setspaceafterinitial{8.2mm plus 0em minus 0em}
\setspacebeforeinitial{4.2mm plus 0em minus 0em}
\def\greinitialformat#1{{\fontsize{40}{40}\selectfont #1}}
\gresetfirstlineaboveinitial{\small \textcolor{red}{Ps 145a}}{}
\setaboveinitialseparation{0.72mm}
\setsecondannotation{\small 8. T.}

\includescore{psalmi/144/inaeternum.tex}
\vspace{0.3cm}
\crot{Psalm 145a}

\begin{quote}
\begin{verse}
 Ich will dich rühmen, mein Gott und König, *\\
und deinen Namen preisen \textit{immer und ewig;}\\
\vin ich will dich preisen Tag für Tag *\\
\vin und deinen Namen loben immer und ewig. \\
Groß ist der Herr und hoch zu loben, *\\
seine Größe ist unerforschlich. \\
\vin Ein Geschlecht verkünde dem andern den \\ \vin Ruhm deiner Werke *\\
\vin und erzähle von deinen gewaltigen Taten. \\
Sie sollen vom herrlichen Glanz deiner \\ Hoheit reden; *\\
ich will deine Wunder besingen.\\
\vin Sie sollen sprechen von der Gewalt\\ \vin  deiner erschreckenden Taten; *\\
\vin ich will von deinen großen Taten berichten.\\
Sie sollen die Erinnerung an deine große \\ Güte wecken *\\
und über deine Gerechtigkeit jubeln.\\
\vin Der Herr ist gnädig und barmherzig, *\\
\vin langmütig und reich an Gnade. \\
Der Herr ist gütig zu allen, *\\
sein Erbarmen waltet über all seinen Werken. \\
\vin Danken sollen dir, Herr, all deine Werke *\\
\vin und deine Frommen dich preisen. \\
Sie sollen von der Herrlichkeit deines \\ Königtums reden, *\\
sollen sprechen von deiner Macht, \\
\vin den Menschen deine machtvollen \\ \vin Taten verkünden *\\
\vin und den herrlichen Glanz deines Königtums. \\
Dein Königtum ist ein Königtum für\\ ewige Zeiten, *\\
deine Herrschaft währt von Geschlecht\\ zu Geschlecht\textit{e}.\\ 
\end{verse}

\end{quote}

\vspace{0.3cm}

\setspaceafterinitial{3.2mm plus 0em minus 0em}
\setspacebeforeinitial{4.2mm plus 0em minus 0em}
\def\greinitialformat#1{{\fontsize{40}{40}\selectfont #1}}
\gresetfirstlineaboveinitial{\small \textcolor{red}{Ps 145b}}{}
\setaboveinitialseparation{0.72mm}
\setsecondannotation{\small 4. T.}

\includescore{psalmi/144/custoditdominus.tex}
\vspace{0.3cm}
\crot{Psalm 145b}
\begin{quote}
\begin{verse}
 Der Herr ist treu in all seinen Worten, *\\ voll Huld in \d all seinen Taten\\
\vin Der Herr stützt alle, die fallen, *\\
\vin und richtet \d alle Gebeugten auf. \\
Aller Augen warten auf dich *\\
und du gibst ihnen Sp\d eise zur rechten Zeit. \\
\vin Du öffnest deine Hand *\\
\vin und sättigst alles, was lebt, nach\\ \vin d\d einem Gefallen. \\
Gerecht ist der Herr in allem, was er tut, *\\
voll Huld in \d all seinen Werken. \\
\vin Der Herr ist allen, die ihn anrufen, nahe, *\\
\vin allen, die zu ihm \d aufrichtig rufen. \\
Die Wünsche derer, die ihn fürchten,\\ erfüllt er, *\\
er hört ihr Schr\d eien und rettet sie. \\
\vin \textit{Alle, die ihn lieben, behütet der Herr,} *\\
\vin doch alle Fr\d evler vernichtet er. \\
Mein Mund verkünde das Lob des Herrn. *\\
Alles, was lebt, preise seinen heiligen Namen \\\d immer und ewig! \\
\end{verse}

 \end{quote}





\vspace{0.3cm}

\setspaceafterinitial{4.2mm plus 0em minus 0em}
\setspacebeforeinitial{4.2mm plus 0em minus 0em}
\def\greinitialformat#1{{\fontsize{40}{40}\selectfont #1}}
\gresetfirstlineaboveinitial{\small \textcolor{red}{Ap 15, 3-4}}{}
\setaboveinitialseparation{0.72mm}
\setsecondannotation{\small iv. T.}

\includescore{cantica/cn/omnesgentes.tex}

\cant{Ap 15, 3-4}

\begin{quote}
\begin{verse}


Groß und wunderbar sind deine Taten,*\\
Herr und Gott, du Herrscher über d\d ie ganze\\ Schöpfung!\\
\vin Gerecht und zuverlässig sind deine Wege,*\\
\vin du K\d önig der Völker.\\
Wer wird dich nicht fürchten, Herr, †\\
wer wird  deinen Namen nicht preisen?*\\
Denn du \d allein bist heilig.\\
\vin Ja, \textit{alle Völker werden kommen und\\ \vin  beten dich an},*\\
\vin denn offenbar geworden sind deine\\ \vin  g\d erechten Taten.\\
\end{verse}
\end{quote}


\medskip
 
\noindent{\rot{Resp.br.} Adjutorium nostrum {\rot{et Hymnus}  Deus Creator \rot{vel} Lucis Creator \rot{p. 32 - 34.}}

\vspace{0.3cm}

\medskip


\begin{flushleft}

\versik{Dirigátur, Dómine, orátio mea.}{Sicut incénsum in conspéctu tuo.}

\medskip
{\rm{
\versik{Herr, mein Gebet werde gelenkt.}{Wie Weihrauch vor dein Angesicht.}
}}
\end{flushleft}



\medskip

\setspaceafterinitial{5.2mm plus 0em minus 0em}
\setspacebeforeinitial{4.2mm plus 0em minus 0em}
\def\greinitialformat#1{{\fontsize{40}{40}\selectfont #1}}
\gresetfirstlineaboveinitial{\small \textcolor{red}{Magni}}{}
\setaboveinitialseparation{0.72mm}
\setsecondannotation{\small viij. T.}

\includescore{cantica/evangelica/suscepit.tex}

\medskip

\rot{Canticum} Magnificat \rot{p. 128.}

\vspace{0.5cm}


Nach dem Magnificat folgen die Fürbitten, das Vater Unser und das Tagesgebet.
Das Chorgebet endet mit einer Antiphon zu Ehren der heiligen Jungfrau.

\newpage