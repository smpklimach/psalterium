\thispagestyle{plain}

\kapklein{\rot{F}eria \rot{S}ecunda}
\kkap{FERIA SECUNDA}


\section[VIGILIAE]{AD VIGILIAS}


\setspaceafterinitial{5.2mm plus 0em minus 0em}
\setspacebeforeinitial{4.2mm plus 0em minus 0em}
\def\greinitialformat#1{{\fontsize{40}{40}\selectfont #1}}
\gresetfirstlineaboveinitial{\small \textcolor{red}{Invitat.}}{Invitat.}
\setaboveinitialseparation{0.72mm}
%\setsecondannotation{\small Ps. 125}

%\includescore{invitatoria/veniteii.tex}

\medskip

\begin{sloppypar}
{\noindent{\rot{Invit.} Kommt herbei, * jubelt dem Herrn.}}
\end{sloppypar}

\bigskip

\noindent{\rot{Hymnus} Æterne rerum Conditor \rot{vel} Nocte surgentes \rot{ut in Dominica pp. 1-3.}}

\section{VIGILIA I}

\begin{sloppypar}

{\noindent{1. Ant.} Hebt euch ihr ewigen Tore * und es zieht ein der König der Herrlichkeit. \rot{Ps. 24}\\
2. Ant. Meine Augen * schauen immer auf den Herrn. \rot{Ps. 25}\\
3. Ant. Deine Barmherzigkeit o Herr * steht mir immer vor Augen. \rot{Ps. 26}\\}
\end{sloppypar}

\medskip

\begin{flushleft}

\versik{Der Herr leitet die Demütigen nach seinem Recht.}{Und lehrt die Sanftmütigen seine Wege.}

\end{flushleft}


\section{VIGILIA II}

\begin{sloppypar}
{\noindent{1. Ant.} Du bist unser Helfer * in all unseren Nöten. \rot{Ps. 46}\\
2. Ant. Jubelt Gott zu * mit lauter Stimme. \rot{Ps. 47}\\
3. Ant. Groß ist der Herr *  und hoch zu loben.\\ \rot{Ps. 48}}
\end{sloppypar}

\begin{flushleft}

\versik{Erleuchte, Herr, meine Augen.}{Und ich denke nach über die Wunder deines Gesetzes.}

\end{flushleft}


\section{VIGILIA III}

\begin{sloppypar}

{\noindent{1. Ant.} Sei mir, Herr, * ein schützender Schild.\\ \rot{Ps. 71}\\
2. Ant. In ihm jubeln alle Stämme der Erde, * alle Völker preisen ihn. \rot{Ps. 72}\\}
\end{sloppypar}

\medskip

\begin{flushleft}

\versik{Mein Sohn, achte auf meine Weisung.}{Ebne dem Herrn die Wege.}

\end{flushleft}


\section{VIGILIA IV}

\begin{sloppypar}

{\noindent{1. Ant.} Gepriesen sei der Herr * in Ewigkeit.\\ \rot{Ps. 89, 20-53}\\
2. Ant. Wunderbar bist du, Herr, * in der Höhe.\\ \rot{Ps. 93}\\
3. Ant. Erhebe dich in deiner Macht * der du richtest die Erde. \rot{Ps. 94}\\}
\end{sloppypar}

\medskip

\begin{flushleft}

\versik{Gib mir Einsicht, damit ich deine Gebote halte.}{Und sie bewahre von ganzem Herzen.}

\end{flushleft}


\section[LAUDES]{AD LAUDES}

\setspaceafterinitial{4.2mm plus 0em minus 0em}
\setspacebeforeinitial{4.2mm plus 0em minus 0em}
\def\greinitialformat#1{{\fontsize{40}{40}\selectfont #1}}
\gresetfirstlineaboveinitial{\small \textcolor{red}{ Ps 3}}{}
\setaboveinitialseparation{0.72mm}
\setsecondannotation{\small ij. T.}

\includescore{psalmi/3/ps3.tex}

\vspace{0.3cm}

\psal{3}

\begin{quote}
\begin{verse}
Herr, wie zahlreich sind meine Bedränger; *\\
so viele stehen gegen mich auf.\\ 
\vin Viele gibt es, die von mir sagen: *\\ 
\vin «Er findet keine Hilfe bei Gott.» \\
Du aber, Herr, bist ein Schild für mich, *\\
du bist meine Ehre und richtest mich auf.\\ 
\vin \textit{Ich habe laut zum Herrn gerufen; *\\ 
\vin da erhörte er mich von seinem \\
\vin heiligen Berg}e.\\ 
Ich lege mich nieder und schlafe ein, *\\
ich wache wieder auf, denn\\
der Herr beschützt mich.\\ 
\vin Viele Tausende von Kriegern\\ 
\vin fürchte ich nicht, *\\ 
\vin wenn sie mich ringsum belagern.\\ 
Herr, erhebe dich, *\\
mein Gott, bring mir Hilfe! \\ 
\vin Denn all meinen Feinden hast du \\ 
\vin den Kiefer zerschmettert, *\\ 
\vin hast den Frevlern die Zähne zerbrochen.\\
Beim Herrn findet man Hilfe.*\\
Auf dein Volk komme dein Segen! \\
\end{verse}
\end{quote}

\vspace{0.3cm}
\setspaceafterinitial{5.2mm plus 0em minus 0em}
\setspacebeforeinitial{4.2mm plus 0em minus 0em}
\def\greinitialformat#1{{\fontsize{40}{40}\selectfont #1}}
\gresetfirstlineaboveinitial{\small \textcolor{red}{Ps 5}}{}
\setaboveinitialseparation{0.72mm}
\setsecondannotation{\small viij. T.}
\includescore{psalmi/5/ps5.tex}

\vspace{0.3cm}

\psal{5}

\begin{quote}
\begin{verse}
 \textit{Höre meine Worte, Herr,} *\\ 
achte auf mein Seufzen!\\ 
\vin Vernimm mein lautes Schreien,\\ 
\vin mein König und mein Gott, *\\ 
\vin denn ich flehe zu dir.\\  
Herr, am Morgen hörst du mein Rufen, *\\ 
am Morgen rüst ich das Opfer zu, \\
halte Ausschau nach dir.\\ 
\vin Denn du bist kein Gott, dem das\\ 
\vin Unrecht gefällt; *\\ 
\vin der Frevler darf nicht bei dir weilen.\\  
Wer sich brüstet, besteht nicht\\
vor deinen Augen; *\\ 
denn dein Hass trifft alle, die Böses tun.\\ 
\vin Du lässt die Lügner zugrunde gehn, *\\ 
\vin Mörder und Betrüger sind dem Herrn\\ 
\vin ein Gräuel.\\ 
Ich aber darf dein Haus betreten *\\ 
dank deiner großen Güte,\\ 
\vin ich werfe mich nieder in Ehrfurcht *\\ 
\vin vor deinem heiligen Tempel.\\ 
Leite mich, Herr, in deiner Gerechtigkeit, †\\
meinen Feinden zum Trotz; *\\  
ebne deinen Weg vor mir!\\ 
\vin Aus ihrem Mund kommt kein wahres Wort, *\\ 
\vin ihr Inneres ist voll Verderben.\\  
Ihre Kehle ist ein offenes Grab, *\\  
aalglatt ist ihre Zunge.\\ 
\vin Gott, lass sie dafür büßen; *\\ 
\vin sie sollen fallen durch ihre eigenen Ränke.\\  
Verstoße sie wegen ihrer vielen Verbrechen; *\\  
denn sie empören sich gegen dich.\\ 
\vin Doch alle sollen sich freuen, \\ 
\vin die auf dich vertrauen, *\\ 
\vin und sollen immerfort jubeln.\\  
Beschütze alle, die deinen Namen lieben, *\\  
damit sie dich rühmen.\\ 
\vin Denn du, Herr, segnest den Gerechten. *\\ 
\vin Wie mit einem Schild deckst du ihn\\ 
\vin mit deiner Gnade.\\ 
\end{verse}
\end{quote}

\newpage

\setspaceafterinitial{5.2mm plus 0em minus 0em}
\setspacebeforeinitial{4.2mm plus 0em minus 0em}
\def\greinitialformat#1{{\fontsize{40}{40}\selectfont #1}}
\gresetfirstlineaboveinitial{\small \textcolor{red}{1 Chr 29}}{}
\setaboveinitialseparation{0.72mm}
\setsecondannotation{\small j. T.}
\includescore{cantica/ca/1chr29.tex}


\cant{1 Chr 29,10-13}

\begin{verse}[\versewidth]

Gepriesen bist du, Herr †\\
Gott unseres Vaters Israel, *\\
von Ewigkeit zu Ewigkeit!\\
\vin Dein, Herr, sind Größe und Kraft, †\\
\vin Ruhm und Glanz und Hoheit; *\\
dein ist alles im Himmel und auf Erden.\\
\vin Herr, dein ist das Königtum. *\\
\vin Du erhebst dich als Haupt über alles.\\
Reichtum und Ehre kommen von dir; *\\
du bist der Herrscher über das All.\\
\vin In deiner Hand liegen Kraft und Stärke; *\\
\vin von deiner Hand kommt alle Größe\\ 
\vin und Macht.\\
Darum danken \textit{wir} dir\textit{, unser Gott,} *\\
und \textit{rühmen deinen herrlichen Namen.}\\

\end{verse}


\rot{vel}


\setspaceafterinitial{5.2mm plus 0em minus 0em}
\setspacebeforeinitial{4.2mm plus 0em minus 0em}
\def\greinitialformat#1{{\fontsize{40}{40}\selectfont #1}}
\gresetfirstlineaboveinitial{\small \textcolor{red}{Sir 36}}{}
\setaboveinitialseparation{0.72mm}
\setsecondannotation{\small 3. T.}
\includescore{cantica/ca/sir36.tex}

\medskip

\begin{sloppypar}
{\noindent\rm{Ant. Zeige uns, Herr, das Licht deiner Barmherzigkeit.}} 
\end{sloppypar}

\medskip
\medskip



\cant{Sir 36,1-7.13-16}

\begin{quote}
\begin{verse}
Rette uns, du G\d ott des Alls, *\\  
und wirf deinen Schrecken auf alle Völker!\\ 
\vin Schwing deine Hand\\ 
\vin g\d egen das fremde Volk, *\\ 
\vin damit es deine mächtigen Taten sieht.\\ 
Wie du dich an uns vor ihren Augen\\
als h\d eilig bezeugt hast, *\\ 
so verherrliche dich an ihnen vor unseren Augen,\\ 
\vin damit sie erkennen, wie w\d ir  es erkannten: *\\ 
\vin Es gibt keinen Gott außer dir.\\ 
Erneuere die Zeichen, wiederh\d ole die Wunder, *\\ 
zeige die Macht deiner Hand\\
und die Kraft deines rechten Armes!\\ 
\vin Sammle alle St\d ämme Jakobs, *\\ 
\vin verteil den Erbbesitz wie in den\\ 
\vin Tagen der Vorzeit!\\  
Hab Erbarmen mit dem Volk, \\
das d\d einen Namen trägt, *\\ 
mit Israel, den du deinen Erstgeborenen\\
nanntest.\\ 
\vin Hab Erbarmen mit deiner h\d eiligen Stadt, *\\ 
\vin mit Jerusalem, dem Ort, wo du wohnst.\\ 
Erfülle Zion mit d\d einem Glanz *\\ 
und deinen Tempel mit deiner Herrlichkeit!\\ 
\end{verse}
\end{quote}

\medskip

\setspaceafterinitial{5.2mm plus 0em minus 0em}
\setspacebeforeinitial{4.2mm plus 0em minus 0em}
\def\greinitialformat#1{{\fontsize{40}{40}\selectfont #1}}
\gresetfirstlineaboveinitial{\small \textcolor{red}{Ps 36}}{}
\setaboveinitialseparation{0.72mm}
\setsecondannotation{\small viij. T.}

% and finally we include the score. The file must be in the same directory as this one.
\includescore{psalmi/35/ps35.tex}


\medskip
\smallskip


\psal{36}

\begin{quote}
\begin{verse}
Der Frevler spricht:\\
«Ich bin entschlossen zum Bösen.» *\\
In seinen Augen gibt es kein\\
Erschrecken vor Gott.\\ 
\vin Er gefällt sich darin, *\\ 
\vin sich schuldig zu machen und zu hassen.\\
Die Worte seines Mundes sind\\
Trug und Unheil; *\\
er hat es aufgegeben, weise und gut zu handeln.\\ 
\vin Unheil plant er auf seinem Lager, †\\ 
\vin er betritt schlimme Wege *\\ 
\vin und scheut nicht das Böse.\\ 
\textit{Herr, deine Güte reicht,\\
so weit der Himmel ist}, *\\
deine Treue, so weit die Wolken zieh\textit{e}n.\\ 
\vin Deine Gerechtigkeit steht wie\\ 
\vin die Berge Gottes, *\\ 
\vin deine Urteile sind tief wie das Meer.\\
Herr, du hilfst Menschen und Tieren. *\\
Gott, wie köstlich ist deine Huld! \\ 
\vin Die Menschen bergen sich\\ 
\vin im Schatten deiner Flügel, †\\ 
\vin sie laben sich am Reichtum deines Hauses; *\\ 
\vin du tränkst sie mit dem Strom deiner Wonnen.\\ 
Denn bei dir ist die Quelle des Lebens, *\\
in deinem Licht schauen wir das Licht.\\ 
\vin Erhalte denen, die dich kennen, deine Huld *\\ 
\vin und deine Gerechtigkeit den Menschen\\ 
\vin mit redlichem Herzen!\\
Lass mich nicht kommen unter den Fuß\\
der Stolzen; *\\
die Hand der Frevler soll mich nicht vertreiben.\\
\vin Dann brechen die Bösen zusammen, *\\ 
\vin sie werden niedergestoßen und können \\ 
\vin nie wieder aufstehn.\\ 

\end{verse}
\end{quote}

\medskip

\setspaceafterinitial{4.2mm plus 0em minus 0em}
\setspacebeforeinitial{4.2mm plus 0em minus 0em}
\resp

\includescore{responsoria_diebusferialibus/respbrsana.tex}

\medskip

\begin{sloppypar}
{\noindent\rm{\rot{Resp.} Heile meine Seele, denn ich habe vor dir gesündigt. Ich habe gesagt: Herr, erbarme dich meiner.}}
\end{sloppypar}

\medskip
\bigskip

\noindent{\rot{Hymnus} Splendor paternæ \rot{vel} Ecce iam noctis \rot{ut in Dominica p. 15 - 16.}}

\medskip

\begin{flushleft}

\versik{Repléti sumus mane misericórdia tua.}{Exultávimus, et delectáti sumus.}

\medskip
{\rm{
\versik{Erfüllt sind wir am Morgen mit deiner Huld.}{Wir sind ausgelassen und fröhlich.}
}}
\end{flushleft}



\setspaceafterinitial{4.2mm plus 0em minus 0em}
\setspacebeforeinitial{4.2mm plus 0em minus 0em}
\def\greinitialformat#1{{\fontsize{40}{40}\selectfont #1}}
\gresetfirstlineaboveinitial{\footnotesize \textcolor{red}{Bendic}}{}
\setaboveinitialseparation{0.72mm}
\setsecondannotation{\small vij. T.}

\includescore{cantica/evangelica/erexitdominus.tex}

\smallskip

\rot{Canticum} Benedictus \rot{p. 196.}



\section[HORA TERTIA]{AD TERTIAM}

\rot{Hymnus} Nunc sancte nobis Spiritus \rot{p. 190.}
\vspace{0.3cm}

\setspaceafterinitial{5.2mm plus 0em minus 0em}
\setspacebeforeinitial{4.2mm plus 0em minus 0em}
\def\greinitialformat#1{{\fontsize{40}{40}\selectfont #1}}
\gresetfirstlineaboveinitial{\small \textcolor{red}{Ps 119b}}{}
\setaboveinitialseparation{0.72mm}
\setsecondannotation{\small ij. T.}
\includescore{psalmi/118/118vdeducmedomine.tex}

\newpage



\medskip

\includescore{psalmi/tertia/horatertia_secunda.tex}


\begin{flushleft}

\versik{Adiútor meus es tu, ne me reícias.}{Neque derelínquas me, Deus salútis meæ.}

\medskip

{\rm{
\versik{Du bist mein Helfer, verstoß mich nicht.}{Und lass mich nicht im Stich, Gott, meines Heiles.}
}}
\end{flushleft}

\section[HORA SEXTA]{AD SEXTAM}

\rot{Hymnus} Rector potens, verax Deus \rot{p. 192.}

\vspace{0.3cm}

\setspaceafterinitial{5.2mm plus 0em minus 0em}
\setspacebeforeinitial{4.2mm plus 0em minus 0em}
\def\greinitialformat#1{{\fontsize{40}{40}\selectfont #1}}
\gresetfirstlineaboveinitial{\small \textcolor{red}{ Ps 7 8 9a}}{}
\setaboveinitialseparation{0.72mm}
\setsecondannotation{\small j. T.}

\includescore{psalmi/7/ps7dominedeus789.tex}



\includescore{psalmi/sexta/sextpsalmisecunda}

\begin{flushleft}

\versik{Benedícam Dóminum in omni témpore.}{Semper laus eius in ore meo.}

\medskip
{\rm{
\versik{Ich will den Herrn allezeit preisen.}{Immer sei sein Lob in meinem Mund.}
}}
\end{flushleft}

\section[HORA NONA]{AD NONAM}

\rot{Hymnus} Rerum Deus tenax vigor \rot{p. 194.}

\vspace{0.3cm}

 \setspaceafterinitial{4.2mm plus 0em minus 0em}
\setspacebeforeinitial{4.2mm plus 0em minus 0em}
\def\greinitialformat#1{{\fontsize{40}{40}\selectfont #1}}
\gresetfirstlineaboveinitial{\small \textcolor{red}{Ps  123sq }}{}
\setaboveinitialseparation{0.72mm}
\setsecondannotation{\small viij. T.}

\includescore{psalmi/122/quihabitas.tex}

\medskip

\includescore{psalmi/nona/nonpsalmisecunda}


\begin{flushleft}

\versik{Rédime me, Dómine, et miserére mei.}{In ecclésiis benedícam Dómino.}

\medskip
{\rm{
\versik{Erlöse mich und sei mir gnädig.}{Den Herrn will ich preisen in der Gemeinde.}
}}
\end{flushleft}



\section[VESPERAE]{AD VESPERAS}

\setspaceafterinitial{4.2mm plus 0em minus 0em}
\setspacebeforeinitial{4.2mm plus 0em minus 0em}
\def\greinitialformat#1{{\fontsize{40}{40}\selectfont #1}}
\gresetfirstlineaboveinitial{\small \textcolor{red}{Ps 113}}{}
\setaboveinitialseparation{0.72mm}
\setsecondannotation{\small 7. T.}

\includescore{psalmi/112/sitnomendomini.tex}
\vspace{0.3cm}

\psal{113}

\begin{quote}

\begin{verse}
Lobet, ihr Kn\d echte des Herrn, *\\
lobt den N\d amen des Herr\textit{e}n!\\ 
\vin Der \textit{ Name des H\d errn sei gepriesen} *\\ 
\vin von nun an b\d is \textit{in Ewigkeit.}\\
Vom Aufgang der Sonne b\d is zum Untergang *\\
sei der Name des H\d err{\textit{e}}n gelobt.\\ 
\vin Der Herr ist erhaben über \d alle Völker, *\\ 
\vin seine Herrlichkeit überr\d agt die Himmel.\\ 
Wer gleicht dem H\d errn, unserm Gott, *\\
im Himmel \d und auf Erden,\\ 
\vin ihm, der \d in der Höhe thront, *\\ 
\vin der hinabschaut \d in die Tiefe, \\
der den Schwachen aus dem St\d aub emporhebt *\\
und den Armen erhöht, d\d er im Schmutz liegt?\\ 
\vin Er gibt ihm einen S\d itz bei den Edlen, *\\ 
\vin bei den Edlen s\d eines Volkes.\\ 
Die Frau, die kinderlos war, lässt er\\
im H\d ause wohnen; *\\
sie wird Mutter und freut sich an \d ihren Kindern.\\ 

\end{verse}
\end{quote}

\vspace{0.3cm}


\setspaceafterinitial{4.2mm plus 0em minus 0em}
\setspacebeforeinitial{4.2mm plus 0em minus 0em}
\def\greinitialformat#1{{\fontsize{40}{40}\selectfont #1}}
\gresetfirstlineaboveinitial{\small \textcolor{red}{Ps 114}}{}
\setaboveinitialseparation{0.72mm}
\setsecondannotation{\small 1. T. irr.}

\includescore{psalmi/113/afacie.tex}
\vspace{0.3cm}



\psal{114}

\begin{quote}
\begin{verse}
Als Israel aus Ägypten \d auszog, *\\
Jakobs Haus aus dem Volk mit fr\d emder Sprache, \\ 
\vin da wurde Juda Gottes H\d eiligtum, *\\ 
\vin Israel das Gebiet s\d einer Herrschaft.\\  
Das Meer sah \d es und floh, *\\
der Jord\d an wich zurück.\\ 
\vin Die Berge hüpften wie W\d idder, *\\ 
\vin die Hügel wie j\d unge Lämmer.\\  
Was ist mit dir, Meer, dass \d du fliehst, *\\
und mit dir, Jordan, dass \d du zurückweichst?\\ 
\vin Ihr Berge, was hüpft ihr wie W\d idder, *\\ 
\vin und ihr Hügel, wie j\d unge Lämmer?\\  
Vor dem Herrn \textit{erbebe, du \d Erde, *\\
vor dem Antlitz des G\d ottes} Jakobs,\\ 
\vin der den Fels zur Wasserflut w\d andelt *\\ 
\vin und Kieselgestein zu quell\d endem Wasser.\\  
\end{verse}
\end{quote}

\newpage

\setspaceafterinitial{7.2mm plus 0em minus 0em}
\setspacebeforeinitial{4.2mm plus 0em minus 0em}
\def\greinitialformat#1{{\fontsize{40}{40}\selectfont #1}}
\gresetfirstlineaboveinitial{\small \textcolor{red}{Ps 115}}{}
\setaboveinitialseparation{0.72mm}
\setsecondannotation{\small 1. T. irr.}

\includescore{psalmi/114/nosquivivimus.tex}

\vspace{0.3cm}

\psal{115}

\begin{quote}
\begin{verse}
Nicht uns, o Herr, bring zu Ehr\d en, †\\
nicht uns, sondern deinen N\d amen, *\\  
in deiner H\d uld und Treue! \\ 
\vin Warum sollen die Völker s\d agen: *\\ 
\vin «Wo \d ist denn ihr Gott?» \\
Unser Gott ist im H\d immel; *\\ 
alles, was ihm gefällt, d\d as vollbringt er.\\ 
\vin Die Götzen der Völker sind nur \\ 
\vin Silber \d und Gold, *\\ 
\vin ein Machw\d erk von Menschenhand.\\ 
Sie haben einen Mund und r\d eden nicht, *\\ 
Aug\d en und sehen nicht;\\ 
\vin sie haben Ohren und h\d ören nicht, *\\ 
\vin eine Nas\d e und riechen nicht;\\
mit ihren Händen können sie nicht greif\d en, †\\
mit den Füßen nicht g\d ehen, *\\  
sie bringen keinen Laut hervor aus \d ihrer Kehle.\\ 
\vin Die sie gemacht haben, sollen\\ 
\vin ihrem Machwerk gl\d eichen, *\\ 
\vin alle, die den Götz\d en vertrauen.\\ 
Israel, vertrau auf den H\d err\textit{e}n! *\\ 
Er ist für euch H\d elfer und Schild.\\ 
\vin Haus Aaron, vertrau auf den H\d err\textit{e}n! *\\ 
\vin Er ist für euch H\d elfer und Schild.\\ 
Alle, die ihr den Herrn fürchtet, \\
vertraut auf den H\d err\textit{e}n! *\\ 
Er ist für euch H\d elfer und Schild.\\ 
\vin Der Herr denkt an uns, er wird uns segn\d en, †\\ 
\vin er wird das Haus Israel s\d egnen, *\\ 
\vin  er wird das Haus \d Aaron segnen.\\ 
Der Herr wird alle segnen, die ihn f\d ürchten, *\\ 
segnen Klein\d e und Große.\\ 
\vin Es mehre euch d\d er Herr, *\\ 
\vin euch und \d eure Kinder.\\  
Seid gesegnet vom H\d err\textit{e}n, *\\ 
der Himmel und Erd\d e gemacht hat.\\ 
\vin Der Himmel ist der Himmel des H\d err\textit{e}n, *\\ 
\vin die Erde aber gab \d er den Menschen.\\  
Tote können den Herrn nicht mehr l\d oben, *\\ 
keiner, der ins Schweig\d en hinabfuhr.\\ 
\vin \textit{Wir aber preisen den H\d err\textit{e}n} *\\ 
\vin von nun an b\d is in Ewigkeit.\\ 

\end{verse}
\end{quote}

\medskip
\vspace{0.3cm}
\setspaceafterinitial{5.2mm plus 0em minus 0em}
\setspacebeforeinitial{4.2mm plus 0em minus 0em}
\def\greinitialformat#1{{\fontsize{40}{40}\selectfont #1}}
\gresetfirstlineaboveinitial{\small \textcolor{red}{ Eph 1}}{}
\setaboveinitialseparation{0.72mm}
\setsecondannotation{\small 3. T.}

\includescore{cantica/cn/sanguisjesu.tex}

\newpage

\cant{Eph 1,3-10}

\begin{quote}
\begin{verse}
Gepr\d iesen sei Gott,*\\
der Gott und Vater unseres Herrn\\
Jesus Christus:\\
\vin Er hat uns mit allem Segen \\ 
\vin seines G\d eistes gesegnet *\\
\vin durch unsere Gemeinschaft mit Christus\\ 
\vin im Himmel.\\
Denn in ihm hat er uns erwählt vor der\\
Ersch\d affung der Welt,*\\
damit wir heilig und untadelig leben vor Gott;\\
\vin er hat uns aus Liebe\\ 
\vin im Voraus d\d azu bestimmt,*\\
\vin seine Söhne zu werden \textit{durch Jesus Christus,}\\
und nach seinem gnädigen Willen zu \d ihm \\
zu gelangen, *\\
zum Lob seiner herrlichen Gnade.\\
\vin Er hat sie uns geschenkt in seinem geliebten\\ 
\vin Sohn; †\\
\vin \textit{durch sein Blut haben wir} die Erlösung \\ 
\vin \textit{die Verg\d ebung der Sünden},*\\
\vin nach dem Reichtum seiner Gnade. \\
Durch sie hat er uns r\d eich beschenkt *\\
mit aller Weisheit und Einsicht \\
\vin und hat uns das Geheimnis seines\\ 
\vin W\d illens kundgetan,* \\
\vin wie er es gnädig im Voraus bestimmt hat: \\
Er hat beschlossen, \\
die Fülle der Zeiten heraufzuführen, †\\
in Christus \d alles zu vereinen, *\\
alles, was im Himmel und auf Erden ist. \\!

\end{verse}
\end{quote}

\medskip



\setspaceafterinitial{4.2mm plus 0em minus 0em}
\setspacebeforeinitial{4.2mm plus 0em minus 0em}
\resp

\includescore{responsoria_diebusferialibus/respbradiutorium.tex}

\medskip

\begin{sloppypar}
{\noindent\rm{\rot{Resp.} Unsere Hilfe ist im Namen des Herrn. Der Himmel und Erde gemacht hat.}}
\end{sloppypar}

\vspace{0.3cm}

\noindent{\rot{Hymnus} Deus Creator \rot{vel} Lucis Creator \rot{ ut in Dominica p. 32 -33}

\vspace{0.3cm}

\begin{flushleft}

\versik{Dirigátur, Dómine, orátio mea.}{Sicut incénsum in conspéctu tuo.}

\medskip
{\rm{
\versik{Lenke, Herr, mein Gebet.}{Wie Weihrauch zu deinem Angesicht.}
}}
\end{flushleft}

\medskip



\setspaceafterinitial{5.2mm plus 0em minus 0em}
\setspacebeforeinitial{4.2mm plus 0em minus 0em}
\def\greinitialformat#1{{\fontsize{40}{40}\selectfont #1}}
\gresetfirstlineaboveinitial{\footnotesize \textcolor{red}{Magni}}{}
\setaboveinitialseparation{0.72mm}
\setsecondannotation{\small {\textcolor{white}{v}}v. T.}

\includescore{cantica/evangelica/exultet.tex}
\vspace{0.3cm}

\rot{Canticum} Magnificat \rot{p. 200.}



\newpage