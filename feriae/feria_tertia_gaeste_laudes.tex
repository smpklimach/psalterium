\kkap{DIENSTAG}

\section[DIENSTAG]{AD LAUDES}

\setspaceafterinitial{5.2mm plus 0em minus 0em}
\setspacebeforeinitial{4.2mm plus 0em minus 0em}
\def\greinitialformat#1{{\fontsize{40}{40}\selectfont #1}}
\gresetfirstlineaboveinitial{\small \textcolor{red}{ Ps 43}}{}
\setaboveinitialseparation{0.72mm}
\setsecondannotation{\small vj. T.}

\includescore{psalmi/42/ps42.tex}

\vspace{0.6cm}
\psal{43}

\begin{quote}
\begin{verse}
Verschaff mir Recht, o Gott, †\\
und führe meine Sache gegen ein \\
treuloses Volk! *\\  
Rette mich vor bösen und t\d ückischen Menschen!\\ 
\vin Denn du bist mein starker Gott. *\\ 
\vin Warum hast d\d u mich verstoßen? \\
Warum muss ich trauernd umhergehen, *\\  
v\d on meinem Feind bedrängt?\\ 
\vin Sende dein Licht und deine Wahrheit, *\\ 
\vin dam\d it sie mich leiten; \\
sie sollen mich führen zu deinem heiligen Berg *\\  
und z\d u deiner Wohnung. \\ 
\vin So will ich zum Altar Gottes treten,\\ 
\vin zum Gott meiner Freude. *\\ 
\vin Jauchzend will ich dich auf der Harf\d e loben,\\ 
\vin Gott, mein Gott. \\
Meine Seele, warum bist du betrübt *\\ 
und bist s\d o unruhig in mir? \\ 
\vin Harre auf Gott; denn\\ 
\vin ich werde ihm noch danken, *\\ 
\vin  meinem \textit{Gott und Retter,\\ 
\vin \d auf den ich schaue.}\\

\end{verse}
\end{quote}


\vspace{0.6cm}
\setspaceafterinitial{5.2mm plus 0em minus 0em}
\setspacebeforeinitial{4.2mm plus 0em minus 0em}
\def\greinitialformat#1{{\fontsize{40}{40}\selectfont #1}}
\gresetfirstlineaboveinitial{\small \textcolor{red}{Ps 57}}{}
\setaboveinitialseparation{0.72mm}
\setsecondannotation{\small viij. T.}
\includescore{psalmi/56/ps56.tex}

\vspace{0.6cm}
\psal{57}

\begin{quote}
\begin{verse}
 Sei mir gnädig, o Gott, sei mir gnädig; *\\
\textit{denn ich flüchte mich zu dir.}\\ 
\vin Im Schatten deiner Flügel\\ 
\vin finde ich Zuflucht, *\\ 
\vin bis das Unheil vorübergeht.\\ 
Ich rufe zu Gott, dem Höchsten, *\\
zu Gott, der mir beisteht.\\ 
\vin Er sende mir Hilfe vom Himmel; †\\ 
\vin meine Feinde schmähen mich. *\\ 
\vin Gott sende seine Huld und Treue.\\
Ich muss mich mitten unter Löwen lagern, *\\
die gierig auf Menschen sind.\\ 
\vin Ihre Zähne sind Spieße und Pfeile, *\\ 
\vin ein scharfes Schwert ihre Zunge.\\ 
Sie haben meinen Schritten ein Netz gelegt *\\ 
und meine Seele gebeugt.\\ 
\vin Sie haben mir eine Grube gegraben; *\\ 
\vin doch fielen sie selbst hinein.\\
Erheb dich über die Himmel, o Gott! *\\
Deine Herrlichkeit erscheine\\
über der ganzen Erde.\\
\vin Mein Herz ist bereit, o Gott, †\\ 
\vin mein Herz ist bereit, *\\ 
\vin ich will dir singen und spielen.\\ 
Wach auf, meine Seele! †\\
Wacht auf, Harfe und Saitenspiel! *\\ 
Ich will das Morgenrot wecken.\\ 
\vin Ich will dich vor den Völkern preisen, Herr, *\\ 
\vin dir vor den Nationen lobsingen.\\ 
Denn deine Güte reicht,\\
so weit der Himmel ist, *\\
deine Treue, so weit die Wolken ziehn.\\ 
\vin Erheb dich über die Himmel, o Gott; *\\ 
\vin deine Herrlichkeit erscheine\\ 
\vin über der ganzen Erde.\\

\end{verse}
\end{quote}


\vspace{0.6cm}

\setspaceafterinitial{5.2mm plus 0em minus 0em}
\setspacebeforeinitial{4.2mm plus 0em minus 0em}
\def\greinitialformat#1{{\fontsize{40}{40}\selectfont #1}}
\gresetfirstlineaboveinitial{\small \textcolor{red}{Tob 13}}{}
\setaboveinitialseparation{0.72mm}
\setsecondannotation{\small vij. T.}
\includescore{cantica/ca/tob13.tex}


\cant{Tob 13,2-5.7-9}

\begin{quote}
\begin{verse}
Gepriesen sei Gott, der in \d Ewigkeit lebt, *\\
sein Königtum s\d ei gepriesen.\\ 
\vin Er züchtigt und hat auch wieder Erbarmen; †\\ 
\vin er führt hinab in die Unterwelt \\ 
\vin und führt auch w\d ieder zum Leben.  *\\ 
\vin Niemand kann seiner M\d acht entfliehen.\\ 
Bekennt euch zu ihm vor allen Völkern,\\
ihr K\d inder Israels; *\\ 
denn er selbst hat uns unter die V\d ölker zerstreut.\\ 
\vin Verkündet dort seine erh\d abene Größe, *\\ 
\vin preist ihn laut vor \d allem, was lebt.\\ 
Denn er ist \d unser Herr und Gott, *\\ 
er ist unser Vater in \d alle Ewigkeit.\\ 
\vin Er züchtigt uns wegen \d unserer Sünden, *\\ 
\vin doch hat er auch w\d ieder Erbarmen.\\ 
Wenn ihr dann seht, was er für euch tut,  †\\
bekennt euch laut und \d offen zu ihm! *\\ 
Preist den Herrn der Gerechtigkeit,\\
\textit{rühmt den \d ewigen König!}\\ 
\vin Ich bekenne mich zum Herrn im L\d and \\ 
\vin der Verbannung, *\\ 
\vin ich bezeuge den Sündern seine Macht\\ 
\vin und erh\d abene Größe.\\ 
Kehrt um, ihr Sünder,  †\\
tut, was recht ist in s\d einen Augen. *\\ 
Vielleicht ist er gnädig und hat mit \d euch\\
Erbarmen.\\ 
\vin Ich will meinen Gott rühmen,\\ 
\vin den K\d önig des Himmels, *\\ 
\vin meine Seele freut sich\\
\vinüber die erhabene Größe m\d eines Gottes.\\

\end{verse}
\end{quote}



\rot{vel}


\setspaceafterinitial{5.2mm plus 0em minus 0em}
\setspacebeforeinitial{4.2mm plus 0em minus 0em}
\def\greinitialformat#1{{\fontsize{40}{40}\selectfont #1}}
\gresetfirstlineaboveinitial{\small \textcolor{red}{Is 38}}{}
\setaboveinitialseparation{0.72mm}
\setsecondannotation{\small j. T.}
\includescore{cantica/ca/is38.tex}

\cant{Is 38,10-14.17-20}

\begin{quote}
\begin{verse}
Ich sagte: In der Mitte meiner Tage †\\
muss ich hinab zu den Pforten der Unterwelt, *\\
man raubt mir den Rest meiner Jahre.\\
\vin Ich sagte: Ich darf den Herrn nicht mehr\\ 
\vin schauen im Land der Lebenden,*\\
\vin keinen Menschen mehr sehen bei den\\ 
\vin Bewohnern der Erde.\\
Meine Hütte bricht man über mir ab,*\\
man schafft sie weg wie das Zelt eines Hirten.\\
\vin Wie ein Weber hast du mein Leben \\ 
\vin zu Ende gewoben, *\\
\vin du schneidest mich ab\\ 
\vin wie ein fertig gewobenes Tuch.\\
Vom Anbruch des Tages bis in die Nacht\\
gibst du mich preis; *\\
bis zum Morgen schreie ich um Hilfe.\\
\vin Meine Augen blicken ermattet nach oben: *\\
\vin Ich bin in Not, Herr. Steh mir bei!\\
\textit{Du hast mich\\
aus meiner bitteren Not gerettet, †\\
du hast mich\\
vor dem tödlichen Abgrund bewahrt}; *\\
denn all meine Sünden\\
warfst du hinter deinen Rücken.\\
\vin Ja, in der Unterwelt dankt man dir nicht, †\\ 
\vin die Toten loben dich nicht; *\\
\vin wer ins Grab gesunken ist,\\ 
\vin kann nichts mehr von deiner Güte erhoffen.\\
Nur die Lebenden danken dir,\\
wie ich am heutigen Tag. *\\
Von deiner Treue erzählt der Vater den Kindern.\\
\vin Der Herr war bereit, mir zu helfen. *\\
\vin Wir wollen singen und spielen im Haus des\\ \vin  Herrn, 
solange wir leben! \\!

\end{verse}
\end{quote}

\medskip

\setspaceafterinitial{5.2mm plus 0em minus 0em}
\setspacebeforeinitial{4.2mm plus 0em minus 0em}
\def\greinitialformat#1{{\fontsize{40}{40}\selectfont #1}}
\gresetfirstlineaboveinitial{\small \textcolor{red}{Ps 63}}{}
\setaboveinitialseparation{0.72mm}
\setsecondannotation{\small vij. T.}


\includescore{psalmi/62/ps62.tex}

\vspace{0.5cm}
\psal{63}

\begin{quote}
\begin{verse}
Gott, du mein G\d ott, dich suche ich, *\\ 
meine Seele d\d ürstet nach dir.\\ 
\vin Nach dir schm\d achtet mein Leib *\\ 
\vin wie dürres, lechzendes L\d and ohne Wasser.\\  
\textit{Darum halte ich Ausschau\\
nach d\d ir} im Heiligtum, *\\ 
\textit{um deine Macht und Herrlichk\d eit zu sehen.}\\ 
\vin Denn deine Huld ist b\d esser als das Leben; *\\ 
\vin darum preisen dich m\d eine Lippen.\\ 
Ich will dich r\d ühmen mein Leben lang, *\\ 
in deinem Namen die H\d ände erheben.\\ 
\vin Wie an Fett und Mark wird s\d att \\ 
\vin meine Seele, *\\ 
\vin mit jubelnden Lippen soll mein M\d und \\ 
\vin dich preisen.\\  
Ich denke an dich auf n\d ächtlichem Lager *\\ 
und sinne über dich nach, w\d enn ich wache.\\ 
\vin Ja, du wurdest m\d eine Hilfe; *\\ 
\vin jubeln kann ich im Schatten d\d einer Flügel.\\  
Meine S\d eele hängt an dir, *\\ 
deine rechte H\d and hält mich fest.\\ 
\vin \textcolor{red}{\bf{(}}Viele trachten mir ohne Grund\\ 
\vin n\d ach dem Leben, *\\ 
\vin aber sie müssen hinabfahren\\ 
\vin in die T\d iefen der Erde.\\ 
Man gibt sie der Gew\d alt des Schwertes preis, *\\ 
sie werden eine Beute d\d er Schakale.\textcolor{red}{\bf{)}}\\ \vin 
Der König aber freue sich an Gott. †\\ 
\vin Wer bei ihm schwört, d\d arf sich rühmen. *\\ 
\vin  Doch allen Lügnern\\
\vin wird der M\d und verschlossen.\\  

\end{verse}
\end{quote}

\noindent\rot{Resp.br.} Sana animam \rot{und Hymnus} Splendor paternæ \rot{oder} Ecce iam noctis \rot{wie am Montag pp. 18ff.}\\


\begin{flushleft}

\versik{Repléti sumus mane misericórdia tua.}{Exultávimus, et delectáti sumus.}

\medskip

{\rm{
\versik{Am Morgen sind wir erfüllt von deiner Huld.}{Wir sind (voll) Jubel und Frohsinn.}
}}
\end{flushleft}

\medskip


\setspaceafterinitial{4.2mm plus 0em minus 0em}
\setspacebeforeinitial{4.2mm plus 0em minus 0em}
\def\greinitialformat#1{{\fontsize{40}{40}\selectfont #1}}
\gresetfirstlineaboveinitial{\footnotesize \textcolor{red}{Benedic}}{}
\setaboveinitialseparation{0.72mm}
\setsecondannotation{\small vij. T.}

\includescore{cantica/evangelica/salutemexinimicis.tex}
\vspace{0.3cm}
\rot{Canticum} Benedictus \rot{p. 64.}