
\thispagestyle{plain}

\kapklein{\rot{C}antica}
\kkap{cantica}


\section{commune dedicationis ecclesiae}

Ps. 24, 84, 87.

\cantvig{Ies 66,10-14}
\begin{verse}[\versewidth]

Freut euch mit Jerusalem, der heiligen Stadt!*\\
Jubelt in ihr, alle, die ihr sie lieb habt!\\
\vin Seid fröhlich und frohlockt mit ihr,*\\
\vin alle, die ihr über sie traurig wart!\\
Saugt euch satt an ihrer tröstenden Brust,*\\
trinkt und labt euch an den Brüsten ihrer Herrlichkeit!\\

\vin Denn so spricht der Herr: †\\
\vin Seht, ich leite den Frieden zu ihr wie einen Strom,*\\
\vin wie einen rauschenden Bach\\ \vin die Herrlichkeit der Völker.\\
Ihr werdet gestillt, auf den Armen getragen,*\\
ihr werdet auf den Knien geschaukelt.\\
\vin Wie einen, den die Mutter tröstet,†\\
\vin so will ich selber euch trösten,*\\
\vin ja, ihr findet Trost in Jerusalem.\\
Ihr werdet es sehn,†\\
und euer Herz wird sich freuen:*\\
Die Hand des Herrn wird sich an seinen Knechten offenbaren.\\ 
 
\end{verse}

\newpage

\section{commune beatae mariae virginis}

Ps. 24, 46, 87.

\cantvig{Eccl. 15,1-6}

\begin{verse}[\versewidth]

Wahrlich, so handelt, wer \\den Herrn fürchtet und ehrt,*\\
Wer an der Weisung festhält,\\
wird die Weisheit erlangen.\\
\vin Sie geht ihm entgegen wie eine Mutter,*\\
\vin sie nimmt ihn auf wie eine junge Gattin.\\
Sie nährt ihn mit dem Brot der Klugheit,*\\
sie tränkt ihn mit dem Wasser der Einsicht.\\
\vin Er stützt sich auf sie und kommt nicht zu Fall,*\\
\vin er vertraut ihr und wird nicht zuschanden.\\
Sie erhöht ihn über seine Gefährten,*\\
sie öffnet ihm den Mund in der Gemeinde.\\
\vin Jubel und Freude läßt sie ihn finden,*\\
\vin sie verleiht ihm einen unvergänglichen Namen.\\
\end{verse}


\section{commune apostolorum}

Ps. 19a (2-7), 64, 97.\\

\cantvig{Ies 55,1-5}



\begin{verse}[\versewidth]
Auf, ihr Dürstenden! Kommt alle zum Wasser!*\\
Auch wer kein Geld hat, soll kommen.\\
\vin Kauft Getreide und esst,†\\
\vin kommt und kauft ohne Geld,*\\
\vin kauft Wein und Milch ohne Bezahlung!\\
Warum bezahlt ihr mit Geld, was nicht nährt,*\\
mit dem Lohn eurer Mühe, was euch nicht satt macht?\\
\vin Hört doch auf mich,†\\
\vin dann bekommt ihr das Beste zu essen,*\\
\vin dann könnt ihr euch laben an fetten Speisen.\\
Neigt euer Ohr und kommt zu mir!*\\
Hört, dann werdet ihr leben!\\
\vin Ich schließe mit euch einen ewigen Bund*\\
\vin getreu der beständigen Huld, die ich \\ \vin David erwiesen.\\
Seht doch: Ich machte ihn\\ zum Zeugen für die Völker,*\\
zum Fürsten und Gebieter der Nationen.\\
\vin Auch du wirst Völker rufen, die du nicht kennst,*\\
\vin Völker, die dich nicht kennen, eilen her zu dir\\
um des Herrn, deines Gottes,\\ willen, des Heiligen Israels,*\\
denn er machte dich herrlich.
\end{verse}

\hspace{2cm}

\section{commune martyrum}

\rot{Pro pluribus:}

Ps. 2, 11, 33b (12-22)\\

\cantvig{Eccl. 51,1-6}
\begin{verse}[\versewidth]

Ich will dich preisen, Herr und König,*\\
du Gott meines Heils, ich will dich loben.\\
\vin Ich will deinen Namen verkünden,*\\
\vin du Hort meines Lebens.\\
Denn du hast mich errettet vom Tod,†\\
du hast meinen Leib bewahrt vor der Grube,*\\
hast meinen Fuß dem Griff der Unterwelt entrissen.\\
\vin Du hast mich befreit\\ 
\vin von der Geißel der verleumderischen Zunge,*\\
\vin von den Lippen treuloser Lügner.\\
Gegen meine Widersacher tratest du mir zur Seite,*\\
in deiner großen Huld hast du mir geholfen\\
\vin aus der Schlinge derer, die auf mein Fallen lauern,*\\
\vin aus den Händen jener, die nach\\ \vin meinem Leben trachten.\\
Aus vielen Nöten hast du mich erlöst:*\\
aus der Bedrängnis der Flammen,\\  die mich umringten,\\
\vin aus dem Feuer, das brennt, \\ \vin ohne dass man es schürt,*\\
\vin aus dem Schoss der Fluten, die nicht\\ \vin bedrohen mit Wasser,\\
von ruchlosen Lippen und Erfindern von Lüge,*\\
von den Pfeilen der falschen Zunge.\\

\end{verse}

\rot{Pro uno vel una:}

Ps. 2, 33a (1-11), 33b (12-22).\\

\cantvig{Eccl. 51,7-12}

\begin{verse}[\versewidth]
 Schon war ich dem Tode nahe*\\
und mein Leben den Tiefen der Unterwelt.\\
\vin Ich blickte ringsum und fand keinen Helfer,*\\
\vin nach einem Beistand spähte ich, \\ \vin doch gab es keinen.\\
Da dachte ich an das Erbarmen des Herrn,*\\
an die Taten seiner Huld, die seit ewig bestehen.\\
\vin Allen, die auf ihn vertrauen, hilft er *\\
\vin aus jeder Gefahr erlöst er sie.\\
So erhob ich von der Erde meine Stimme,*\\
von den Toren der Unterwelt schrie ich.\\
\vin Ich rief: „O Herr, du bist doch mein Vater,*\\
\vin mein Gott und der Held, der mich rettet.\\
Verlass mich nicht am Tag der Not,*\\
am Tag der Vernichtung und der Verwüstung.\\
\vin Deinen Namen will ich allezeit loben,* \\
\vin im Gebet will ich an dich denken.“\\
Da hat der Herr meine Stimme gehört*\\
und auf mein Flehen geachtet.\\
\vin Von allem Unheil hat er mich erlöst,*\\
\vin am Tag der Not mich gerettet.\\
Darum danke ich dem Herrn und lobe ihn,*\\
ich preise seinen Namen.\\

\end{verse}




\section{commune pastorum}

Ps. 21, 92a (2-9), 92b (10-16).\\

\cantvig{Ies 61,1-9}

\begin{verse}[\versewidth]
 


Der Geist des Herrn ruht auf mir,*\\
denn der Herr hat mich gesalbt.\\
\vin Er hat mich gesandt,†\\
\vin frohe Botschaft zu bringen den Armen,*\\
\vin zu heilen, die gebrochenen Herzens sind,\\
Entlassung zu verkünden den Gefangenen *\\
und den Gefesselten Befreiung, \\
\vin auszurufen ein Gnadenjahr des Herrn,*\\
\vin einen Tag der Vergeltung für unseren Gott,\\
alle Trauernden zu trösten,*\\
zu erfreuen die Trauernden Zions.\\
\vin Ihr werdet heißen „Priester des Herrn“,*\\
\vin „Diener unseres Gottes“ wird man euch nennen.\\
Ich, der Herr, gebe ihnen in Treue den Lohn,*\\
ich schließe mit ihnen einen ewigen Bund.\\
\vin Ihr Geschlecht wird unter \\ \vin den Nationen bekannt sein,*\\
\vin und ihre Sprösslinge inmitten der Völker.\\
Alle, die sie sehen, werden erkennen:*\\
Sie sind ein Geschlecht, das der Herr gesegnet hat.\\

 
 
\end{verse}




\section{commune virginum}

Ps. 21, 92a (2-9), 92b (10-16).\\

\cantvig{Eccl. 14,20-27}
\begin{verse}[\versewidth]

Selig der Mensch, der nachsinnt über die Weisheit,*\\
der sich bemüht um Einsicht:\\
\vin Er richtet seinen Sinn auf ihre Wege*\\
\vin und achtet auf ihre Pfade.\\
Er geht ihr nach wie ein Späher,*\\
er lauert am Eingang ihrer Wohnung.\\
\vin Er schaut durch ihre Fenster *\\
\vin und horcht an ihren Türen.\\
Bei ihrem Haus schlägt er sein Lager auf *\\
und macht die Stricke seines \\ Zeltes fest an ihrer Mauer.\\
\vin An ihrer Seite stellt er sein Zelt auf *\\
\vin und wohnt in guter Nachbarschaft.\\
Er baut sein Nest in ihrem Laub *\\
und nächtigt in ihren Zweigen.\\
\vin In ihrem Schatten sucht er Zuflucht vor der Hitze*\\
\vin und wohnt im Schutz ihres Hauses.\\
\end{verse}

\section{commune virorum sanctorum}

Ps. 21, 92a (2-9), 92b (10-16).\\

\rot{Canticum ut in Commune Pastorum.}

\section{commune mulierum sanctarum}

Ps. 19a (2-7), 45a (2-10), 45b (11-18).\\

\cantvig{Sap 9,1.9-19}

\begin{verse}[\versewidth]
 Gott der Väter und Her des Erbarmens,*\\
durch dein Wort hast du das All geschaffen.\\
\vin Bei dir ist die Weisheit, die deine Wege kennt,*\\
\vin als du die Welt erschufst, war sie zugegen.\\
Sie weiß, woran du Gefallen hast *\\
und was recht ist nach deinen Geboten.\\
\vin So sende sie herab vom heiligen Himmel *\\
\vin und schicke sie vom Thron deiner Herrlichkeit,\\
damit sie mir beisteht und alle Mühe mit mir teilt,*\\
und damit ich erkenne, woran du Gefallen hast.\\
\vin Denn alles weiß und versteht sein,†\\
\vin besonnen wird sie mich leiten in meinem Tun,*\\
\vin in ihrem Lichtglanz wird sie mich schützen.\\
Welcher Mensch kann Gottes Plan erkennen,*\\
wer begreift, was der Wille des Herrn ist?\\
\vin Wer hat je deine Pläne erkannt, †\\
\vin wenn du ihm nicht Weisheit gegeben,*\\
\vin wenn du ihn nicht deinen heiligen Geist\\
\vin aus der Höhe gesandt hast.\\
Nur so wurden die Pfade der \\ Erdenbewohner geebnet,†\\
die Menschen lernten, woran du Gefallen hast:*\\
sie wurden gerettet durch die Weisheit.\\

\end{verse}




