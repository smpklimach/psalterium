

\setspaceafterinitial{2.2mm plus 0em minus 0em}
\setspacebeforeinitial{2.2mm plus 0em minus 0em}

\def\greinitialformat#1{%
{\fontsize{39}{39}\selectfont #1}%
}

%\redlines

\gresetfirstlineaboveinitial{\small \bf viij. T.}{\small \bf viij. T.}
%\setaboveinitialseparation{0.52mm}
%\setsecondannotation{\footnotesize {{ iv. T.}}}

% We type a text in the top right corner of the score:
%\commentary{{\small \emph{Cf. Is. 30, 19 . 30 ; Ps. 79}}}

% and finally we include the score. The file must be in the same directory as this one.
\includescore{ca/dan3,52.tex}
\vspace{0.3cm}
\begin{center}
 \textcolor{red}{\large \bf Dan 3,52-57}\\
Omnis creatura laudet Dominum\\
\textit{\small Laudem dicite Deo nostro, omnes servi eius. (Ap 19,5)}
\end{center}
\begin{verse}[\versewidth]
Benedíctus es, Dómine Deus patrum nostrórum, *\\
et laudábilis et superexaltátus in s\'{æ}cula;\\!
\vin et benedíctum nomen glóriæ tuæ sanctum, *\\
\vin et superlaudábile et superexaltátum in s\'{æ}cula.\\!
Benedíctus es in templo sancto glóriæ tuæ, *\\
et superlaudábilis et supergloriósus in s\'{æ}cula.\\!
\vin Benedíctus es in throno regni tui, *\\
\vin et superlaudábilis et superexaltátus in s\'{æ}cula.\\!
Benedíctus es, qui intuéris abýssos †\\
sedens super chérubim, *\\
et laudábilis et superexaltátus in s\'{æ}cula.\\!
\vin Benedíctus es in firmaménto cæli, *\\
\vin et laudábilis et gloriósus in s\'{æ}cula.\\!
Benedícite, ómnia ópera Dómini, Dómino; *\\
laudáte et superexaltáte eum in s\'{æ}cula.\\
\end{verse}
\vspace{1cm}


