
\thispagestyle{plain}

\kapklein{\rot{C}antica}
\kkap{cantica}


\section{cantica evangelica}


\cantben{Lc 1,68-79}

\begin{verse}[\versewidth]


Gepriesen sei der Herr, der Gott Israels,*\\
denn heimgesucht hat er sein Volk\\
und ihm Erlösung erschaffen.\\
\vin Er ließ uns erstehn einen machtvollen Retter*\\
\vin im Hause seines Knechtes David,\\
wie er verheißen hat durch den Mund seiner Heiligen,*\\
durch die Propheten der Vorzeit:\\
 \vin Er werde uns vor unseren Feinden erretten*\\
\vin und aus der Hand all derer, die uns hassen,\\
er werde sich unserer Väter erbarmen †\\
und seines heiligen Bundes gedenken,*\\
des Eides, den er geschworen unserm Vater Abraham:\\
\vin uns zu geben, daß wir ihm furchtlos dienen,*\\
\vin befreit aus der Hand der Feinde,\\
in Heiligkeit und Gerechtigkeit vor seinem Angesicht*\\
all unsre Tage.\\

\vin Und du, mein Kind, wirst „Prophet\\ 
\vin des Höchsten“ genannt, †\\
\vin denn du wirst dem Herrn vorangehn*\\
\vin und ihm die Wege bereiten:\\
seinem Volk zu schenken die Erkenntnis des Heiles*\\
in der Vergebung der Sünden\\
\vin durch unseres Gottes erbarmende Liebe,*\\
\vin in der uns heimsucht das aufstrahlende Licht aus der Höhe,\\
um denen zu scheinen,\\
die in Finsternis sitzen und im Schatten des Todes,*\\
 unsere Füße zu lenken auf den Weg des Friedens.\\

 
\end{verse}

\vspace{0.6cm}


\cantmag{Lc 1,46-55}


\begin{verse}[\versewidth]
 
Meine Seele preist die Größe des Herrn,*\\
es jubelt mein Geist über Gott, meinen Heiland.\\
\vin Denn seine niedrige Magd hat er in Gnaden angesehn.*\\
\vin Siehe, von nun an preisen mich selig alle Geschlechter.\\
Denn Großes hat an mir getan der Mächtige,*\\
und sein Name ist heilig.\\
\vin Sein Erbarmen währt von\\ 
\vin Geschlecht zu Geschlecht*\\
\vin über denen, die ihn fürchten.\\

Mit seinem Arm vollbrachte er machtvolle Taten,*\\
er hat zerstreut, die im Herzen voll Hochmut sind.\\
\vin Die Mächtigen hat er vom Thron gestürzt*\\
\vin und hat erhöht die Niedrigen.\\
Die Hungrigen hat er erfüllt mit Gütern*\\
und Reiche weggeschickt mit leeren Händen.\\
\vin Er hat sich angenommen seines Knechtes Israel*\\
\vin eingedenk seines Erbarmens,\\
wie er es unseren Vätern verheißen hat*\\
Abraham und seinem Stamm auf ewig.\\
\end{verse}

\vspace{0.6cm}


\cantnun{Lc 2,29-32}

\begin{verse}
 
Nun entläßt du, o Herr, deinen Knecht*\\
nach deinem Wort in Frieden.\\
\vin Denn meine Augen haben dein Heil gesehen,*\\
\vin das du bereitet hast vor allen Völkern:\\
Licht, das den Heiden Offenbarung schenkt,*\\
und Herrlichkeit für dein Volk Israel.\\
\end{verse}









\section{dominica}

%\cantvig{Eccl 42,15-25;43,28-30}
\cantvig{Eccl 42;43}
\begin{verse}[\versewidth]

Der Werke Gottes will ich nun gedenken,*\\
was ich gesehen habe, will ich erzählen:\\
\vin Durch Gottes Wort sind seine Werke entstanden;*\\
\vin Werk seines Wohlwollens ist seine Lehre.\\
Über allem strahlt die leuchtende Sonne:*\\
das Werk des Herrn ist voll von seiner Herrlichkeit.\\
\vin Nicht einmal die Engel Gottes*\\
\vin können all seine Wunder erzählen.\\ 
Gott gibt seinen Heerscharen Kraft,*\\
damit sie bestehen können vor seiner Herrlichkeit.\\
\vin Die Meerestiefe und das \\
\vin Menschenherz erforscht er,*\\
\vin ihre Geheimnisse kennt er alle.\\
Denn der Höchste hat Kenntnis von allem,*\\
er sieht voraus, was bis in Ewigkeit geschehen wird.\\
\vin Vergangenes und Künftiges – er macht es kund,*\\
\vin die Rätsel des Verborgenen enthüllt er.\\
Ihm mangelt es an keiner Einsicht,*\\
kein Ding bleibt ihm verborgen.\\
\vin Seine machtvolle Weisheit hat festen Bestand:*\\
\vin er ist ja derselbe seit ewigen Zeiten.\\
Nichts ist hinzuzufügen, nichts zu entfernen.*\\
Er braucht keinen, der ihn berate.\\
\vin Voll Anmut sind alle seine Werke,*\\
\vin bis hin zum kleinesten Funken\\
\vin und zur flüchtigen Erscheinung.\\
Sie alle leben und bestehen für immer,*\\
für jedes Bedürfnis ist alles bereitgestellt.\\
\vin Jedes Ding ist vom andern verschieden,*\\
\vin und keins von denen, die er schuf, ist entbehrlich.\\
Eines ergänzt das andere in seinem Wert.*\\
Wer kann sich satt sehn an ihrer Herrlichkeit?\\

\vin Wir wollen ihn lobpreisen, †\\
\vin ergründen können wir ihn nicht -*\\
\vin er ist ja noch größer als all seine Werke.\\
Überaus ehrfurchtgebietend ist der Herr,*\\
und wunderbar ist sein machtvolles Wirken.\\
\vin Die ihr den Herrn lobpreist, erhebt die Stimme, †\\
\vin singt, so gut ihr nur könnt:*\\
\vin es wird niemals genügen.\\
Die ihr ihn erhebt, bietet alle Kraft auf, †\\
werdet nicht müde:*\\
ihr kommt nie an ein Ende!\\
 
\end{verse}

\vspace{0.6cm}

\cantlaui{Dan 3,57-88}
\begin{verse}[\versewidth]
 Preiset den Herrn, all ihr Werke des Herrn;*\\
 lobt und erhebt ihn in Ewigkeit!\\
\vin Preiset den Herrn, ihr Engel des Herrn;*\\
\vin ihr hohen Himmel, lobpreiset ihn!\\
Preist den Herrn, all ihr Wasser droben am Himmel;*\\
all ihr Mächte lobpreiset ihn!\\
\vin Preiset den Herrn, Sonne und Mond;*\\
\vin ihr Sterne am Himmel, lobpreiset ihn!\\
Preiset den Herrn, jeder Regen und Tau,*\\
all ihr Stürme, lobpreiset ihn!\\
\vin Preiset den Herrn, Feuer und Glut,*\\
\vin Frost und Hitze, lobpreiset ihn!\\
Preiset den Herrn, ihr Tropfen des Taus,†\\
ihr Flocken des Schnees,*\\
Eis und Kälte, lobpreiset ihn!\\
\vin Preiset den Herrn, ihr Kristalle von Schnee und Rauhreif,*\\
\vin ihr Blitze und Wolken, lobpreiset ihn!\\
Preiset den Herrn, ihr Nächte und Tage,*\\
Licht und Dunkel, lobpreiset ihn!\\
\vin Du, Erde,  preise den Herrn,*\\
\vin lob und erheb ihn in Ewigkeit!\\
Preiset den Herrn, ihr Berge und Hügel,*\\
alles, was sproßt auf der Erde, lobpreise ihn!\\
\vin Preiset den Herrn, ihr Quellen,*\\
\vin ihr Meere und Ströme, lobpreiset ihn!\\
Preiset den Herrn, ihr Ungeheuer der See †\\
und alles, was sich regt in den Wassern,*\\
all ihr Vögel des Himmels, lobpreiset Ihn!\\
\vin Preiset den Herrn, all ihr Tiere, wilde und zahme,*\\
\vin all ihr Menschen lobpreiset Ihn!\\

Du, Israel, preise den Herrn,*\\
lob und erheb ihn in Ewigkeit!\\
\vin Preiset den Herrn, ihr Priester des Herrn,*\\
\vin ihr seine Knechte, lobpreiset Ihn!\\
Preiset den Herrn, ihr Heiligen †\\
und ihr gebeugten Herzen,*\\
ihr Geister und Seelen der Gerechten, lobpreiset Ihn!\\

\vin Preiset den Herrn, Hananja, Asarja und Mischael,*\\
\vin lobt und erhebt ihn in Ewigkeit!\\
Denn er entriss uns der Unterwelt,*\\
er hat uns errettet aus der Hand des Todes.\\
\vin Er befreite uns aus dem lodernden Ofen,*\\
\vin er hat uns erlöst aus der Mitte des Feuers.\\

Laßt uns preisen den Vater und den Sohn
mit dem Heiligen Geist,*\\
ihn loben und erheben in Ewigkeit!\\

\end{verse}


\vspace{0.6cm}

\cantlauii{Dan 3,52-57}



\begin{verse}[\versewidth]
 Gepriesen bist du, Herr, du Gott unserer Väter,*\\
gelobt und hoch erhoben in Ewigkeit.\\
\vin Gepriesen ist deiner Herrlichkeit heiliger Name,*\\
\vin gelobt und hoch erhoben in Ewigkeit.\\
Gepriesen bist du im Tempel deiner heiligen Herrlichkeit,*\\
gelobt und hoch erhoben in Ewigkeit.\\
\vin Gepriesen bist du auf dem Thron deiner Herrschaft,*\\
\vin gelobt und hoch erhoben in Ewigkeit.\\
Gepriesen bist du, der auf den Kerubim thront †\\
und niederschaut in die Tiefen des Abgrunds,*\\
gelobt und hoch erhoben in Ewigkeit,\\
\vin Gepriesen bist du, der in der Feste des Himmels,*\\
\vin gelobt und hoch erhoben in Ewigkeit.\\
Gepriesen ist der Vater und der Sohn mit dem Heiligen Geiste,*\\
gelobt und hoch erhoben in Ewigkeit.\\

 \vin Preiset den Herrn, all ihr Werke des Herrn;*\\
 \vin lobt und erhebt ihn in Ewigkeit!\\

\end{verse}

\vspace{0.6cm}

\cantves{Ap 19,1-8}
\begin{verse}[\versewidth]
 Das Heil und die Herrlichkeit und die Macht
gebührt unserem Gott,*\\
denn wahr und gerecht sind seine Gerichte:\\
\vin Er hat die große Hure gerichtet, †\\
\vin die mit ihrer Unzucht die Erde verdarb;*\\
\vin er rächte das Blut seiner Knechte an ihren Händen.\\
Preist unsern Gott, all seine Knechte*\\
und alle, die ihn fürchten, Kleine wie Große!\\
\vin Denn der Herr, unser Gott, der Allherrschende,*\\
\vin hat angetreten seine Herrschaft.\\
Laßt uns frohlocken und jauchzen*\\
und ihm die Ehre erweisen.\\
\vin Denn gekommen ist die Hochzeit des Lammes, †\\
\vin die Braut ist bereit,*\\
\vin sie durfte sich kleiden in strahlendes Linnen.\\
\end{verse}


\section{feria secunda}


\cantlaui{1 Chr 29,10-13}

\begin{verse}[\versewidth]

Gepriesen bist du, Herr †\\
Gott unsres Vaters Israel,*\\
von Ewigkeit zu Ewigkeit!\\
\vin Dein, o Herr, sind Größe und Kraft, †\\
\vin Ruhm und Glanz und Hoheit;*\\
dein ist alles im Himmel und auf Erden.\\
\vin Dein, o Herr, ist das Königtum.*\\
\vin Du erhebst dich als Haupt über alles.\\
Von dir kommen Reichtum und Ehre;*\\
du bist Herrscher über alles.\\
\vin In deiner Hand liegen Kraft und Stärke;*\\
\vin von deiner Hand kommt alle Macht und Größe:\\
Herr, unser Gott, wir danken dir,*\\
wir rühmen deinen herrlichen Namen.\\
 
\end{verse}

\vspace{0.6cm}



\cantlauii{Eccl 36,1-7.13-16}

\begin{verse}[\versewidth]
 Hilf uns, du Gott des Alls,*\\
 und lege deinen Schrecken auf alle Völker!\\
\vin  Wider das fremde Volk erhebe deine Hand:*\\
 \vin lass es deine mächtigen Taten sehen.\\
Wie du dich  vor ihren Augen als heilig bezeugt hast an uns,*\\
so verherrliche dich vor unseren Augen an ihnen,\\
\vin damit sie erkennen, wie wir es erkannten,*\\
\vin dass außer dir kein Gott ist.\\
Erneuere die Zeichen,†\\
wiederhole die Wunder,*\\
zeige die Macht deiner Hand und die Kraft deiner Rechten.\\

\vin Sammle alle Stämme Jakobs,*\\
\vin verteile das Erbe wie in den Tagen der Vorzeit.\\
Erbarme dich des Volkes, das deinen Namen trägt,*\\
Israel, das du deinen Erstgeborenen nanntest.\\
\vin Erbarme dich deiner heiligen Stadt,*\\
\vin Jerusalem, der Stätte deiner Wohnung.\\
Erfülle den Zion mit deinem Glanz *\\
und deinen Tempel mit deiner Herrlichkeit! \\
\end{verse}

\vspace{0.6cm}

\cantves{Eph 1,3-10}

\begin{verse}[\versewidth]
Gepriesen sei Gott,*\\
der Gott und Vater unseres Herrn Jesus Christus:\\
\vin Er hat uns gesegnet in Christus*\\
\vin mit der Fülle geistlichen Segens im Himmel.\\
In Christus hat er uns erwählt vor der Erschaffung der Welt,*\\
heilig und makellos zu sein vor seinem Angesicht.\\
\vin In Liebe hat er uns dazu vorherbestimmt,*\\
\vin seine Kinder zu werden durch Jesus Christus,\\
nach dem Wohlgefallen seines Willens, †\\
zum Lob seiner herrlichen Gnade,*\\
mit der er uns beschenkt hat in\\
seinem geliebten Sohne.\\
\vin In ihm und durch sein Blut †\\
\vin haben wir die Erlösung,*\\
\vin die Vergebung der Sünden\\
nach dem Reichtum seiner Gnade,†\\
die er ausgoß über uns in Fülle,*\\
in aller Weisheit und Einsicht.\\
\vin Er hat uns das Geheimnis seines Willens kundgetan*\\
\vin nach seinem gnädigen Ratschluß, den er im voraus gefaßt hat:\\
Heraufzuführen die Fülle der Zeiten, †\\
und alles, was im Himmel ist und auf der Erde,*\\
unter einem Haupt zu vereinen in Christus.\\

\end{verse}


\section{feria tertia}

\cantlaui{Tob 13,1-2,6-10}
\begin{verse}[\versewidth]
 Gepriesen sei Gott, der in Ewigkeit lebt,*\\
gepriesen sein Königtum!\\
\vin Er ist es, der züchtigt und der sich erbarmt, †\\
\vin er führt zu den Toten hinab und er führt auch herauf;*\\
\vin seiner Hand kann niemand entrinnen.\\
Wenn ihr euch zu ihm bekehrt, †\\
aus ganzem Herzen und mit ganzer Seele,*\\
um vor ihm das Wahre zu tun,\\
\vin dann wendet auch er sich euch zu*\\
\vin und wird sein Angesicht nicht mehr vor euch verbergen.\\
Wenn ihr dann seht, was er für euch getan hat*\\
dann preiset ihn mit lauter Stimme!\\
\vin Dankt dem Herrn der Gerechtigkeit,*\\
\vin rühmt den ewigen König!\\
Ich preise den Herrn im Land der Verbannung,*\\
ich künde seine Macht und Größe einem Volk von Sündern.\\
\vin Kehrt um, ihr Sünder! Tut, was vor ihm recht ist!*\\
\vin Wer weiß,ob er nicht gnädig ist und sich euer erbarmt?\\


Ich will meinen Gott erheben, †\\
meine Seele lobe den König des Himmels,*\\
frohlocken soll sie über seine Größe.\\

\end{verse}

\vspace{0.6cm}

\cantlauii{Is 38,10-14.17-20}

\begin{verse}[\versewidth]
 Ich sagte: †\\
In der Mitte meiner Tage muss ich dahingehn,*\\
ich bin entboten zu den Toren der Unterwelt\\
Für den Rest meiner Jahre.\\
\vin Ich sagte: †\\
\vin Ich werde den Herrn nicht mehr schaun\\
im Lande der Lebenden,*\\
keinen Menschen mehr sehn bei den Bewohnern der Erde.\\
\vin Abgebrochen wurde meine Hütte,*\\
\vin mir weggetragen wie ein Hirtenzelt.\\
Ich wob wie ein Weber mein Leben zu Ende*\\
da schnitt er mich vom Webstuhl ab.\\
\vin Noch ehe der Tag zur Nacht wird,*\\
\vin führst du mein Ende herbei.\\
Bis zum Morgen schrie ich um Hilfe*\\
da zermalmte er wie ein Löwe mein Gebein.\\
\vin Noch ehe der Tag zur Nacht wird,*\\
\vin Führst du mein Ende herbei.\\
Ich blicke verschmachtend zur Höhe:*\\
Herr, ich bin in Bedrängnis, tritt für mich ein!\\

\vin Siehe, zum Heil wurde mir die Bitternis; †\\
\vin vor der Grube der Vernichtung hast du meine Seele bewahrt,*\\
hast hinter dich geworfen all meine Sünden.\\
\vin Denn die Unterwelt kann dich nicht rühmen, †\\
\vin der Tod kann dich nicht loben.*\\
\vin Wer in die Grube fuhr, hofft nicht auf deine Treue.\\
Der Lebende, nur der Lebende preist dich, †\\
wie ich es heute tue.*\\
Der Vater verkündet den Kindern deine Treue:\\
\vin Der Herr war bereit, mich zu retten! †\\
\vin Darum rühmen wir die Saiten im Hause des Herrn*\\
\vin alle Tage unseres Lebens.\\

\end{verse}

\vspace{0.6cm}

\cantves{Ap 4,11;5,9.10.12}
\begin{verse}[\versewidth]
Herr, unser Gott, du bist würdig,\\
Macht zu empfangen und Ehre und Herrlichkeit.\\
\vin Denn du hast alle Dinge geschaffen,*\\
\vin durch deinen Willen ist alles geworden.\\
Ja, du bist würdig, †\\
Das Buch zu nehmen und seine Siegel zu öffnen,*\\
denn du wurdest geschlachtet.\\
\vin Du hast uns für Gott mit deinem Blut erkauft †\\
\vin aus jedem Stamm und jeder Sprache,*\\
\vin aus jedem Volk und jedem Geschlechte.\\
Du machtest uns zu einem Königtum †\\
und zu Priestern für unseren Gott,*\\
zu herrschen über die Erde.\\

\vin Würdig ist das Lamm, das geschlachtet ist, †\\
\vin Macht zu empfangen,\\ 
\vin Reichtum, Weisheit und Stärke,*\\
\vin Ehre und Lobpreis und Herrlichkeit.\\

\end{verse}

\section{feria quarta}


\cantlaui{Jdt 16,1f.13-15}
\begin{verse}[\versewidth]
 Stimmt meinem Gott ein Lied an mit schallenden Pauken,*\\
singt für den Herrn mit klingenden Zimbeln.\\
\vin Preist ihn und singt sein Lob, †\\
\vin erhebt seinen Namen, rufet ihn an!*\\
\vin Denn der Herr ist ein Gott, der den Kriegen ein Ende setzt.\\
Ich singe meinem Gott ein neues Lied: †\\
Herr, du bist groß und herrlich,*\\
Wunderbar in deiner Kraft, unübertrefflich!\\
\vin Dienen muß dir deine ganze Schöpfung;*\\
\vin Denn du befahlst,  - da trat sie ins Dasein.\\
Du sandtest deinen Geist, den Bau zu errichten.*\\
Niemand kann deiner Stimme widerstehen.\\
\vin Berge und Meere wanken in ihrem Grund, †\\
\vin Felsen zerschmelzen vor\\ 
\vin deinem Antlitz wie Wachs,*\\
\vin doch denen, die dich fürchten, erzeigst du dich gnädig.\\
\end{verse}


\vspace{0.6cm}

\cantlauii{Sam 1,1-10}

\begin{verse}[\versewidth]

 Mein Herz ist fröhlich im Herrn,*\\
Erhöht ist meine Macht durch meinen Gott.\\
\vin Weit öffnet sich mein Mund gegen meine Feinde;*\\
\vin Denn ich freue mich deiner Hilfe.\\
Keiner ist heilig wie der Herr, †\\
Denn außer dir ist keiner,*\\
Keiner ist ein Fels wie unser Gott.\\
\vin Redet nicht immer so vermessen,*\\
\vin Kein freches Wort entfahre eurem Mund.\\
Denn ein wissender Gott ist der Herr,*\\
Von ihm werden die Taten gewogen.\\

\vin Zerbrochen wird der Bogen der Helden,*\\
\vin die Wankenden aber gürten sich mit Kraft.\\
Um Brot verdingen sich die Satten,*\\
doch die Hungrigen können feiern für immer.\\
\vin Die Unfruchtbare – siebenmal gebiert sie,*\\
\vin doch die an Kindern reich war, welkt dahin.\\
Der Herr macht tot und macht lebendig.*\\
er führt ins Totenreich hinab und führt auch herauf.\\
\vin Der Herr macht tot und macht lebendig,*\\
\vin er erniedrigt und er erhöht.\\
Den Geringen richtet er auf aus dem Staub,*\\
aus dem Schmutz erhebt er den Armen.\\ 
\vin Er gibt ihm einen Sitz bei den Edlen,*\\
\vin den Thron der Ehre läßt er ihn erben.\\
Denn dem Herrn gehören die Säulen der Erde;*\\
er hat den Erdkreis auf sie gegründet.\\
\vin Die Schritte seiner Frommen hütet er,*\\
die Frevler aber verstummen im Dunkel.\\
Denn niemand ist stark durch eigene Kraft.*\\
\vin Wer gegen den Herrn streitet, wird zerbrechen.\\
\vin Der Höchste läßt im Himmel den Donner erdröhnen,*\\
der Herr richtet die Enden der Erde.\\
Seinem König gebe er Kraft*\\
und erhöhe die Macht seines Gesalbten.\\
\end{verse}


\vspace{0.6cm}

\cantves{Col 1, (12)15-20}
\begin{verse}[\versewidth]

Er ist das Bild des unsichtbaren Gottes,*\\
der Erstgeborene der ganzen Schöpfung.\\
\vin Denn in ihm wurde alles erschaffen †\\
\vin im Himmel und auf Erden -*\\
\vin alles ist durch ihn und auf ihn hin geschaffen.\\
Er selber ist vor allem, †\\
in ihm hat alles Bestand.*\\
Er ist das Haupt des Leibes, der Kirche.\\ 
\vin Er ist der Ursprung, †\\
\vin der Erstgeborene von den Toten,*\\
\vin auf dass er den Vorrang habe in allem.\\
Denn es hat Gott gefallen, *\\
mit seiner ganzen Fülle in ihm zu wohnen\\
\vin und durch ihn alles mit sich zu versöhnen,*\\
\vin was auf der Erde und was im Himmel ist:\\
denn er stiftete Frieden*\\
durch das Blut seines Kreuzes.\\
\end{verse}


\section{feria quinta}

\cantlaui{Ier 31,10-14}

\begin{verse}[\versewidth]


Hört, ihr Völker, das Wort des Herrn,*\\
verkündet es auf den fernsten Inseln und sagt:\\ 
\vin Er, der Israel zerstreut hat, wird es auch sammeln*\\
\vin und hüten wie ein Hirt seine Herde.\\
Denn der Herr wird Jakob erlösen*\\
und ihn befreien aus der Hand des Stärkeren.\\
\vin Sie kommen und jubeln auf Zions Höhe,†\\
\vin sie strahlen vor Freude über die Gaben des Herrn,*\\ 
\vin über Korn, Wein und Öl, über Lämmer und Rinder.\\ 
Sie werden wie ein bewässerter Garten sein*\\ 
und nie mehr verschmachten.\\
\vin Dann freut sich das Mädchen beim Reigentanz,*\\
\vin Jung und Alt sind fröhlich.\\ 
Ich verwandle ihre Trauer in Jubel,*\\
tröste und erfreue sie nach ihrem Kummer.\\
\vin Ich labe die Priester mit Opferfett*\\
\vin und mein Volk wird satt an meinen Gaben.\\

\end{verse}




\cantlauii{Is 12,1-6}

\begin{verse}[\versewidth]
 
O Herr, ich will dich preisen:†\\
Du hast mir gezürnt, doch dein Zorn\\ hat sich gewendet,*\\
und du hast mich getröstet.\\
\vin Siehe, der Gott meines Heiles!*\\
\vin Ich bin voll Vertrauen und fürchte mich nicht.\\
Denn Gott, der Herr ist meine Stärke,\\
mein Lied ist er,*\\
er ist mein Retter geworden.\\

\vin Ihr werdet Wasser schöpfen mit Freude *\\
\vin aus den Quellen des Heiles.\\

An jenem Tage werdet ihr sagen:*\\
Preiset den Herrn, ruft seinen Namen aus!\\

\vin Von seinen Taten erzählt den Völkern!*\\
\vin Verkündet: Sein Name ist erhaben!\\

Lobsingt dem Herrn, denn machtvoll \\ hat er sich erwiesen;*\\
die ganze Erde soll es wissen.\\

\vin Jauchzt und jubelt, ihr Bewohner von Zion!*\\
\vin Denn groß ist in eurer Mitte der Heilige Israels!\\
\end{verse}

\vspace{0.6cm}

%\cantves{Ap 11,17-18;12,10-12}
\cantves{Ap 11;12}

\begin{verse}[\versewidth]
 Wir danken dir, Herr, allherrschender Gott,*\\
der ist und der war.\\ 
\vin Denn du hast deine große Macht ergriffen*\\
\vin und angetreten deine Herrschaft.\\
Die Völker gerieten in Zorn, †\\
da ist dein Zorn gekommen,*\\
es kam die Zeit, die Toten zu richten,\\
\vin die Zeit, den Lohn zu geben*\\
\vin deinen Knechten, den Propheten und Heiligen,\\
allen, die deinen Namen fürchten,*\\
den Kleinen wie den Großen,\\
\vin die Zeit, ins Verderben zu bringen,*\\
\vin die die Erde verderben.\\

Jetzt ist er da, der rettende Sieg †\\
und die Macht und die Herrschaft unseres Gottes*\\
und die Vollmacht seines Gesalbten.\\
\vin Denn gestürzt ist, der unsere Brüder verklagte,*\\
\vin der Tag und Nacht sie verklagte vor unserem Gott.\\
Sie haben ihn besiegt durch das Blut des Lammes*\\
und durch ihr Wort und Zeugnis,\\ 
\vin sie hielten ihr Leben nicht fest,*\\
\vin gaben es hin im Tode.\\
Darum jauchzet, ihr Himmel*\\
und alle, die darin wohnen!\\
\vin Wehe dem Land und dem Meer: †\\
\vin denn Satan fuhr hinab zu euch in großem Zorne,*\\
\vin weil er weiß: er hat nur eine kurze Frist.\\

\end{verse}


\section{feria sexta}

\cantlaui{Is 45,15-25}
\begin{verse}[\versewidth]
 Wahrlich, du bist ein verborgener Gott,*\\
Gott Israels, du Retter!\\
\vin Die sich Götzen schmieden, werden beschämt und zuschanden,*\\
\vin mit Schmach bedeckt gehen sie alle dahin.\\
Doch Israel wird gerettet vom Herrn,*\\
wird auf ewig gerettet;\\
\vin ihr werdet nicht beschämt und nicht zuschanden*\\
\vin auf immer und ewig.\\
Denn so spricht der Herr, der die Himmel geschaffen,*\\
der Gott, der die Erde gemacht und gebildet und der sie erhält\\
\vin -er hat sie nicht geschaffen als Öde,*\\
\vin er machte sie, damit man auf ihr wohne -:\\
„Ich bin der Herr,*\\
ich und sonst keiner.\\
\vin Ich habe nicht geredet im Verborgenen,*\\
\vin nicht in einem finsteren Winkel der Erde,\\
zu den Nachkommen Jakobs sagte ich nicht:*\\
„Sucht mich in der Öde“.\\
\vin Ich, der Herr, rede, was wahr ist.\\
\vin ich künde, was recht ist.\\

Versammelt euch, kommt alle und tretet herzu,*\\
ihr Entronnenen unter den Völkern!\\
\vin Legt eure Sache dar und bringt sie vor,*\\
\vin haltet miteinander Rat!\\
Wer hat dies hören lassen von alters her,*\\
wer hat es vor Zeiten schon kundgetan?\\
\vin War nicht ich es, der Herr?\\
\vin Es gibt ja keinen Gott außer mir,*\\
\vin es gibt neben mir keinen Gott, der gerecht ist und rettet!\\
Wendet euch zu mir und laßt euch retten, †\\
all ihr Enden der Erde,*\\
denn ich bin Gott – und sonst keiner.\\
\vin Ich habe bei mir selbst geschworen, †\\
\vin Wahrheit ging aus von meinem Munde,*\\
\vin ein Wort, das niemals widerrufen wird:\\
„Vor mir wird jedes Knie sich beugen, †\\
und jede Zunge wird mit einem Schwur bekennen:*\\
Im Herrn allein ist Heil und Stärke!“\\

\vin Es werden zu ihm kommen und sich schämen*\\
\vin alle, die gegen ihn entbrannten.\\
Alle Nachkommen Israels*\\
erlangen Heil und Ruhm im Herrn.\\

\end{verse}

\vspace{0.6cm}

%\cantlauii{Hab 3,2-4.13a.15-19}
\cantlauii{Hab 3,2-19}
\begin{verse}[\versewidth]
 

Herr, ich habe die Kunde vernommen,*\\
Herr, ich habe dein Werk geschaut.\\
\vin Inmitten der Jahre laß es geschehen, †\\
\vin inmitten der Jahre tue es kund,*\\
\vin im Zorn sei eingedenk deines Erbarmens!\\
Gott kommt heran von Teman,*\\
der Heilige vom Gebirge Paran.\\
\vin Seine Hoheit überkleidet den Himmel,*\\
\vin die Erde ist von seinem Ruhm erfüllt.\\
Wie das Licht ist sein Glanz, † \\
von seiner Hand gehen Strahlen aus:*\\
die Hülle seiner Macht.\\
\vin So ziehst du aus, dein Volk zu retten,*\\
\vin zu retten deinen Gesalbten.\\ 
Du bahnst mit deinen Rossen im Meer einen Weg,*\\
im Brausen der gewaltigen Wasser.\\
\vin Ich hab es vernommen, mir zittert der Leib,*\\
\vin die Lippen beben mir vor solcher Kunde,\\
mir fuhr der Fraß in die Glieder,*\\
es wankt mir der Schritt.\\
\vin Doch abwarten will ich in Ruhe den Tag der Bedrängnis,*\\
\vin der hereinbrechen wird über das Volk, das uns plündert.\\
Der Feigenbaum grünt nicht,*\\
an den Reben ist kein Ertrag,\\
\vin die Frucht des Ölbaums bleibt aus,*\\
\vin die Äcker bringen kein Brot,\\
die Schafe sind aus der Hürde verschwunden,*\\
und in den Ställen steht kein Rind mehr.\\ 
\vin Dennoch will ich über den Herrn frohlocken,*\\
\vin Will jubeln über den Gott meines Heils.\\
Der Herr, mein Gott, ist meine Kraft. †\\
Er macht meine Füße schnell wie die Füße der Hinden,*\\
Auf Höhen läßt er mich schreiten.\\
\end{verse}




\vspace{0.6cm}

\cantves{Ap 15, 3-4}
\begin{verse}[\versewidth]
Gross und wunderbar sind deine Werke,*\\
Herr, Gott, Allherrschender.\\
\vin Gerecht und wahr sind deine Wege,*\\
\vin du König der Völker.\\
Wer sollte dich nicht fürchten, oh Herr, †\\
und deinen Namen nicht preisen?*\\
Denn du allein bist heilig.\\
\vin Ja, alle Völker werden kommen und vor dir huldigen,*\\
\vin denn offenbar geworden sind deine gerechten Taten.\\
\end{verse}

\section{sabbato}

%\cantlaui{Ex 15,1-4a,8-13.17-18}
\cantlaui{Ex 15,1-18}
\begin{verse}[\versewidth]


Ich singe dem Herrn ein Lied, †\\
denn er ist hoch erhaben.*\\
Rosse und Wagen warf er ins Meer.\\
\vin Meine Stärke und mein Lied ist der Herr,*\\
\vin er ist mir zum Retter geworden.\\
Er ist mein Gott, ihn will ich preisen,*\\
meines Vaters Gott, ihn will ich erheben.\\
\vin Der Herr ist ein Held im Kriege,*\\
\vin „Herr“ ist sein Name.\\
Ins Meer warf er die Wagen des Pharao †\\
und all seine Streitmacht.*\\
Seine besten Kämpfer versanken im Schilfmeer.\\
\vin Deine Rechte, o Herr, ist gewaltig an Stärke,*\\
\vin deine Rechte, o Herr, zerschmettert den Feind.\\
In deiner Hoheit Fülle wirfst du deine Gegner nieder.*\\
Du sendest die Glut deines Zorns, die frißt sie wie Stoppeln.\\
\vin Du schnaubtest vor Wut, da türmten\\ 
\vin sich die Wasser, †\\
\vin wie ein Damm standen aufrecht die Wogen,* \\
\vin die Fluten erstarrten mitten im Meer.\\
Es sagte der Feind: †\\
„Ich will sie verfolgen, ich hole sie ein.*\\
Ich teile die Beute, ich stille meine Gier.\\
\vin Ich zücke mein Schwert,*\\
\vin meine Hand wird sie vertreiben.“\\
Da schnaubtest du Sturm - †\\
und schon begrub sie das Meer,*\\
sie versanken wie Blei in gewaltigen Wassern.\\

\rot{divisio}\\

Wer ist wie du, Herr, unter den Göttern? †\\
Wer ist wie du, gewaltig in Heiligkeit,*\\
schrecklich in ruhmvollen Taten, Wunder vollbringend?\\
\vin Du strecktest deine Rechte aus,*\\
\vin da verschlang sie die Erde.\\
Du lenktest das Volk, das du erlöst hast, in Güte*\\
und führtest sie mit Macht zu deiner heiligen Wohnung.\\
\vin Als die Völker das hörten, erbebten sie,*\\
\vin die Philister packte der Schrecken.\\
Es erschraken die Fürsten von Edom, †\\
Zittern ergriff die Mächtigen Moabs,*\\
die Bewohner Kanaans verzagten alle.\\
\vin Über sie brach Furcht und Entsetzen herein.*\\
\vin Sie erstarrten zu Stein vor der Macht deines Armes,\\
während dein Volk hindurchzog, o Herr,*\\
während das Volk hindurchzog, das du erworben.\\
\vin Du brachtest sie hin und pflanztest sie ein*\\
\vin Auf dem Berg deines Erbes:\\
Einen Ort, wo du thronst, Herr, hast du bereitet,*\\
Ein Heiligtum, Herr, haben deine Hände gegründet.\\
\vin Der Herr ist König*\\
\vin für immer und ewig!\\
 
\end{verse}
 
\end{verse}



\vspace{0.6cm}


\cantlauii{Dtn 32,1-12}

\begin{verse}[\versewidth]
 
Hört, ihr Himmel, denn ich will reden,*\\
es lausche die Erde den Worten meines Mundes!\\
\vin Wie Regen riesle meine Lehre,*\\
\vin wie Tau ergieße sich meine Rede,\\
wie Regenschauer auf grünende Saaten,*\\
wie Perlen von Tau auf den Fluren.\\
\vin Denn ausrufen will ich den Namen des Herrn:*\\
\vin Gebt unserem Gott die Ehre!\\
Er ist der Fels! Vollkommen ist sein Tun,*\\
recht sind all seine Wege.\\
\vin Ein Gott der Treue – ohne Falsch:*\\
\vin er ist gerecht und gerade.\\
Mißratene Kinder fielen von ihm ab,*\\
ein Geschlecht, das verkehrt und verdreht ist.\\
\vin So also vergiltst du dem Herrn,*\\
\vin du Volk voller Torheit und Unverstand?\\
Ist er nicht dein Vater, der dich geschaffen?*\\
Er selbst hat dich gemacht und hingestellt. \\

Gedenke der Tage der Vorzeit,*\\
Achte auf die Jahre der vergangnen Geschlechter.\\
\vin Frag deinen Vater, er wird dir's erzählen,*\\
\vin frage die Alten, sie werden es dir sagen. \\
Als der Höchste einst den Nationen zuwies ihr Erbland*\\
und voneinander schied die Kinder der Menschen,\\
\vin da legte er fest die Grenzen der Völker*\\
\vin nach der Zahl der Götter.\\
Aber der Anteil des Herrn ist sein Volk,*\\
Jakob ist sein ihm zugemessenes Erbe.\\
\vin Er las ihn auf in der Wüste,*\\
\vin In der Öde, im Geheul der Wildnis.\\
Er hegte ihn und gab auf ihn acht,*\\
er hütete ihn wie den Stern seines Auges.\\
\vin Wie ein Adler, der seine Brut zum Fluge aufstört*\\
\vin und schwebt über seinen Jungen,\\
so spreiteter die Flügel und nahm ihn auf*\\
und trug ihn auf seinen Schwingen.\\
\vin Der Herr allein hat ihn geleitet,*\\
\vin kein fremder Gott stand ihm zur Seite.\\

 
\end{verse}


\vspace{0.6cm}


\cantves{Phil 2,6-11}

\begin{verse}[\versewidth]


Er, der in Gottes Gestalt war,*\\
hielt nicht daran fest, Gott gleich zu sein,\\
\vin sondern er hat sich selbst entäußert*\\
\vin und nahm eines Sklaven Gestalt an:\\
uns Menschen wurde er gleich*\\
und seiner Erscheinung nach als Mensch erfunden.\\
\vin Er hat sich selbst erniedrigt †\\
\vin und wurde gehorsam bis zum Tod,*\\
\vin ja, bis zum Tod am Kreuze.\\
Darum hat Gott ihn auch erhöht †\\
und ihm verliehen den Namen,*\\
der über alle Namen erhaben ist.\\
\vin Damit im Namen Jesu jedes Knie sich beuge*\\
\vin im Himmel und auf Erden und in der Unterwelt\\
und jede Zunge bekenne: †\\
„Jesus Christus ist der Herr!“\\
zur Ehre Gottes des Vaters.\\	

\end{verse}