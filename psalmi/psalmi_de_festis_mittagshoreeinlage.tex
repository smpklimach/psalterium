

\section[]{FESTPSALMEN MITTAGSHORE}


\setspaceafterinitial{5.2mm plus 0em minus 0em}
\setspacebeforeinitial{4.2mm plus 0em minus 0em}
\def\greinitialformat#1{{\fontsize{40}{40}\selectfont #1}}
\gresetfirstlineaboveinitial{\small \textcolor{red}{Hochfest}}{}
\setaboveinitialseparation{0.72mm}
\setsecondannotation{\small j. T.}
\includescore{appendices/rectorpotens1.tex}

\vspace{1cm}

\setspaceafterinitial{4.2mm plus 0em minus 0em}
\setspacebeforeinitial{4.2mm plus 0em minus 0em}
\def\greinitialformat#1{{\fontsize{40}{40}\selectfont #1}}
\gresetfirstlineaboveinitial{\small \textcolor{red}{Fest}}{}
\setaboveinitialseparation{0.72mm}
\setsecondannotation{\small vij. T.}
\includescore{appendices/rectorpotens2.tex}

\vspace{0.2cm}

\begin{sloppypar}
{\noindent\rm Allmächtiger Lenker, wahrhafter Gott,
der du leitest den Wechsel der Dinge,
den Morgen mit Helle erleuchtest
und den Mittag in Hitze.}
\end{sloppypar}






{\setlength{\columnsep}{0.5cm}

\begin{multicols}{2}
\begin{verse}[\versewidth]
 

{\small{\frot{E}xstíngue flammas lítium\\
aufer calórem nóxium, \\
confer salútem córporum\\
verámque pacem córdium.\\!

\frot{P}ræsta, Pater piíssime,\\ 
Patríque compar Unice, \\
cum Spíritu Paráclito\\
regnans per omne s\'{æ}culum.\\ \frot{A}men.\\!}}

\end{verse}

\columnbreak
 
\begin{verse}[\versewidth]
 
{\footnotesize\rm{Lösche aus die Flammen des Streites,\\
entferne die schädliche Wärme,\\
gib dem Körper Wohlbefinden\\
und dem Herzen wahre Freude.\\
{\textcolor{white}x}\\!

	
	    
Verleihe es, gütigster Vater\\
und du, Eingeborener gleich dem Vater\\
mit dem Tröster, dem Hl. Geist\\
herrschend in alle Ewigkeit.\\ Amen.\\!}}


\end{verse} 
\end{multicols}
}

\vspace{1cm}

\begin{quote}
\begin{verse}

\psal{120}

\smallskip

Ich rief zum Herrn in meiner Not, *\\
und er hat mich erhört.\\
\vin Herr, rette mein Leben vor Lügnern, *\\
\vin rette es vor falschen Zungen!\\
Was soll er dir tun, was alles dir antun, *\\
du falsche Zunge?\\
\vin Scharfe Pfeile von Kriegerhand *\\
\vin und glühende Ginsterkohlen dazu.\\
Weh mir, dass ich als Fremder\\
in Meschech bin *\\
und bei den Zelten von Kedar wohnen muss!\\
\vin Ich muss schon allzu lange wohnen *\\
\vin bei Leuten, die den Frieden hassen.\\
Ich verhalte mich friedlich;\\
doch ich brauche nur zu reden, *\\
dann suchen sie Hader und Streit.\\!

\end{verse}

\newpage

\begin{verse}

\psal{121}

\smallskip


Ich hebe meine Augen auf zu den Bergen: *\\
Woher kommt mir Hilfe?\\
\vin Meine Hilfe kommt vom Herrn, *\\
\vin der Himmel und Erde gemacht hat.\\
Er lässt deinen Fuß nicht wanken; *\\
er, der dich behütet, schläft nicht.\\
\vin Nein, der Hüter Israels *\\
\vin schläft und schlummert nicht.\\
Der Herr ist dein Hüter, \\
der Herr gibt dir Schatten; *\\
er steht dir zur Seite.\\
\vin Bei Tag wird dir die Sonne nicht schaden *\\
\vin noch der Mond in der Nacht.\\
Der Herr behüte dich vor allem Bösen, *\\
er behüte dein Leben.\\
\vin Der Herr behüte dich,\\
\vin wenn du fortgehst und wiederkommst, *\\
\vin von nun an bis in Ewigkeit.\\!

\end{verse}

\begin{verse}

\psal{122}

\smallskip

Ich freute mich, als man mir sagte: *\\
``Zum Haus des Herrn wollen wir pilgern.''\\
\vin Schon stehen wir\\
\vin in deinen Toren, Jerusalem: †\\
\vin Jerusalem, du starke Stadt, *\\
\vin dicht gebaut und fest gefügt.\\
Dorthin ziehen die Stämme hinauf, \\
die Stämme des Herrn, †\\
wie es Israel geboten ist, *\\
den Namen des Herrn zu preisen.\\
\vin Denn dort stehen Throne bereit\\
\vin für das Gericht, *\\
\vin die Throne des Hauses David.\\
Erbittet für Jerusalem Frieden! *\\
Wer dich liebt, sei in dir geborgen.\\
\vin Friede wohne in deinen Mauern, *\\
\vin in deinen Häusern Geborgenheit.\\
Wegen meiner Brüder und Freunde *\\
will ich sagen: In dir sei Friede.\\
\vin Wegen des Hauses des Herrn,\\ 
\vin unseres Gottes, *\\
\vin will ich dir Glück erflehen.\\!

\end{verse}
\end{quote}