

\setspaceafterinitial{2.2mm plus 0em minus 0em}
\setspacebeforeinitial{2.2mm plus 0em minus 0em}

\def\greinitialformat#1{%
{\fontsize{39}{39}\selectfont #1}%
}

%\redlines

\gresetfirstlineaboveinitial{\small \bf 1. T.}{\small \bf 1. T.}
%\setaboveinitialseparation{0.52mm}
%\setsecondannotation{\footnotesize {{ iv. T.}}}

% We type a text in the top right corner of the score:
%\commentary{{\small \emph{Cf. Is. 30, 19 . 30 ; Ps. 79}}}

% and finally we include the score. The file must be in the same directory as this one.
\includescore{148/ps148.tex}
\vspace{0.3cm}
\begin{center}
 \textcolor{red}{\large \bf Psalmus 148}\\
Præconium Domini creatoris\\
\textit{\small Sedenti in throno et Agno: benedictio et honor et gloria et potestas in sæcula sæculorum. (Ap 5,13)}
\end{center}
\begin{verse}[\versewidth]
Laudáte Dóminum de cælis, *\\
laudáte eum, in excélsis.\\!
\vin Laudáte eum, omnes ángeli eius, *\\
\vin laudáte eum, omnes virtútes eius.\\!
Laudáte eum, sol et luna, *\\
laudáte eum, omnes stellæ et lumen.\\!
\vin Laudáte eum, cæli cælórum, *\\
\vin et aquæ omnes quæ super cælos sunt.\\!
Laudent nomen Dómini, *\\
quia ipse mandávit, et creáta sunt.\\!
\vin Státuit ea in ætérnum, et in s\'{æ}culum s\'{æ}culi; *\\
\vin præcéptum pósuit, et non præteríbit.\\!
Laudáte Dóminum de terra, *\\
dracónes et omnes abýssi,\\!
\vin ignis, grando, nix, fumus, *\\
\vin spíritus procellárum qui facit verbum eius,\\!
montes et omnes colles, *\\
ligna fructífera et omnes cedri,\\!
\vin béstiæ et univérsa pécora, *\\
\vin serpéntes et vólucres pennátæ.\\!
Reges terræ et omnes pópuli, *\\
príncipes et omnes iúdices terræ,\\!
\vin iúvenes et vírgines, *\\
\vin senes cum iunióribus,\\!
laudent nomen Dómini, *\\
quia exaltátum est nomen eius solíus.\\!
\vin Conféssio eius super cælum et terram, *\\
\vin et exaltávit cornu pópuli sui.\\!
Hymnus ómnibus sanctis eius, *\\
fíliis Israel, pópulo qui propínquus est ei.\\
\end{verse}
\vspace{1cm}


