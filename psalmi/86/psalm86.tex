

\setspaceafterinitial{2.2mm plus 0em minus 0em}
\setspacebeforeinitial{2.2mm plus 0em minus 0em}

\def\greinitialformat#1{%
{\fontsize{39}{39}\selectfont #1}%
}

%\redlines

\gresetfirstlineaboveinitial{\small \bf v. T.}{\small \bf v. T.}
%\setaboveinitialseparation{0.52mm}
%\setsecondannotation{\footnotesize {{ iv. T.}}}

% We type a text in the top right corner of the score:
%\commentary{{\small \emph{Cf. Is. 30, 19 . 30 ; Ps. 79}}}

% and finally we include the score. The file must be in the same directory as this one.
\includescore{86/gloriosadicta86.tex}
\vspace{0.3cm}
\begin{center}
 \textcolor{red}{\large \bf Psalmus 86}\\
Ierusalem omnium gentium mater\\
\textit{\small Illa quæ sursum Ierusalem libera est, quæ est mater nostra. (Gal 1,26)}
\end{center}
\begin{verse}[\versewidth]
Fundaménta eius in móntibus sanctis; †\\
díligit Dóminus portas Sion *\\
super ómnia tabernácula Iacob.\\!
\vin Gloriósa dicta sunt de te, *\\
\vin cívitas Dei.\\!
Memor ero Rahab et Babylónis inter sciéntes me; †\\
ecce Philist\'{æ}a et Tyrus cum Æthiópia, *\\
hi nati sunt illic.\\!
\vin Et de Sion dicétur: „Hic et ille natus est in ea, *\\
\vin et ipse firmávit eam Altíssimus.“\\!
Dóminus réferet in librum populórum: *\\
„Hi nati sunt illic.“\\!
\vin Et cantant sicut choros ducéntes: *\\
\vin „Omnes fontes mei in te.“\\
\end{verse}
\vspace{1cm}


