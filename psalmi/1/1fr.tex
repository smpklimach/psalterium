\setspaceafterinitial{2.2mm plus 0em minus 0em}
\setspacebeforeinitial{2.2mm plus 0em minus 0em}

\def\greinitialformat#1{%
{\fontsize{39}{39}\selectfont #1}%
}

%\redlines

\gresetfirstlineaboveinitial{\small \bf iv. T.}{\small \bf iv. T.}
%\setaboveinitialseparation{0.52mm}
%\setsecondannotation{\footnotesize {{ iv. T.}}}

% We type a text in the top right corner of the score:
%\commentary{{\small \emph{Cf. Is. 30, 19 . 30 ; Ps. 79}}}

% and finally we include the score. The file must be in the same directory as this one.
%\includescore{111/fidelia.tex}
\vspace*{\fill}

\begin{center}
 \textcolor{red}{\large \bf Psaume 1}\\
De duabus hominum viis\\
\textit{\small Beati qui, sperantes in crucem, in aquam descenderunt. (Ex auctore quodam II sæculi)}
\end{center}
\begin{verse}[\versewidth]
Heureux l’homme qui ne marche pas selon le conseil des méchants, $\dagger$\\
Qui ne s’arrête pas sur la voie des pécheurs, $\ast$\\
Et qui ne s’assied pas en compagnie des moqueurs,\\!
\vin  Mais qui trouve son plaisir dans la loi de l’Eternel, $\ast$\\
\vin  Et qui la médite jour et nuit!\\!
Il est comme un arbre planté près d’un courant d’eau, $\ast$\\
Qui donne son fruit en sa saison, \\!
\vin  Et dont le feuillage ne se flétrit point: $\ast$\\
\vin  Tout ce qu’il fait lui réussit. \\!
Il n’en est pas ainsi des méchants: $\ast$\\
Ils sont comme la paille que le vent dissipe. \\!
\vin  C’est pourquoi les méchants ne résistent pas au jour du jugement, $\ast$\\
\vin  Ni les pécheurs dans l’assemblée des justes;\\!
Car l’Eternel connaît la voie des justes, $\ast$\\
Et la voie des pécheurs mène à la ruine.

\end{verse}
\newpage


