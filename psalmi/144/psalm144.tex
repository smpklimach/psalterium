

\setspaceafterinitial{7.2mm plus 0em minus 0em}
\setspacebeforeinitial{2.2mm plus 0em minus 0em}

\def\greinitialformat#1{%
{\fontsize{39}{39}\selectfont #1}%
}

%\redlines

\gresetfirstlineaboveinitial{\small \bf 8. T.}{\small \bf 8. T.}
%\setaboveinitialseparation{0.52mm}
%\setsecondannotation{\footnotesize {{ iv. T.}}}

% We type a text in the top right corner of the score:
%\commentary{{\small \emph{Cf. Is. 30, 19 . 30 ; Ps. 79}}}

% and finally we include the score. The file must be in the same directory as this one.
\includescore{144/inaeternum.tex}
\vspace{0.3cm}
\begin{center}
 \textcolor{red}{\large \bf Psalmus 144 A}\\
Laus divinæ maiestatis\\
\textit{\small Iustus es, Domine, qui es et qui eras. (Ap 16,5)}
\end{center}
\begin{verse}[\versewidth]
Exaltábo te, Deus meus rex, †\\
et benedícam nómini tuo *\\
in s\'{æ}culum et in s\'{æ}culum s\'{æ}culi.\\
\vin Per síngulos dies benedícam tibi, †\\
\vin et laudábo nomen tuum *\\
\vin in s\'{æ}culum et in s\'{æ}culum s\'{æ}culi.\\
Magnus Dóminus et laudábilis nimis, *\\
et magnitúdinis eius non est investigátio.\\
\vin Generátio generatióni laudábit ópera tua, *\\
\vin et poténtiam tuam pronuntiábunt.\\
Magnificéntiam glóriæ maiestátis tuæ loquéntur, *\\
et mirabília tua enarrábunt.\\
\vin Et virtútem terribílium tuórum dicent, *\\
\vin et magnitúdinem tuam narrábunt.\\
Memóriam abundántiæ suavitátis tuæ eructábunt, *\\
et iustítia tua exsultábunt.\\
\vin Miserátor et miséricors Dóminus, *\\
\vin longánimis et multæ misericórdiæ.\\
Suávis Dóminus univérsis, *\\
et miseratiónes eius super ómnia ópera eius.\\
\vin Confiteántur tibi, Dómine, ómnia ópera tua; *\\
\vin et sancti tui benedícant tibi.\\
Glóriam regni tui dicant, *\\
et poténtiam tuam loquántur,\\
\vin ut notas fáciant fíliis hóminum poténtias tuas, *\\
\vin et glóriam magnificéntiæ regni tui.\\
Regnum tuum regnum ómnium sæculórum, *\\
et dominátio tua in omnem generatiónem et generatiónem.\\
\end{verse}
\vspace{1cm}



\setspaceafterinitial{2.2mm plus 0em minus 0em}
\setspacebeforeinitial{2.2mm plus 0em minus 0em}

\def\greinitialformat#1{%
{\fontsize{39}{39}\selectfont #1}%
}

%\redlines

\gresetfirstlineaboveinitial{\small \bf 4. T.}{\small \bf 4. T.}
%\setaboveinitialseparation{0.52mm}
%\setsecondannotation{\footnotesize {{ iv. T.}}}

% We type a text in the top right corner of the score:
%\commentary{{\small \emph{Cf. Is. 30, 19 . 30 ; Ps. 79}}}

% and finally we include the score. The file must be in the same directory as this one.
\includescore{144/custoditdominus2.tex}
\vspace{0.3cm}
\begin{center}
 \textcolor{red}{\large \bf Psalmus 144 B}\\
Laus divinæ maiestatis\\
\textit{\small Iustus es, Domine, qui es et qui eras. (Ap 16,5)}
\end{center}
\begin{verse}[\versewidth]
Fidélis Dóminus in ómnibus verbis suis, *\\
et sanctus in ómnibus opéribus suis.\\!
\vin Allevat Dóminus omnes qui córruunt, *\\
\vin et érigit omnes depréssos.\\!
Oculi ómnium in te sperant, *\\
et tu das illis escam in témpore opportúno.\\!
\vin Aperis tu manum tuam, *\\
\vin et imples omne ánimal in beneplácito.\\!
Iustus Dóminus in ómnibus viis suis, *\\
et sanctus in ómnibus opéribus suis.\\!
\vin Prope est Dóminus ómnibus invocántibus eum, *\\
\vin ómnibus invocántibus eum in veritáte.\\!
Voluntátem timéntium se fáciet, †\\
et deprecatiónem eórum exáudiet, *\\
et salvos fáciet eos.\\!
\vin Custódit Dóminus omnes diligéntes se, *\\
\vin et omnes peccatóres dispérdet.\\!
Laudatiónem Dómini loquétur os meum, †\\
et benedícat omnis caro nómini sancto eius *\\
in s\'{æ}culum et in s\'{æ}culum s\'{æ}culi.\\
\end{verse}
\vspace{1cm}


