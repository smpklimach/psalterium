


\begin{quote}
\begin{verse}

\psal{9b}

\smallskip
Der Herr aber thront für ewig; *\\
er stellt seinen Thron auf zum Gericht\textit{e}.\\
\vin Er richtet den Erdkreis gerecht, *\\
\vin er spricht den Völkern das Urteil,\\
\vin das sie verdienen.\\
So wird der Herr für den Bedrückten zur Burg,*\\
zur Burg in Zeiten der Not.\\
\vin Darum vertraut dir,\\ 
\vin wer deinen Namen kennt;*\\
\vin denn du, Herr,\\ 
\vin verlässt keinen, der dich sucht.\\
Singt dem Herrn, der thront auf dem Zion, *\\
verkündet unter den Völkern seine Taten!\\
\vin Denn er, der jede Blutschuld rächt,\\ 
\vin denkt an die Armen, *\\
\vin und ihren Notschrei vergisst er nicht.\\
Sei mir gnädig in meiner Not; *\\
Herr, sieh doch, wie sie mich hassen!\\
\vin Führ mich herauf\\ 
\vin von den Pforten des Todes, †\\
\vin damit ich all deinen Ruhm verkünde\\ 
\vin in den Toren von Zion *\\
\vin und frohlocke, weil du mir hilfst.\\
Völker versanken der Grube,\\
die sie selber gegraben; *\\
im Netz, das sie heimlich gelegt,\\
hat ihr Fuß sich verfangen.\\
\vin Kundgetan hat sich der Herr: \\ 
\vin Er hielt sein Gericht; *\\
\vin im eigenen Werk\\ 
\vin hat sich der Frevler verstrickt.\\
Hinabfahren müssen die Frevler\\
zum Totenreich, *\\
alle Heiden, die Gott vergessen.\\
\vin Doch der Arme ist nicht auf ewig vergessen,*\\
\vin des Elenden Hoffnung\\ 
\vin ist nicht für immer verloren.\\
\textit{Erheb dich, Herr,\\
damit nicht der Mensch triumphiert}, *\\
damit die Völker gerichtet werden\\
vor deinem Angesicht.\\
\vin Wirf Schrecken auf sie, o Herr! *\\
\vin Erkennen sollen die Völker:\\ 
\vin Sie sind nur Menschen.\\!


\end{verse}

\begin{verse}

\psal{10}
\smallskip

Herr, warum bleibst du so fern, *\\
verbirgst dich in Zeiten der Not?\\
\vin In seinem Hochmut\\ 
\vin quält der Frevler die Armen. *\\
\vin Er soll sich fangen in den Ränken,\\ 
\vin die er selbst ersonnen hat.\\
Denn der Frevler rühmt sich nach Herzenslust,*\\
er raubt, er lästert und verachtet den Herrn.\\
\vin Überheblich sagt der Frevler: †\\
\vin ``Gott straft nicht. Es gibt keinen Gott.'' *\\
\vin So ist sein ganzes Denken.\\
Zu jeder Zeit glückt ihm sein Tun. †\\
Hoch droben und fern von sich\\
wähnt er deine Gerichte. *\\
All seine Gegner faucht er an. \\
\vin Er sagt in seinem Herzen:\\ 
\vin ``Ich werde niemals wanken. *\\
\vin Von Geschlecht zu Geschlecht\\
\vin trifft mich kein Unglück.\\
Sein Mund ist voll Fluch \\ 
und Trug und Gewalttat; *\\
auf seiner Zunge sind Verderben und Unheil.\\
\vin Er liegt auf der Lauer in den Gehöften †\\
\vin und will den Schuldlosen heimlich ermorden; *\\
\vin seine Augen spähen aus nach dem Armen.\\
Er lauert im Versteck wie ein Löwe im Dickicht, †\\
er lauert darauf, den Armen zu fangen; *\\
er fängt den Armen\\ 
und zieht ihn in sein Netz.\\
\vin Er duckt sich und kauert sich nieder, *\\
\vin seine Übermacht bringt die Schwachen zu Fall.\\
Er sagt in seinem Herzen: ''Gott vergisst es,*\\
er verbirgt sein Gesicht, er sieht es niemals.``\\


\crot{divisio}


Herr, steh auf, Gott, erheb deine Hand, *\\
vergiss die Gebeugten nicht!\\
\vin Warum darf der Frevler Gott verachten, *\\
\vin und in seinem Herzen sagen:\\ 
\vin ''Du strafst nicht? ``\\
Du siehst es ja selbst; *\\
denn du schaust auf Unheil und Kummer.\\
\vin Der Schwache vertraut sich dir an; *\\
\vin du bist den Verwaisten ein Helfer.\\
Zerbrich den Arm des Frevlers und des Bösen,*\\
bestraf seine Frevel,\\
so dass man von ihm nichts mehr findet.\\
\vin Der Herr ist König für immer und ewig, *\\
\vin in seinem Land gehen die Heiden zugrunde.\\
Herr, du hast die Sehnsucht\\
der Armen gestillt, *\\
du stärkst ihr Herz, du hörst auf sie:\\
\vin Du verschaffst den Verwaisten\\ 
\vin und Bedrückten ihr Recht. *\\
\vin Kein Mensch mehr verbreite \\ \vin  Schrecken im Land\textit{e}.\\!

\end{verse}
\end{quote}