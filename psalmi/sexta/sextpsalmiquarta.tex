



\begin{quote}
\begin{verse}


\psal{11}
\smallskip
Beim Herrn finde ich Zuflucht. †\\
Wie könnt ihr mir sagen:*\\
In die Berge flieh wie ein Vogel?\\ \vin
Schon spannen die Frevler den Bogen, †\\ \vin
sie legen den Pfeil auf die Sehne, um \\ \vin aus dem Dunkel *\\ \vin
zu treffen die Menschen mit \\ \vin redlichem Herzen.\\
Gerät alles ins Wanken,*\\
was kann da der Gerechte noch tun?\\ \vin
Der Herr weilt in seinem heiligen Tempel,*\\ \vin
der Thron des Herrn ist im Himmel.\\
Seine Augen schauen herab,*\\
seine Blicke prüfen die Menschen.\\ \vin
Der Herr prüft Gerechte und Frevler;*\\ \vin
wer Gewalttat liebt, den hasst er aus \\ \vin tiefster Seele.\\
Auf die Frevler lasse er Feuer und \\ Schwefel regnen;*\\
sengender Wind sei ihr Anteil.\\ \vin
Denn der Herr ist gerecht, er liebt\\ \vin gerechte Taten;*\\ \vin
wer rechtschaffen ist, darf sein\\ \vin  Angesicht schauen.\\


\end{verse}




\begin{verse}

\psal{12}

\smallskip

Hilf doch, o Herr, die Frommen \\ schwinden dahin, *\\
unter den Menschen gibt es keine Treue mehr.\\ \vin
Sie lügen einander an, einer den andern, *\\ \vin
mit falscher Zunge und zwiespältigem \\ \vin  Herzen reden sie.\\
Der Herr vertilge alle falschen Zungen, *\\
jede Zunge, die vermessen redet.\\ \vin
Sie sagen: Durch unsre Zunge sind \\ \vin  wir mächtig; *\\ \vin
unsre Lippen sind unsre Stärke. Wer ist \\ \vin
 uns überlegen?\\
Die Schwachen werden unterdrückt, die \\ Armen seufzen. †\\
Darum spricht der Herr: Jetzt stehe ich auf, *\\ 
dem Verachteten bringe ich Heil.\\
\vin Die Worte des Herrn sind lautere Worte, †\\
\vin Silber, geschmolzen im Ofen, von \\ \vin Schlacken geschieden,*\\
\vin geläutert siebenfach.\\ 
Du, Herr, wirst uns behüten *\\
und uns vor  diesen Leuten 
für immer erretten,\\
\vin auch wenn die Frevler frei umhergehen *\\
\vin und unter den Menschen die Gemeinheit\\ \vin groß wird.\\

\end{verse}




\begin{verse}

\psal{13}
\smallskip
 Wie lange noch, Herr, vergisst du mich ganz? *\\
Wie lange noch verbirgst du dein \\Gesicht vor mir?\\ \vin
Wie lange noch muss ich Schmerzen ertragen \\  \vin in meiner Seele, †\\ \vin
in meinem Herzen Kummer Tag für Tag? *\\ \vin 
Wie lange noch darf mein Feind über \\ \vin mich triumphieren?\\
Blick doch her, erhöre mich, \textit{Herr,} mein Gott,\\ *
\textit{erleuchte meine Augen}, damit ich nicht \\ entschlafe und sterbe,\\ \vin
damit mein Feind nicht sagen kann: *   \\ \vin ``Ich habe ihn überwältigt``,\\
damit meine Gegner nicht jubeln, *\\weil ich ihnen  erlegen bin.\\
\vin Ich aber baue auf deine Huld, *\\ \vin
mein Herz soll über deine Hilfe frohlocken.\\ 
Singen will ich dem Herrn, *\\weil er mir  Gutes getan hat.\\
\end{verse}

\newpage

\begin{verse}

\psal{15}

\smallskip
 Herr, wer darf Gast sein in deinem Zelt,*\\
wer darf weilen auf deinem heiligen Berg\textit{e}?\\ \vin
Der makellos lebt,*\\ \vin
und das Rechte tut;\\
der von Herzen die Wahrheit sagt  *\\
und mit seiner Zunge nicht verleumdet;\\ \vin
der seinem Freund nichts Böses antut *\\ \vin
und seinen Nächsten nicht schmäht;\\
der den Verworfenen verachtet, doch alle,*\\
die den Herrn fürchten, in Ehren hält;\\ \vin
der sein Versprechen nicht ändert,*\\ \vin
das er seinem Nächsten geschworen hat;\\
der sein Geld nicht auf Wucher ausleiht *\\
und nicht zum Nachteil des Schuldlosen\\ Bestechung annimmt.\\ \vin
Wer sich danach richtet,*\\ \vin
der wird niemals wanken.\\
\end{verse}


\end{quote}