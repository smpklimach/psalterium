

\setspaceafterinitial{2.2mm plus 0em minus 0em}
\setspacebeforeinitial{2.2mm plus 0em minus 0em}

\def\greinitialformat#1{%
{\fontsize{39}{39}\selectfont #1}%
}

%\redlines

\gresetfirstlineaboveinitial{\small \bf viij. T.}{\small \bf viij. T.}
%\setaboveinitialseparation{0.52mm}
%\setsecondannotation{\footnotesize {{ iv. T.}}}

% We type a text in the top right corner of the score:
%\commentary{{\small \emph{Cf. Is. 30, 19 . 30 ; Ps. 79}}}

% and finally we include the score. The file must be in the same directory as this one.
\includescore{136/hymnumcantate.tex}
\vspace{0.3cm}
\begin{center}
 \textcolor{red}{\large \bf Psalmus 136}\\
Super flumina Babylonis\\
\textit{\small Hanc corporalem populi captivitatem referre in exemplum spiritalis captivitatis oportet. (S. Hilarius)}
\end{center}
\begin{verse}[\versewidth]
Super flúmina Babylónis, illic sédimus et flévimus, *\\
cum recordarémur Sion.\\
\vin In salícibus in médio eius *\\
\vin suspéndimus cítharas nostras.\\
Quia illic rogavérunt nos, qui captívos duxérunt nos, *\\
verba cantiónum,\\
\vin et qui affligébant nos, lætítiam: *\\
\vin „Cantáte nobis de cánticis Sion.“\\
Quómodo cantábimus cánticum Dómini *\\
in terra aliéna?\\
\vin Si oblítus fúero tui, Ierúsalem, *\\
\vin obliviscátur sui déxtera mea;\\
adh\'{æ}reat lingua mea fáucibus meis, *\\
si non memínero tui,\\
\vin si non præposúero Ierúsalem *\\
\vin in cápite lætítiæ meæ.\\
Memor esto, Dómine, advérsus fílios Edom *\\
diéi Ierúsalem;\\
\vin qui dicébant: „Exinaníte, exinaníte *\\
\vin usque ad fundaméntum in ea.“\\
Fília Babylónis devástans, *\\
beátus qui retríbuet tibi retributiónem tuam quam retribuísti nobis,\\
\vin beátus qui tenébit *\\
\vin et allídet párvulos tuos ad petram.\\
\end{verse}
\vspace{1cm}


